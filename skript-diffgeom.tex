%%
%% Skript Differentialgeometrie im Wintersemester 12/13
%% Zur Vorlesung von Dr. Grensing am KIT Karlsruhe
%%
%% Mitschrieb und Textsatz von Jan-Bernhard Kordaß.
%%
\documentclass[a4paper, twoside, 11pt]{scrbook}

\usepackage[utf8x]{inputenc}
\usepackage[T1]{fontenc}
\usepackage{lmodern}

\usepackage[ngerman]{babel}
\usepackage[a4paper, top=2.5cm, bottom=3cm, left=2.5cm, right=4.5cm]{geometry}

\usepackage{fancyhdr} % erlaubt mehr Optionen in Kopf- und Fusszeile

% header configuration
%\pagestyle{fancy}
%\fancyhf{
%\lhead[\thepage]{\rightmark}}
%\rhead[\nouppercase{\leftmark}]{\thepage}		

\usepackage{xcolor} % Farben
\usepackage{marginnote} % Randnotizen
\usepackage{enumitem} % Fuer mehr Einstellungmoeglichkeiten bei Aufzaehlungen
\usepackage{xifthen} % Erlaubt die Verwendung von if-then-else Befehlen im Code
\usepackage{index} % Index erzeugen
\newindex{default}{idx}{ind}{Stichwortverzeichnis}
\usepackage{xspace} % intelligende Leerzeichen bei Macros
\usepackage[normalem]{ulem} % unterstreichen von Text
\usepackage{cancel} % schraeg durchstreichen von Text

\renewcommand{\CancelColor}{\color{gray}} % Farbe zum schraegen Druchstreichen in grau

\definecolor{rltred}{rgb}{0.75,0,0}
\definecolor{rltgreen}{rgb}{0,0.5,0}
\definecolor{rltblue}{rgb}{0,0,0.75}

%sichere Fraben, die sich auch bei einem SW-Druck unterscheiden lassen (Platzhalter momentan)
\definecolor{color1}{cmyk}{1,0,0,0} %cyan
\definecolor{color2}{rgb}{0,1,0} %green

\usepackage[hyperindex=true]{hyperref} % Verweise als Hyperlinks
\hypersetup{
  pdftitle={Differentialgeometrie Dr. Grensing},
  pdfsubject={Differentialgeometrie Geometrie},
  pdfkeywords={Differentialgeometrie Grensing},
  pdfproducer={pdfLaTeX},
  pdfpagemode={UseOutlines},
  colorlinks=true,
  bookmarksopen=true,
  bookmarksnumbered=true,
  urlcolor=rltblue,
  filecolor=rltgreen,
  linkcolor=rltblue,
  backref=true,
  pagebackref=true,
  pdfpagemode=None,
  citecolor=rltblue
}

% vertausche die Theta, Phi, Rho und Epsilon mit ihrer "var" Version
%\newcommand{\swapcmd}[2]{
%	\let\temp\#1
%	\left\#1\#2
%	\let\#2\temp
%}
\let\temp\phi
\let\phi\varphi
\let\varphi\temp

\let\temp\theta
\let\theta\vartheta
\let\vartheta\temp

\let\temp\epsilon
\let\epsilon\varepsilon
\let\varepsilon\temp

\let\temp\rho
\let\rho\varrho
\let\varrho\temp

\usepackage{tikz} % Fuer Zeichnungen in TikZ
\usetikzlibrary{matrix,arrows,calc,intersections, positioning, patterns, decorations.text, decorations.pathmorphing}

% neue Befehle fuer haeufig benutzte TikZ Formen; erstes Argument steht fuer die Position, Zweites fuer die Groesse
\newcommand{\tikzrichtung}[3][1]{ % zeichnet eine rote Linie von einem Punkt in eine Richtung mit rotem Knoten am Ende
	\draw[red] #2 -- ($#2 + #1*#3$) circle(0.05);
}
\newcommand{\tikzschnuller}[2][1]{
	% definiere die Knoten relativ zum ersten Knoten skaliert mit dem Faktor
	\coordinate (schnuller1) at #2; \coordinate (schnuller2) at ($(schnuller1)+#1*(-1.75,-0.75)$); \coordinate (schnuller3) at ($(schnuller1)+#1*(-2.5,-2.25)$); \coordinate (schnuller4) at ($(schnuller1)+#1*(0,-2)$); \coordinate (schnuller5) at ($(schnuller1)+#1*(1.75,-0.25)$);
    %\fill (schnuller1) circle (0.05) (schnuller2) circle (0.05) (schnuller3) circle (0.05) (schnuller4) circle (0.05) (schnuller5) circle (0.05);
    
    % die Richtungsvektoren der Bezier Tangenten fuer die einzelnen Knoten (der Erste und der letzte haben keine Tangente)
    \coordinate (ctrls1) at ($#1*(1.25,0.25)$); \coordinate (ctrls2) at ($-0.5*(ctrls1)$); \coordinate (ctrls4) at ($#1*(1,-1)$); \coordinate (ctrls3) at ($-0.5*(ctrls4)$); \coordinate (ctrls6) at ($#1*(1,1.5)$); \coordinate (ctrls5) at ($-0		.33*(ctrls6)$);
	% die eigentlichen Tangenten
    \coordinate (tang1) at ($(schnuller2)+(ctrls1)$); \coordinate (tang2) at ($(schnuller2)+(ctrls2)$); \coordinate (tang3) at ($(schnuller3)+(ctrls3)$); \coordinate (tang4) at ($(schnuller3)+(ctrls4)$); \coordinate (tang5) at ($(schnuller4)+(ctrls5)$); \coordinate (tang6) at ($(schnuller4)+(ctrls6)$);
    %\fill[red] (tang1) circle (0.05); \fill[red] (tang2) circle (0.05); \fill[red] (tang3) circle (0.05); \fill[red] (tang4) circle (0.05); \fill[red] (tang5) circle (0.05); \fill[red] (tang6) circle (0.05);
    %\draw[red] (tang1) -- (tang2); \draw[red] (tang3) -- (tang4); \draw[red] (tang5) -- (tang6);
	
	\draw (schnuller1) ..controls(schnuller1) and (tang1).. (schnuller2) ..controls(tang2) and (tang3).. (schnuller3) ..controls(tang4) and (tang5).. (schnuller4) ..controls(tang6) and (schnuller5).. (schnuller5);
	
	% zeichne nun das Loch in der Mitte
	\def\angle{20} % Rotationswinkel
	\coordinate (c) at ($#2+#1*(-1.25,-1.25)$); % Mittelpunkt der Ellipse die den unteren Bogen bildet
	\begin{scope}
		\clip[rotate=\angle] ($(c)-#1*(1,0.6)$) rectangle ($(c)+#1*(1,-0.1)$);
		\path[draw,rotate=\angle,name path=l] (c) ellipse(#1*1 and #1*0.5);
	\end{scope}
	\path[name path=u,rotate=\angle] ($(c)-#1*(0,0.5)$) ellipse(#1*0.75 and #1*0.5);
	\path[name intersections={of=u and l}];
	\begin{scope}
		\clip[rotate=\angle] (intersection-1) rectangle ($(intersection-2)+#1*(0,0.5)$);
		\draw[rotate=\angle] ($(c)-#1*(0,0.5)$) ellipse(#1*0.75 and #1*0.5);
	\end{scope}		
}
%\newcommand{\tikzkartoffel}[2][1]{}		
\newcommand{\tikzsegel}[2][1]{
	% definiere die Knoten relativ zum ersten Knoten skaliert mit dem Faktor
	\coordinate (segel1) at #2; \coordinate (segel2) at ($(segel1)+#1*(4,1.5)$); \coordinate (segel3) at ($(segel1)+#1*(2,-0.5)$);
	%\fill (segel1) circle (0.05) (segel2) circle (0.05) (segel3) circle (0.05);
	
	% die Richtungsvektoren der Bezier Tangenten fuer die einzelnen Knoten (der Erste und der letzte haben keine Tangente)
	\coordinate (ctrls1) at ($#1*(0.75,1.5)$); \coordinate (ctrls2) at ($#1*(-0.75,0.25)$); \coordinate (ctrls3) at ($#1*(-0.5,-0.25)$); \coordinate (ctrls4) at ($#1*(0.25,1)$); \coordinate (ctrls5) at ($#1*(-0.375,0.375)$); \coordinate (ctrls6) at ($#1*(0.75,0.125)$);
	% die eigentlichen Tangenten
	\coordinate (tang1) at ($(segel1)+(ctrls1)$); \coordinate (tang2) at ($(segel2)+(ctrls2)$); \coordinate (tang3) at ($(segel2)+(ctrls3)$); \coordinate (tang4) at ($(segel3)+(ctrls4)$); \coordinate (tang5) at ($(segel3)+(ctrls5)$); \coordinate (tang6) at ($(segel1)+(ctrls6)$);
%	\fill[red] (tang1) circle (0.05); \fill[red] (tang2) circle (0.05); \fill[red] (tang3) circle (0.05); \fill[red] (tang4) circle (0.05); \fill[red] (tang5) circle (0.05); \fill[red] (tang6) circle (0.05);
 %   \draw[red] (tang1) -- (segel1) -- (tang6); \draw[red] (tang2) -- (segel2) -- (tang3); \draw[red] (tang4) -- (segel3) -- (tang5);
	
	\draw (segel1) ..controls(tang1) and (tang2).. (segel2) ..controls(tang3) and (tang4).. (segel3) ..controls(tang5) and (tang6).. (segel1) --cycle;
}
\newcommand{\tikztorus}[2][1]{
	% \draw[step=0.25,gray!15] (-6,-1) grid (6,5); \draw[step=0.5,gray!30] (-6,-1) grid (6,5); \fill (0,0) circle(0.1); %Hilfsgitter
	% zuerst die aeussere Ellips
	\draw[] #2  ellipse (#1*2 and #1*1);
	
	% dann das Loch
	\begin{scope}
      \clip ($#2 - #1*(1, 0.5)$) rectangle ($#2 + #1*(1, 1)$);
      \path[draw,name path=gkreis] ($#2 + #1*(0,0.75)$) ellipse (#1*1.25 and #1*1);
    \end{scope}
    \path[name path=kkreis] ($#2 - #1*(0,0.5)$) ellipse (#1*1 and #1*0.75);
    \path[name intersections={of=gkreis and kkreis}];
    \begin{scope}
      \clip (intersection-1) rectangle ($(intersection-2)+(0,0.5)$);
      \draw ($#2 - #1*(0,0.5)$) ellipse (#1*1 and #1*0.75);
    \end{scope}
    
    % definiere Werte auf die wir in der restlichen Zeichnung zurueckgreifen koennen
	\def\torusbreite{#1*2}
	\def\torushoehe{#1*1}
	\def\torusdicke{#1*0.75}
	\coordinate (torusUntenLoch) at ($#2 - #1*(0,0.25)$);
	\coordinate (torusUnten) at ($#2 - #1*(0,1)$);
}

\usepackage[toc]{glossaries} % Symbolverzeichnis
\glossarystyle{treehypergroup}
\makeglossaries

% Mathe Pakete
\usepackage{amsmath}
\usepackage{amssymb}
\usepackage{stmaryrd}
\usepackage{bm} % fette Mathe Zeichen
%\usepackage{amsthm}
\usepackage[hyperref,amsmath,thmmarks,thref]{ntheorem}

% common mathematical operators and sets
\DeclareMathOperator{\aff}{aff} % affine Huelle
\DeclareMathOperator{\ddet}{det} % Determinante
\DeclareMathOperator{\diam}{diam} % diameter
\DeclareMathOperator{\dist}{dist} % distance
\DeclareMathOperator{\ddim}{dim} % dimension
\DeclareMathOperator{\dR}{dR} % deRahm
\DeclareMathOperator{\ggT}{ggT} % goesster gemeinsamer Teiler
\DeclareMathOperator{\id}{id} % identity
\DeclareMathOperator{\inh}{inh} % Inhalt
\DeclareMathOperator{\grad}{grad} % Gradient
\DeclareMathOperator{\kgV}{kgV} % kleinstes gemeinsames Vielfaches
\DeclareMathOperator{\mspan}{span} % Lineare Huelle
\DeclareMathOperator{\n}{n} % Umlaufzahl
\DeclareMathOperator{\offen}{offen}
\DeclareMathOperator{\pr}{pr}
\DeclareMathOperator{\res}{res} % Residuum
\DeclareMathOperator{\rg}{rg} % rank (i)
\DeclareMathOperator{\scal}{scal} % Skalarkruemmung
\DeclareMathOperator{\sgn}{sgn} % Signum
\DeclareMathOperator{\spur}{spur} % Spur
\DeclareMathOperator{\supp}{supp} % support
\DeclareMathOperator{\sternf}{sternf}
\DeclareMathOperator{\tr}{tr} % Spur

\DeclareMathOperator{\Abb}{Abb} % maps
\DeclareMathOperator{\Aut}{Aut} % automorphisms
\DeclareMathOperator{\Bild}{Bild}
\DeclareMathOperator{\Charakteristik}{char}
\DeclareMathOperator{\Charakt}{char}
\DeclareMathOperator{\D}{D} % Jacobi matrix or derivative
\DeclareMathOperator{\Diff}{Diff}
\DeclareMathOperator{\End}{End} % endomorphisms
\DeclareMathOperator{\Gl}{GL} % general linear group
\DeclareMathOperator{\GL}{GL} % general linear group
\DeclareMathOperator{\Gr}{Gr}
\DeclareMathOperator{\Graph}{Graph}
\DeclareMathOperator{\Hom}{Hom} % homomorphisms
\DeclareMathOperator{\Id}{id} % identity
\DeclareMathOperator{\Inn}{Inn} % Untergruppe der inneren Automorphismen
\DeclareMathOperator{\Kern}{Kern}
\DeclareMathOperator{\Oo}{O} % Matrizen sie mit ihrer Transponierten multipiziert die Einheitsmatrix ergeben
\DeclareMathOperator{\Relation}{\scriptstyle\mathrm{R}} % custom Relation
\DeclareMathOperator{\Rang}{Rang} % rank (ii)
\DeclareMathOperator{\SL}{SL} % Matrizen mit Deteminante 1
\DeclareMathOperator{\Stab}{Stab} % Stabilisator
\DeclareMathOperator{\Sym}{Sym} % symmetric group
\DeclareMathOperator{\T}{T} % tangent bundle

\DeclareMathOperator{\ric}{ric} % Ricci Tensor
\DeclareMathOperator{\Ric}{Ric} % Ricci Tensor field

\newcommand{\Zentrum}[1]{\ensuremath{\mathrm Z(#1)}} % Zentrum einer Gruppe
\newcommand{\Ordnung}[1][]{ % Ordnung einer Gruppe
  \ifthenelse{\isempty{#1}}{
    \#
  }{
    \left|#1\right|
  }
}

% X als Malzeichen
\newcommand{\X}{\times}

% ein schoener aussehender Faktorraum anstatt einfach nur A/B
\newcommand{\FakRaum}[2]{
	\raisebox{0.7ex}{\ensuremath{#1}}
	\ensuremath{\mkern-3mu}\big/\ensuremath{\mkern-3mu}
	\raisebox{-0.6ex}{\ensuremath{#2}}}
\newcommand{\smallFakRaum}[2]{
	\scriptsize{\raisebox{0.7ex}{\ensuremath{#1}}
	\ensuremath{\mkern-3mu}\ / \ensuremath{\mkern-3mu}
	\raisebox{-0.6ex}{\ensuremath{#2}}}}

%Realteil und Imaginaerteil
\renewcommand{\Re}{\ensuremath{\operatorname{Re}}} % <-- sollte man da nicht besser \DeclareMathOperator verwenden?
% [kann man nicht, weil \Re und \Im schon deklariert sind
% einen "\ReDeclareMathOperator" Befehl gibt es nicht. JB]
\renewcommand{\Im}{\ensuremath{\operatorname{Im}}}

% \DeclareMathOperator{\Real}{Re} % real part
% \DeclareMathOperator{\Imag}{Im} % imaginary part

% canonic sets
\DeclareMathOperator{\C}{\mathbb{C}}
\DeclareMathOperator{\F}{\mathbb{F}}
\DeclareMathOperator{\K}{\mathbb{K}}
\DeclareMathOperator{\N}{\mathbb{N}}
\DeclareMathOperator{\Q}{\mathbb{Q}}
\DeclareMathOperator{\R}{\mathbb{R}}
\DeclareMathOperator{\RP}{\mathbb{RP}} % real projection plane
\DeclareMathOperator{\Tor}{\mathbb{T}} % torus
\DeclareMathOperator{\Z}{\mathbb{Z}}
\DeclareMathOperator{\B}{\mathbb{B}} % unit ball

%  geschwungene Buchstaben
\DeclareMathOperator{\calD}{\mathcal{D}}
\DeclareMathOperator{\calI}{\mathcal{I}}
\DeclareMathOperator{\calJ}{\mathcal{J}}
\DeclareMathOperator{\calL}{\mathcal{L}}
\DeclareMathOperator{\calT}{\mathcal{T}}
\DeclareMathOperator{\calV}{\mathcal{V}}

% Redeclare \P (Prim or Propability) and put the old, reversed "breakline P" in \BreakLineP
\let\BreakLineP\P
\renewcommand{\P}{\ensuremath{\mathbb{P}}}

% Differentialoperatoren als Brüche
\newcommand{\dop}{\mathrm{d}}	
\newcommand{\difffrac}[3][]{\ifthenelse{\isempty{#1}}{\frac{\dop #2}{\dop #3}}{\left. \frac{\dop #2}{\dop #3} \right|_{#1}}}
\newcommand{\pdifffrac}[3][]{\ifthenelse{\isempty{#1}}{\frac{\partial #2}{\partial #3}}{\left. \frac{\partial #2}{\partial #3} \right|_{#1}}}

% stellt einen großen vertikalen Strich an einen Term, nuetzlich in Bruechen
\newcommand{\bigvert}[1]{\left. #1 \right|}

% quotient space or group
\newcommand{\modulo}[1]{\ensuremath{/_{\displaystyle #1}}}

% declaring Index for group theory
\newcommand{\Index}[2]{\ensuremath{(#1 \SlimDdot #2)}}


% canonic environments
\newcounter{thmglobal}
%\swapnumbers
\theoremstyle{plain}

%%%%%%%%%%%%%%%%%%%%%%%%%%%%%%%%%%%%%%%%%%%%%%%%%%%%%%%%%%%%%%%%%%%%%%%%%%%%%%%%%%%%%%%%%%%%%%%%%%%%%%%%%%%%%%%%%%%%%%%%%%%%%%%%%%%%%%%%%%%%%%%%%%%%%%%%%%%%%%%%%%%%%%%%

\makeatletter

% Options
\newboolean{enableDeepNumbering}
\setboolean{enableDeepNumbering}{false}

\DeclareOption{deepnum}{
  \setboolean{enableDeepNumbering}{true}
}

\newboolean{enableMarginThm}
\setboolean{enableMarginThm}{false}

\DeclareOption{marginthm}{
  \setboolean{enableMarginThm}{true}
}

\ProcessOptions\relax

% call makeindex for an index register
%\makeindex

% Name language settings

% theorem names, ngerman
\newcommand{\cmLangThmSatz}{Satz\xspace}
\newcommand{\cmLangThmLemma}{Lemma\xspace}
\newcommand{\cmLangThmKor}{Korollar\xspace}
\newcommand{\cmLangThmProp}{Proposition\xspace}

\newcommand{\cmLangThmDfn}{Definition\xspace}
\newcommand{\cmLangThmBsp}{Beispiel\xspace}

\newcommand{\cmLangThmBem}{Bemerkung\xspace}

% short forms, ngerman
\newcommand{\cmLangThmShortSatz}{Satz\xspace}
\newcommand{\cmLangThmShortLemma}{Lemma\xspace}
\newcommand{\cmLangThmShortKor}{Kor\xspace}
\newcommand{\cmLangThmShortProp}{Prop\xspace}

\newcommand{\cmLangThmShortDfn}{Def\xspace}
\newcommand{\cmLangThmShortBsp}{Bsp\xspace}

\newcommand{\cmLangThmShortBem}{Bem\xspace}


\theoremstyle{plain}
\newtheorem{Dfn}{\cmLangThmDfn}[chapter]
\newtheorem{Satz}[Dfn]{Satz}
\newtheorem{Lemma}[Dfn]{\cmLangThmLemma}
\newtheorem{Kor}[Dfn]{\cmLangThmKor}
\newtheorem{Prop}[Dfn]{\cmLangThmProp}
\theorembodyfont{\normalfont}
\newtheorem{Bsp}[Dfn]{\cmLangThmBsp}
\newtheorem{Bem}[Dfn]{\cmLangThmBem}
\newtheorem{Aufg}{Aufgabe}
\newtheorem{Loes}{L\"osung}

\theoremstyle{nonumberplain}
\newtheorem{dfn}{\cmLangThmDfn}
\newtheorem{satz}{Satz}
\newtheorem{lemma}{\cmLangThmLemma}
\newtheorem{kor}{\cmLangThmKor}
\newtheorem{prop}{\cmLangThmProp}

\newtheorem{bsp}{\cmLangThmBsp}
\newtheorem{bem}{\cmLangThmBem}

\theoremsymbol{\ensuremath{\Box}}
\theorembodyfont{\normalfont}
\newtheorem{bew}{Beweis}

\theoremsymbol{}
\theoremstyle{empty}
\newtheorem{emptythm}{}% druckt nur den optionalen Namen aus

\theoremstyle{break}

% Add unnumbered Theorems, use amsthm style in both style modes
\theoremstyle{plain}
%\newtheorem*{satz*}{\cmLangThmSatz}
%\newtheorem*{lemma}{\cmLangThmLemma}
%\newtheorem*{kor*}{\cmLangThmKor}
%\newtheorem*{prop*}{\cmLangThmProp}

\theoremstyle{definition}
%\newtheorem*{dfn}{\cmLangThmDfn}
%\newtheorem*{bsp*}{\cmLangThmBsp}

\theoremstyle{remark}
%\newtheorem*{bem*}{\cmLangThmBem}
\newtheorem*{beh*}{Behauptung}


% 2-level numbering$
\numberwithin{thmglobal}{section}

% check if 3-level numbering is enabled
\ifthenelse{\boolean{enableDeepNumbering}}{
  \numberwithin{thmglobal}{subsection}
}{}


% some other customisations

% changing enumerations
\setlist[enumerate]{label=(\arabic*), itemsep=0cm, leftmargin=2cm}
\setlist[itemize]{itemsep=0cm} %\setlist[itemize]{itemsep=0cm, leftmargin=2cm}

% replace the slim emptyset symbol
\let\emptyset\varnothing

% set line distances
\linespread{1.1}

% Add a ':' for mathmode with tiny whitespaces around
\newcommand{\SlimDdot}{\ensuremath{\mathrm{:}}}


% headline and cover generation commands

% generate a simple headline
% usage: \CmHeadline[date]{title}{topic}{author}
\newcommand{\CmHeadline}[4][]{
  \begin{minipage}[t]{\textwidth}
    \huge{\textbf{#2}}\\
    \large{#3, #4}\relax
    \ifthenelse{\isempty{#1}}{}{\relax\large{, #1}}
  \end{minipage}
}

% generate a simple cover page
% usage: \CmCover[type(,skript)]{title}{subtitle}{date}
\newcommand{\CmCover}[4][]{
  \thispagestyle{empty}
  \begin{titlepage}
    \begin{center}
      \begin{minipage}[b]{0.8\textwidth}
	\vspace*{5cm}
        \ifthenelse{\isempty{#1}}{
          % Default cover arrangement
          \Huge{\textbf{#2}}\\[0.5cm]
          \huge{#3}\\[0.8cm]
          \Large{#4}
        }{
          \ifthenelse{\equal{#1}{skript}}{
            % Cover for lecture scripts
            \huge{#3}\\[0.5cm]
            \Huge{\textbf{#2}}\\[0.5cm]
            \Large{#4}
          }{}
        }
      \end{minipage}
    \end{center}
  \end{titlepage}
  \pagebreak
}


% indexing support

% Print and index given text
% usage: \CmIndex{[(optionally put another text for the index in here)]{(text to print and add to index)}
\newcommand{\CmIndex}[2][]{\ifthenelse{\isempty{#1}}{\index{#2}}{\index{#1}}#2}

% Highlight(bold) and index the given text
% usage: \CmMark[(optionally put another text for the index in here)]{(text to highlight and add to index)}
\newcommand{\CmMark}[2][]{\textbf{\CmIndex[#1]{#2}}}


% sectioning support

% Prints a description for a section in italic, bold. Most likely to use right under \section.
% usage: \CmSectionDescription{(short description of section contents)}
\newcommand{\CmSectionDescription}[1]{
  \vspace{-0.3cm}
  \hangindent=0.4cm
  \hangafter=0
  \begin{itshape}
    \textbf{#1}
  \end{itshape}
  \vspace{0.3cm}
}

% Starts a new paragraph inside of a theorem environment (as defined above)
% usage \CmSubThm[(paragraph title)]
\newenvironment{CmSubThm}[1]{
  \begin{itemize}[leftmargin=0.5cm,label=]
  \item
    \ifthenelse{\isempty{#1}}{}{
      \hspace{-0.5cm}(\textit{#1})\\[0.2cm]
    }
  }{
  \end{itemize}
}


% svg updater
% needs shell escape option

% Checks if the given image file has been modified and a custom command (if possible) via command line to generate something new.
\newcommand{\CmExecuteIfFileNewer}[3]{
  \ifnum\pdfstrcmp{\pdffilemoddate{#1}}
  {\pdffilemoddate{#2}}>0
  {\immediate\write18{#3}}\fi
}

% Tries to include an image file and checks if the given one has been modified. If so it calls inkscape (if possible) via command line to generate new pdf und pdf_tex files from the corresponding svg.
\newcommand{\CmIncludeSvg}[1]{
  \def\svg@filepath{}
  
  % check if corrosponding svg file exists
  \IfFileExists{#1.svg}
  {
    \def\svg@filepath{#1}
  }{
    % if it does not, search in the graphicspath for it
    \expandafter\@tfor\expandafter\currentsvgpath\expandafter:\expandafter=\Ginput@path\do{
      \IfFileExists{\currentsvgpath#1.svg}{
        \edef\svg@filepath{\currentsvgpath #1}
      }{}
    }
  }
  % if something was found, include the graphic, TODO: Make it work correctly
  \ifthenelse{\isundefined{\svg@filepath} \OR \isempty{\svg@filepath}}{
    \PackageError{canonicalmath}{Image file not found!}
  }{
    \PackageWarning{FilePath}{|\svg@filepath|}
    \CmExecuteIfFileNewer{\svg@filepath.svg}{\svg@filepath.pdf}{inkscape -z -D --file=\svg@filepath.svg --export-pdf=\svg@filepath.pdf --export-latex}
    \input{\svg@filepath.pdf_tex}
  }
}

% Tries to include an image file on the center of the margin at the current position.
\newcommand{\CmMarginSvg}[3][0cm]{
  \marginnote{
    \centering
    \def\svgwidth{#3}
    \CmIncludeSvg{#2}
  }[#1]
}

% Tries to include an image file centering it at the current position
\newcommand{\CmPutSvg}[3][0cm]{
  \begin{figure}[h!]
    \vspace{#1}
    \centering
    \def\svgwidth{#3}
    \CmIncludeSvg{#2}
  \end{figure}
}
\makeatother


%%%%%%%%%%%%%%%%%%%%%%%%%%%%%%%%%%%%%%%%%%%%%%%%%%%%%%%%%%%%%%%%%%%%%%%%%%%%%%%%%%%%%%%%%%%%%%%%%%%%%%%%%%%%%%%%%%%%%%%%%%%%%%%%%%%%%%%%%%%%%%%%%%%%%%%%%%%%%%%%%%%%%%%%

%\usepackage{canonicalsync}
%\CsUsePackage[/home/JB/Projects/tex-package-canonical-sync/]{canonicalsync}
%\CsUsePackageWithOptions[/home/JB/Projects/tex-package-canonical-math/]{canonicalmath}{marginthm}

\usepackage{mathtools}

\usepackage{graphicx}
\usepackage{float}
\usepackage{transparent}
\usepackage{wrapfig}

\graphicspath{{img/}}

\parindent0pt

% Befehl fuer Anfuerungszeichen unten und oben
\newcommand{\quot}[1]{\textrm{\glqq}{#1}\textrm{\grqq}}



\setlist[enumerate]{label=(\arabic*), itemsep=0cm, leftmargin=1cm}

%--------------------------------------------------------------------
%-------------------- Eintraege fuer das Glossar --------------------
%--------------------------------------------------------------------

\newglossaryentry{topologischer Raum}{name=Topologischer Raum,description={Eine Menge \ensuremath{X} zusammen mit einer Topologie \ensuremath{T}, das hei\ss t einem Mengensystem das offene Teilmengen von \ensuremath{X} definiert, wobei die leere Menge, die Grundmenge, der Durchschnitt endlich vieler offener Mengen und die Vereinigung beliebig vieler offener Mengen offen sind},text={topologischer Raum}}

\newglossaryentry{Hausdorff-Raum}{name=Hausdorff-Raum,description={Ein topologischer Raum \ensuremath{M}, in dem es f"ur alle \ensuremath{x, y \in M, x \ne y} disjunkte offene Umgebungen \ensuremath{U(x)} und \ensuremath{U(y)} gibt, es werden also alle paarweise verschiedenen Punkte \ensuremath{x, y} durch Umgebungen getrennt.},text={Hausdorff-Raum}}

%--------------------------------------------------------------------
%-------------------- Hier beginnt das Skript -----------------------
%--------------------------------------------------------------------

\begin{document}

\CmHeadline{Vorlesung Differentialgeometrie}{Gehalten von Dr. Grensing}{Wintersemester 2012/13}

\vspace{0.5cm}

Version \textbf{0.01} \quad Build: \today

\paragraph{Wichtiger Hinweis:}
Dies ist eine Zusammenfassung der Vorlesung "`Differentialgeometrie"' von Dr. Grensing im Wintersemester 2012/13 am KIT und dient lediglich dazu die Inhalte für meine eigene Verwendung besser zusammenzufassen und aufzubereiten. Es besteht weder eine Garantie über Vollständigkeit, noch Korrektheit der enthaltenen Aussagen.\\

Bei Anmerkungen bzw. beim Auffinden von Fehlern schicken Sie bitte eine E-Mail an
\begin{center}
  jan-bernhard.kordass@student.kit.edu
\end{center}

%%
%% 1. Vorlesung 16.10.12
%% 
%% Skript Differentialgeometrie im Wintersemester 12/13
%% Zur Vorlesung von Dr. Grensing am KIT Karlsruhe
%%
%% Mitschrieb und Textsatz von Jan-Bernhard Kordaß.
%%

\section*{"Ubersicht}

\begin{itemize}
\item Mannigfaltigkeiten, Tangentialvektoren
\item Kovariante Ableitung
\item Riemannsche Metriken
\item Krümmung
\item Jacobifelder
\item Satz von Bonnet
\end{itemize}

\section{Differenzierbare Mannigfaltigkeiten}

\begin{dfn*}
  Eine $n$-dimensionale \CmMark{topologische Mannigfaltigkeit} $M$ ist ein topologischer Hausdorff-Raum mit einer abzählbaren Basis der Topologie in dem zu jedem Punkt $p \in M$ eine offene Menge $U$ mit $p \in U$ existiert und ein Hom"oomorphismus $\phi \colon U \to V$ auf eine offene Menge $V \subset \R^{n}$.

% Abbildung 1-1
%\CmPutSvg{1-1-topologische-mf}{8.5cm}
\begin{center}\begin{tikzpicture}[font=\scriptsize]
	\draw[->] (-1.5,0) to[out=20, in=160]node[above,font=\scriptsize]{$\varphi' \circ \varphi^{-1}$} (1.5,0);
	
	\draw[->] (-4,-0.5) -- (-2,-0.5); \draw[->] (-3.75,-0.75) -- (-3.75, 1.25); \node[font=\scriptsize] at (-4, 1.25) {$\R^n$};
	\draw[->] (2,-0.5) -- (4,-0.5); \draw[->] (2.25,-0.75) -- (2.25, 1.25); \node[font=\scriptsize] at (2, 1.25) {$\R^m$};
	
	\node[font=\scriptsize] at (0,2) {$U \cap U' \ne 0$};
	
	\draw (-4.25, 1.75) to[out=70,in=180] (-1.75,3) to[out=300,in=90] (-1.25, 1.25) to[out=180,in=340] (-4.25, 1.75) -- cycle; \node at (-1.25,3) {$M$};
	\filldraw[fill=gray!20] (-2.75,2) circle(0.4); \node[font=\scriptsize] at (-3.25,2.25) {$U$};
	\filldraw[fill=gray!20] (-3,0.25) circle (0.5); \node at (-2.25, 0.5) {$V$};
	\draw[->] (-2.75,1.5) to[out=280,in=80] node[right]{$\varphi$} (-2.75,0.75);
			
	\draw (1.75, 1.75) to[out=70,in=180] (4.25,3) to[out=300,in=90] (4.75, 1.25) to[out=180,in=340] (1.75, 1.75) -- cycle; \node at (4.75,3) {$M$};
	\filldraw[fill=gray!20] (3.55,2.25) circle(0.6); \node[font=\scriptsize] at (2.75,2.25) {$U'$};
		
	\coordinate (ctrl0up) at ($(2.5,-0.25) + 0.2*(0.5,2)$); \coordinate (ctrl0down) at ($(2.5,-0.25) + 0.2*(0,-1.5)$);
	\coordinate (ctrl1down) at ($(3,0.25) - 0.1*(0.5,1)$); \coordinate (ctrl1up) at ($(3,0.25) + 0.1*(0.5,1)$);
	\coordinate (ctrl2down) at ($(3,0.7) - 0.1*(0.5,1)$); \coordinate (ctrl2up) at ($(3,0.7) + 0.1*(0.5,1)$);
	\coordinate (ctrl3down) at ($(4,0.5) + 0.3*(-0.5,1)$); \coordinate (ctrl3up) at ($(4,0.5) - 0.3*(-0.25,1)$);
	\coordinate (ctrl4down) at ($(3.75,-0.3) + 0.2*(0.8,1)$); \coordinate (ctrl4up) at ($(3.75,-0.3) - 0.2*(0.7,0.75)$);
	\begin{scope}
		\fill[gray!20] (2.5,-0.25) ..controls(ctrl0up) and (ctrl1down).. (3,0.25) ..controls(ctrl1up) and (ctrl2down).. (3,0.7) ..controls(ctrl2up) and (ctrl3down).. (4,0.5) ..controls(ctrl3up) and (ctrl4down).. (3.75,-0.3) ..controls(ctrl4up) and (ctrl0down).. (2.5,-0.25); \node at (4.25, 0.5) {$V'$};
		\clip(2.5,-0.25) ..controls(ctrl0up) and (ctrl1down).. (3,0.25) ..controls(ctrl1up) and (ctrl2down).. (3,0.7) ..controls(ctrl2up) and (ctrl3down).. (4,0.5) ..controls(ctrl3up) and (ctrl4down).. (3.75,-0.3) ..controls(ctrl4up) and (ctrl0down).. (2.5,-0.25); \node at (4.25, 0.5) {$V'$};
		\fill[gray] (2,0) circle (1);
		 (2.5,-0.25) ..controls(ctrl0up) and (ctrl1down).. (3,0.25) ..controls(ctrl1up) and (ctrl2down).. (3,0.7) ..controls(ctrl2up) and (ctrl3down).. (4,0.5) ..controls(ctrl3up) and (ctrl4down).. (3.75,-0.3) ..controls(ctrl4up) and (ctrl0down).. (2.5,-0.25); \node at (4.25, 0.5) {$V'$};
	\end{scope}
	\draw  (2.5,-0.25) ..controls(ctrl0up) and (ctrl1down).. (3,0.25) ..controls(ctrl1up) and (ctrl2down).. (3,0.7) ..controls(ctrl2up) and (ctrl3down).. (4,0.5) ..controls(ctrl3up) and (ctrl4down).. (3.75,-0.3) ..controls(ctrl4up) and (ctrl0down).. (2.5,-0.25) -- cycle; \node at (4.25, 0.5) {$V'$};
	\draw[->] (3.5,1.5) to[out=280,in=80] node[right]{$\varphi'$} (3.5,0.75);
\end{tikzpicture}\end{center}

  $\varphi' \circ \varphi^{-1}$ ist ein Hom"oomorphismus offener Mengen des $\R^n$ bzw. $\R^m$. Nach dem Satz von Brouwer (1912) gilt dann $m = n$. Damit ist die Dimension einer zusammenh"angenden topologischen Mannigfaltigkeit eindeutig definiert.\\

  Die Abbildung $\varphi \colon U \to V \subset \R^n$ hei\ss t \CmMark{Karte} von $M$ um $p$, die Menge $U$ hei\ss t \CmMark{Kartengebiet}.\\

  Eine Menge von Karten $\mathcal A = \{(\varphi_{\alpha}, U_{\alpha}) \mid \alpha \in J \}$ hei\ss t \CmMark{Atlas} von $M$, falls $\bigcup_{\alpha \in J}U_{\alpha} = M$.\\

  Ein Atlas $\mathcal A$ von $M$ hei\ss t $C^k$-Atlas, wenn für alle $\alpha, \beta \in J$ mit $U_{\alpha} \cap U_{\beta} \neq \emptyset$ der sogenannte \CmMark{Kartenwechsel}:
  \begin{align*}
    \varphi_{\beta} \circ \varphi_{\alpha}^{-1}\colon \varphi_{\alpha}(U_{\alpha} \cap U_{\beta}) \to \varphi_{\beta}(U_{\alpha} \cap U_{\beta})
  \end{align*}
  ein $C^k$-Diffeomorphismus ist.\\

  % Abbildung 1-2
  %\CmPutSvg{1-2-kartenwechsel}{8cm}
  \begin{center}\begin{tikzpicture}[font=\scriptsize]
  	\draw[->] (-1.5,0) to[out=20, in=160]node[above,font=\scriptsize]{$\varphi_\beta \circ \varphi^{-1}_\alpha$} (1.5,0);
	
	\draw[->] (-4,-0.5) -- (-2,-0.5); \draw[->] (-3.75,-0.75) --node[left]{$\R^n$} (-3.75, 1.25);
	\draw[->] (2,-0.5) -- (4,-0.5); \draw[->] (2.25,-0.75) -- (2.25, 1.25);
	
	\draw[thick]  (-0.25, 3) to[out=0,in = 150] (2,2.5) -- (1.75, 1.5) to[out=190,in=350] (-1.75, 1.5) to[out=90,in=180] (-0.25, 3) -- cycle; \node at (2.25,2.75) {$M$};
	
	\begin{scope}
		\clip (0.25,2.25) circle(0.5);
		\clip (-0.25,2) circle(0.5);
		\fill[gray!20] (0,2) circle(1);
	\end{scope}
	\draw (0.25,2.25) circle(0.5) (-0.25,2) circle(0.5); \node at (-1, 2.25) {$U_\alpha$}; \node at (1,2.5) {$U_\beta$};
	
	\draw[->] (-0.5,2) to[out=180,in=75] node[left]{$\varphi_\alpha$} (-3,0.25);
	\draw[->] (0.5,2.25) to[out=0,in=105] node[right]{$\varphi_\beta$} (3,0.25);
  \end{tikzpicture}\end{center}


  Eine Karte $\psi \colon U \to V$ von $M$ hei\ss t \CmMark{verträglich} mit einem $C^k$-Atlas $\mathcal A = \{(\varphi_{\alpha},U_{\alpha}) \mid \alpha \in J\}$ wenn jeder Kartenwechsel
  \begin{align*}
    \varphi_{\alpha} \circ \psi(U \cap U_{\alpha}) \to \varphi_{\alpha}(U \cap U_{\alpha})
  \end{align*}
  ein $C^k$-Diffeomorphismus ist, i.e. $\mathcal A' = \mathcal A \cup \{(\psi, U)\}$ ist ebenfalls ein $C^k$-Atlas.\\

  Die Menge aller mit $\mathcal A$ verträglichen Karten ist ein \CmMark{maxmaler $C^k$-Atlas}. Jeder maximale Atlas enthält alle mit ihm verträglichen Karten. Ein maximaler $C^k$-Atlas hei\ss t auch \CmMark{$C^k$-differenzierbare Struktur}.

\end{dfn*}

\begin{dfn}[differenzierbare Mannigfaltigkeit]
  Eine \CmMark{differenzierbare Mannigfaltigkeit} der Klasse $C^k$ ist eine topologische Mannigfaltigkeit zusammen mit einer $C^{k}$-differenzierbaren Struktur.\\
\end{dfn}

\begin{bsp}
  Einige Beispiele f"ur glatte Mannigfaltigkeiten:
  \begin{enumerate}%[1)]
  \item $M = \R^n, \mathcal A = \{(\Id_{\R^n},\R^n)\}$
  \item $M \subset \R^n$ offen, $\mathcal A = \{(\imath_{M},M)\}$
  \item $S^1 \subset \R^2$ ist eine eindimensionale $C^{\infty}$-Mannigfaltigkeit:
    \begin{align*}
      U = \{(\sin t, \cos t) \mid t \in (0,2\pi)\}
    \end{align*}

    % Abbildung 1-3
    \marginnote{\begin{center}\begin{tikzpicture}[font=\footnotesize]
    		%\draw[step=0.25,gray!15] (-1,-1) grid (1,1); \draw[step=0.5,gray!30] (-1,-1) grid (1,1); \fill (0,0) circle(0.1); %Hilfsgitter
		\draw (0,0) circle (1); \draw[dashed] (0,0) circle (1.1); \draw[dotted] (0,0) circle (0.9); \node at (1,1) {$S^1$};
		\filldraw[fill=white] (-1,0) circle (0.1) (1,0) circle (0.1);
    \end{tikzpicture}\\
    \textcolor{gray}{$S^1$ Einheitskreis}
    \end{center}}[-2cm]
    % \CmMarginSvg[-2cm]{1-3-karten-der-s1}{3cm}

    ist offen in $S^1$ und die Kartenabbildung
    \begin{align*}
      \varphi \colon (\sin t, \cos t) \mapsto t
    \end{align*}
    ist ein Hom"oomorphismus.
    \begin{align*}
      \varphi' \colon U' = \{(\sin t, \cos t) \mid t \in (-\pi,\pi)\} \to (-\pi,\pi)
    \end{align*}
    ebenfalls. $\mathcal A = \{(\varphi, U), (\varphi',U')\}$ ist ein Atlas von $S^1$, denn $U \cup U' = S^1$.
    \begin{align*}
      & \varphi' \circ \varphi^{-1} \colon \varphi(U \cap U') \to \varphi'(U \cap U')\\
      & (0,\pi)\cup(\pi,2\pi) \to (-\pi,0)\cup(0,\pi), t \mapsto \begin{cases}
        t & 0 < t < \pi\\
        t-2\pi & \pi < t < 2\pi
      \end{cases}
    \end{align*}

  \item Jeder reelle Vektorraum endlicher Dimension ist in kanonischer Weise eine $C^{\infty}$-Mannigfaltigkeit.\\
    W"ahle eine Basis $\{v_1, \ldots, v_n\}$ von $V$. Diese definiert mit
    \begin{align*}
      \varphi\left(\sum\lambda_iv_i\right) = (\lambda_1, \ldots, \lambda_n)
    \end{align*}
    eine Bijektion auf $\R^n$. Damit erhält man eine globale Karte von $V$.
    Der zugehörige Atlas h"angt nicht von der Wahl der Basis ab, denn ist $\{w_1, \ldots, w_n\}$ eine weitere Basis von $V$ und $\psi(\sum \lambda_iw_i) = (\lambda_1, \ldots, \lambda_n)$ eine weitere Karte, so ist $\varphi \circ \psi^{-1}$ als Endomorphismus des $\R^n$ schon $C^{\infty}$.

  \item $S^n = \{(x^0, x^1, \ldots, x^n) \mid \sum_{i = 0}^n(x^{i})^2 = 1\}$.\\

    % Abbildung 1.4
    %\CmMarginSvg{1-4-s3-sphaere}{3.5cm}
    \marginnote{\begin{center}\begin{tikzpicture}[font=\scriptsize]
    		%\draw[step=0.25,gray!15] (-1,-1) grid (1,1); \draw[step=0.5,gray!30] (-1,-1) grid (1,1); \fill (0,0) circle(0.1); %Hilfsgitter
		% Koordinatenachsen mit Beschriftung
		\draw[->] (0,-1.25) -- (0,1.5) node[left]{$x^0$}; \draw[->] (-1.25,0) -- (1.5,0) node[below]{$x^1$}; \draw[->] (1,1) -- (-1.25,-1.25) node[right]{$x^2$}; \node at (1.25, 1.5) {$S^2 \subset \R^3$};
		% Kreis, Ellipse und Gerade (verwende Namen um Schnittpunkt bestimmen zu koennen)
		\path[draw, thick, name path=kreis] (0,0) circle (1) ellipse(1 and 0.5); \path[draw,name path=gerade] (0,1) -- (1,-1.25);
		% Punkte N und p
		\filldraw[fill=white] (0,1) circle (0.05) node[anchor=south west,xshift=-2,yshift=-1.5]{$N$} ($(0,1)+0.35*(1,-1.25)-0.35*(0,1)$) circle (0.05) node[right]{$p$};
		% Punkt phi(p) bei Schnittpunkt von Gerade und Kreis
		\path [name intersections={of=kreis and gerade}]; \filldraw[fill=white] (intersection-2) circle(0.05) node[right]{$\varphi(p)$};
	\end{tikzpicture}\end{center}}%[3.5cm]
    
    Betrachte den Nordpol $N = (1,0,\ldots,0)$ und den S"udpol $S = (-1,0,\ldots,0)$ und die Abbildung
    \begin{align*}
      & \varphi \colon U = S^{n}\setminus\{N\} \to \R^n, x \mapsto \left(\frac{x^1}{1-x^0}, \ldots, \frac{x^{n}}{1-x^0}\right),\\
      & \psi \colon U' = S^{n} \setminus \{S\} \to \R^n, x \mapsto \left(\frac{x^1}{1+x^0}, \ldots, \frac{x^n}{1+x^0}\right)
    \end{align*}

    Aufgabe: Zeige, dass $(\varphi, U), (\psi, U')$ einen $C^{\infty}$-Atlas auf $S^n$ definiert.

  \end{enumerate}
\end{bsp}

\begin{dfn}[Differenzierbare Abbildungen]
Eine stetige Abbildung $f \colon M \to N$ zwischen glatten Mannigfaltigkeiten $M$ und $N$ hei\ss t \CmMark{glatt} ($C^{\infty}$-differenzierbar), wenn es zu jedem $p \in M$ Karten $(\varphi, U)$ in $M$ um $p$ und geeignete $(\varphi', U')$ in $N$ um $f(p)$ gibt, so dass $\varphi' \circ f\circ\varphi^{-1}$ glatt ist.
% Abbildung 1-5
%\CmPutSvg{1-5-glatte-abb}{9cm}
\begin{center}\begin{tikzpicture}[font=\scriptsize]
	%\draw[step=0.25,gray!15] (-5,-1) grid (5,5); \draw[step=0.5,gray!30] (-5,-1) grid (5,5); \fill (0,0) circle(0.1); %Hilfsgitter
	
	% Die Abbildungspfeile
	\draw[->] (-1.5,0) to[out=20, in=160]node[above]{$\varphi' \circ f \circ \varphi$} (1.5,0);
	\draw[->] (-1,2) --node[above]{$f$} (1.5,2);
	
	% Die Achsen
	\draw[->] (-4.5,-0.5) -- (-2,-0.5); \draw[->] (-4.25,-0.75) --node[left]{$\R^n$} (-4.25, 1.25);
	\draw[->] (2,-0.5) -- (4.5,-0.5); \draw[->] (2.25,-0.75) --node[left]{$\R^m$} (2.25, 1.25);
	
	% Die Blasen
	\draw[thick] (-4.25, 1.75) to[out=70,in=180] (-1.75,3) to[out=300,in=90] (-1.25, 1.25) to[out=180,in=340] (-4.25, 1.75) -- cycle; \node[font=\normalfont] at (-1.25,3) {$M$};
	\draw[thick] (1.75, 1.75) to[out=70,in=180] (4.25,3) to[out=300,in=90] (4.75, 1.25) to[out=180,in=340] (1.75, 1.75) -- cycle; \node[font=\normalfont] at (4.75,3) {$N$};
	
	% Die linke Kartoffel (zuerst werden die Punkte definiert, dann die Richtungsvektoren der Splines, dann die Kartoffel selbst)
	\coordinate (kartoffel0l) at (-3.25,1.75); \coordinate (kartoffel1l) at (-3.25,2.5); \coordinate (kartoffel2l) at (-2.25,2.25); \coordinate (kartoffel3l) at (-2.5,1.75);
	\coordinate (ctrlk0l) at (-0.25,0.5); \coordinate (ctrlk1l) at (0.5,0.25); \coordinate (ctrlk2l) at (-0.25,1); \coordinate (ctrlk3l) at (2,0.25);
	\draw (kartoffel0l) ..controls($(kartoffel0l)+0.5*(ctrlk0l)$) and ($(kartoffel1l)-0.3*(ctrlk1l)$).. (kartoffel1l) ..controls($(kartoffel1l)+0.6*(ctrlk1l)$) and($(kartoffel2l)+0.45*(ctrlk2l)$).. (kartoffel2l) ..controls($(kartoffel2l)-0.25*(ctrlk2l)$) and ($(kartoffel3l)+0.15*(ctrlk3l)$).. (kartoffel3l)  ..controls($(kartoffel3l)-0.1*(ctrlk3l)$) and ($(kartoffel0l)-0.9*(ctrlk0l)$).. (kartoffel0l); \node at (-3.5,2.25) {$U$};
	% Der Punkt in der Kartoffel, der Pfeils raus und der Kreis
	\draw[->] (-2.75,2) node[right]{$p$} to[out=280,in=80] node[right]{$\varphi$} (-2.75,0); \fill (-2.75,2) circle (0.05);
	\draw (-3,0.25) circle(0.5); \node at (-3.5,0.75) {$V$};
	
	% Die rechte Kartoffel
	\coordinate (kartoffel0r) at (3.25,1.75); \coordinate (kartoffel1r) at (3.5,2.5); \coordinate (kartoffel2r) at (4.5,2.25); \coordinate (kartoffel3r) at (4.25,1.5);
	\coordinate (ctrlk0r) at (-0.25,0.5); \coordinate (ctrlk1r) at (-0.25,0.25); \coordinate (ctrlk2r) at (-0.25,0.5); \coordinate (ctrlk3r) at (0.25,0);
	\draw (kartoffel0r) ..controls($(kartoffel0r)+0.5*(ctrlk0r)$) and ($(kartoffel1r)-(ctrlk1r)$).. (kartoffel1r) ..controls($(kartoffel1r)+(ctrlk1r)$) and ($(kartoffel2r)+(ctrlk2r)$).. (kartoffel2r) ..controls($(kartoffel2r)-(ctrlk2r)$) and ($(kartoffel3r)+(ctrlk3r)$).. (kartoffel3r) ..controls($(kartoffel3r)-(ctrlk3r)$) and ($(kartoffel0r)-(ctrlk0r)$).. (kartoffel0r); \node at (3.25,2.25) {$U'$};
	
	\draw[->] (3.75,2) node[right]{$f(p)$} to[out=280,in=80] node[right]{$\varphi'$} (3.75,0); \fill (3.75,2) circle (0.05);
	\draw (3.5,0.25) circle(0.5); \node at (3,0.75) {$V'$};
\end{tikzpicture}

\textcolor{red}{Sollte das in der Zeichnung beim unteren Pfeil nicht $\varphi'\circ f \circ \varphi^{-1}$ hei\ss en?}\end{center}
Die Menge aller glatten Abbildungen von $M$ nach $N$ wird $C^{\infty}(M,N)$ genannt.

\end{dfn}

\textbf{Konvention}: Ab jetzt seien zunächst alle Mannigfaltigkeiten, wie auch alle Abbildungen als glatt vorrausgesetzt.

\begin{bem}
  Da Kartenwechsel $C^{\infty}$ sind, gilt obige Bedingung automatisch für alle Karten von $M$ und $N$ (evtl. nach Einschränkung).
\end{bem}

\begin{bsp}
  Es folgen zwei Beispiele für diffenrenzierbare Abbildungen:
  \begin{enumerate}
  \item $(\varphi,U) \in \mathcal A \Rightarrow \varphi \in C^{\infty}(U,\R^n)$, denn
    \begin{align*}
      \Id_{\R^n}\circ \varphi \circ \varphi^{-1} = \varphi \circ \varphi^{-1} \in C^{\infty}.
    \end{align*}
  \item $f \in C^{\infty}(M,N), \ g \in C^{\infty}(N,P) \Rightarrow g \circ f \in C^{\infty}(M,P)$, denn
    \begin{align*}
      \varphi_p \circ g \circ f \circ \varphi^{-1}_m = (\varphi_p \circ g \circ \varphi_n^{-1}) \circ (\varphi_n \circ f \circ \varphi_m^{-1}) \in C^{\infty}.
    \end{align*}
  \end{enumerate}
\end{bsp}

\begin{dfn}[Diffeomorphismus]
  Eine Abbildung $f \colon M \to N$ hei\ss t \CmMark{Diffeomorphismus}, wenn $f$ bijektiv ist und $f$, sowie $f^{-1}$ $C^{\infty}$-Abbildungen von $M$ nach $N$ sind. Insbesondere haben $M$ und $N$ in diesem Fall dieselbe Dimension.\\

Die Menge der Diffeomorphismen von $M$ nach $N$ wird mit $\Diff(M,N)$ bezeichnet. Die Menge der Diffeomorphismen von $M$ nach $M$ wird mit $\Diff(M)$ bezeichnet. $(\Diff(M), \circ)$ ist eine Gruppe.

\end{dfn}

%%% Local Variables: 
%%% mode: latex
%%% TeX-master: "../skript-diffgeom"
%%% End: 

%% 
%% 2. Vorlesung <2012-10-19 Fri>, Fortsetung
%% 
%% Skript Differentialgeometrie im Wintersemester 12/13
%% Zur Vorlesung von Dr. Grensing am KIT Karlsruhe
%% 
%% Mitschrieb und Textsatz von Jan-Bernhard Kordaß.
%% 

\chapter{Tangentialvektoren und Tangentialräume}

% Abbildung 2-1
\CmMarginSvg{2-1-tangentialvektoren-motivation}{3.5cm}

Betrachte in der nebenstehenden Abbildung eine differenzierbare \gls{Kurve} $c \colon (-\varepsilon,\varepsilon) \to S^2$ mit $c(0) = p$. Dann gilt:
\begin{align*}
  0 = \difffrac[t=0]{}{t} \left<c(t),c(t)\right> = 2\left<\dot c(0),c(0)\right> = 2 \left<\dot c(0),p\right> 
  \Rightarrow \dot c(0) \in p^{\perp}.
\end{align*}

% Bemerke $1 = \left<c(t),c(t)\right>$

\textcolor{red}{(Ich habe Punkte "uber einige der $c$ gesetzt, bitte "uberpr"ufen)}Es sei $M$ eine glatte Mannigfaltigkeit und es seien glatte Kurven $c_i\colon (-\varepsilon_i,\varepsilon_i) \to M$ mit $c_1(0) = c_2(0) = p \in M$ gegeben.

Die Kurven heißen \CmMark{äquivalent}, wenn es eine Karte $(\varphi,U)$ von $M$ und $p$ gibt, so dass gilt
\begin{align*}
  \difffrac[t=0]{}{t}(\varphi \circ c_1) = \difffrac[t=0]{}{t}(\varphi \circ c_2)
\end{align*}

\begin{Lemma}
  Der oben definierte Begriff der Äquivalenz ist unabhängig von der Wahl der Karte.
\end{Lemma}

\begin{proof}
  Es sei $(\psi,V)$ eine weitere Karte von $M$ um $p$. Dann gilt:
  \begin{align*}
    \difffrac[t=0]{}{t}(\psi\circ c_1) & = \difffrac[t=0]{}{t}(\psi\circ\varphi^{-1}\circ\varphi \circ c_1) = \D (\psi \circ \varphi^{-1})|_{\varphi(p)} \cdot \difffrac[t=0]{}{t}(\varphi \circ c_1)\\
    & = \D(\psi \circ \varphi^{-1})|_{\varphi(p)} \cdot \difffrac[t=0]{}{t}(\varphi \circ c_2) = \ldots = \difffrac[t=0]{}{t}(\psi \circ c_2).
  \end{align*}
\end{proof}

\begin{Dfn}[Geometrische Definition des Tangentialraums]
  Es sei $M$ eine glatte Mannigfaltigkeit und $p \in M$. Ein (geometrischer) \CmMark{Tangentialvektor} an $M$ in $p$ ist eine Äquivalenzklasse von Kurven $c$ mit $c(0) = p$. Die Menge
  \begin{align*}
    \T_{p}^{\text{geo}}M = \{ [c] \mid c \colon (-\varepsilon,\varepsilon) \to M \text{ glatt}, c(0) = p\}
  \end{align*}
  heißt (geometrischer) \CmMark{Tangentialraum} an $M$ in $p$.
\end{Dfn}

\begin{Bem}
  Mit den Bezeichnungen wie oben ist die folgende Abbildung bijektiv:
  \begin{align*}
    A \colon \T_p^{\text{geo}}M \to \R^n, [c] \mapsto \difffrac[t=0]{}{t}(\varphi \circ c).
  \end{align*}
\end{Bem}

\begin{proof}
  Zu jedem Vektor $v \in \R^n$ sei $B(v) = [t \mapsto \varphi^{-1}(\varphi(p) + tv)]$ die Äquivalenzklasse der abgebildeten Kurve auf der Mannigfaltigkeit.

  % Abbildung 2-2
  \CmPutSvg{2-2-beweis-bijektivitaet-tpm-rn}{10cm}

  \[ A B(v) = \difffrac[t=0]{}{t}(\varphi \circ B(v)) = \difffrac[t=0]{}{t}(\varphi \circ (\varphi^{-1}(\varphi(p) + tv)) = \difffrac[t=0]{}{t}(\varphi(p) + tv) = v. \]
  \[ B A (\underbrace{[c]}_{\ni c}) = B(v_c) = [t \mapsto \varphi^{-1}(\varphi(p) + tv_c)] \text{ wobei } v_c = \difffrac[t=0]{}{t}(\varphi \circ c). \]
  Die Kurven $c$ und $t \mapsto \varphi^{-1}(\varphi(p) + tv_c)$ sind äquivalent, also ist $B A[c] = [c]$ und somit $A$ bijektiv.
\end{proof}

Damit erhält $\T_p^{\text{geo}}M$ die Struktur eines reellen Vektorraumes vermöge der folgenden Verknüpfung:
\begin{align*}
  \lambda[c_1] + \mu[c_2] = A^{-1}(\lambda A[c_1]+ \mu A[c_2]).
\end{align*}
Dabei gilt $\lambda[c_1]+\mu[c_2] = [c]$ für $c(t) = \varphi^{-1}(\varphi(p) + t(\lambda v_1 + \mu v_2))$ mit $v_i = \difffrac[t=0]{}{t}(\varphi \circ c_i)$.

\begin{Lemma}
  Die oben definierte Lineare Struktur ist unabhängig von der Wahl der Karte.
\end{Lemma}

\begin{proof}
  Es sei $(\psi, V)$ eine Karte von $M$ um $p$ und $A'[c] = \difffrac[t]{}{t}(\psi \circ c)$. Dann gilt:
  \begin{align*}
    A A'^{-1}(v) & = \difffrac[t=0]{}{t}(\varphi \circ (\psi^{-1} (\psi(p) + tv)))\\
    & = \D(\varphi \circ \psi^{-1})|_{\psi(p)} \cdot \difffrac[t=0]{}{t}(\psi \circ \psi^{-1}(\varphi(p) + tv)) = \D (\varphi \circ \varphi^{-1}) \cdot v.
  \end{align*}
  Also ist $A A'^{-1}$ linear,
  \begin{align*}
    A'^{-1}(\lambda A'[c_1] + \mu A'[c_2]) & = A^{-1}(A A'^{-1}(\lambda A'[c_1] + \mu A'[c_2]))\\
    & = A^{-1} (\lambda A A'^{-1}[c_1] + \mu A A'^{-1} [c_2])\\
    & = A^{-1}(\lambda A [c_1] + \mu A [c_2]).
  \end{align*}
\end{proof}


% 3. Vorlesung <2012-10-23 Tue>

\paragraph{Motivation: Richtungsableitungen im $\R^n$}\hfill
\begin{Bem}
  
  Für $f,g \in C^{\infty}(\R^n), \ x,y \in \R^n$ ist die \CmMark{Richtungsableitung} wie folgt definiert:
  \begin{align*}
    \partial_vf(x) = \D f|x \cdot v = \difffrac[t=0]{}{t}f(x+tv).
  \end{align*}
  Diese erfüllt die Leibniz-Regel:
  \begin{align*}
    \partial_v(fg)(x) = \partial_vf(x)\cdot g(x) + f(x) \cdot \partial_v(g)(x).
  \end{align*}
\end{Bem}

\begin{Dfn}[Algebraische Definition des Tangentialraumes]
  Es sei $M$ eine glatte Mannigfaltigkeit und $p\in M$. Ein (algebraischer) \CmMark{Tangentialvektor} an $M$ in $p$ ist eine Lineare Abbildung $X_p \colon C^{\infty}(M) \to \R$, welche die Leibniz-Regel erfüllt:
  \begin{align*}
    X_p(fg) = X_p(f) \cdot g(p) + f(p) \cdot X_p(g).
  \end{align*}

  Die algebraischen Tangentialvektoren bilden einen reellen Vektorraum $\T_p^{\text{alg}}M$, den Tangentialraum an $M$ in $p$.
\end{Dfn}

\begin{Lemma}
  Es sei U eine Umgebung von $p \in M$. Dann existiert eine Umgebung $V \subset U$ von $p$ und eine glatte reellwertige Funktion $\sigma \in C^{\infty}(M)$ mit den Eigenschaften $\sigma|_V = 1$ und $\supp(\sigma) \subset U$.
\end{Lemma}

%%%
%%% Abbildung 2-3
%%% 
\textcolor{red}{Abbildung 2-3}


\begin{proof}
  Man kann o.E. annehmen, dass $U$ Kartengebiet einer Karte $\varphi$ von $M$ um $p$ ist und $\varphi(p) = 0 \in \R^n$.\\

  Es sei $\varepsilon > 0$ so, dass $\overline B_c(0) \subset \varphi(U)$. \\

  %%% 
  %%% Abbildung 2-4
  %%% 
  \textcolor{red}{Abbildung 2-4}

  Ist dann $\eta$ eine glatte Funktion auf $\R$ mit $\eta \equiv 1$ auf $\left[\frac{-\varepsilon^{2}}{2},\frac{\varepsilon^2}{2}\right]$ und $\eta \equiv 0$ auf $\R \setminus (-\varepsilon^2,\varepsilon^2)$, so hat für $U_1 = \varphi^{-1}(B_{\frac{\varepsilon}{2}}(0))$ die Funktion
  \begin{align*}
    \sigma(q) =
    \begin{cases}
      \eta(\|\varphi(q)\|^2) & \text{ für } q \in U_1\\
      0 & \text{ sonst }
    \end{cases}.
  \end{align*}
  die gewünschten Eigenschaften.
\end{proof}

% Lemma 2.
\begin{Lemma}
Für alle $X_p\in\T_p^{\text{alg}}M$ gilt:
\begin{enumerate}[label=(\roman*),widest=ii]
\item $X_p(f) = 0$ falls $f$ in einer Umgebung von $p$ konstant ist.
\item $X_p(f) = X_p(g)$ falls $f$ und $g$ auf einer Umgebung übereinstimmen.
\end{enumerate}
\end{Lemma}

\begin{proof}\begin{enumerate}[label=(\roman*),widest=ii,leftmargin=*]
\item[(ii)]
	Es sei $U$ eine Umgebung von $p$ mit $f|_U = g|_U$. Ist dann $\sigma$ wie in \textcolor{red}{Lemma 2.5}, so gilt $\sigma f = \sigma g$ und aus
	\begin{align*}
		X_p(\sigma)f(p)+\sigma(p)X_p(f) = X_{p}(\sigma f) = X_p(\sigma g) = X_p(\sigma) g(p) + \sigma(p) X_p(g)
	\end{align*}
	folgt $X_p(f) = X_p(g)$.\\
\item[(i)]
	Wegen der $\R$-Linearität und (ii) genügt es $f \equiv 1$ zu betrachten. Es gilt
	\begin{align*}
		X_p(1) = X_p(1 \cdot 1) = X_p(1) \cdot 1 + 1 \cdot X_p(1) = 2 \cdot X_p(1),
	\end{align*}
	also $X_p(1) = 0$.
\end{enumerate}\end{proof}

\begin{Bem}
  Also gilt für $f \in C^{\infty}(M)$ und $g \in C^{\infty}(U)$ direkt:
  \begin{align*}
    & \sigma g =
    \begin{cases}
      \sigma g|_U & \textcolor{red}{\sigma g \in C^{\infty}(M)}\\
      0 & \text{ sonst }
    \end{cases},\\
    & \sigma g \in C^{\infty}(M) 
    \Rightarrow X_p(g) = X_p(\sigma g).
  \end{align*}
  Für eine Karte $\varphi \colon U \to V$ von $M$ und $p$ seien algebraische Tangentialvektoren definiert:
  \begin{align*}
    \pdifffrac[p]{}{x^i} \in \T_p^{\text{alg}}M, \pdifffrac[p]{}{x^i}(f) = \partial_i(f \circ \varphi^{-1})(\varphi(p)) = \D(f \circ \varphi^{-1})|_{\varphi(p)}e_i.
  \end{align*}
\end{Bem}

% Satz 2.7
\begin{Satz}
  Die Vektoren $\pdifffrac[p]{}{x^1},\ldots,\pdifffrac[p]{}{x^n}$ bilden eine Basis von $T_p^{\text{alg}}M$.
\end{Satz}

% Lemma 2.8
\begin{Lemma}
  Es sei $x_0 \in \R^n$ und $g \in C^{\infty}(B_{\rho}(x_0))$.
  Dann existieren glatte Funktionen $h_i \in C^{\infty}(B_{\rho}(x_0))$ mit $h_i(x_0) = \partial_ig(x_0)$ und 
  \begin{align*}
    g(x) = g(x_0) + \sum_{i=1}^n(x^i-x_0^i)h_i(x).
  \end{align*}
\end{Lemma}

\begin{bew}[Beweis des Satzes]
Die $j$-te Komponente $\varphi^j$ der Karte ist glatt und es gilt:
\begin{align*}
	\pdifffrac[p]{}{x^i}(\varphi^j) = \partial_i(\varphi^j \circ \varphi^i)(\varphi(p)) = \partial_ix^j(\varphi(p)) = \delta_i^j.
\end{align*}
Damit sind die Vektoren linear unabhängig.

Es sei $X_p\in \T_p^{\text{alg}}M$ und $f \in C^{\infty}(M)$.
Für $x_0=\varphi(p) \in \R^n, \ B_{\rho}(x_0) \subset \varphi(U)$ und für $g = f \circ \varphi^{-1}|_{B_{\rho}(x_0)}$ gilt mit den Bezeichnungen wie im letzten Lemma:
\begin{align*}
	X_p(f) & = X_p(g \circ \varphi) = X_p(g(\varphi(p)) + \sum \left(\varphi^i - \varphi(p)^i)(h_i \circ \varphi) \right)\\
	& = \underbrace{X_p(g(\varphi(p)))}_{\mathclap{=0}} + \sum X_p((\varphi^i-\varphi(p)^i)(h_i \circ \varphi))\\
	& = \sum X_p(\varphi^i)(h_i\circ\varphi)(p) - X_p(\varphi(p)^i)(h_i\circ \varphi)(p) + \sum (q^i-\varphi(p)^i)(p) X_p(h_i \circ \varphi)\\
	& = \sum_{i=1}^n X_p(\varphi^i)\underbrace{(h_i \circ \varphi)(p)}_{\mathclap{=h_i(\varphi(p) = h_i(x_0) = \partial_ig(x_0) = \partial_i(f\circ \varphi^{-1})(\varphi(p)) = \pdifffrac[p]{}{x^i}(f)}}\\
	& = \sum_{i=1}^nX_p(\varphi^i)\pdifffrac[p]{}{x^i}(f).
\end{align*}
\end{bew}

\begin{Bem}
  Ist $X_p=\sum \xi^i\pdifffrac[p]{}{x^i}$, so gilt $\xi^i = X_p(\varphi^i)$.
\end{Bem}

\begin{bew}[Beweis des Lemmas]
Es gilt:
\begin{align*}
	g(x) - g(x_0) = \int_0^1\difffrac{}{t}g(tx + (1-t)x_0)dt = \sum_{i=1}^n(x^i-x_0^i)\underbrace{\int_0^1\partial_ig(tx + (1-t)x_0) dt}_{=: h_i(x)}.
\end{align*}
\end{bew}

% Satz 2.9
\begin{Satz}[Äquivalenz der Tangentialraumbegriffe]
  Die Abbildung
  \begin{align*}
    J_p \colon \T_p^{\text{geo}}M \to \T_p^{\text{alg}}M, \ J_{p}[c](f) = \difffrac[t=0]{}{t}(f\circ c)
  \end{align*}
  ist ein linearer \gls{Isomorphismus} \quot{$c(0)(f)$}.
\end{Satz}

\begin{bew}
  Wegen
  \begin{align*}
    J_p[c](f)& = \difffrac[t=0]{}{t}(f\circ c) = \difffrac[t=0]{}{t}(f \circ \varphi^{-1} \circ \varphi \circ c)\\
    &  = \D(f \circ \varphi^{-1})|_{\varphi(p)} \difffrac[t=0]{}{t} (\varphi \circ c) = \D (f\circ \varphi^{-1})|_{\varphi(p)}A[c]
  \end{align*}
  ist $J_p = \D(\cdot)\circ A$ linear.

  Ist $[c] \in \Kern J_p$, so folgt aus $0 = J_p[c](\varphi^i) = \difffrac[t=0]{}{t}(\varphi^i \circ c)$, dass $\difffrac[t=0]{}{t}(\varphi \circ c) = 0$ gilt, also $[c] = 0$. Damit ist $J_p$ injektiv, also ein Isomorphismus.
\end{bew}

\begin{Bem}
  \begin{enumerate}[label=\arabic*)]
  \item Ist $X_p = \sum \xi^i\pdifffrac[p]{}{x^i}$, so gilt $X_p = \dot c(0)$ für $c(t) = \varphi^{-1}(\varphi(p) + t\xi)$.
\item Für jede glatte Kurve $c$ durch $p$ ist $\difffrac[t=0]{}{t}(\varphi \circ c)$ der Koeffizientenvektor von $\dot c(0)$ in der Basis $\pdifffrac[p]{}{x^i}$.
  \end{enumerate}
\end{Bem}


% Satz 2.10
\begin{Satz}[Transformationsverhalten bei Kartenwechsel]
  Es seien $\varphi$ und $\psi$ Karten in $M$ um $p$ und es bezeichnen $\pdifffrac[p]{}{x^i}$ und $\pdifffrac[p]{}{y^i}$ die damit assoziierten Basen von $\T_pM$. Dann gilt
  \begin{align*}
    \pdifffrac[p]{}{x^i} = \sum_j \partial_i(\psi^j \circ \varphi^{-1})(\varphi(p)) \pdifffrac[p]{}{y^j}.
  \end{align*}
Es sei $X_p = \sum \xi^i \pdifffrac[p]{}{x^i} = \sum \eta^i\pdifffrac[p]{}{y^i}$. Dann gilt:
\begin{align*}
  \eta^j = \sum \partial_i(\psi^j \circ \varphi^{-1})(\varphi(p))\xi^i \text{ bzw. }
  \eta = \D(\psi \circ \varphi^{-1})(\varphi(p))\xi.
\end{align*}
\end{Satz}

\begin{proof}
  Es gelte $\pdifffrac[p]{}{x^i} = \sum \alpha_i^j\pdifffrac[p]{}{y^j}$ und nach obiger Bemerkung zum vorletzten Satz gilt:
  \begin{align*}
    \alpha_i^j = \pdifffrac[p]{}{x^i}(\psi^j) = \partial_i(\psi^j \circ \varphi^{-1})(\varphi(p))
  \end{align*}
\end{proof}

\textcolor{red}{Kapitel 2 hat drei Nummerierungen zu viel, bitte die Nummerierung korrigieren}
%%% Local Variables: 
%%% mode: latex
%%% TeX-master: "../skript-diffgeom"
%%% End: 


%% 
%% 4. Vorlesung <2012-10-26 Fri>
%% 
%% Skript Differentialgeometrie im Wintersemester 12/13
%% Zur Vorlesung von Dr. Grensing am KIT Karlsruhe
%% 
%% Mitschrieb und Textsatz von Jan-Bernhard Kordaß.
%% 

\section{Differentiale}

% Abb 4/1

Es seien $M$ und $N$ Mannigfaltigkeiten und $\Phi \colon M \to N$ eine glatte Abbildung.
Sind $p \in M$ und $X_p \in \T_pM$ , so ist 
\begin{align*}
  \Phi_{*p}X_p \colon C^{\infty}(N) \to \R, f \mapsto X_p(\underbrace{f \circ \Phi}_{\in C^{\infty}(N)}).
\end{align*}
ein Tangentialvektor an $N$ in $\Phi(p)$:
\begin{align*}
  \Phi_{*p}X_p(fg) & = X_p((f \circ \Phi)(g \circ \Phi)) = X_p(f \circ \Phi)(g \circ \Phi)(p) + (f \circ \Phi)(p)X_p(g \circ \Phi)\\
  & = \Phi_{*p}X_p(f)q(\Phi(p)) + f(\Phi(p)) \Phi_{*p}X_p(g).
\end{align*}
\begin{center}\begin{tikzpicture}[font=\scriptsize]
	%\draw[step=0.25,gray!15] (-6,-1) grid (6,5); \draw[step=0.5,gray!30] (-6,-1) grid (6,5); \fill (0,0) circle(0.1); %Hilfsgitter
	
	% Die Abbildungspfeile
	\draw[->] (-1.5,0) to[out=20, in=160]node[above]{$\psi' \circ \Phi \circ \varphi^{-1}$}node[below]{diff'bar} (1.5,0);
	\draw[->] (-1.5,3) to[out=20, in=160]node[above]{$\Phi$} (1.5,3);
	
	% Die Achsen
	\draw[->,thick] (-5.5,-0.5) -- (-2,-0.5); \draw[->,thick] (-5.25,-0.75) -- (-5.25, 1.25); \node[font=\normalfont] at (-2,1.25) {$\R^m$};
	\draw[->,thick] (2,-0.5) -- (5.5,-0.5); \draw[->,thick] (2.25,-0.75) -- (2.25, 1.25); \node[font=\normalfont] at (5.5,1.25) {$\R^n$};
	
	% das linke Ding
	\coordinate (ding0) at (-3.5,4.5); \coordinate (ding1) at (-4.25,3.5); \coordinate (ding2) at (-4.5,2); \coordinate (ding3) at (-2.75,2.5); \coordinate (ding4) at (-1.5,4);
	\coordinate (ctrld0) at (0.5,-0.25); \coordinate (ctrld1) at (0.75,0.5); \coordinate (ctrld2) at (-0.5,0.5); \coordinate (ctrld3) at (0.25,0.5); \coordinate (ctrld4) at (-0.75,0); 
	\draw[thick] (ding0) ..controls($(ding0)+(ctrld0)$) and ($(ding1)+(0.75,0.5)$).. (ding1) ..controls($(ding1)-(ctrld1)$) and($(ding2)+(ctrld2)$).. (ding2) ..controls($(ding2)-(ctrld2)$) and ($(ding3)-2*(ctrld3)$).. (ding3) ..controls($(ding3)+(ctrld3)$) and ($(ding4)+(ctrld4)$).. (ding4);
	% das Loch in der Mitte nicht vergessen, es besteht aus zwei geclipten Kreisen
	\begin{scope}
		\clip (-4.25,2.5) rectangle (-2.75,3);
		%\draw[thick] (-4.25,3) to[out=330,in=180] (-3.5,2.75) to[out=0,in=210] (-2.75,3);
		\path[draw,thick,name path=gkreis] (-3.5,4.25) circle (1.5);
	\end{scope}
	\path[name path=kkreis] (-3.5,2) circle(1);
	\path[name intersections={of=gkreis and kkreis}];
	\begin{scope}
		\clip (intersection-1) rectangle ($(intersection-2)+(0,0.5)$);
		\draw[thick]  (-3.5,2) circle(1);
	\end{scope}
	
	% das rechte Ding
	\draw[thick] (4, 3)  ellipse (2 and 1);
	% und das Loch
	\begin{scope}
		\clip (3, 2.75) rectangle (5, 4);
		\path[draw,thick,name path=gkreis] (4,3.75) ellipse (1.25 and 1);
	\end{scope}
	\path[name path=kkreis] (4,2.5) ellipse (1 and 0.75);
	\path[name intersections={of=gkreis and kkreis}];
	\begin{scope}
		\clip (intersection-1) rectangle ($(intersection-2)+(0,0.5)$);
		\draw[thick] (4,2.5) ellipse (1 and 0.75);
	\end{scope}
	
	% die linke Kartoffel
	\coordinate (kartoffel0) at (-4.25,2.5); \coordinate (kartoffel1) at (-4.25,2); \coordinate (kartoffel2) at (-4,2.125); \coordinate (kartoffel3) at (-3.75,2); \coordinate (kartoffel4) at (-3.25,2.5); \coordinate (kartoffel5) at (-3.75,2.5); 
	\coordinate (ctrlk0) at (0.25,0.25); \coordinate (ctrlk1235) at (0.25,0); \coordinate (ctrlk4) at (0,0.25);
	\draw (kartoffel0) ..controls($(kartoffel0)-(ctrlk0)$) and ($(kartoffel1)-0.5*(ctrlk1235)$).. (kartoffel1) ..controls($(kartoffel1)+0.5*(ctrlk1235)$) and ($(kartoffel2)-(ctrlk1235)$).. (kartoffel2) ..controls($(kartoffel2)+(ctrlk1235)$) and ($(kartoffel3)-0.5*(ctrlk1235)$).. (kartoffel3) ..controls($(kartoffel3)+2*(ctrlk1235)$) and ($(kartoffel4)-(ctrlk4)$).. (kartoffel4) ..controls($(kartoffel4)+(ctrlk4)$) and ($(kartoffel5)+(ctrlk1235)$).. (kartoffel5) ..controls($(kartoffel5)-0.5*(ctrlk1235)$) and ($(kartoffel0)+(ctrlk0)$).. (kartoffel0) -- cycle;
	\fill (-3.75,2.25) circle (0.05) node[right]{$p$};
	
	% die rechte Kartoffel (Kreis)
	\draw (4,2.5) ellipse (0.5 and 0.25); \fill (4,2.5) circle (0.05) node[right]{$q$};
	
	% die beiden Umgebungen unten
	\draw (-3.75,0.25) circle (0.5); \fill (-3.75,0.25) circle (0.05) node[right]{$x$};
	\draw (4,0.25) circle (0.5); \fill (4,0.25) circle (0.05) node[right]{$y$};
	
	% Abbildungspfeile
	\draw[->] (-4, 2.25) to[out=250,in=110] node[left]{$\varphi$} (-4,0.75);
	\draw[->] (3.75, 2.5) to[out=250,in=110] node[left]{$\psi$} (3.75,0.75);
\end{tikzpicture}\end{center}

% Definition 3.1
\begin{dfn}
  Die lineare Abbildung $\Phi_{*p} \colon \T_pM \to \T_{\Phi(p)}N$ heißt das \CmMark{Differential} von $\Phi$ in $p$. Der Rang von $\Phi_{*p}$ bezeichnet man als den Rang von $\Phi$ in $p$.
\end{dfn}

% Lemma 3.2
\begin{lemma}[Differentiale in lokalen Koordinaten]
  Sind $\varphi$ und $\psi$ Karten von $M$ und $N$ um $p$ und $\Phi(p) = q$, sowie $\pdifffrac[p]{}{x^i}$ und $\pdifffrac[q]{}{y^i}$ die Standardbasen von $\T_pM$ und $\T_qN$ bezüglich der Karten $\varphi$ und $\psi$, so gilt:
  \begin{align*}
    \Phi_{*p}\pdifffrac[p]{}{x^i} = \sum \partial_i(\psi^j \circ \Phi \circ \varphi^{-1})(\varphi(p))\pdifffrac[q]{}{y^j}.
  \end{align*}
  Die partielle Ableitung $\partial_i(\psi^j \circ \Phi \circ \varphi^{-1})(\varphi(p))$ bezeichnet man auch kurz $\frac{\partial \Phi^j}{\partial x^i}(p)$.
\end{lemma}

\begin{bem}
  Aus der Linearität von $\Phi_{*p}$ folgt, dass für $X_p = \sum \xi^i\pdifffrac[p]{}{x^i} \in \T_pM$ und $\Phi_{*p}X_p = \sum \eta^j\pdifffrac[q]{}{y^j}$ gilt:
  \begin{align*}
    \eta^j = \sum \frac{\partial \Phi^j}{\partial x^i}\xi^i \text{, beziehungsweise } \eta = \D(\psi \circ \Phi \circ \varphi^{-1})\xi.
  \end{align*}
\end{bem}

\begin{proof}
  \begin{align*}
    \underbrace{\left(\Phi_{*p}\pdifffrac[p]{}{x^i}\right)}_{\textcolor{red}{\in \T_qN}}(\psi^j) = \pdifffrac[p]{}{x^i}(\psi^j \circ \Phi) = \partial_i (\psi^j \circ \Phi \circ \varphi^{-1})(\varphi(p)) = \frac{\partial \Phi^j}{\partial y^i}(p).
  \end{align*}
\end{proof}

\begin{bem}[Charakterisierung durch Kurven]
  Ist $[c] \in \T_p\textcolor{red}{M}$, so gilt für $f \in C^{\infty}(M)$:
  \begin{align*}
    \Phi_{*p}[c](f) = [c](f \circ \Phi) = \difffrac[t=0]{}{t}(\underbrace{f \circ \Phi \circ c)}_{\substack{\text{glatte Kurve}\\ \text{auf }N}} = [\Phi \circ c](f)
  \end{align*}
  also $\Phi_{*p}[c] = [\Phi \circ c]$.
\end{bem}

\begin{bem}[Tangentialräume an Untermannigfaltigkeiten der $\R^n$]
  Ist $U$ eine Untermannigfaltigkeit in $\R^n$ mit den Eigenschaften
  \begin{itemize}
  \item $F \colon U \to M \cap F(U)$ ein Homöomorphismus,
  \item $\D F|_x\colon \R^m \to \R^{m+k}$ injektiv für alle $x \in U$.
  \end{itemize}
  Dann ist $\psi = F^{-1}$ eine Karte von $M$. Es bezeichnen $\pdifffrac[p]{}{y^i}$ die Standardbasis bezüglich $\psi$ und $\pdifffrac[x]{}{x^i}$ die Standardbasis bezüglich der kanonischen Karte $\Id_{\R^m}$ des $\R^m$.\\
  Dann gilt für $g \in C^{\infty}(M)$ beliebig:
  \begin{align*}
    & \pdifffrac[p]{}{y^i}(g) = \partial_i(g \circ \psi^{-1})(\underbrace{\psi(p)}_{=x}) = \partial_i(g \circ F)(x) = F_{*x}\left(\pdifffrac[p]{}{x_i}\right)(f).\\
    & F_{*x}\left(\pdifffrac[p]{}{x^i}\right) = F_{*x}[t \mapsto x + te_i] = [t \mapsto F(x+te_i)] \sim \difffrac[t=0]{}{t}F(x+te_i) = \D F|_x(e_i) = \partial_iF|_x.\\
    & \T_pM \quot{=} \left<\partial_1F|_x, \ldots, \partial_m F|_x\right).
\end{align*}
%$\pdifffrac[a]{b}{c}, \pdifffrac[a]{b}{}, \pdifffrac[a]{}{c}, \pdifffrac{b}{c}, \pdifffrac{}{c}, \pdifffrac{b}{}$
\end{bem}

% Abb 4/2

\begin{bem*}[Eigenschaften des Differentials]\hfill
  \begin{itemize}
  \item (Kettenregel) Sind $\Phi \colon M \to N$ und $\Psi \colon N \to P$ glatt, so gilt:
    \begin{align*}
      (\Psi \circ \Phi)_{*p} = \Psi_{*\Phi(p)} \circ \Phi_{*p}.
    \end{align*}
  \item Ist $\textcolor{red}{\Phi} \colon M \to N$ ein Diffeomorphismus, so ist $\Phi_{*p}$ ein Vektorraumisomorphismus. % Verwendet die Kettenregel
  \item (Satz von der Umkehrabbildung) Ist $\Phi \colon M \to N$ glatt und $\Phi_{*p}$ bijektiv, so existieren Umgebungen $U$ von $p$ und $V$ von $\Phi(p)$, so dass $\Phi|_{U} \colon U \to V$ ein Diffeomorphismus ist.
  \end{itemize}
\end{bem*}

% Definition 2.3
\begin{dfn}[Reguläre Punkte, Submersion, Immersion]
  Es sei $\Phi \colon M \to N$ glatt.
  \begin{itemize}
  \item Es Punkt $p \in M$ heißt \CmMark{regulärer Punkt} von $\Phi$, wenn $\Phi_{*p}$ surjektiv ist. Ein Punkt $q \in N$ heißt regulärer Wert, wenn jeder Punkt $p \in \Phi^{-1}(q)$ regulär ist.
  \item Die Abbildung $\Phi$ heißt \CmMark{Submersion}, wenn $\Phi$ surjektiv ist und alle $p \in M$ reguläre Punkte sind.
  \item Die Abbildung $\Phi$ heißt \CmMark{Immersion}, wenn für alle $p \in M$ $\Phi_{*p}$ injektiv ist.
  \item Die Abbildung $\Phi$ heißt \CmMark{Einbettung}, wenn $\Phi$ Immersion und Homöomorphismus auf sein Bild ist.
  \end{itemize}
\end{dfn}

\begin{bsp}
  \begin{itemize}
  \item Betrachte eine Abbildung $\Phi$

    % Abb 4/3

    Immersion: $\difffrac{}{t}$ Basis von $\T_x\R$, $\Phi_{*x}(\difffrac{}{t}) \quot{=} \difffrac{}{t}\Phi$
  \item $\R \to \R^2 \cong \C, t \mapsto e^{it}$ ist eine Immersion aber ebenfalls nicht injektiv.
  \item $\R \to S^1 \subset \C, t \mapsto e^{it}$ ist Immersion und Submersion.
  \item $\R \to S^1 \times \R, t \mapsto (e^{it},t)$ ist eine Einbettung.

    % Abb 4/4

  \item Ist $M \subset N$ Untermannigfaltigkeit, so ist $\imath \colon M \hookrightarrow N$ eine Einbettung.% INKLUSIONSSPFEIL
  \end{itemize}
\end{bsp}

% Satz 3.4
\begin{satz}
  Es seien $M$ und $N$ glatte Mannigfaltigkeiten, $\Phi \colon M \to N$ eine glatte Abbildung und $p \in M$, sowie $q = \Phi(p)$. Es bezeichnen $m$ und $n$ die Dimensionen von $M$ und $N$ und $r$ den Rang von $\Phi$ in $p$. Dann gelten folgende Aussagen:
  \begin{itemize}
  \item Zu jeder Karte $\psi$ von $N$ um $q$ mit $\psi(q) = 0$ existiert eine Karte $\alpha$ von $M$ um $p$ mit $\alpha(p) = 0$ und glatte Funktionen $f^{r+1},\ldots,f^n$ mit
    \begin{align*}
      \left(\psi \circ \Phi \circ \alpha^{-1}\right)\left(x^1,\ldots, x^m\right) = \left(x^1, \ldots, x^{r}, f^{r+1}(x), \ldots, f^n(x)\right).
    \end{align*}
  \item Falls der Rang von $\Phi$ auf einer Umgebung von $p$ konstant $r$ ist, so existieren Karten $\alpha$ um $p$ mit $\alpha(p) = 0$ und $\beta$ um $q$ mit $\beta(q) = 0$, so dass
    \begin{align*}
      \left(\beta \circ \Phi \circ \alpha^{-1}\right)\left(x^1, \ldots, x^m\right) = \left(x^1, \ldots, x^r, 0, \ldots, 0\right).
    \end{align*}
  \end{itemize}
\end{satz} 

% Korollar 3.5
\begin{kor}
  \begin{enumerate}[label=(\roman*)]
  \item Falls $\Phi$ auf einer offenen Umgebung von $P = \Phi^{-1}(q)$ konstanten Rang $r$ hat, so ist $P$ eine Untermannigfaltigkeit der Kodimension $r$.
  \item Ist $q$ ein regulärer Wert von $\Phi$, so ist $P = \Phi^{-1}(q)$ eine Untermannigfaltigkeit von $M$ der Kodimension $n$.\\
    Beispiel: $\|\cdot\|^{-1}(1) = S^n \supset \R^{n+1} \to \R, x \mapsto \|n\|$.
  \item Ist $\Phi_{*p}$ injektiv, so existiert eine Umgebung $U$ von $p$, so dass $\Phi(U) = Q \subset N$ eine Untermannigfaltigkeit von $N$ ist.
  \item Ist $\Phi$ eine Einbettung, so ist $Q = \Phi(M)$ eine $m$-dimensionale Untermannigfaltigkeit von $M$ und $\Phi \colon M \to Q$ ist ein Diffeomorphismus.
  \end{enumerate}
\end{kor}

\begin{proof}
  ad (i): Ist $p \in P = \Phi^{-1}(q)$. Nach der zweiten Aussage des vorrangegangenen Satzes existieren Karten $(\alpha,U), (\beta, V)$ mit 
  \begin{align*}
    (\beta \circ \Phi \circ \alpha^{-1})(x^1,\ldots,x^n) = (x^1, \ldots,x^r, 0, \ldots, 0)
  \end{align*}
  und es gilt:
  \begin{align*}
    \alpha(P \cap U) & = (\alpha \circ \Phi^{-1} \circ \beta^{-1})(0) \\
    & = \{x \in \alpha (U) \mid x^1 = \ldots = x^r = 0\} = \alpha(U) \cap \{0\} \times \R^{m-r}.
  \end{align*} 
  ad (ii): Ist $q$ ein regulärer Wert von $\Phi$, so existieren nach dem ersten Teil des vorigen Satzes Karten $\psi,\alpha$ mit 
  \begin{align*}
    (\psi \circ \Phi \circ \alpha^{-1})(x^1, \ldots, x^m) = (x^1, \ldots, x^n) \ m \geq n = r \quad \forall x \in \alpha(U).
  \end{align*}
  Es gilt also für alle $u \in U$:
  \begin{align*}
    \Rang \Phi_{*u} = \Rang \D(\psi \circ \Phi \circ \alpha^{-1})|_x = \Rang
    \left(\begin{array}{ccc|c}
      1 &  & 0 & \\
      & \ddots & & 0 \\
      0 & & 1 & 
    \end{array}\right)
    = \textcolor{red}{n}
  \end{align*}
  Damit folgt die Behauptung aus (i).\\
  ad (iii): $\Phi_{*p}$ ist injektiv $\Rightarrow r = m \leq n$. Nach Wahl von Karten wie in (ii):
  \begin{align*}
    & (\psi \circ \Phi \circ \alpha^{-1})(x^1, \ldots, x^m) = (x^1, \ldots, x^m,f^{m+1}(x), \ldots, f^n(x))\\
    & \Rang \Phi_{*u} = \Rang 
    \left( \begin{array}{ccc}
      1 & & 0 \\
        & \ddots &  \\
      0 & & 1 \\
      \hline
        & 0      & 
      \end{array} \right)
    = m % Matrix 4/5
  \end{align*}
  Nach der ersten Aussage des letzten Satzes erhalten wir spezielle Karten:
  \begin{align*}
    (\beta \circ \Phi \circ \alpha^{-1})(x^1, \ldots, x^m) = (x^1, \ldots, x^m, 0, \ldots, 0) \in \R^{m} \times \{0\},
  \end{align*}
  wobei $\beta$ eine adaptierte Karte für $\Phi(U) = Q$ ist.\\
  (iv) folgt aus (iii).
\end{proof}

%%% Local Variables: 
%%% mode: latex
%%% TeX-master: "../skript-diffgeom"
%%% End: 



%-_-_-_-_-_-_-_-_-_-_-_-_-_-_ Anhang -_-_-_-_-_-_-_-_-_-_-_-_-_-_-_-_

\appendix

%-_-_-_-_-_-_-_-_-_-_-_-_-_-_ Uebungen -_-_-_-_-_-_-_-_-_-_-_-_-_-_-_-_

\chapter{"Ubungen}

% Die Benennung der "section" so aendern, dass "\"Ubung 123 vom " am Anfang steht
% Der Code ist fast genau der vom Anfang der Praeambel, dort steht die Erklaerung
\renewcommand*{\othersectionlevelsformat}[3]{\ifstr{#1}{section}{\"Ubung\ #3\ vom\ }{#3\autodot\enskip}}

% Das Format der "section" in Kopfzeile der rechten Seiten
%\renewcommand*{\sectionmarkformat}{\"Ubung \thesection\autodot\ vom\enskip}

\setcounter{section}{-1}

\section{22. Oktober 2012}
\setcounter{Aufg}{0} %Damit die Aufgaben jedes Mal bei Aufgabe 1 anfangen
\setcounter{Loes}{0}

\begin{dfn}[Gra\ss mann-Manningfaltigkeiten]
Sei $k \le n$ und $\Gr_k(\R^n) = \{ V \subseteq \R^n | \ddim V = k\}$
\end{dfn}

\emph{Behauptung:} $\Gr_k(\R^n)$ ist eine glatte Manningfaltigkeit.

\begin{bem}
F"ur $k = 1$ ist $\Gr_1(\R^n) = \R \P^n$
\end{bem}

$X_0 \in \Gr_k(\R^n) \Rightarrow \R^n = X_0 \oplus X_0^\perp$, $X_0 = \mspan\{e_1,\ldots ,e_k\}$ \marginnote{\begin{tikzpicture}
\draw[->] (-1.5, 0) -- (1.5,0) node[below]{$X_0$};
\draw[->] (0,-1.5) -- (0,1.5) node[left]{$X_0^\perp$};
\draw (-1,-1) --(1,1) node[below]{$Y$};
\end{tikzpicture}}

Definiere $U_{X_0} := \{Y \in \Gr_k(\R^n) | Y \cap X_0^\perp = \{0\}\}$. F"ur $Y \in U_{X_0}$ gilt dann: $\pr_{X_0}(Y) = X_0 \Rightarrow \pr_{X_0}$ ist ein Isomorphismus
	\[X_0 \xrightarrow{(\pr_{X_0}|_Y)^{-1}} Y \xrightarrow{\pr_{X_0^\perp}} X_0^\perp \]
Definiere
	\[ \varphi_{X_0}: \left\{\begin{array}{ccl} U_{X_0} &\to& \Hom(\underbrace{X_0, X_0^\perp}_{\cong \R^{k \cdot (n-k)}}) \\
		Y &\mapsto& \pr_{X_0^\perp} \circ (\pr_{X_0}|_Y)^{-1} \end{array}\right.\]
	\[ \varphi_{X_0}^{-1}: \left\{\begin{array}{ccl} \Hom(X_0, X_0^\perp) &\to& U_{X_0} \\
		f &\mapsto& \Graph(f) = \{x + xf | x \in X_0\} \end{array}\right.\]
\emph{Zu zeigen:}\begin{enumerate}
\item
	$U_{X_0}$ ist offen
\item
	$\varphi_{X-0}, \varphi_{X_0}^{-1}$ sind beide stetig
\item
	$\varphi_{X-0} \circ \varphi_{X_0}^{-1}$ ist glatt
\item
	$\Gr_k(\R^n)$ ist Hausdorffsch und hat eine abz"ahlbare Basis der Topologie
\end{enumerate}

\textbf{Welche Topologie eigentlich?} Sei $V = \{ (v_1,\ldots, v_k) \in (\R^n)^k | v_1,\ldots, v_k$ linear unabh"angig$\}$ und $\pi: V \to \Gr_k(\R^n), (v_1,\ldots ,v_k) \mapsto \mspan\{v_1,\ldots ,v_k\}$. Topologie auf $\Gr_k(\R^n)$: induziert von der Quotientopologie auf $V\modulo{\sim\pi}$, also
	\[U \subset \Gr_k(\R^n) \text{ offen } \Leftrightarrow \pi^{-1}(U) \text{ offen} \]
$V$ ist offen in $(\R^n)^k$: $V = \widetilde{\ddet}^{-1}(\R^{\left(\begin{smallmatrix}n \\ k\end{smallmatrix}\right)} \setminus \{0\})$ mit $\widetilde{\ddet}(v_1,\ldots,v_k) = (\ddet(k \times k\text{-Untermatrizen}))$

\emph{Zu zeigen:} $\pi^{-1}(U_{X_0})$ offen

$\pi^{-1}(U_{X_0}) = \{(v_1,\ldots ,v_k) \in V |\ \pr_{X_0}|_{\mspan\{v_i\}} \text{ hat vollen Rang}\} = \{(v_1,\ldots ,v_k) \in V|\ \pr_{X_0}(V-i) \text{ sind linear unabh"angig} \} = (\widetilde{\ddet} \circ (\pr_{X_0},\ldots ,\pr_{X_0}))^{-1}(\R^{\left(\begin{smallmatrix}n \\ k\end{smallmatrix}\right)} \textcolor{red}{\setminus \{0\}})$

$\Rightarrow U_{Y_0}$ ist offen.
\begin{description}[font=\normalfont\bfseries]
\item[zu 2)]\begin{description}[font=\normalfont\itshape]
	\item[Behauptung:] f"ur alle $Y \in U_{X_0}$ gibt es genau eine Basis $(y_1,\ldots ,y_k)$ von $Y$ sodass $\pr_{X_0}(y_i) = x_i$ f"ur eine feste Orthonormalbasis $(x_1,\ldots ,x_k)$ von $X_0$. Bezeichnet $B(Y)$ diese Basis, so ist $B: U_{X_0} \to V$ stetig
	\item[Beweis:] Existenz und Eindeutigkeit $\checkmark$ ($\pr_{X_0}$ ist Isomorphismus)
		
		F"ur $(v_1,\ldots ,v_k) \in \pi^{-1}(U_{X_0})$ ist $B \circ \pi(v_1,\ldots ,v_k) = ((\pr_{X_0}|_{\mspan\{v_1,\ldots ,v_k\}})^{-1} X_i)_{i \le k}$. Die Darstellungsmatrix von $(\pr_{X_0}|_{\mspan\{v_1,\ldots ,v_k\}})^{-1}$ bez"uglich $\{x_i\}, \{y_i\}$ h"angt stetig von den $v_i$ ab. Daraus folgt dass $B \circ \pi|_{\pi^{-1}(U_{X_0}}$ stetig ist, womit auch $B$ stetig ist. Es gilt:
			\[ B(Y)_i = \underbrace{x_i}_{\in X_0} + \underbrace{\varphi_{X_0}(Y)_{X_i}}_{\in X_0^\perp} \qquad \text{(*)} \]
		$\Rightarrow \varphi_{X_0}(Y)_{x_i}$ h"angt stetig von $Y$ ab.
		
		$\Rightarrow$ Darstellende Matrix von $\varphi_{X_0}(Y)$ h"angt stetig von $Y$ ab $\Rightarrow \varphi_{X_0}$ ist stetig
		
		(*) $\Rightarrow B(\varphi_{X_0}^{-1}(A))_i = x_i + Ax_i \Rightarrow B \circ \varphi_{X_0}^{-1}$ ist stetig (sogar glatt)
			\[ \varphi_{X_0}^{-1} = (\pi \circ B) \circ \varphi_{X_0}^{-1} \text{ ist stetig} \]
	\end{description}
\item[zu 3)]
	$\varphi_{X_0} \circ \varphi_{\tilde X_0}^{-1} = \varphi_{X_0} \circ \pi \circ (\underbrace{B_{\tilde X_0} \circ \varphi_{\tilde X_0}^{-1}}_{\text{ist glatt, s. o.}})$ ist glatt.
	
	$\varphi_{X_0} \circ \pi$ ist glatt, da $\varphi_{X_0} \circ \pi(v_1,\ldots ,v_k)(x_i) = (\underbrace{B_{X_0} \circ \pi}_{\substack{\text{glatt (Darst.}\\ \text{aus Beh.)}}})(v_1,\ldots ,v_k) - x_i$
\item[zu 4)]
	Abz"ahlbare Basis der Topologie wird von $V$ geerbt. \emph{Hausdorffsch}: Seien $X_0 \ne \tilde X_0 \in \Gr_k(\R^n) \xRightarrow[\text{"Ub. Aufg.}]{\text{L. A.}} \exists Z \subseteq \R^n, \ddim Z = n-k: Z \cap X_0 = \{0\} = Z \cap \tilde X_0, U_{\underbrace{Z^\perp}_{k\text{-dim}}} \ni X_0, \tilde X_0$
	
	\emph{Alternativ:} Sei $w \in X_0 \setminus \tilde X_0$ und $d_w^2: \Gr_k(\R^n) \to \R, Y \mapsto (\dist(w, Y))^2 \Rightarrow d_w^2(X_0) = 0, d_w^2(\tilde X_0) > 0$. Falls $d_w^2$ stetig ist, gilt: $(d_w^2)^{-1}((-\infty, \frac{d_w^2(\tilde X_0)}{2}))$ und $(d_w^2)^{-1}((\frac{d_w^2(\tilde X_0)}{2}, \infty))$ trennen und sind offen.
\end{description}

%-_-_-_-_-_-_-_-_-_-_-_-_-_-_ Stichwortverzeichnis -_-_-_-_-_-_-_-_-_-_-_-_-_-_-_-_

\printindex

%-_-_-_-_-_-_-_-_-_-_-_-_-_-_ Symbolverzeichnis -_-_-_-_-_-_-_-_-_-_-_-_-_-_-_-_

\printglossaries

%-_-_-_-_-_-_-_-_-_-_-_-_-_-_ Literaturverzeichnis -_-_-_-_-_-_-_-_-_-_-_-_-_-_-_-_

\end{document}
