%%
%% Skript Differentialgeometrie im Wintersemester 12/13
%% Zur Vorlesung von Dr. Grensing am KIT Karlsruhe
%%
%% Hauptdokument
%%

\documentclass[paper=A4, twoside, chapterprefix=true, bibliography=totoc, headsepline]{scrbook}

% Abstand zwischen zwei Textbloecken
\setlength\parskip{\smallskipamount}

% Nummerierung der Paragraphen anpassen (sonst kommt etwas wie "Definition 2.9.1" heraus)
\renewcommand{\thesection}{\arabic{section}}

% Aendert die Kapitelbeschriftung in der Kopfzeile der linken Seiten
\renewcommand*{\chaptermarkformat}{\chapappifchapterprefix{\ }\thechapter:\enskip}
\renewcommand*{\sectionmarkformat}{\thesection\autodot\enskip}

% Einheitliche Schriftart (KOMA Script verwendet fuer einige Ueberschriften eine serifenlose Schrift, mischt also
% Schriftarten. Ich habe mir die Argumente dafuer durchgelsen und war nicht ueberzeugt. Wenn jemand, der mehr als
% ich von der Materie versteht, anderer Meinung ist kann er diese Zeilen hier einfach auskommentieren)
\setkomafont{chapter}{\Huge\bfseries\rmfamily}
\setkomafont{chapterentry}{\bfseries\rmfamily}
\setkomafont{disposition}{\bfseries\rmfamily}
\setkomafont{descriptionlabel}{\bfseries\rmfamily}

\usepackage[utf8x]{inputenc}
\usepackage[T1]{fontenc}
\usepackage{lmodern}

\usepackage[ngerman]{babel}
\usepackage[top=2.5cm, bottom=3cm, left=2.5cm, right=4.5cm]{geometry}

%\usepackage[nobottomtitles]{titlesec}


% Importiert den Abgek"urzten Hash der aktuellen Git ID
%%% This file is generated by Makefile.
%%% Do not edit this file!\n%%%
	\gdef\GITAbrHash{98aff2a}	\gdef\GITAuthorDate{Tue Jan 15 23:43:34 2013 +0100}	\gdef\GITAuthorName{Jan-Bernhard Kordaß}\gdef\GITVersionTag{
0.1
}


% Lade saemtliche erweiterte LaTeX Konfigurationen
%%
%% Skript Differentialgeometrie im Wintersemester 12/13
%% Zur Vorlesung von Dr. Grensing am KIT Karlsruhe
%%
%% Preambel
%%

\usepackage{fancyhdr} % erlaubt mehr Optionen in Kopf- und Fusszeile

% header configuration
%\pagestyle{fancy}
%\fancyhf{
%\lhead[\thepage]{\rightmark}}
%\rhead[\nouppercase{\leftmark}]{\thepage}		

\usepackage{xcolor} % Farben
\usepackage{marginnote} % Randnotizen
\usepackage{enumitem} % Fuer mehr Einstellungmoeglichkeiten bei Aufzaehlungen
\usepackage{xifthen} % Erlaubt die Verwendung von if-then-else Befehlen im Code
\usepackage{index} % Index erzeugen
\newindex{default}{idx}{ind}{Stichwortverzeichnis}
\usepackage{xspace} % intelligende Leerzeichen bei Macros
\usepackage[normalem]{ulem} % unterstreichen von Text
\usepackage{cancel} % schraeg durchstreichen von Text
\usepackage{units} % schoenere Schreibweise fuer Einheiten mit Bruechen, laedt auch das nicefrac Paket

\renewcommand{\CancelColor}{\color{gray}} % Farbe zum schraegen Druchstreichen in grau

\definecolor{rltred}{rgb}{0.75,0,0}
\definecolor{rltgreen}{rgb}{0,0.5,0}
\definecolor{rltblue}{rgb}{0,0,0.75}

%sichere Fraben, die sich auch bei einem SW-Druck unterscheiden lassen (Platzhalter momentan)
\definecolor{color1}{cmyk}{1,0,0,0} %cyan
\definecolor{color2}{rgb}{0,1,0} %green

\usepackage[hyperindex=true]{hyperref} % Verweise als Hyperlinks
\hypersetup{
	pdftitle={Differentialgeometrie Dr. Grensing},
	pdfsubject={Differentialgeometrie Geometrie},
	pdfkeywords={Differentialgeometrie Grensing},
	pdfproducer={pdfLaTeX},
	pdfpagemode={UseOutlines},
	colorlinks=true,
	bookmarksopen=true,
	bookmarksnumbered=true,
	urlcolor=rltblue,
	filecolor=rltgreen,
	linkcolor=rltblue,
	backref=true,
	pagebackref=true,
	pdfpagemode=None,
	citecolor=rltblue
}

% vertausche die Theta, Phi, Rho und Epsilon mit ihren "var" Versionen
%\newcommand{\swapcmd}[2]{
%	\let\temp\#1
%	\left\#1\#2
%	\let\#2\temp
%}
\let\temp\phi
\let\phi\varphi
\let\varphi\temp

\let\temp\theta
\let\theta\vartheta
\let\vartheta\temp

\let\temp\epsilon
\let\epsilon\varepsilon
\let\varepsilon\temp

\let\temp\rho
\let\rho\varrho
\let\varrho\temp


%%
%% Fuer Zeichnungen in TikZ
%%

\usepackage{tikz} 
\usetikzlibrary{matrix,arrows,calc,intersections, through, positioning, patterns, decorations.text, decorations.pathmorphing, decorations.markings, decorations.pathreplacing}

% neue Befehle fuer haeufig benutzte TikZ Formen; erstes Argument steht fuer die Position, Zweites fuer die Groesse
\newcommand{\tikzrichtung}[3][1]{ % zeichnet eine rote Linie von einem Punkt in eine Richtung mit rotem Knoten am Ende
	\draw[red] #2 -- ($#2 + #1*#3$) circle(0.05);
}
\newcommand{\tikzgitter}[3][0.25]{ %Hilfsgitter, das optionale Argument steht fuer die kleine Maschenweite, die Grosse ist doppelt so gross
	\draw[step=#1,gray!15] #2 grid #3;
	\draw[step=2*#1,gray!30] #2 grid #3;
	\fill (0,0) circle(0.1); 
}
\newcommand{\tikzschnuller}[2][1]{
	% definiere die Knoten relativ zum ersten Knoten skaliert mit dem Faktor
	\coordinate (schnuller1) at #2; \coordinate (schnuller2) at ($(schnuller1)+#1*(-1.75,-0.75)$); \coordinate (schnuller3) at ($(schnuller1)+#1*(-2.5,-2.25)$); \coordinate (schnuller4) at ($(schnuller1)+#1*(0,-2)$); \coordinate (schnuller5) at ($(schnuller1)+#1*(1.75,-0.25)$);
    %\fill (schnuller1) circle (0.05) (schnuller2) circle (0.05) (schnuller3) circle (0.05) (schnuller4) circle (0.05) (schnuller5) circle (0.05);
    
    % die Richtungsvektoren der Bezier Tangenten fuer die einzelnen Knoten (der Erste und der letzte haben keine Tangente)
    \coordinate (ctrls1) at ($#1*(1.25,0.25)$); \coordinate (ctrls2) at ($-0.5*(ctrls1)$); \coordinate (ctrls4) at ($#1*(1,-1)$); \coordinate (ctrls3) at ($-0.5*(ctrls4)$); \coordinate (ctrls6) at ($#1*(1,1.5)$); \coordinate (ctrls5) at ($-0		.33*(ctrls6)$);
	% die eigentlichen Tangenten
    \coordinate (tang1) at ($(schnuller2)+(ctrls1)$); \coordinate (tang2) at ($(schnuller2)+(ctrls2)$); \coordinate (tang3) at ($(schnuller3)+(ctrls3)$); \coordinate (tang4) at ($(schnuller3)+(ctrls4)$); \coordinate (tang5) at ($(schnuller4)+(ctrls5)$); \coordinate (tang6) at ($(schnuller4)+(ctrls6)$);
    %\fill[red] (tang1) circle (0.05); \fill[red] (tang2) circle (0.05); \fill[red] (tang3) circle (0.05); \fill[red] (tang4) circle (0.05); \fill[red] (tang5) circle (0.05); \fill[red] (tang6) circle (0.05);
    %\draw[red] (tang1) -- (tang2); \draw[red] (tang3) -- (tang4); \draw[red] (tang5) -- (tang6);
	
	\draw (schnuller1) ..controls(schnuller1) and (tang1).. (schnuller2) ..controls(tang2) and (tang3).. (schnuller3) ..controls(tang4) and (tang5).. (schnuller4) ..controls(tang6) and (schnuller5).. (schnuller5);
	
	% zeichne nun das Loch in der Mitte
	\def\angle{20} % Rotationswinkel
	\coordinate (c) at ($#2+#1*(-1.25,-1.25)$); % Mittelpunkt der Ellipse die den unteren Bogen bildet
	\begin{scope}
		\clip[rotate=\angle] ($(c)-#1*(1,0.6)$) rectangle ($(c)+#1*(1,-0.1)$);
		\path[draw,rotate=\angle,name path=l] (c) ellipse(#1*1 and #1*0.5);
	\end{scope}
	\path[name path=u,rotate=\angle] ($(c)-#1*(0,0.5)$) ellipse(#1*0.75 and #1*0.5);
	\path[name intersections={of=u and l}];
	\begin{scope}
		\clip[rotate=\angle] (intersection-1) rectangle ($(intersection-2)+#1*(0,0.5)$);
		\draw[rotate=\angle] ($(c)-#1*(0,0.5)$) ellipse(#1*0.75 and #1*0.5);
	\end{scope}		
}
%\newcommand{\tikzkartoffel}[2][1]{}		
\newcommand{\tikzsegel}[2][1]{
	% definiere die Knoten relativ zum ersten Knoten skaliert mit dem Faktor
	\coordinate (segel1) at #2; \coordinate (segel2) at ($(segel1)+#1*(4,1.5)$); \coordinate (segel3) at ($(segel1)+#1*(2,-0.5)$);
	%\fill (segel1) circle (0.05) (segel2) circle (0.05) (segel3) circle (0.05);
	
	% die Richtungsvektoren der Bezier Tangenten fuer die einzelnen Knoten (der Erste und der letzte haben keine Tangente)
	\coordinate (ctrls1) at ($#1*(0.75,1.5)$); \coordinate (ctrls2) at ($#1*(-0.75,0.25)$); \coordinate (ctrls3) at ($#1*(-0.5,-0.25)$); \coordinate (ctrls4) at ($#1*(0.25,1)$); \coordinate (ctrls5) at ($#1*(-0.375,0.375)$); \coordinate (ctrls6) at ($#1*(0.75,0.125)$);
	% die eigentlichen Tangenten
	\coordinate (tang1) at ($(segel1)+(ctrls1)$); \coordinate (tang2) at ($(segel2)+(ctrls2)$); \coordinate (tang3) at ($(segel2)+(ctrls3)$); \coordinate (tang4) at ($(segel3)+(ctrls4)$); \coordinate (tang5) at ($(segel3)+(ctrls5)$); \coordinate (tang6) at ($(segel1)+(ctrls6)$);
%	\fill[red] (tang1) circle (0.05); \fill[red] (tang2) circle (0.05); \fill[red] (tang3) circle (0.05); \fill[red] (tang4) circle (0.05); \fill[red] (tang5) circle (0.05); \fill[red] (tang6) circle (0.05);
 %   \draw[red] (tang1) -- (segel1) -- (tang6); \draw[red] (tang2) -- (segel2) -- (tang3); \draw[red] (tang4) -- (segel3) -- (tang5);
	
	\draw (segel1) ..controls(tang1) and (tang2).. (segel2) ..controls(tang3) and (tang4).. (segel3) ..controls(tang5) and (tang6).. (segel1) --cycle;
}
\newcommand{\tikztorus}[2][1]{
%	\tikzgitter{(-6,-1)}{(6,5)}
	% zuerst die aeussere Ellipse
	\draw[] #2  ellipse (#1*2 and #1*1);
	
	% dann das Loch
	\begin{scope}
      \clip ($#2 - #1*(1, 0.5)$) rectangle ($#2 + #1*(1, 1)$);
      \path[draw,name path=gkreis] ($#2 + #1*(0,0.75)$) ellipse (#1*1.25 and #1*1);
    \end{scope}
    \path[name path=kkreis] ($#2 - #1*(0,0.5)$) ellipse (#1*1 and #1*0.75);
    \path[name intersections={of=gkreis and kkreis}];
    \begin{scope}
      \clip (intersection-1) rectangle ($(intersection-2)+(0,0.5)$);
      \draw ($#2 - #1*(0,0.5)$) ellipse (#1*1 and #1*0.75);
    \end{scope}
    
    % definiere Werte auf die wir in der restlichen Zeichnung zurueckgreifen koennen
	\def\torusbreite{#1*2}
	\def\torushoehe{#1*1}
	\def\torusdicke{#1*0.75}
	\coordinate (torusUntenLoch) at ($#2 - #1*(0,0.25)$);
	\coordinate (torusUnten) at ($#2 - #1*(0,1)$);
}

\usepackage[toc]{glossaries} % Symbolverzeichnis
\glossarystyle{treehypergroup}
\makeglossaries

% Mathe Pakete
\usepackage{amsmath}
\usepackage{amssymb}
\usepackage{stmaryrd}
\usepackage{bm} % fette Mathe Zeichen
%\usepackage{amsthm}
\usepackage[hyperref,amsmath,thmmarks,thref]{ntheorem}

% common mathematical operators and sets
\DeclareMathOperator{\aff}{aff} % affine Huelle
\DeclareMathOperator{\cs}{cs} % allgemeiner Cosinus
\DeclareMathOperator{\ct}{ct} % allgemeiner Cotangens
\DeclareMathOperator{\ddet}{det} % Determinante
\DeclareMathOperator{\diam}{diam} % diameter
\DeclareMathOperator{\dist}{dist} % distance
\DeclareMathOperator{\ddim}{dim} % dimension
\DeclareMathOperator{\dR}{dR} % deRahm
\DeclareMathOperator{\eukl}{eukl} % euklidisch
\DeclareMathOperator{\ggT}{ggT} % goesster gemeinsamer Teiler
\DeclareMathOperator{\id}{id} % identity
\DeclareMathOperator{\inh}{inh} % Inhalt
\DeclareMathOperator{\grad}{grad} % Gradient
\DeclareMathOperator{\kgV}{kgV} % kleinstes gemeinsames Vielfaches
\DeclareMathOperator{\mspan}{span} % Lineare Huelle
\DeclareMathOperator{\n}{n} % Umlaufzahl
\DeclareMathOperator{\offen}{offen}
\DeclareMathOperator{\pr}{pr}
\DeclareMathOperator{\res}{res} % Residuum
\DeclareMathOperator{\rg}{rg} % rank (i)
\DeclareMathOperator{\scal}{scal} % Skalarkruemmung
\DeclareMathOperator{\sgn}{sgn} % Signum
\DeclareMathOperator{\sn}{sn} % allgemeiner Sinus
\DeclareMathOperator{\spur}{spur} % Spur
\DeclareMathOperator{\supp}{supp} % support
\DeclareMathOperator{\sternf}{sternf}
\DeclareMathOperator{\tr}{tr} % Spur

\DeclareMathOperator{\Abb}{Abb} % maps
\DeclareMathOperator{\Aut}{Aut} % automorphisms
\DeclareMathOperator{\Bild}{Bild}
\DeclareMathOperator{\Charakteristik}{char}
\DeclareMathOperator{\Charakt}{char}
\DeclareMathOperator{\D}{D} % Jacobi matrix or derivative
\DeclareMathOperator{\Diff}{Diff}
\DeclareMathOperator{\End}{End} % endomorphisms
\DeclareMathOperator{\Gl}{GL} % general linear group
\DeclareMathOperator{\GL}{GL} % general linear group
\DeclareMathOperator{\Gr}{Gr}
\DeclareMathOperator{\Graph}{Graph}
\DeclareMathOperator{\Hh}{H} % Hessesche
\DeclareMathOperator{\Hom}{Hom} % homomorphisms
\DeclareMathOperator{\Id}{id} % identity
\DeclareMathOperator{\Ind}{Ind} % Index
\DeclareMathOperator{\Inn}{Inn} % Untergruppe der inneren Automorphismen
\DeclareMathOperator{\Iso}{Iso}
\DeclareMathOperator{\Kern}{Kern}
\DeclareMathOperator{\Oo}{O} % Matrizen sie mit ihrer Transponierten multipiziert die Einheitsmatrix ergeben
\DeclareMathOperator{\Relation}{\scriptstyle\mathrm{R}} % custom Relation
\DeclareMathOperator{\Rang}{Rang} % rank (ii)
\DeclareMathOperator{\SL}{SL} % Matrizen mit Deteminante 1
\DeclareMathOperator{\Stab}{Stab} % Stabilisator
\DeclareMathOperator{\Sym}{Sym} % symmetric group
\DeclareMathOperator{\T}{T} % tangent bundle

\DeclareMathOperator{\ric}{ric} % Ricci Tensor
\DeclareMathOperator{\Ric}{Ric} % Ricci Tensor field

\newcommand{\Zentrum}[1]{\ensuremath{\mathrm Z(#1)}} % Zentrum einer Gruppe
\newcommand{\Ordnung}[1][]{ % Ordnung einer Gruppe
  \ifthenelse{\isempty{#1}}{
    \#
  }{
    \left|#1\right|
  }
}

% X als Malzeichen
\newcommand{\X}{\times}

% ein schoener aussehender Faktorraum anstatt einfach nur A/B
\newcommand{\FakRaum}[2]{
	\raisebox{0.7ex}{\ensuremath{#1}}
	\ensuremath{\mkern-3mu}\big/\ensuremath{\mkern-3mu}
	\raisebox{-0.6ex}{\ensuremath{#2}}}
\newcommand{\smallFakRaum}[2]{
	\scriptsize{\raisebox{0.7ex}{\ensuremath{#1}}
	\ensuremath{\mkern-3mu}\ / \ensuremath{\mkern-3mu}
	\raisebox{-0.6ex}{\ensuremath{#2}}}}

%Realteil und Imaginaerteil
\renewcommand{\Re}{\ensuremath{\operatorname{Re}}} % <-- sollte man da nicht besser \DeclareMathOperator verwenden?
% [kann man nicht, weil \Re und \Im schon deklariert sind
% einen "\ReDeclareMathOperator" Befehl gibt es nicht. JB]
\renewcommand{\Im}{\ensuremath{\operatorname{Im}}}

% \DeclareMathOperator{\Real}{Re} % real part
% \DeclareMathOperator{\Imag}{Im} % imaginary part

% canonic sets
\DeclareMathOperator{\C}{\mathbb{C}}
\DeclareMathOperator{\F}{\mathbb{F}}
%\DeclareMathOperator{\F}{\mathbb{H}}
\DeclareMathOperator{\K}{\mathbb{K}}
\DeclareMathOperator{\N}{\mathbb{N}}
\DeclareMathOperator{\Q}{\mathbb{Q}}
\DeclareMathOperator{\R}{\mathbb{R}}
\DeclareMathOperator{\RP}{\mathbb{RP}} % real projection plane
\DeclareMathOperator{\Tor}{\mathbb{T}} % torus
\DeclareMathOperator{\Z}{\mathbb{Z}}
\DeclareMathOperator{\B}{\mathbb{B}} % unit ball

%  geschwungene Buchstaben
\DeclareMathOperator{\calD}{\mathcal{D}}
\DeclareMathOperator{\calI}{\mathcal{I}}
\DeclareMathOperator{\calJ}{\mathcal{J}}
\DeclareMathOperator{\calL}{\mathcal{L}}
\DeclareMathOperator{\calT}{\mathcal{T}}
\DeclareMathOperator{\calV}{\mathcal{V}}

% Redeclare \P (Prim or Propability) and put the old, reversed "breakline P" in \BreakLineP
\let\BreakLineP\P
\renewcommand{\P}{\ensuremath{\mathbb{P}}}
\let\umlautsH\H % long Hungarian umlaut (double acute)
\renewcommand{\H}{\ensuremath{\mathbb{H}}}

% Differentialoperatoren als Brüche
\newcommand{\dop}{\mathrm{d}}	
\newcommand{\difffrac}[3][]{\ifthenelse{\isempty{#1}}{\frac{\dop #2}{\dop #3}}{\left. \frac{\dop #2}{\dop #3} \right|_{#1}}}
\newcommand{\pdifffrac}[3][]{\ifthenelse{\isempty{#1}}{\frac{\partial #2}{\partial #3}}{\left. \frac{\partial #2}{\partial #3} \right|_{#1}}}

% stellt einen großen vertikalen Strich an einen Term, nuetzlich in Bruechen
\newcommand{\bigvert}[1]{\left. #1 \right|}

% quotient space or group
\newcommand{\modulo}[1]{\ensuremath{/_{\displaystyle #1}}}

% declaring Index for group theory
\newcommand{\Index}[2]{\ensuremath{(#1 \SlimDdot #2)}}


% canonic environments
\newcounter{thmglobal}
%\swapnumbers
\theoremstyle{plain}

%%%%%%%%%%%%%%%%%%%%%%%%%%%%%%%%%%%%%%%%%%%%%%%%%%%%%%%%%%%%%%%%%%%%%%%%%%%%%%%%%%%%%%%%%%%%%%%%%%%%%%%%%%%%%%%%%%%%%%%%%%%%%%%%%%%%%%%%%%%%%%%%%%%%%%%%%%%%%%%%%%%%%%%%

\makeatletter

% Options
\newboolean{enableDeepNumbering}
\setboolean{enableDeepNumbering}{false}

\DeclareOption{deepnum}{
  \setboolean{enableDeepNumbering}{true}
}

\newboolean{enableMarginThm}
\setboolean{enableMarginThm}{false}

\DeclareOption{marginthm}{
  \setboolean{enableMarginThm}{true}
}

\ProcessOptions\relax

% call makeindex for an index register
%\makeindex

% Name language settings

% theorem names, ngerman
\newcommand{\cmLangThmSatz}{Satz\xspace}
\newcommand{\cmLangThmLemma}{Lemma\xspace}
\newcommand{\cmLangThmKor}{Korollar\xspace}
\newcommand{\cmLangThmProp}{Proposition\xspace}

\newcommand{\cmLangThmDfn}{Definition\xspace}
\newcommand{\cmLangThmBsp}{Beispiel\xspace}

\newcommand{\cmLangThmBem}{Bemerkung\xspace}

% short forms, ngerman
\newcommand{\cmLangThmShortSatz}{Satz\xspace}
\newcommand{\cmLangThmShortLemma}{Lemma\xspace}
\newcommand{\cmLangThmShortKor}{Kor\xspace}
\newcommand{\cmLangThmShortProp}{Prop\xspace}

\newcommand{\cmLangThmShortDfn}{Def\xspace}
\newcommand{\cmLangThmShortBsp}{Bsp\xspace}

\newcommand{\cmLangThmShortBem}{Bem\xspace}


\theoremstyle{plain}
\newtheorem{Dfn}{\cmLangThmDfn}[chapter]
\newtheorem{Satz}[Dfn]{Satz}
\newtheorem{Lemma}[Dfn]{\cmLangThmLemma}
\newtheorem{Kor}[Dfn]{\cmLangThmKor}
\newtheorem{Prop}[Dfn]{\cmLangThmProp}
\theorembodyfont{\normalfont}
\newtheorem{Bsp}[Dfn]{\cmLangThmBsp}
\newtheorem{Bem}[Dfn]{\cmLangThmBem}
\newtheorem{Aufg}{Aufgabe}
\newtheorem{Loes}{L\"osung}

\theoremstyle{nonumberplain}
\newtheorem{dfn}{\cmLangThmDfn}
\newtheorem{satz}{Satz}
\newtheorem{lemma}{\cmLangThmLemma}
\newtheorem{kor}{\cmLangThmKor}
\newtheorem{prop}{\cmLangThmProp}

\newtheorem{bsp}{\cmLangThmBsp}
\newtheorem{bem}{\cmLangThmBem}

\theoremsymbol{\ensuremath{\Box}}
\theorembodyfont{\normalfont}
\newtheorem{bew}{Beweis}

\theoremsymbol{}
\theoremstyle{empty}
\newtheorem{emptythm}{}% druckt nur den optionalen Namen aus

\theoremstyle{break}

% Add unnumbered Theorems, use amsthm style in both style modes
\theoremstyle{plain}
%\newtheorem*{satz*}{\cmLangThmSatz}
%\newtheorem*{lemma}{\cmLangThmLemma}
%\newtheorem*{kor*}{\cmLangThmKor}
%\newtheorem*{prop*}{\cmLangThmProp}

\theoremstyle{definition}
%\newtheorem*{dfn}{\cmLangThmDfn}
%\newtheorem*{bsp*}{\cmLangThmBsp}

\theoremstyle{remark}
%\newtheorem*{bem*}{\cmLangThmBem}
\newtheorem*{beh*}{Behauptung}


% 2-level numbering$
\numberwithin{thmglobal}{section}

% check if 3-level numbering is enabled
\ifthenelse{\boolean{enableDeepNumbering}}{
  \numberwithin{thmglobal}{subsection}
}{}


% some other customisations

% changing enumerations
\setlist[enumerate]{label=(\arabic*), itemsep=0cm, leftmargin=2cm}
\setlist[itemize]{itemsep=0cm} %\setlist[itemize]{itemsep=0cm, leftmargin=2cm}

% replace the slim emptyset symbol
%\let\emptyset\varnothing

% set line distances
\linespread{1.1}

% Add a ':' for mathmode with tiny whitespaces around
\newcommand{\SlimDdot}{\ensuremath{\mathrm{:}}}


% headline and cover generation commands

% generate a simple headline
% usage: \CmHeadline[date]{title}{topic}{author}
\newcommand{\CmHeadline}[4][]{
  \begin{minipage}[t]{\textwidth}
    \huge{\textbf{#2}}\\
    \large{#3, #4}\relax
    \ifthenelse{\isempty{#1}}{}{\relax\large{, #1}}
  \end{minipage}
}

% generate a simple cover page
% usage: \CmCover[type(,skript)]{title}{subtitle}{date}
\newcommand{\CmCover}[4][]{
  \thispagestyle{empty}
  \begin{titlepage}
    \begin{center}
      \begin{minipage}[b]{0.8\textwidth}
	\vspace*{5cm}
        \ifthenelse{\isempty{#1}}{
          % Default cover arrangement
          \Huge{\textbf{#2}}\\[0.5cm]
          \huge{#3}\\[0.8cm]
          \Large{#4}
        }{
          \ifthenelse{\equal{#1}{skript}}{
            % Cover for lecture scripts
            \huge{#3}\\[0.5cm]
            \Huge{\textbf{#2}}\\[0.5cm]
            \Large{#4}
          }{}
        }
      \end{minipage}
    \end{center}
  \end{titlepage}
  \pagebreak
}


% indexing support

% Print and index given text
% usage: \CmIndex{[(optionally put another text for the index in here)]{(text to print and add to index)}
\newcommand{\CmIndex}[2][]{\ifthenelse{\isempty{#1}}{\index{#2}}{\index{#1}}#2}

% Highlight(bold) and index the given text
% usage: \CmMark[(optionally put another text for the index in here)]{(text to highlight and add to index)}
\newcommand{\CmMark}[2][]{\textbf{\CmIndex[#1]{#2}}}
%\newcommand{\CmMark}[2][]{\textbf{\CmIndex[#1]{#2}}\marginnote{\scriptsize{#2}}} % <-- Das ist nur ein Versuch die definierten Begriffe in den Rand zu stellen


% sectioning support

% Prints a description for a section in italic, bold. Most likely to use right under \section.
% usage: \CmSectionDescription{(short description of section contents)}
\newcommand{\CmSectionDescription}[1]{
  \vspace{-0.3cm}
  \hangindent=0.4cm
  \hangafter=0
  \begin{itshape}
    \textbf{#1}
  \end{itshape}
  \vspace{0.3cm}
}

% Starts a new paragraph inside of a theorem environment (as defined above)
% usage \CmSubThm[(paragraph title)]
\newenvironment{CmSubThm}[1]{
  \begin{itemize}[leftmargin=0.5cm,label=]
  \item
    \ifthenelse{\isempty{#1}}{}{
      \hspace{-0.5cm}(\textit{#1})\\[0.2cm]
    }
  }{
  \end{itemize}
}


% svg updater
% needs shell escape option

% Checks if the given image file has been modified and a custom command (if possible) via command line to generate something new.
\newcommand{\CmExecuteIfFileNewer}[3]{
  \ifnum\pdfstrcmp{\pdffilemoddate{#1}}
  {\pdffilemoddate{#2}}>0
  {\immediate\write18{#3}}\fi
}

% Tries to include an image file and checks if the given one has been modified. If so it calls inkscape (if possible) via command line to generate new pdf und pdf_tex files from the corresponding svg.
\newcommand{\CmIncludeSvg}[1]{
  \def\svg@filepath{}
  
  % check if corrosponding svg file exists
  \IfFileExists{#1.svg}
  {
    \def\svg@filepath{#1}
  }{
    % if it does not, search in the graphicspath for it
    \expandafter\@tfor\expandafter\currentsvgpath\expandafter:\expandafter=\Ginput@path\do{
      \IfFileExists{\currentsvgpath#1.svg}{
        \edef\svg@filepath{\currentsvgpath #1}
      }{}
    }
  }
  % if something was found, include the graphic, TODO: Make it work correctly
  \ifthenelse{\isundefined{\svg@filepath} \OR \isempty{\svg@filepath}}{
    \PackageError{canonicalmath}{Image file not found!}
  }{
    \PackageWarning{FilePath}{|\svg@filepath|}
    \CmExecuteIfFileNewer{\svg@filepath.svg}{\svg@filepath.pdf}{inkscape -z -D --file=\svg@filepath.svg --export-pdf=\svg@filepath.pdf --export-latex}
    \input{\svg@filepath.pdf_tex}
  }
}

% Tries to include an image file on the center of the margin at the current position.
\newcommand{\CmMarginSvg}[3][0cm]{
  \marginnote{
    \centering
    \def\svgwidth{#3}
    \CmIncludeSvg{#2}
  }[#1]
}

% Tries to include an image file centering it at the current position
\newcommand{\CmPutSvg}[3][0cm]{
  \begin{figure}[h!]
    \vspace{#1}
    \centering
    \def\svgwidth{#3}
    \CmIncludeSvg{#2}
  \end{figure}
}
\makeatother


%%%%%%%%%%%%%%%%%%%%%%%%%%%%%%%%%%%%%%%%%%%%%%%%%%%%%%%%%%%%%%%%%%%%%%%%%%%%%%%%%%%%%%%%%%%%%%%%%%%%%%%%%%%%%%%%%%%%%%%%%%%%%%%%%%%%%%%%%%%%%%%%%%%%%%%%%%%%%%%%%%%%%%%%

\usepackage{mathtools}

\usepackage{graphicx}
\usepackage{float}
\usepackage{transparent}
\usepackage{wrapfig}

\graphicspath{{img/}}

\parindent0pt

% Befehl fuer Anfuerungszeichen unten und oben
\newcommand{\quot}[1]{\textrm{\glqq}{#1}\textrm{\grqq}}



\setlist[enumerate]{label=(\arabic*), itemsep=0cm, leftmargin=1cm}
 
% Lade Eintraege des Glossars
%%
%% Skript Differentialgeometrie im Wintersemester 12/13
%% Zur Vorlesung von Dr. Grensing am KIT Karlsruhe
%%
%% Glossar
%%

\newglossaryentry{Dualraum}{
	name=Dualraum,
	description={Die zu einem Vektorraum \ensuremath{V} "uber einem K"orper \ensuremath{K} geh"orende Menge aller linearen Abbildungen von \ensuremath{V} nach \ensuremath{K}. Der Dualraum selbst ist ebenfalls ein Vektorraum mit Skalarmultiplikation mit Elementen aus \ensuremath{K}},
	text={Dualraum}
}

\newglossaryentry{GL}{
	name={GL\ensuremath{_n}},
	description={Allgemeine lineare Gruppe, Gruppe aller regul"aren $n \X n$-Matrizen mit Koeffizienten aus einem K"orper $K$},text={\ensuremath{\Gl}},
	symbol={\ensuremath{\Gl}},
	sort=GL
}

\newglossaryentry{Hausdorff-Raum}{
	name=Hausdorff-Raum,
	description={Ein topologischer Raum \ensuremath{M}, in dem es f"ur alle \ensuremath{x, y \in M, x \ne y} disjunkte offene Umgebungen \ensuremath{U(x)} und \ensuremath{U(y)} gibt, es werden also alle paarweise verschiedenen Punkte \ensuremath{x, y} durch Umgebungen getrennt},
	text={Hausdorff-Raum}
}

\newglossaryentry{Homoeomorphismus}{
	name={Hom"oomorphismus},
	description={Eine bijektive, stetig differenzierbare Abbildung zwischen zwei Objekten, deren Umkehrabbildung ebenfalls stetig differenzierbar ist},
	text={Hom{\"o}omorphsimus},
	sort={Homoomorphismus}
}

\newglossaryentry{topologischer Raum}{
	name=Topologischer Raum,
	description={Eine Menge \ensuremath{X} zusammen mit einer Topologie \ensuremath{T}, das hei\ss t einem Mengensystem das offene Teilmengen von \ensuremath{X} definiert, wobei die leere Menge, die Grundmenge, der Durchschnitt endlich vieler offener Mengen und die Vereinigung beliebig vieler offener Mengen offen sind},
	text={topologischer Raum}
}

\newglossaryentry{Topologie}{
	name=Topologie,
	description={Ein Mengensystem das Teilmengen einer Grundmenge als offene Mengen definiert, wobei die leere Menge und die Grundmenge selbst offen sind und der Durchschnitt endlich vieler offener Mengen und die Vereinigung beliebig vieler offener Mengen wieder offen sind}
}

\newglossaryentry{spur}{
	name={Spur},
	description={Die Summe aller Diagonalelemente einer quadratischen Matrix. F"ur eine $n \X n$-Matrix $A$ gilt also $\spur A = a_{11} + \ldots + a_{nn}$},text={\ensuremath{\spur}},
	symbol={\ensuremath{\spur} oder \ensuremath{\tr}},
	sort=spur
}

\newglossaryentry{supp}{
	name={Tr\"ager},
	description={Die abgeschlossene H"ulle $\supp(f) = \overline{\{ x \in I \mid f(x) \ne 0 \}}$ der Nichtnullstellenmenge einer Funktion $f: I \to \R$},text={\ensuremath{\supp}},
	symbol={\ensuremath{\supp}},
	sort=Traeger
}

\newglossaryentry{Diffeomorphismus}{
	name={Diffeomorphismus},
	description={Eine bijektive, stetige Abbildung zwischen zwei Objekten, deren Umkehrabbildung ebenfalls stetig ist}
}

\newglossaryentry{Homomorphismus}{
	name={Hompmorphismus},
	description={asdf}
}

\newglossaryentry{Endomorphismus}{
	name={Endomorphismus},
	description={Ein Homomorphismus einer Struktur in sich selbst}
}

\newglossaryentry{Isomorphismus}{
	name={Isomorphismus},
	description={Ein Homomorphismus einer Struktur in sich selbst}
}

\newglossaryentry{Produkttopologie}{
	name={Produkttopologie},
	description={F"ur eine Menge von topologischen R"aumen $X_i$ mit $i \in I$ f"ur eine Indexmenge $I$ und ihr kartesisches Produkt $X = \Pi_{i \in I} X_i$ ist $p_i: X \to X_i$ die kanonische Projektion. Die Produkttopologie auf $X$ ist definiert als die Topologie mit den wenigsten offenen Mengen, bezüglich der alle Projektionen $p_i$ stetig sind}
}

\newglossaryentry{Kurve}{
	name={Kurve},
	description={asdf}
}



% Informationen zum Skript
\subject{inoffizielles Skript}
\title{Differentialgeometrie}
\subtitle{Gehalten von Dr. S. Grensing im Wintersemester 2012/13}
\author{getippt von Aleksandar Sandic\thanks{\href{mailto:aleksandar.sandic@student.kit.edu}{Aleksandar.Sandic@student.kit.edu}} und Jan-Bernhard Korda\ss\thanks{\href{mailto:jan-bernhard.kordass@student.kit.edu}{Jan-Bernhard.Kordass@student.kit.edu}}}


%%
%% Inhalt
%% 

\begin{document}

\maketitle

% Inhaltsverzeichnis
\pdfbookmark[1]{Inhaltsverzeichnis}{contents}
\setlength\parskip{0.6pt}
\tableofcontents

\chapter*{Vorwort}
\addcontentsline{toc}{chapter}{Vorwort}
\setlength\parskip{\smallskipamount} Skript Version \textbf{\GITVersionTag}
(git: \GITAbrHash) \quad Build: \today

\section*{\"Uber dieses Skript}
Dies ist eine Mitschrieb der Vorlesung \quot{Differentialgeometrie} von Dr. S. Grensing gehalten im Wintersemester 2012/13 am Karlsruher Institut f"ur Technologie.
Dr. Grensing ist nicht f"ur den Inahlt verantwortlich und es besteht weder eine Garantie "uber Vollst"andigkeit, noch Korrektheit der enthaltenen Aussagen.

\section*{Wer}
Das Skript wurde in Zusammenarbeit von Jan-Bernhard Korda"s und Aleksandar Sandic getippt.
Bei Anmerkungen bzw. beim Auffinden von Fehlern schickt bitte eine E-Mail an
\begin{center}
  \href{mailto:jan-bernhard.kordass@student.kit.edu}{jan-bernhard.kordass@student.kit.edu}\\
  \href{mailto:aleksandar.sandic@student.kit.edu}{aleksandar.sandic@student.kit.edu}
\end{center}

\section*{Wo}
Momentan ist das Skript noch im Aufbau und der Quellcode befindet sich auf GitHub.
Nach Ende der Vorlesung wenn das Skript fertig ist werden wir es im Mitschiebwiki hosten.
Wer den Quellcode nicht kompilieren kann oder m"ochte kann auch ein fertiges PDF herunterladen, es handelt sich aber unter Umst"anden nicht um die aktuellste Version.

Aktuellsten Version auf GitHub: \url{https://github.com/Tarcvar/Skript-DiffGeom}\\
Fertiges PDF: \url{https://dl.dropbox.com/u/22321777/skript-diffgeom.pdf}\\
Seite der Vorlesung: \url{http://www.math.kit.edu/iag5/edu/difgeo2012w/de}\\
Mitschiebwiki: \url{http://mitschriebwiki.nomeata.de/}

\section*{\textcolor{red}{Hinweise f"ur Autoren}}
\textcolor{red}{(dieser Teil wird nach der Fertig"-stellung entfernt)}

Hier ist eine Liste an Konventionen f"ur das Skript.
Diese Regeln sind nicht in Stein gemei\ss elt und wenn jemand anderer Meinung ist k"onnen wir uns absprechen und die Regeln "andern, aber ansonsten sollte jeder der am Skript mitarbeitet sich nach M"oglichkeit an sie halten.
Selbst wenn wir unsere Meinung anschlie"send "andern ist es viel leichter die "Anderungen vorzunehmen wenn alles einheitlich ist.

\begin{description}[font=\normalfont\itshape]
\item[Anf"uhrungszeichen:]
	sie k"onnen nicht direkt im Code gesetzt werden, man muss entweder den genauen Befehl kennen oder man nimmt einfach unseren eigenen \verb|\quot{}| Befehl.
\item[Abs"atze:]
	benutzt nicht \verb|\\| f"ur neue Abs"atze, dieser Befehl ist dazu da um einen Zeilenumbruch zu erzwingen.
	F"ur einen neuen Absatz f"ugt eine leere Zeile ein (also zweimal die Enter Taste dr"ucken), der Unterschied ist dass dadurch ein etwas gr"o\ss erer Abstand zwischen Abs"atzen eingef"ugt wird der die Unterscheidung leichter macht (der Abstand wird durch \verb|parskip| bestimmt).
	Es ist in der Literatur "ublich die erste Zeile eines Absatzes einzur"ucken (\verb|parindent|), allerdings sind unsere Abs"atze daf"ur zu kurz, es w"urde h"asslich aussehen alle paar Zeilen eine Einr"uckung zu haben.
	Nach einem Theorem wird automatisch ein noch gr"o"serer Abstand eingef"ugt, es gibt also keinen Grund an der Stelle etwas zu "andern.
\item[Theoreme:]
	das \verb|ntheorem| Paket stellt theoremartige Umgebungen zur Verf"ugung.
	Unsere Konvention besagt dass die Befehle aller nummerierten Theoreme mit Gro"sbuchstaben anfangen sollen (zum Beispiel \verb|Dfn| f"ur Definitionen).
	F"ur alle nummerierten Theoreme gibt es auch eine nicht nummerierte Version mit Kleinbuchstaben (also \verb|dfn|).
	F"ur Beweise sollte \verb|bew| anstelle von \verb|proof| verwendet werden.
	Bei jedem Theorem kann man als optionalen Parameter einen speziellen Namen vergeben, wie \verb|\begin{Satz}[Satz des Pythagoras]|.
	Wenn in der Vorlesung keine Umgebung verwendet wurde dann benutzt auch keine im Skript, wenn wir "uberall Bemerkung auf Bemerkung haben bl"aht es das Skript auf und st"ort den Lesefluss.
	Bitte nummeriert nur Theoreme die auch in der Vorlesung nummeriert wurden.
\item[Spezielle Theoreme:]
	wenn eine spezielle Theoremumgebung ben"otig wird, wie zum Beispiel \quot{Anwendung}, f"ur die es sich nicht lohnen w"urde eine eigene Theoremumgebung zu definieren, kann die \verb|\emptythm| Umgebung verwendet werden.
	Gebt dabei den gew"unschten Namen als optionales Argument ein, also \verb|\begin{emptythm}[Anwendung]|.
	Der Name wird dabei ganz normal gesetzt, also ohne Klammern wie sonst bei optionalen Argumenten "ublich.
\item[Listen und Aufz"ahlungen:]
	die Umgebungen \verb|itemize|, \verb|enumerate| und \verb|description| k"onnen dank des \verb|enumitem| Pakets stark beeinflusst werden.
	Im Allgemeinen sind die Standardeinstellungen in Ordnung, allerdings m"ussen wir manchmal ein Ausnahme machen und Listen "andern.
	Die h"aufigste Option wird \verb|label=| sein um die Nummerierung zu "andern.
	Haltet euch an den Standard der Vorlesung.
	Eintr"age sind standardm"a\ss ig leicht einger"uckt, das ist gut f"ur die Lesbarkeit, aber bei mehr als einer oder zwei Zeilen pro Eintrag wird es h"asslich.
	Mit der Option \verb|leftmargin=*| wird der Listeineintrag ganz an den Rand geschoben (beziehungsweise soweit es geht bei Unterlisten), das spart Platz und sieht besser aus.
	Die Schriftart von Labels bei der \verb|description| Umgebung kann mit der Option \verb|font=| eingestellt werden.
\item[Zeichungen:]
	sollten vorzugsweise in TikZ gschrieben werden anstatt ein externes Programm zu verwenden.
	TikZ liefert saubere Ergebnisse, passt nahtlos in das Design, erzeugt keine weiteren Abh"angigkeiten und die Zeichnungen k"onnen jederzeit von jedem ge"andert oder korrigiert werden.
	F"ur h"aufig verwendete Formen wurden eigene Befehle geschrieben, mehr dazu findet man im Quellcode.
\item[Umlaute und "s:]
	je nach Kodierung kann man diese Zeichen entweder direkt im Code benutzen oder man schreibt \verb|"a|, \verb|"o|, \verb|"u| und \verb|"s|.
	Wir k"onnen f"ur's Erste bei normalen Umlauten bleiben, ich w"urde aber trotzdem am Ende einmal Suchen\&Ersetzen durchlaufen lassen.
\item[Farben:]
	Blau ist f"ur Hyperlinks reserviert, rot f"ur Fehler und Unsicherheiten, grau f"ur Anmerkungen und Notizen f"ur den Leser.
	Der Vorteil von blau ist dass es bei einem Schwarz-Wei"s Ausdruck nicht von schwarz unterscheidbar ist.
\item[Randnotizen:]
	k"onnen mit \verb|\marginnote{}| gesetzt werden (beachtet dass keine Leerzeilen im Befehl erlaubt sind).
	Randnotizen sind gut f"ur Anmerkungen oder kleine Zeichnungen.
	Wir k"onnen bei Bedarf auch einstellen dass bei Definitionen der definierte Begriff im Rand daneben steht.
	Das \verb|ntheorem| Paket erlaubt es auch die Nummer von Theormen in den Rand zu stellen, das w"are auch eine "Uberlegung wert.
\item[Glossareintr"age:]
	das \verb|glossaries| Paket erlaubt es ein Glossar zu erstellen.
	Hier sollten alle Begriffe stehen die \emph{nicht} in dieser Vorlesung definiert wurden aber verwendet werden.
	Die Idee ist dass der Leser manchmal nicht die genaue Definition wei"s und es ist praktisch wenn man eine kleine Ged"achtnisst"utze im Skript hat.
\end{description}

\subsection*{Erstellung \textcolor{red}{*wichtig*}}
Abschlie"send noch eine Bemerkung zum Erstellen einer \quot{neuen} Version, nach einer "Anderung.
Damit im Skript die korrekte Git Versions-ID geladen wird und weiterhin sichergestellt wird, dass alle Referenzen verf"ugbar sind, die Bibliographie und auch die Eintr"age des Stichwortverzeichnisses aktuell sind, sei ausdr"ucklich empfohlen das beigelegte \verb|Makefile| zu verwenden (zumindest unter UNIX-artigen Systemen, Windows-Nutzer m"ussen all dies manuell sicher stellen).

Dazu navigiert man in der Konsole in das Verzeichnis des Skriptes und ruft \verb|make| via \verb|make -k| auf.

%% 
%% Vorlesungsinhalt
%% 

%%
%% 1. Vorlesung <2012-10-16 Tue>
%% 
%% Skript Differentialgeometrie im Wintersemester 12/13
%% Zur Vorlesung von Dr. Grensing am KIT Karlsruhe
%%
%% Mitschrieb und Textsatz von Jan-Bernhard Kordaß.
%%

\section*{"Ubersicht}

\begin{itemize}
\item Mannigfaltigkeiten, Tangentialvektoren
\item Kovariante Ableitung
\item Riemannsche Metriken
\item Krümmung
\item Jacobifelder
\item Satz von Bonnet
\end{itemize}

\chapter{Differenzierbare Mannigfaltigkeiten}

\begin{dfn}
  Eine $n$-dimensionale \CmMark[Mannigfaltigkeit!topologische]{topologische Mannigfaltigkeit} $M$ ist ein topologischer \gls{Hausdorff-Raum} mit einer abzählbaren Basis der \gls{Topologie} in dem zu jedem Punkt $p \in M$ eine offene Menge $U$ mit $p \in U$ existiert und ein \gls{Homoeomorphismus} $\phi \colon U \to V$ auf eine offene Menge $V \subset \R^{n}$.

% Abbildung 1-1
%\CmPutSvg{1-1-topologische-mf}{8.5cm}
\begin{center}\begin{tikzpicture}[font=\scriptsize]
	\draw[->] (-1.5,0) to[out=20, in=160]node[above,font=\scriptsize]{$\phi' \circ \phi^{-1}$} (1.5,0);
	
	\draw[->] (-4,-0.5) -- (-2,-0.5); \draw[->] (-3.75,-0.75) -- (-3.75, 1.25); \node[font=\scriptsize] at (-4, 1.25) {$\R^n$};
	\draw[->] (2,-0.5) -- (4,-0.5); \draw[->] (2.25,-0.75) -- (2.25, 1.25); \node[font=\scriptsize] at (2, 1.25) {$\R^m$};
	
	\node[font=\scriptsize] at (0,2) {$U \cap U' \ne 0$};
	
	\draw (-4.25, 1.75) to[out=70,in=180] (-1.75,3) to[out=300,in=90] (-1.25, 1.25) to[out=180,in=340] (-4.25, 1.75) -- cycle; \node at (-1.25,3) {$M$};
	\filldraw[fill=gray!20] (-2.75,2) circle(0.4); \node[font=\scriptsize] at (-3.25,2.25) {$U$};
	\filldraw[fill=gray!20] (-3,0.25) circle (0.5); \node at (-2.25, 0.5) {$V$};
	\draw[->] (-2.75,1.5) to[out=280,in=80] node[right]{$\phi$} (-2.75,0.75);
			
	\draw (1.75, 1.75) to[out=70,in=180] (4.25,3) to[out=300,in=90] (4.75, 1.25) to[out=180,in=340] (1.75, 1.75) -- cycle; \node at (4.75,3) {$M$};
	\filldraw[fill=gray!20] (3.55,2.25) circle(0.6); \node[font=\scriptsize] at (2.75,2.25) {$U'$};
		
	\coordinate (ctrl0up) at ($(2.5,-0.25) + 0.2*(0.5,2)$); \coordinate (ctrl0down) at ($(2.5,-0.25) + 0.2*(0,-1.5)$);
	\coordinate (ctrl1down) at ($(3,0.25) - 0.1*(0.5,1)$); \coordinate (ctrl1up) at ($(3,0.25) + 0.1*(0.5,1)$);
	\coordinate (ctrl2down) at ($(3,0.7) - 0.1*(0.5,1)$); \coordinate (ctrl2up) at ($(3,0.7) + 0.1*(0.5,1)$);
	\coordinate (ctrl3down) at ($(4,0.5) + 0.3*(-0.5,1)$); \coordinate (ctrl3up) at ($(4,0.5) - 0.3*(-0.25,1)$);
	\coordinate (ctrl4down) at ($(3.75,-0.3) + 0.2*(0.8,1)$); \coordinate (ctrl4up) at ($(3.75,-0.3) - 0.2*(0.7,0.75)$);
	\begin{scope}
		\fill[gray!20] (2.5,-0.25) ..controls(ctrl0up) and (ctrl1down).. (3,0.25) ..controls(ctrl1up) and (ctrl2down).. (3,0.7) ..controls(ctrl2up) and (ctrl3down).. (4,0.5) ..controls(ctrl3up) and (ctrl4down).. (3.75,-0.3) ..controls(ctrl4up) and (ctrl0down).. (2.5,-0.25); \node at (4.25, 0.5) {$V'$};
		\clip(2.5,-0.25) ..controls(ctrl0up) and (ctrl1down).. (3,0.25) ..controls(ctrl1up) and (ctrl2down).. (3,0.7) ..controls(ctrl2up) and (ctrl3down).. (4,0.5) ..controls(ctrl3up) and (ctrl4down).. (3.75,-0.3) ..controls(ctrl4up) and (ctrl0down).. (2.5,-0.25); \node at (4.25, 0.5) {$V'$};
		\fill[gray] (2,0) circle (1);
		 (2.5,-0.25) ..controls(ctrl0up) and (ctrl1down).. (3,0.25) ..controls(ctrl1up) and (ctrl2down).. (3,0.7) ..controls(ctrl2up) and (ctrl3down).. (4,0.5) ..controls(ctrl3up) and (ctrl4down).. (3.75,-0.3) ..controls(ctrl4up) and (ctrl0down).. (2.5,-0.25); \node at (4.25, 0.5) {$V'$};
	\end{scope}
	\draw  (2.5,-0.25) ..controls(ctrl0up) and (ctrl1down).. (3,0.25) ..controls(ctrl1up) and (ctrl2down).. (3,0.7) ..controls(ctrl2up) and (ctrl3down).. (4,0.5) ..controls(ctrl3up) and (ctrl4down).. (3.75,-0.3) ..controls(ctrl4up) and (ctrl0down).. (2.5,-0.25) -- cycle; \node at (4.25, 0.5) {$V'$};
	\draw[->] (3.5,1.5) to[out=280,in=80] node[right]{$\phi'$} (3.5,0.75);
\end{tikzpicture}\end{center}

  $\phi' \circ \phi^{-1}$ ist ein Hom"oomorphismus offener Mengen des $\R^n$ bzw. $\R^m$. Nach dem \href{http://de.wikipedia.org/wiki/Fixpunktsatz_von_Brouwer}{Satz von Brouwer} (1912) gilt dann $m = n$. Damit ist die Dimension einer zusammenh"angenden topologischen Mannigfaltigkeit eindeutig definiert.\\

  Die Abbildung $\phi \colon U \to V \subset \R^n$ hei\ss t \CmMark{Karte} von $M$ um $p$, die Menge $U$ hei\ss t \CmMark{Kartengebiet}.\\

  Eine Menge von Karten $\mathcal A = \{(\phi_{\alpha}, U_{\alpha}) \mid \alpha \in J \}$ hei\ss t \CmMark{Atlas} von $M$, falls $\bigcup_{\alpha \in J}U_{\alpha} = M$.\\

  Ein Atlas $\mathcal A$ von $M$ hei\ss t $C^k$-Atlas, wenn für alle $\alpha, \beta \in J$ mit $U_{\alpha} \cap U_{\beta} \neq \emptyset$ der sogenannte \CmMark{Kartenwechsel}:
  \begin{align*}
    \phi_{\beta} \circ \phi_{\alpha}^{-1}\colon \phi_{\alpha}(U_{\alpha} \cap U_{\beta}) \to \phi_{\beta}(U_{\alpha} \cap U_{\beta})
  \end{align*}
  ein $C^k$-\gls{Diffeomorphismus} ist.

  \begin{center}\begin{tikzpicture}[font=\scriptsize]
  	\draw[->] (-1.5,0) to[out=20, in=160]node[above,font=\scriptsize]{$\phi_\beta \circ \phi^{-1}_\alpha$} (1.5,0);
	
	\draw[->,thick] (-4,-0.5) -- (-2,-0.5); \draw[->,thick] (-3.75,-0.75) --node[left]{$\R^n$} (-3.75, 1.25);
	\draw[->,thick] (2,-0.5) -- (4,-0.5); \draw[->,thick] (2.25,-0.75) -- (2.25, 1.25);
	
	\draw[thick]  (-0.25, 3) to[out=0,in = 150] (2,2.5) -- (1.75, 1.5) to[out=190,in=350] (-1.75, 1.5) to[out=90,in=180] (-0.25, 3) -- cycle; \node at (2.25,2.75) {$M$};
	
	\begin{scope}
		\clip (0.25,2.25) circle(0.5);
		\clip (-0.25,2) circle(0.5);
		\fill[gray!20] (0,2) circle(1);
	\end{scope}
	\draw (0.25,2.25) circle(0.5) (-0.25,2) circle(0.5); \node at (-1, 2.25) {$U_\alpha$}; \node at (1,2.5) {$U_\beta$};
	
	\draw[->] (-0.5,2) to[out=180,in=75] node[left]{$\phi_\alpha$} (-3,0.25);
	\draw[->] (0.5,2.25) to[out=0,in=105] node[right]{$\phi_\beta$} (3,0.25);
  \end{tikzpicture}\end{center}


  Eine Karte $\psi \colon U \to V$ von $M$ hei\ss t \CmMark[Karte!vertr{\"a}gliche]{verträglich} mit einem $C^k$-Atlas $\mathcal A = \{(\phi_{\alpha},U_{\alpha}) \mid \alpha \in J\}$ wenn jeder Kartenwechsel
  \begin{align*}
    \phi_{\alpha} \circ \psi^{-1}: \psi(U \cap U_{\alpha}) \to \phi_{\alpha}(U \cap U_{\alpha})
  \end{align*}
  ein $C^k$-Diffeomorphismus ist, also $\mathcal A' = \mathcal A \cup \{(\psi, U)\}$ ebenfalls ein $C^k$-Atlas ist. Die Menge aller mit $\mathcal A$ verträglichen Karten ist ein \CmMark[Atlas!maximaler]{maximaler $C^k$-Atlas}. Jeder maximale Atlas enthält alle mit ihm verträglichen Karten. Ein maximaler $C^k$-Atlas hei\ss t auch \CmMark{$C^k$-differenzierbare Struktur}.

\end{dfn}

\begin{Dfn}[differenzierbare Mannigfaltigkeit]
  Eine \CmMark[Mannigfaltigkeit!differenzierbare]{differenzierbare Mannigfaltigkeit} der Klasse $C^k$ ist eine topologische Mannigfaltigkeit zusammen mit einer $C^{k}$-differenzierbaren Struktur.\\
\end{Dfn}

\begin{bsp}
  Einige Beispiele f"ur glatte Mannigfaltigkeiten:
  \begin{enumerate}[leftmargin=*,label=\arabic*)]
  \item $M = \R^n, \mathcal A = \{(\Id_{\R^n},\R^n)\}$
  \item $M \subset \R^n$ offen, $\mathcal A = \{(\imath_{M},M)\}$
  \item $S^1 \subset \R^2$ ist eine eindimensionale $C^{\infty}$-Mannigfaltigkeit:
    \begin{align*}
      U = \{(\sin t, \cos t) \mid t \in (0,2\pi)\}
    \end{align*}

    % Abbildung 1-3
    \marginnote{\begin{center}\begin{tikzpicture}[font=\footnotesize]
    		%\draw[step=0.25,gray!15] (-1,-1) grid (1,1); \draw[step=0.5,gray!30] (-1,-1) grid (1,1); \fill (0,0) circle(0.1); %Hilfsgitter
		\draw (0,0) circle (1); \draw[dashed] (0,0) circle (1.1); \draw[dotted] (0,0) circle (0.9); \node at (1,1) {$S^1$};
		\filldraw[fill=white] (-1,0) circle (0.1) (1,0) circle (0.1);
    \end{tikzpicture}\\
    \textcolor{gray}{$S^1$ Einheitskreis}
    \end{center}}[-2cm]
    % \CmMarginSvg[-2cm]{1-3-karten-der-s1}{3cm}

    ist offen in $S^1$ und die Kartenabbildung
    \begin{align*}
      \phi \colon (\sin t, \cos t) \mapsto t
    \end{align*}
    ist ein Hom"oomorphismus.
    \begin{align*}
      \phi' \colon U' = \{(\sin t, \cos t) \mid t \in (-\pi,\pi)\} \to (-\pi,\pi)
    \end{align*}
    ebenfalls. $\mathcal A = \{(\phi, U), (\phi',U')\}$ ist ein Atlas von $S^1$, denn $U \cup U' = S^1$.
    \begin{align*}
      & \phi' \circ \phi^{-1} \colon \phi(U \cap U') \to \phi'(U \cap U')\\
      & (0,\pi)\cup(\pi,2\pi) \to (-\pi,0)\cup(0,\pi), t \mapsto \begin{cases}
        t & 0 < t < \pi\\
        t-2\pi & \pi < t < 2\pi
      \end{cases}
    \end{align*}

  \item Jeder reelle Vektorraum endlicher Dimension ist in kanonischer Weise eine $C^{\infty}$-Mannigfaltigkeit.\\
    W"ahle eine Basis $\{v_1, \ldots, v_n\}$ von $V$. Diese definiert mit
    \begin{align*}
      \phi\left(\sum\lambda_iv_i\right) = (\lambda_1, \ldots, \lambda_n)
    \end{align*}
    eine Bijektion auf $\R^n$. Damit erhält man eine globale Karte von $V$.
    Der zugehörige Atlas h"angt nicht von der Wahl der Basis ab, denn ist $\{w_1, \ldots, w_n\}$ eine weitere Basis von $V$ und $\psi(\sum \lambda_iw_i) = (\lambda_1, \ldots, \lambda_n)$ eine weitere Karte, so ist $\phi \circ \psi^{-1}$ als \gls{Endomorphismus} des $\R^n$ schon $C^{\infty}$.

  \item $S^n = \{(x^0, x^1, \ldots, x^n) \mid \sum_{i = 0}^n(x^{i})^2 = 1\}$.\\

    % Abbildung 1.4
    %\CmMarginSvg{1-4-s3-sphaere}{3.5cm}
    \marginnote{\begin{center}\begin{tikzpicture}[font=\scriptsize]
    		%\draw[step=0.25,gray!15] (-1,-1) grid (1,1); \draw[step=0.5,gray!30] (-1,-1) grid (1,1); \fill (0,0) circle(0.1); %Hilfsgitter
		% Koordinatenachsen mit Beschriftung
		\draw[->] (0,-1.25) -- (0,1.5) node[left]{$x^0$}; \draw[->] (-1.25,0) -- (1.5,0) node[below]{$x^1$}; \draw[->] (1,1) -- (-1.25,-1.25) node[right]{$x^2$}; \node at (1.25, 1.5) {$S^2 \subset \R^3$};
		% Kreis, Ellipse und Gerade (verwende Namen um Schnittpunkt bestimmen zu koennen)
		\path[draw, thick, name path=kreis] (0,0) circle (1) ellipse(1 and 0.45); \path[draw,name path=gerade] (0,1) -- (1,-1.25);
		% Punkte N und p
		\filldraw[fill=white] (0,1) circle (0.05) node[anchor=south west,xshift=-2,yshift=-1.5]{$N$} ($(0,1)+0.35*(1,-1.25)-0.35*(0,1)$) circle (0.05) node[right]{$p$};
		% Punkt phi(p) bei Schnittpunkt von Gerade und Kreis
		\path [name intersections={of=kreis and gerade}]; \filldraw[fill=white] (intersection-2) circle(0.05) node[right]{$\phi(p)$};
	\end{tikzpicture}\end{center}}%[3.5cm]
    
    Betrachte den Nordpol $N = (1,0,\ldots,0)$ und den S"udpol $S = (-1,0,\ldots,0)$ und die Abbildung
    \begin{align*}
      & \phi \colon U = S^{n}\setminus\{N\} \to \R^n, x \mapsto \left(\frac{x^1}{1-x^0}, \ldots, \frac{x^{n}}{1-x^0}\right),\\
      & \psi \colon U' = S^{n} \setminus \{S\} \to \R^n, x \mapsto \left(\frac{x^1}{1+x^0}, \ldots, \frac{x^n}{1+x^0}\right)
    \end{align*}

    Aufgabe: Zeige, dass $(\phi, U), (\psi, U')$ einen $C^{\infty}$-Atlas auf $S^n$ definiert.

  \end{enumerate}
\end{bsp}

\begin{Dfn}[Differenzierbare Abbildungen]
Eine stetige Abbildung $f \colon M \to N$ zwischen glatten Mannigfaltigkeiten $M$ und $N$ hei\ss t \CmMark{glatt} ($C^{\infty}$-differenzierbar), wenn es zu jedem $p \in M$ Karten $(\phi, U)$ in $M$ um $p$ und geeignete $(\phi', U')$ in $N$ um $f(p)$ gibt, so dass $\phi' \circ f\circ\phi^{-1}$ glatt ist.
% Abbildung 1-5
%\CmPutSvg{1-5-glatte-abb}{9cm}
\begin{center}\begin{tikzpicture}[font=\scriptsize]
	%\draw[step=0.25,gray!15] (-5,-1) grid (5,5); \draw[step=0.5,gray!30] (-5,-1) grid (5,5); \fill (0,0) circle(0.1); %Hilfsgitter
	
	% Die Abbildungspfeile
	\draw[->] (-1.5,0) to[out=20, in=160]node[above]{$\phi' \circ f \circ \phi^{-1}$} (1.5,0);
	\draw[->] (-1,2) --node[above]{$f$} (1.5,2);
	
	% Die Achsen
	\draw[->,thick] (-4.5,-0.5) -- (-2,-0.5); \draw[->,thick] (-4.25,-0.75) --node[left]{$\R^n$} (-4.25, 1.25);
	\draw[->,thick] (2,-0.5) -- (4.5,-0.5); \draw[->,thick] (2.25,-0.75) --node[left]{$\R^m$} (2.25, 1.25);
	
	% Die Blasen
	\draw[thick] (-4.25, 1.75) to[out=70,in=180] (-1.75,3) to[out=300,in=90] (-1.25, 1.25) to[out=180,in=340] (-4.25, 1.75) -- cycle; \node[font=\normalfont] at (-1.25,3) {$M$};
	\draw[thick] (1.75, 1.75) to[out=70,in=180] (4.25,3) to[out=300,in=90] (4.75, 1.25) to[out=180,in=340] (1.75, 1.75) -- cycle; \node[font=\normalfont] at (4.75,3) {$N$};
	
	% Die linke Kartoffel (zuerst werden die Punkte definiert, dann die Richtungsvektoren der Splines, dann die Kartoffel selbst)
	\coordinate (kartoffel0l) at (-3.25,1.75); \coordinate (kartoffel1l) at (-3.25,2.5); \coordinate (kartoffel2l) at (-2.25,2.25); \coordinate (kartoffel3l) at (-2.5,1.75);
	\coordinate (ctrlk0l) at (-0.25,0.5); \coordinate (ctrlk1l) at (0.5,0.25); \coordinate (ctrlk2l) at (-0.25,1); \coordinate (ctrlk3l) at (2,0.25);
	\draw (kartoffel0l) ..controls($(kartoffel0l)+0.5*(ctrlk0l)$) and ($(kartoffel1l)-0.3*(ctrlk1l)$).. (kartoffel1l) ..controls($(kartoffel1l)+0.6*(ctrlk1l)$) and($(kartoffel2l)+0.45*(ctrlk2l)$).. (kartoffel2l) ..controls($(kartoffel2l)-0.25*(ctrlk2l)$) and ($(kartoffel3l)+0.15*(ctrlk3l)$).. (kartoffel3l)  ..controls($(kartoffel3l)-0.1*(ctrlk3l)$) and ($(kartoffel0l)-0.9*(ctrlk0l)$).. (kartoffel0l); \node at (-3.5,2.25) {$U$};
	% Der Punkt in der Kartoffel, der Pfeils raus und der Kreis
	\draw[->] (-2.75,2) node[right]{$p$} to[out=280,in=80] node[right]{$\phi$} (-2.75,0); \fill (-2.75,2) circle (0.05);
	\draw (-3,0.25) circle(0.5); \node at (-3.5,0.75) {$V$};
	
	% Die rechte Kartoffel
	\coordinate (kartoffel0r) at (3.25,1.75); \coordinate (kartoffel1r) at (3.5,2.5); \coordinate (kartoffel2r) at (4.5,2.25); \coordinate (kartoffel3r) at (4.25,1.5);
	\coordinate (ctrlk0r) at (-0.25,0.5); \coordinate (ctrlk1r) at (-0.25,0.25); \coordinate (ctrlk2r) at (-0.25,0.5); \coordinate (ctrlk3r) at (0.25,0);
	\draw (kartoffel0r) ..controls($(kartoffel0r)+0.5*(ctrlk0r)$) and ($(kartoffel1r)-(ctrlk1r)$).. (kartoffel1r) ..controls($(kartoffel1r)+(ctrlk1r)$) and ($(kartoffel2r)+(ctrlk2r)$).. (kartoffel2r) ..controls($(kartoffel2r)-(ctrlk2r)$) and ($(kartoffel3r)+(ctrlk3r)$).. (kartoffel3r) ..controls($(kartoffel3r)-(ctrlk3r)$) and ($(kartoffel0r)-(ctrlk0r)$).. (kartoffel0r); \node at (3.25,2.25) {$U'$};
	
	\draw[->] (3.75,2) node[right]{$f(p)$} to[out=280,in=80] node[right]{$\phi'$} (3.75,0); \fill (3.75,2) circle (0.05);
	\draw (3.5,0.25) circle(0.5); \node at (3,0.75) {$V'$};
\end{tikzpicture}

% Korrektur
% \textcolor{red}{Sollte das in der Zeichnung beim unteren Pfeil nicht $\phi'\circ f \circ \phi^{-1}$ hei\ss en?}
% Ja! sollte es! (JB, <2012-11-9 Fri>)

\end{center}

Die Menge aller glatten Abbildungen von $M$ nach $N$ wird $C^{\infty}(M,N)$ genannt.
\end{Dfn}

\begin{emptythm}[Konvention:]
Ab jetzt seien zunächst alle Mannigfaltigkeiten, wie auch alle Abbildungen als glatt vorrausgesetzt.
\end{emptythm}

\begin{bem}
  Da Kartenwechsel $C^{\infty}$ sind, gilt obige Bedingung automatisch für alle Karten von $M$ und $N$ (evtl. nach Einschränkung).
\end{bem}

\begin{bsp}
  Es folgen zwei Beispiele für differenzierbare Abbildungen:
  \begin{enumerate}
  \item $(\phi,U) \in \mathcal A \Rightarrow \phi \in C^{\infty}(U,\R^n)$, denn
    \begin{align*}
      \Id_{\R^n}\circ \phi \circ \phi^{-1} = \phi \circ \phi^{-1} \in C^{\infty}.
    \end{align*}
  \item $f \in C^{\infty}(M,N), \ g \in C^{\infty}(N,P) \Rightarrow g \circ f \in C^{\infty}(M,P)$, denn
    \begin{align*}
      \phi_p \circ g \circ f \circ \phi^{-1}_m = (\phi_p \circ g \circ \phi_n^{-1}) \circ (\phi_n \circ f \circ \phi_m^{-1}) \in C^{\infty}.
    \end{align*}
  \end{enumerate}
\end{bsp}

\begin{Dfn}[Diffeomorphismus]
Eine Abbildung $f \colon M \to N$ hei\ss t \CmMark{Diffeomorphismus}, wenn $f$ bijektiv ist und $f$, sowie $f^{-1}$ $C^{\infty}$-Abbildungen von $M$ nach $N$ sind. Insbesondere haben $M$ und $N$ in diesem Fall dieselbe Dimension. Die Menge der Diffeomorphismen von $M$ nach $M$ wird mit $\Diff(M)$ bezeichnet. $(\Diff(M), \circ)$ ist bez"uglich der Hintereinanderausf"uhrung eine Gruppe.
\end{Dfn}


% 2. Vorlesung <2012-10-19 Fri>

\section{Produkte von Mannigfaltigkeiten}

Es seien $M$ und $N$ glatte Mannigfaltigkeiten der Dimensionen $m$ und $n$. Dann hat $M \times N$ versehen mit der \gls{Produkttopologie}, die Struktur einer Mannigfaltigkeit. Da $M$ und $N$ hausdorffsch sind und abzählbare Basen ihrer Topologie besitzen gilt dies auch für $M \times N$.
Sind $(\phi, U)$ und $(\psi, V)$ Karten von $M$ bzw. $N$, so ist $\phi \times \psi$ ein Homöomorphismus von $U \times V$ auf sein offenes Bild in $\R^m \times \R^n \cong \R^{m + n}$.

Seien $\mathcal A = \{(\phi_{\alpha}, U_{\alpha}) \mid \alpha \in \calI \}$ und $\mathcal A' = \{(\psi_{\beta}, V_{\beta}) \mid \beta \in \calJ\}$ $C^{\infty}$-Atlanten von $M$ und $N$. Dann ist $\mathcal B = \{(\phi_{\alpha}\times\psi_{\beta},U_{\alpha}\times V_{\beta}) \mid (\alpha, \beta) \in \calI \times \calJ\}$ ein $C^{\infty}$-Atlas von $M\times N$, denn 
\begin{align*}
  (\phi_{\alpha} \times \psi_{\beta}) \circ (\phi_{\mu} \times \psi_{\nu})^{-1} = (\phi_{\alpha} \circ \phi_{\mu}^{-1}) \times (\psi_{\beta} \circ \psi_{\nu}^{-1})
\end{align*}
ist ein $C^{\infty}$-Diffeomorphismus. Damit ist $M\times N$ in kanonischer Weise eine glatte $(m+n)$-dimensionale Mannigfaltigkeit. Die kanonischen Projektionen $\pi_M\colon M\times N \to M, \ \pi_N\colon M \times N \to N$ und die Abbildung $\tau \colon M \times N \to N \times M, (p,q) \mapsto (q,p)$ sind glatte Abbildungen.

\begin{bsp}\marginnote{\begin{tikzpicture}[scale=0.9,font=\scriptsize]
	%\draw[step=0.25,gray!15] (0,-2) grid (3,2); \draw[step=0.5,gray!30] (0,-2) grid (3,2); \fill (0,0) circle(0.1); %Hilfsgitter
	\tikztorus{(0,0)};
	\draw (0,0) ellipse (1.5 and 0.7*\torushoehe);
	\begin{scope}
		\clip (torusUntenLoch) rectangle ($(torusUnten)-(0.25,0)$);
		\draw ($0.5*(torusUntenLoch) + 0.5*(torusUnten)$) ellipse (0.2 and 0.5*\torusdicke);
	\end{scope}\begin{scope}
		\clip (torusUntenLoch) rectangle ($(torusUnten)+(0.25,0)$);
		\draw[dashed] ($0.5*(torusUntenLoch) + 0.5*(torusUnten)$) ellipse (0.2 and 0.5*\torusdicke);
	\end{scope}
	\node[below,font=\scriptsize] at (torusUnten) {$S^1$}; \node[font=\normalsize] at (-1.5,1) {$T^2$};
	\draw[->] (1.75,1) node[right]{$S^1$} to[out=180,in=450] (1.25,0.5);
\end{tikzpicture}}
Es folgen einige Beispiele für Produkt-Mannigfaltigkeiten:
\begin{enumerate}[label=\arabic*)]
\item
	Zylinder $\R \times S^1$
\item
	$T^n = \bigtimes_{i=1}^nS^1$
  
    $\iota: \R^m \hookrightarrow \R^n$, $(x^1,\ldots ,x^m) \mapsto (x^1,\ldots ,x^m,0, 0,\ldots)$
\end{enumerate}
\end{bsp}


\section{Untermannigfaltigkeiten}

\begin{Dfn}[Untermannigfaltigkeit]\label{def-1-4}
  Es sei $N$ eine glatte Mannigfaltigkeit. Eine Teilmenge $M \subseteq N$ heißt \CmMark{Untermannigfaltigkeit} von $N$, wenn für alle $p \in M$ eine Karte $(\phi, U)$ von $N$ um $p$ existiert, so dass
  \begin{align*}
    \phi(U \cap M) = \phi(U) \cap \underbrace{(\R^m \times \{0\})}_{\mathclap{\{(x^1,\ldots,x^m,0,\ldots,0) \in \R^m\times\R^{n-m} \cong \R^{n} \}}}
  \end{align*}
gilt. Eine solche Karte heißt an $M$ \CmMark[Karte!adaptierte]{adaptierte Karte}. Die Zahl $n-m$ heißt \CmMark{Kodimension} von $M$ in $N$.

% Abbildung 1-7
%\CmPutSvg{1-7-untermf}{8cm}
\begin{center}\begin{tikzpicture}[font=\scriptsize,scale=0.9]
	%\draw[step=0.25,gray!15] (-6,-1) grid (6,5); \draw[step=0.5,gray!30] (-6,-1) grid (6,5); \fill (0,0) circle(0.1); %Hilfsgitter
	
	\draw[->] (1.5,-0.5) -- (5.5,-0.5)node[below]{$\R^m$}; \draw[->] (1.5,-0.5) -- (1.5,2.5)node[left]{$\R^{n-m}$}; \draw[ultra thick] (1.5,-0.5) --node[above]{$\phi(U\cap M)$} (4,-0.5); \node at (4.5,2) {$\R^n$};
	
	\draw[->] (-3,0.75) to[out=330,in=180]node[below]{$\phi$} (1,0);
	
	\coordinate(1) at (-0.25,2.75); \coordinate(2) at (-5.25,1.25); \coordinate(3) at (-4.25,4.25);
	\coordinate(ctrl1) at (0,-1.25); \coordinate(ctrl2) at (1.75,-1); \coordinate(ctrl3) at (-0.5,-0.25);
	\draw (1) ..controls($(1)+(ctrl1)$) and($(2)+(ctrl2)$).. (2) ..controls($(2)-(ctrl2)$) and($(3)+(ctrl3)$).. (3);
	
	\coordinate (a) at (-1.25, 2.75); \coordinate (b) at (-3.5, 1.75); \coordinate (c) at (-4.25, 1.75); \coordinate (d) at (-4.5, 2.25); \coordinate (e) at (-5, 2.5); \coordinate (f) at (-4.25, 3.5);
	\coordinate (ctrlb) at (1,0); \coordinate (ctrlc) at (0.52,-0.25); \coordinate (ctrle) at (0,-0.25);
	\draw[very thick] (a) ..controls(a) and ($(b) + 1.75*(ctrlb)$).. (b) ..controls($(b) - 0.5*(ctrlb)$) and ($(c) + 0.5*(ctrlc)$).. (c) ..controls($(c) - 0.5*(ctrlc)$) and ($(d) + 0.5*(ctrlc)$).. (d) ..controls($(d) - 0.5*(ctrlc)$) and ($(e) + 0.5*(ctrle)$).. (e) ..controls($(e) - 2*(ctrle)$) and(f).. (f);
	
	\filldraw[fill=white] (b) circle(0.05) node[anchor=north west]{$p$}; \draw (b) circle(0.5); \node at (-4,1.25) {$U$}; \node at (-1.75,2) {$M$}; \node at (-5.5,3.5) {$N$};
	
	\coordinate (cntr) at (-3,3);
	\begin{scope}
		\clip ($(cntr)-(1,0.6)$) rectangle ($(cntr)+(1,-0.1)$);
		\draw[name path=l] (cntr) ellipse(1 and 0.5);
	\end{scope}
	\path[name path=u] ($(cntr) - (0,0.5)$) ellipse(0.75 and 0.5);
	\path[name intersections={of=u and l}];
	\begin{scope}
		\clip (intersection-1) rectangle ($(intersection-2)+(0,0.5)$);
		\draw ($(cntr) - (0,0.5)$) ellipse(0.75 and 0.5);
	\end{scope}
\end{tikzpicture}\end{center}

\end{Dfn}

\begin{Lemma}\label{lemma-1-5}
  Es seien $N$ eine $n$-dimensionale glatte Mannigfaltigkeit und $M \subseteq N$ eine $m$-dimensionale Untermannigfaltigkeit von $N$. Bezeichnet $\mathcal A$ einen $C^{\infty}$-Atlas von $N$ und $\pi \colon \R^n \to \R^m, (x^1, \ldots, x^m,\ldots,x^n) \mapsto (x^1, \ldots, x^m)$, so ist
  \begin{align*}
    \mathcal B = \{(\pi \circ \phi|_{U \cap M},U\cap M) \mid (\phi, U) \in \mathcal A \text{ an } M \text{ adaptierte Karte}\}
  \end{align*}
  ein $C^{\infty}$-Atlas von $M$.
\end{Lemma}

\begin{bew}
  Die Hausdorff-Eigenschaft und die Abzählbarkeit der Topologie werden von $N$ auf $M$ vererbt.\\
  Ist $p \in N$, so existiert eine adaptierte Karte $(\phi,U)$ von $N$ um $p$ und $\pi \circ \phi|_{U \cap M}$ ist ein Homöomorphismus von $U \cap M$ auf eine offene Teilmenge des $\R^m$. Jeder Kartenwechsel
  \begin{align*}
    (\pi \circ \phi|_{U \cap M}) \circ (\pi \circ \psi|_{V \cap M})^{-1} = (\pi \circ \phi) \circ (\psi^{-1} \circ \imath) = \pi \circ (\phi \circ \psi^{-1}) \circ \imath
  \end{align*}
  ist ein $C^{\infty}$-Diffeomorphismus.
\end{bew}

\begin{bem}
  Erinnerung: $M \subseteq \R^n$ heißt glatte $n$-dimensionale Untermannigfaltigkeit des $\R^n$, wenn für alle $p \in M$ eine offene Umgebung $U$ und eine Abbildung $\phi \colon U \to \R^n$  mit folgenden Eigenschaften existiert:
  \begin{enumerate}[label=(\roman*),widest=ii]
  \item $\phi \colon U \to \phi(U)$ ist ein Diffeomorphismus auf sein offenes Bild im $\R^{n}$.
  \item $\phi(U \cap M) = \phi(U) \cap (\R^{m} \times \{0\})$.
  \end{enumerate}
Jedes solche $M$ ist eine Untermannigfaltigkeit im Sinne von Definition \ref{def-1-4}, denn jedes $\phi$ wie oben ist wegen (i) eine Karte von $\R^n$ (im Sinne glatter Mannigfaltigkeiten) und wegen (ii) eine an $M$ adaptierte Karte. Also sind mit Lemma \ref{lemma-1-5} glatte Untermannigfaltigkeiten des $\R^n$ glatte Mannigfaltigkeiten (im allgemeineren Sinne).
\end{bem}

%%% Local Variables: 
%%% mode: latex
%%% TeX-master: "../skript-diffgeom"
%%% End: 
 %%
%% 2. Vorlesung <2012-10-19 Fri>, Fortsetung
%% 
%% Skript Differentialgeometrie im Wintersemester 12/13
%% Zur Vorlesung von Dr. Grensing am KIT Karlsruhe
%%
%% Mitschrieb und Textsatz von Jan-Bernhard Kordaß.
%%

\section{Tangentialvektoren und Tangentialräume}

% Abbildung 2-1
\CmMarginSvg{2-1-tangentialvektoren-motivation}{3.5cm}

Betrachte in der nebenstehenden Abbildung eine differenzierbare Kurve $c \colon (-\varepsilon,\varepsilon) \to S^2$ mit $c(0) = p$. Dann gilt:
\begin{align*}
  0 = \difffrac{}{t} \left<c(t),c(t)\right> = 2\left<c(0),c(0)\right> = 2 \left<c(0),p\right> 
  \Rightarrow c(0) \in p^{\perp}.
\end{align*}

% Bemerke $1 = \left<c(t),c(t)\right>$

Es sei $M$ eine glatte Mannigfaltigkeit und es seien glatte Kurven $c_i\colon (-\varepsilon_i,\varepsilon_i) \to M$ mit $c_1(0) = c_2(0) = p \in M$ gegeben.\\

Die Kurven heißen \CmMark{äquivalent}, wenn es eine Karte $(\varphi,U)$ von $M$ und $p$ gibt, so dass 
\begin{align*}
  \difffrac[t=0]{}{t}(\varphi \circ c_1) = \difffrac[t=0]{}{t}(\varphi \circ c_2)
\end{align*}
gilt.

\begin{lemma}
  Der oben definierte Begriff der Äquivalenz ist unabhängig von der Wahl der Karte.
\end{lemma}

\begin{proof}
  Es sei $(\psi,V)$ eine weitere Karte von $M$ um $p$. Dann gilt:
  \begin{align*}
    \difffrac[t=0]{}{t}(\psi\circ c_1) & = \difffrac[t=0]{}{t}(\psi\circ\varphi^{-1}\circ\varphi \circ c_1) = \D (\psi \circ \varphi^{-1})|_{\varphi(p)} \cdot \difffrac[t=0]{}{t}(\varphi \circ c_1)\\
    & = \D(\psi \circ \varphi^{-1})|_{\varphi(p)} \cdot \difffrac[t=0]{}{t}(\varphi \circ c_2) = \ldots = \difffrac[t=0]{}{t}(\psi \circ c_2).
  \end{align*}
\end{proof}

\begin{dfn}[Geometrische Definition des Tangentialraums]
  Es sei $M$ eine glatte Mannigfaltigkeit und $p \in M$. Ein (geometrischer) \CmMark{Tangentialvektor} an $M$ in $p$ ist eine Äquivalenzklasse von Kurven $c$ mit $c(0) = p$. Die Menge
  \begin{align*}
    \T_{p}^{\text{geo}}M = \{ [c] \mid c \colon (-\varepsilon,\varepsilon) \to M \text{ glatt}, c(0) = p\}
  \end{align*}
  heißt (geometrischer) \CmMark{Tangentialraum} an $M$ in $p$.
\end{dfn}

\begin{bem}
  Mit den Bezeichnungen wie oben ist die folgende Abbildung bijektiv:
  \begin{align*}
    A \colon \T_p^{\text{geo}}M \to \R^n, [c] \mapsto \difffrac[t=0]{}{t}(\varphi \circ c).
  \end{align*}
\end{bem}

\begin{proof}
  Zu $v \in \R^n$ sei $B(v) = [t \mapsto \varphi^{-1}(\varphi(p) + tv)]$ - die Äquivalenzklasse der abgebildeten Kurve auf der Mannigfaltigkeit.

  % Abbildung 2-2
  \CmPutSvg{2-2-beweis-bijektivitaet-tpm-rn}{10cm}

  \begin{align*}
    & A B(v) = \difffrac[t=0]{}{t}(\varphi \circ B(v)) = \difffrac[t=0]{}{t}(\varphi \circ (\varphi^{-1}(\varphi(p) + tv)) = \difffrac[t=0]{}{t}(\varphi(p) + tv) = v.\\
    & v_c = \difffrac[t=0]{}{t}(\varphi \circ c)\\
    & B A (\underbrace{[c]}_{\ni c}) = B(v_c) = [t \mapsto \varphi^{-1}(\varphi(p + tv_c)].
  \end{align*}
  Die Kurven $c$ und $t \mapsto \varphi^{-1}(\varphi(p) + tv_c)$ sind äquivalent, also ist $B A[c] = [c]$ und somit $A$ bijektiv.
\end{proof}

Damit erhält $\T_p^{\text{geo}}M$ die Struktur eines reellen Vektorraumes vermöge der folgenden Verknüpfung:
\begin{align*}
  \lambda[c_1] + \mu[c_2] = A^{-1}(\lambda A[c_1]+ \mu A[c_2]).
\end{align*}
Dabei gilt $\lambda[c_1]+\mu[c_2] = [c]$ für $c(t) = \varphi^{-1}(\varphi(p) + t(\lambda v_1 + \mu v_2))$ mit $v_i = \difffrac[t=0]{}{t}(\varphi \circ c_i)$.

\begin{lemma}
  Die oben definierte Lineare Struktur ist unabhängig von der Wahl der Karte.
\end{lemma}

\begin{proof}
  Es sei $(\psi, V)$ eine Karte von $M$ um $p$ und $A'[c] = \difffrac[t]{}{t}(\psi \circ c)$. Dann gilt:
  \begin{align*}
    A A'^{-1}(v) & = \difffrac[t=0]{}{t}(\varphi \circ (\psi^{-1} (\psi(p) + tv)))\\
    & = \D(\varphi \circ \psi^{-1})|_{\psi(p)} \cdot \difffrac[t=0]{}{t}(\psi \circ \psi^{-1}(\varphi(p) + tv)) = \D (\varphi \circ \varphi^{-1}) \cdot v.
  \end{align*}
  Also ist $A A'^{-1}$ linear,
  \begin{align*}
    A'^{-1}(\lambda A'[c_1] + \mu A'[c_2]) & = A^{-1}(A A'^{-1}(\lambda A'[c_1] + \mu A'[c_2]))\\
    & = A^{-1} (\lambda A A'^{-1}[c_1] + \mu A A'^{-1} [c_2])\\
    & = A^{-1}(\lambda A [c_1] + \mu A [c_2]).
  \end{align*}
\end{proof}


% 3. Vorlesung <2012-10-23 Tue>

\paragraph{Motivation: Richtungsableitungen im $\R^n$}\hfill
\begin{bem}
  
  Für $f,g \in C^{\infty}(\R^n), \ x,y \in \R^n$ ist die Richtungsableitung wie folgt definiert:
  \begin{align*}
    \partial_vf(x) = \D f|x \cdot v = \difffrac[t=0]{}{t}f(x+tv).
  \end{align*}
  Diese erfüllt die Leibniz-Regel:
  \begin{align*}
    \partial_v(fg)(x) = \partial_vf(x)\cdot g(x) + f(x) \cdot \partial_v(g)(x).
  \end{align*}
\end{bem}

\begin{dfn}[Algebraische Definition des Tangentialraumes]
  Es sei $M$ eine glatte Mannigfaltigkeit und $p\in M$. Ein (algebraischer) \CmMark{Tangentialvektor} an $M$ in $p$ ist eine Lineare Abbildung $X_p \colon C^{\infty}(M) \to \R$, welche die Leibniz-Regel erfüllt:
  \begin{align*}
    X_p(fg) = X_p(f) \cdot g(p) + f(p) \cdot X_p(g).
  \end{align*}

  Die algebraischen Tangentialvektoren bilden einen reellen Vektorraum $\T_p^{\text{alg}}M$, den Tangentialraum an $M$ in $p$.
\end{dfn}

\begin{lemma}
  Es sei U eine Umgebung von $p \in M$. Dann existiert eine Umgebung $V \subset U$ von $p$ und eine glatte reellwertige Funktion $\sigma \in C^{\infty}(M)$ mit den Eigenschaften $\sigma|_V = 1$ und $\supp(\sigma) \subset U$.
\end{lemma}

% Abbildung 2-3


\begin{proof}
  Man kann o.E. annehmen, dass $U$ Kartengebiet einer Karte $\varphi$ von $M$ um $p$ ist und $\varphi(p) = 0 \in \R^n$.\\

  Es sei $\varepsilon > 0$ so, dass $\overline B_c(0) \subset \varphi(U)$. \\

  % Abbildung 2-4

  Ist dann $\eta$ eine glatte Funktion auf $\R$ mit $\eta \equiv 1$ auf $\left[\frac{-\varepsilon^{2}}{2},\frac{\varepsilon^2}{2}\right]$ und $\eta \equiv 0$ auf $\R \setminus (-\varepsilon^2,\varepsilon^2)$, so hat für $U_1 = \varphi^{-1}(B_{\frac{\varepsilon}{2}}(0)$ die Funktion
  \begin{align*}
    \sigma(q) =
    \begin{cases}
      \eta(\|\varphi(q)\|^2 & \text{ für } q \in U_1\\
      0 & \text{ sonst }
    \end{cases}.
  \end{align*}
  die gewünschten Eigenschaften.
\end{proof}

% Lemma 2.
\begin{lemma}
  Für alle $X_p\in\T_p^{\text{alg}}M$ gilt:
  \begin{enumerate}
  \item $X_p(f) = 0$ falls $f$ in einer Umgebung von $p$ konstant ist.
  \item $X_p(f) = X_p(g)$ falls $f$ und $g$ auf einer Umgebung übereinstimmen.
  \end{enumerate}
\end{lemma}

\begin{proof}
  ad (ii). Es sei $U$ eine Umgebung von $p$ mit $f|_U = g|_U$. Ist dann $\sigma$ wie im Beweis des vorigen Lemmas, so gilt $\sigma f = \sigma g$ und aus
  \begin{align*}
    X_p(\sigma)f(p)+\sigma(p)X_p(f) = X_{p}(\sigma f) = X_p(\sigma g) = X_p(\sigma) g(p) + \sigma(p) X_p(g)
  \end{align*}
  folgt $X_p(f) = X_p(g)$.\\
  ad (i). Wegen der $\R$-Linearität und (ii) genügt es $f \equiv 1$ zu betrachten. Es gilt
  \begin{align*}
    X_p(1) = X_p(1 \cdot 1) = X_p(1) \cdot 1 + 1 \cdot X_p(1) = 2 \cdot X_p(1),
  \end{align*}
  also $X_p(1) = 0$.
\end{proof}

\begin{bem}
  Also gilt für $f \in C^{\infty}(M)$ und $g \in C^{\infty}(U)$ direkt:
  \begin{align*}
    & \sigma g =
    \begin{cases}
      \sigma g|_U & \\
      0 & \text{ sonst }
    \end{cases},\\
    & \sigma g \in C^{\infty}(M) 
    \Rightarrow X_p(g) = X_p(\sigma g).
  \end{align*}
  Für eine Karte $\varphi \colon U \to V$ von $M$ und $p$ seien algebraische Tangentialvektoren definiert:
  \begin{align*}
    \pdifffrac[p]{}{x^i} \in \T_p^{\text{alg}}M, \pdifffrac[p]{}{x^i}(f) = \partial_i(f \circ \varphi^{-1})(\varphi(p)) = \D(f \circ \varphi^{-1})|_{\varphi(p)}e_i.
  \end{align*}
\end{bem}

% Satz 2.7
\begin{satz}
  Die Vektoren $\pdifffrac[p]{}{x^1},\ldots,\pdifffrac[p]{}{x^n}$ bilden eine Basis von $T_p^{\text{alg}}M$.
\end{satz}

% Lemma 2.8
\begin{lemma}
  Es sei $x_0 \in \R^n$ und $g \in C^{\infty}(B_{\rho}(x_0))$.
  Dann existieren glatte Funktionen $h_i \in C^{\infty}(B_{\rho}(x_0))$ mit $h_i(x_0) = \partial_ig(x_0)$ und 
  \begin{align*}
    g(x) = g(x_0) + \sum_{i=1}^n(x^i-x_0^i)h_i(x).
  \end{align*}
\end{lemma}

\begin{proof}
  (Beweis des Satzes).\\

  Die $j$-te Komponente $\varphi^j$ der Karte ist glatt und es gilt:
  \begin{align*}
    \pdifffrac[p]{}{x^i}(\varphi^j) = \partial_i(\varphi^j \circ \varphi^i)(\varphi(p)) = \partial_ix^j(\varphi(p)) = \delta_i^j.
  \end{align*}

  Damit sind die Vektoren linear unabhängig.\\

  Es sei $X_p\in \T_p^{\text{alg}}M$ und $f \in C^{\infty}(M)$.
  Für $x_0=\varphi(p) \in \R^n, \ B_{\rho}(x_0) \subset \varphi(U)$ und für $g = f \circ \varphi^{-1}|_{B_{\rho}(x_0)}$ gilt mit den Bezeichnungen wie im letzten Lemma:
  \begin{align*}
    X_p(f) & = X_p(g \circ \varphi) = X_p(g(\varphi(p)) + \sum \left(\varphi^i - \varphi(p)^i)(h_i \circ \varphi) \right)\\
    & = \underbrace{X_p(g(\varphi(p)))}_{\mathclap{=0}} + \sum X_p((\varphi^i-\varphi(p)^i)(h_i \circ \varphi))\\
    & = \sum X_p(\varphi^i)(h_i\circ\varphi)(p) - X_p(\varphi(p)^i)(h_i\circ \varphi)(p) + \sum (q^i-\varphi(p)^i)(p) X_p(h_i \circ \varphi)\\
    & = \sum_{i=1}^n X_p(\varphi^i)\underbrace{(h_i \circ \varphi)(p)}_{\mathclap{=h_i(\varphi(p) = h_i(x_0) = \partial_ig(x_0) = \partial_i(f\circ \varphi^{-1})(\varphi(p)) = \pdifffrac[p]{}{x^i}(f)}}\\
    & = \sum_{i=1}^nX_p(\varphi^i)\pdifffrac[p]{}{x^i}(f).
  \end{align*}
\end{proof}

\begin{bem}
  Ist $X_p=\sum \xi^i\pdifffrac[p]{}{x^i}$, so gilt $\xi^i = X_p(\varphi^i)$.
\end{bem}

\begin{proof}
  (Beweis des Lemmas). Es gilt:
  \begin{align*}
    g(x) - g(x_0) = \int_0^1\difffrac{}{t}g(tx + (1-t)x_0)dt = \sum_{i=1}^n(x^i-x_0^i)\underbrace{\int_0^1\partial_ig(tx + (1-t)x_0) dt}_{=: h_i(x)}.
  \end{align*}
\end{proof}

% Satz 2.9
\begin{satz}[Äquivalenz der Tangentialraumbegriffe]
  Die Abbildung
  \begin{align*}
    J_p \colon \T_p^{\text{geo}}M \to \T_p^{\text{alg}}M, \ J_{p}[c](f) = \difffrac[t=0]{}{t}(f\circ c)
  \end{align*}
  ist ein linearer Isomorphismus.
\end{satz}

\begin{proof}
  Wegen
  \begin{align*}
    J_p[c](f)& = \difffrac[t=0]{}{t}(f\circ c) = \difffrac[t=0]{}{t}(f \circ \varphi^{-1} \circ \varphi \circ c)\\
    &  = \D(f \circ \varphi^{-1})|_{\varphi(p)} \difffrac[t=0]{}{t} (\varphi \circ c) = \D (f\circ \varphi^{-1})|_{\varphi(p)}A[c]
  \end{align*}
  ist $J_p = \D(\cdot)\circ A$ linear.\\

  Ist $[c] \in \Kern J_p$, so folgt aus $0 = J_p[c](\varphi^i) = \difffrac[t=0]{}{t}(\varphi^i \circ c)$, dass $\difffrac[t=0]{}{t}(\varphi \circ c) = 0$ gilt, also $[c] = 0$.\\

  Damit ist $J_p$ injektiv, also ein Isomorphismus.
\end{proof}

\begin{bem}
  \begin{enumerate}
  \item Ist $X_p = \sum \xi^i\pdifffrac[p]{}{x^i}$,so gilt $X_p = c(0)$ für $c(t) = \varphi^{-1}(\varphi(p) + t\xi)$.
\item Für jede glatte Kurve $c$ durch $p$ ist $\difffrac[t=0]{}{t}(\varphi \circ c)$ der Koeffizientenvektor von $\dot c(0)$ in der Basis $\pdifffrac[p]{}{x^i}$.
  \end{enumerate}
\end{bem}


% Satz 2.10
\begin{satz}[Transformationsverhalten bei Kartenwechsel]
  Es seien $\varphi$ und $\psi$ Karten in $M$ um $p$ und es bezeichnen $\pdifffrac[p]{}{x^i}$ und $\pdifffrac[p]{}{y^i}$ die damit assoziierten Basen von $\T_pM$. Dann gilt
  \begin{align*}
    \pdifffrac[p]{}{x^i} = \sum_j \partial_i(\psi^j \circ \varphi^{-1})(\varphi(p)) \pdifffrac[p]{}{y^j}.
  \end{align*}
Es sei $X_p = \sum \xi^i \pdifffrac[p]{}{x^i} = \sum \eta^i\pdifffrac[p]{}{y^i}$. Dann gilt:
\begin{align*}
  \eta^j = \sum \partial_i(\psi^j \circ \varphi^{-1})(\varphi(p))\xi^i \text{ bzw. }
  \eta = \D(\psi \circ \varphi^{-1})(\varphi(p))\xi.
\end{align*}
\end{satz}

\begin{proof}
  Es gelte $\pdifffrac[p]{}{x^i} = \sum \alpha_i^j\pdifffrac[p]{}{y^j}$ und nach obiger Bemerkung zum vorletzten Satz gilt:
  \begin{align*}
    \alpha_i^j = \pdifffrac[p]{}{x^i}(\psi^j) = \partial_i(\psi^j \circ \varphi^{-1})(\varphi(p))
  \end{align*}
\end{proof}

%%% Local Variables: 
%%% mode: latex
%%% TeX-master: "../skript-diffgeom"
%%% End: 

 %% 
%% 4. Vorlesung <2012-10-26 Fri>
%% 
%% Skript Differentialgeometrie im Wintersemester 12/13
%% Zur Vorlesung von Dr. Grensing am KIT Karlsruhe
%% 
%% Mitschrieb und Textsatz von Jan-Bernhard Kordaß.
%% 

\section{Differentiale}

% Abb 4/1

Es seien $M$ und $N$ Mannigfaltigkeiten und $\Phi \colon M \to N$ eine glatte Abbildung.
Sind $p \in M$ und $X_p \in \T_pM$ , so ist 
\begin{align*}
  \Phi_{*p}X_p \colon C^{\infty}(N) \to \R, f \mapsto X_p(\underbrace{f \circ \Phi}_{\in C^{\infty}(N)}).
\end{align*}
ein Tangentialvektor an $N$ in $\Phi(p)$:
\begin{align*}
  \Phi_{*p}X_p(fg) & = X_p((f \circ \Phi)(g \circ \Phi)) = X_p(f \circ \Phi)(g \circ \Phi)(p) + (f \circ \Phi)(p)X_p(g \circ \Phi)\\
  & = \Phi_{*p}X_p(f)q(\Phi(p)) + f(\Phi(p)) \Phi_{*p}X_p(g).
\end{align*}
\begin{center}\begin{tikzpicture}[font=\scriptsize]
	%\draw[step=0.25,gray!15] (-6,-1) grid (6,5); \draw[step=0.5,gray!30] (-6,-1) grid (6,5); \fill (0,0) circle(0.1); %Hilfsgitter
	
	% Die Abbildungspfeile
	\draw[->] (-1.5,0) to[out=20, in=160]node[above]{$\psi' \circ \Phi \circ \varphi^{-1}$}node[below]{diff'bar} (1.5,0);
	\draw[->] (-1.5,3) to[out=20, in=160]node[above]{$\Phi$} (1.5,3);
	
	% Die Achsen
	\draw[->,thick] (-5.5,-0.5) -- (-2,-0.5); \draw[->,thick] (-5.25,-0.75) -- (-5.25, 1.25); \node[font=\normalfont] at (-2,1.25) {$\R^m$};
	\draw[->,thick] (2,-0.5) -- (5.5,-0.5); \draw[->,thick] (2.25,-0.75) -- (2.25, 1.25); \node[font=\normalfont] at (5.5,1.25) {$\R^n$};
	
	% das linke Ding
	\coordinate (ding0) at (-3.5,4.5); \coordinate (ding1) at (-4.25,3.5); \coordinate (ding2) at (-4.5,2); \coordinate (ding3) at (-2.75,2.5); \coordinate (ding4) at (-1.5,4);
	\coordinate (ctrld0) at (0.5,-0.25); \coordinate (ctrld1) at (0.75,0.5); \coordinate (ctrld2) at (-0.5,0.5); \coordinate (ctrld3) at (0.25,0.5); \coordinate (ctrld4) at (-0.75,0); 
	\draw[thick] (ding0) ..controls($(ding0)+(ctrld0)$) and ($(ding1)+(0.75,0.5)$).. (ding1) ..controls($(ding1)-(ctrld1)$) and($(ding2)+(ctrld2)$).. (ding2) ..controls($(ding2)-(ctrld2)$) and ($(ding3)-2*(ctrld3)$).. (ding3) ..controls($(ding3)+(ctrld3)$) and ($(ding4)+(ctrld4)$).. (ding4);
	% das Loch in der Mitte nicht vergessen, es besteht aus zwei geclipten Kreisen
	\begin{scope}
		\clip (-4.25,2.5) rectangle (-2.75,3);
		%\draw[thick] (-4.25,3) to[out=330,in=180] (-3.5,2.75) to[out=0,in=210] (-2.75,3);
		\path[draw,thick,name path=gkreis] (-3.5,4.25) circle (1.5);
	\end{scope}
	\path[name path=kkreis] (-3.5,2) circle(1);
	\path[name intersections={of=gkreis and kkreis}];
	\begin{scope}
		\clip (intersection-1) rectangle ($(intersection-2)+(0,0.5)$);
		\draw[thick]  (-3.5,2) circle(1);
	\end{scope}
	
	% das rechte Ding
	\draw[thick] (4, 3)  ellipse (2 and 1);
	% und das Loch
	\begin{scope}
		\clip (3, 2.75) rectangle (5, 4);
		\path[draw,thick,name path=gkreis] (4,4) ellipse (1.5 and 1);
	\end{scope}
	\path[name path=kkreis] (4,2.5) ellipse (1 and 0.75);
	\path[name intersections={of=gkreis and kkreis}];
	\begin{scope}
		\clip (intersection-1) rectangle ($(intersection-2)+(0,0.5)$);
		\draw[thick] (4,2.5) ellipse (1 and 0.75);
	\end{scope}
	
	% die linke Kartoffel
	\coordinate (kartoffel0) at (-4.25,2.5); \coordinate (kartoffel1) at (-4.25,2); \coordinate (kartoffel2) at (-4,2.125); \coordinate (kartoffel3) at (-3.75,2); \coordinate (kartoffel4) at (-3.25,2.5); \coordinate (kartoffel5) at (-3.75,2.5); 
	\coordinate (ctrlk0) at (0.25,0.25); \coordinate (ctrlk1235) at (0.25,0); \coordinate (ctrlk4) at (0,0.25);
	\draw (kartoffel0) ..controls($(kartoffel0)-(ctrlk0)$) and ($(kartoffel1)-0.5*(ctrlk1235)$).. (kartoffel1) ..controls($(kartoffel1)+0.5*(ctrlk1235)$) and ($(kartoffel2)-(ctrlk1235)$).. (kartoffel2) ..controls($(kartoffel2)+(ctrlk1235)$) and ($(kartoffel3)-0.5*(ctrlk1235)$).. (kartoffel3) ..controls($(kartoffel3)+2*(ctrlk1235)$) and ($(kartoffel4)-(ctrlk4)$).. (kartoffel4) ..controls($(kartoffel4)+(ctrlk4)$) and ($(kartoffel5)+(ctrlk1235)$).. (kartoffel5) ..controls($(kartoffel5)-0.5*(ctrlk1235)$) and ($(kartoffel0)+(ctrlk0)$).. (kartoffel0) -- cycle;
	\fill (-3.75,2.25) circle (0.05) node[right]{$p$};
	
	% die rechte Kartoffel (Kreis)
	\draw (4,2.5) ellipse (0.5 and 0.25); \fill (4,2.5) circle (0.05) node[right]{$q$};
	
	% die beiden Umgebungen unten
	\draw (-3.75,0.25) circle (0.5); \fill (-3.75,0.25) circle (0.05) node[right]{$x$};
	\draw (4,0.25) circle (0.5); \fill (4,0.25) circle (0.05) node[right]{$y$};
	
	% Abbildungspfeile
	\draw[->] (-4, 2.25) to[out=250,in=110] node[left]{$\varphi$} (-4,0.75);
	\draw[->] (3.75, 2.5) to[out=250,in=110] node[left]{$\psi$} (3.75,0.75);
\end{tikzpicture}\end{center}

% Definition 3.1
\begin{dfn}
  Die lineare Abbildung $\Phi_{*p} \colon \T_pM \to \T_{\Phi(p)}N$ heißt das \CmMark{Differential} von $\Phi$ in $p$. Der Rang von $\Phi_{*p}$ bezeichnet man als den Rang von $\Phi$ in $p$.
\end{dfn}

% Lemma 3.2
\begin{lemma}[Differentiale in lokalen Koordinaten]
  Sind $\varphi$ und $\psi$ Karten von $M$ und $N$ um $p$ und $\Phi(p) = q$, sowie $\pdifffrac{x^i}|_p$ und $\pdifffrac{y^i}|_q$ die Standardbasen von $\T_pM$ und $\T_qN$ bezüglich der Karten $\varphi$ und $\psi$, so gilt:
  \begin{align*}
    \Phi_{*p}\pdifffrac{x^i}|_p = \sum \partial_i(\psi^j \circ \Phi \circ \varphi^{-1})(\varphi(p))\pdifffrac{y^j}|_q.
  \end{align*}
  Die partielle Ableitung $\partial_i(\psi^j \circ \Phi \circ \varphi^{-1})(\varphi(p))$ bezeichnet man auch kurz $\frac{\partial \Phi^j}{\partial x^i}(p)$.
\end{lemma}

\begin{bem}
  Aus der Linearität von $\Phi_{*p}$ folgt, dass für $X_p = \sum \xi^i\pdifffrac{x^i}|_p \in \T_pM$ und $\Phi_{*p}X_p = \sum \eta^j\pdifffrac{y^j}|_q$ gilt:
  \begin{align*}
    \eta^j = \sum \frac{\partial \Phi^j}{\partial x^i}\xi^i \text{, beziehungsweise } \eta = \D(\psi \circ \Phi \circ \varphi^{-1})\xi.
  \end{align*}
\end{bem}

\begin{proof}
  \begin{align*}
    \underbrace{\left(\Phi_{*p}\pdifffrac{x^i}|_p\right)}_{\textcolor{red}{\in \T_qN}}(\psi^j) = \pdifffrac{x^i}|_p(\psi^j \circ \Phi) = \partial_i (\psi^j \circ \Phi \circ \varphi^{-1})(\varphi(p)) = \frac{\partial \Phi^j}{\partial y^i}(p).
  \end{align*}
\end{proof}

\begin{bem}[Charakterisierung durch Kurven]
  Ist $[c] \in \T_p\textcolor{red}{M}$, so gilt für $f \in C^{\infty}(M)$:
  \begin{align*}
    \Phi_{*p}[c](f) = [c](f \circ \Phi) = \difffrac{t}|_{t=0}(\underbrace{f \circ \Phi \circ c)}_{\substack{\text{glatte Kurve}\\ \text{auf }N}} = [\Phi \circ c](f)
  \end{align*}
  also $\Phi_{*p}[c] = [\Phi \circ c]$.
\end{bem}

\begin{bem}[Tangentialräume an Untermannigfaltigkeiten der $\R^n$]
  Ist $U$ eine Untermannigfaltigkeit in $\R^n$ mit den Eigenschaften
  \begin{itemize}
  \item $F \colon U \to M \cap F(U)$ ein Homöomorphismus,
  \item $\D F|_x\colon \R^m \to \R^{m+k}$ injektiv für alle $x \in U$.
  \end{itemize}
  Dann ist $\psi = F^{-1}$ eine Karte von $M$. Es bezeichnen $\pdifffrac{y^i}|_p$ die Standardbasis bezüglich $\psi$ und $\pdifffrac{x^i}|_x$ die Standardbasis bezüglich der kanonischen Karte $\Id_{\R^m}$ des $\R^m$.\\
  Dann gilt für $g \in C^{\infty}(M)$ beliebig:
  \begin{align*}
    & \pdifffrac{y^i}|_p(g) = \partial_i(g \circ \psi^{-1})(\underbrace{\psi(p)}_{=x}) = \partial_i(g \circ F)(x) = F_{*x}\left(\pdifffrac{x_i}|_p\right)(f).\\
    & F_{*x}\left(\pdifffrac{x^i}|_p\right) = F_{*x}[t \mapsto x + te_i] = [t \mapsto F(x+te_i)] \sim \difffrac{t}|_{t=0}F(x+te_i) = \D F|_x(e_i) = \partial_iF|_x.\\
    & \T_pM "=" \left<\partial_1F|_x, \ldots, \partial_m F|_x\right).
\end{align*}
\end{bem}

% Abb 4/2

\begin{bem*}[Eigenschaften des Differentials]\hfill
  \begin{itemize}
  \item (Kettenregel) Sind $\Phi \colon M \to N$ und $\Psi \colon N \to P$ glatt, so gilt:
    \begin{align*}
      (\Psi \circ \Phi)_{*p} = \Psi_{*\Phi(p)} \circ \Phi_{*p}.
    \end{align*}
  \item Ist $\Psi \colon M \to N$ ein Diffeomorphismus, so ist $\Phi_{*p}$ ein Vektorraumisomorphismus. % Verwendet die Kettenregel
  \item (Satz von der Umkehrabbildung) Ist $\Phi \colon M \to N$ glatt und $\Phi_{*p}$ bijektiv, so existieren Umgebungen $U$ von $p$ und $V$ von $\Phi(p)$, so dass $\Phi|_{U} \colon U \to V$ ein Diffeomorphismus ist.
  \end{itemize}
\end{bem*}

% Definition 2.3
\begin{dfn}[Reguläre Punkte, Submersion, Immersion]
  Es sei $\Phi \colon M \to N$ glatt.
  \begin{itemize}
  \item Es Punkt $p \in M$ heißt \CmMark{regulärer Punkt} von $\Phi$, wenn $\Phi_{*p}$ surjektiv ist. Ein Punkt $q \in N$ heißt regulärer Wert, wenn jeder Punkt $p \in \Phi^{-1}(q)$ regulär ist.
  \item Die Abbildung $\Phi$ heißt \CmMark{Submersion}, wenn $\Phi$ surjektiv ist und alle $p \in M$ reguläre Punkte sind.
  \item Die Abbildung $\Phi$ heißt \CmMark{Immersion}, wenn für alle $p \in M$ $\Phi_{*p}$ injektiv ist.
  \item Die Abbildung $\Phi$ heißt \CmMark{Einbettung}, wenn $\Phi$ Immersion und Homöomorphismus auf sein Bild ist.
  \end{itemize}
\end{dfn}

\begin{bsp}
  \begin{itemize}
  \item Betrachte eine Abbildung $\Phi$

    % Abb 4/3

    Immersion: $\difffrac{t}$ Basis von $\T_x\R$, $\Phi_{*x}(\difffrac{t}) "=" \difffrac{t}\Phi$
  \item $\R \to \R^2 \cong \C, t \mapsto e^{it}$ ist eine Immersion aber ebenfalls nicht injektiv.
  \item $\R \to S^1 \subset \C, t \mapsto e^{it}$ ist Immersion und Submersion.
  \item $\R \to S^1 \times \R, t \mapsto (e^{it},t)$ ist eine Einbettung.

    % Abb 4/4

  \item Ist $M \subset N$ Untermannigfaltigkeit, so ist $\imath \colon M \to N$ eine Einbettung.% INKLUSIONSSPFEIL
  \end{itemize}
\end{bsp}

% Satz 3.4
\begin{satz}
  Es seien $M$ und $N$ glatte Mannigfaltigkeiten, $\Phi \colon M \to N$ eine glatte Abbildung und $p \in M$, sowie $q = \Phi(p)$. Es bezeichnen $m$ und $n$ die Dimensionen von $M$ und $N$ und $r$ den Rang von $\Phi$ in $p$. Dann gelten folgende Aussagen:
  \begin{itemize}
  \item Zu jeder Karte $\psi$ von $N$ um $q$ mit $\psi(p) = 0$ existiert eine Karte $\alpha$ von $M$ um $p$ mit $\alpha(p) = 0$ und glatte Funktionen $f^{r+1},\ldots,f^n$ mit
    \begin{align*}
      \left(\psi \circ \Phi \circ \alpha^{-1}\right)(x^1, \ldots, x^{r}, f^{r+1}(x), \ldots, f^n(x)).
    \end{align*}
  \item Falls der Rang von $\Phi$ auf einer Umgebung von $p$ konstant $r$ ist, so existieren Karten $\alpha$ um $p$ mit $\alpha(p) = 0$ und $\beta$ um $q$ mit $\beta(q) = 0$, so dass
    \begin{align*}
      (\beta \circ \Phi \circ \alpha^{-1})(x^1, \ldots, x^m) = (x^1, \ldots, x^r, 0, \ldots, 0).
    \end{align*}
  \end{itemize}
\end{satz} 

% Korollar 3.5
\begin{kor}
  \begin{enumerate}[label=(\roman*)]
  \item Falls $\Phi$ auf einer offenen Umgebung von $P = \Phi^{-1}(q)$ konstanten Rang $r$ hat, so ist $P$ eine Untermannigfaltigkeit der Kodimension $r$.
  \item Ist $q$ ein regulärer Wert von $\Phi$, so ist $P = \Phi^{-1}(q)$ eine Untermannigfaltigkeit von $M$ der Kodimension $n$.\\
    Beispiel: $\|\cdot\|^{-1}(1) = S^n \supset \R^{n+1} \to \R, x \mapsto \|n\|$.
  \item Ist $\Phi_{*p}$ injektiv, so existiert eine Umgebung $U$ von $p$, so dass $\Phi(U) = Q \subset N$ eine Untermannigfaltigkeit von $N$ ist.
  \item Ist $\Phi$ eine Einbettung, so ist $Q = \Phi(M)$ eine $m$-dimensionale Untermannigfaltigkeit von $M$ und $\Phi \colon M \to Q$ ist ein Diffeomorphismus.
  \end{enumerate}
\end{kor}

\begin{proof}
  ad (i): Ist $p \in P = \Phi^{-1}(q)$. Nach der zweiten Aussage des vorrangegangenen Satzes existieren Karten $(\alpha,U), (\beta, V)$ mit 
  \begin{align*}
    (\beta \circ \Phi \circ \alpha^{-1})(x^1,\ldots,x^n) = (x^1, \ldots,x^r, 0, \ldots, 0)
  \end{align*}
  und es gilt:
  \begin{align*}
    \alpha(P \cap U) & = (\alpha \circ \Phi^{-1} \circ \beta^{-1})(0) \\
    & = \{x \in \alpha (U) \mid x^1 = \ldots = x^r = 0\} = \alpha(U) \cap \{0\} \times \R^{m-r}.
  \end{align*} 
  ad (ii): Ist $q$ ein regulärer Wert von $\Phi$, so existieren nach dem ersten Teil des vorigen Satzes Karten $\psi,\alpha$ mit 
  \begin{align*}
    (\psi \circ \Phi \circ \alpha^{-1})(x^1, \ldots, x^m) = (x^1, \ldots, x^n) \ m \geq n = r \quad \forall x \in \alpha(U).
  \end{align*}
  Es gilt also für alle $u \in U$:
  \begin{align*}
    \Rang \Phi_{*u} = \Rang \D(\psi \circ \Phi \circ \alpha^{-1})|_x = \Rang
    \begin{pmatrix}
      1 &        & 0 &  &   & \\
      & \ddots &   &  & 0 & \\
      0 &        & 1 &  &   &
    \end{pmatrix}
    = 0
  \end{align*}
  Damit folgt die Behauptung aus (i).\\
  ad (iii): $\Phi_{*p}$ ist injektiv $\Rightarrow r = m \leq n$. Nach Wahl von Karten wie in (ii):
  \begin{align*}
    & (\psi \circ \Phi \circ \alpha^{-1})(x^1, \ldots, x^m) = (x^1, \ldots, x^m,f^{m+1}(x), \ldots, f^n(x))\\
    & \Rang \Phi_{*u} = \Rang 
    \begin{pmatrix}
      1 &        & 0\\
        & \ddots &  \\
      0 &        & 1\\
        &        &  \\
        & 0      &  \\
        &        &  
    \end{pmatrix}
    = m % Matrix 4/5
  \end{align*}
  Nach der ersten Aussage des letzten Satzes erhalten wir spezielle Karten:
  \begin{align*}
    (\beta \circ \Phi \circ \alpha^{-1})(x^1, \ldots, x^m) = (x^1, \ldots, x^m, 0, \ldots, 0) \in \R^{m} \times \{0\},
  \end{align*}
  wobei $\beta$ eine adaptierte Karte für $\Phi(U) = Q$ ist.\\
  (iv) folgt aus (iii).
\end{proof}

%%% Local Variables: 
%%% mode: latex
%%% TeX-master: "../skript-diffgeom"
%%% End: 

%% 
%% 5. Vorlesung <2012-10-30 Tue>, Fortsetzung
%% 
%% Skript Differentialgeometrie im Wintersemester 12/13
%% Zur Vorlesung von Dr. Grensing am KIT Karlsruhe
%% 
%% Mitschrieb und Textsatz von Jan-Bernhard Kordaß.
%% 

\section{Tangentialbündel und Vektorfelder}

\begin{dfn}[Tangentialbündel]
  Es sei $M$ eine glatte Mannigfaltigkeit. Die Menge $\T M = \dot \bigcup_{p \in M}\T_pM$ zusammen mit der sogenannten kanonischen Projektion $\pi \colon \T M \to M, \T_pM \ni X_p \mapsto p$ heißt das \CmMark{Tangentialbündel} von $M$.
\end{dfn}

\subsection{Das Tangentialbündel als glatte Mannigfaltigkeit}

Es sei $(\varphi, U)$ eine Karte von $M$. Setzt man $\T M|_U = \pi^{-1}(U) = \dot \bigcup_{p\in U}\T_pM$, so ist nach \textcolor{red}{Satz 2.8} % REFERENZ!
die Abbildung
\begin{align*}
  & \overline \varphi \colon \T M|_U \to \varphi(U) \times \R^m \subset \R^{2m}\\
  &\T_p M \ni X_p = \sum \xi^i\pdifffrac[p]{}{x^i} \mapsto (\varphi(p),\xi)
\end{align*}
bijektiv.
Es sei eine Topologie auf $\T M$ dadurch erklärt, dass eine Menge $V \subset \T M$ genau dann offen ist, wenn für alle Karten $(\varphi, U)$ die Menge $\overline \varphi(V \cap \T M|_U)$ offen in $\R^{2m}$ ist. Diese Topologie ist hausdorffsch und besitzt eine abzählbare Basis, da dies für $M$ und $\R^m$ gilt. Nach Konstruktion sind alle $\overline \varphi$ Homöomorphismen. Ist $\mathcal A = \{(\varphi, U\}$ ein $C^{\infty}$-Atlas von $M$, so definiert
\begin{align*}
  \overline A = \{(\overline \varphi, \T M|_U) \mid (\varphi, U) \in \mathcal A\}
\end{align*}
eine glatte Struktur auf $\T M$. Für Karten $(\varphi, U), (\psi, V)$ von $M$ ist der Kartenwechsel
\begin{align*}
  \overline \psi \circ \overline \varphi^{-1} \colon \varphi(U \cap V) \times \R^m &\to \psi(U \cap V) \times \R^m \\ \ (x, \xi) &\mapsto (\psi \circ \varphi^{-1}(x),\D(\psi \circ \varphi^{-1}|_x\xi),
\end{align*}
$\left(X_p = \sum \xi^i \pdifffrac[p]{}{x^i}\right)$ ist glatt. Damit trägt $\T M$ in kanonischer Weise eine glatte Struktur.
Darüber hinaus ist die kanonische Projektion $\pi \colon \T M \to M$ bezüglich dieser glatten Struktur eine Submersion. (Beweis: Übungsaufgabe)

Ist $N$ eine weitere glatte Mannigfaltigkeit und $\Phi \colon M \to N$ glatt, so ist $\Phi_{*} \colon \T M \to \T N, \ X_p \mapsto \Phi_{*p}X_p$ eine glatte Abbildung (ebensfalls Übungsaufgabe).

\begin{dfn}
  Eine stetige Abbildung $X \colon M \to \T M$ mit $\pi \circ X = \Id_M$ heißt \CmMark{Vektorfeld} auf $M$.
  Ist $X$ glatt (als Abbildung zwischen glatten Mannigfaltigkeiten), so heißt $X$ ein \CmMark{glattes Vektorfeld.}
\end{dfn}

\begin{bem}
  Ist $(\varphi, U)$ eine Karte von $M$, so sind die Abbildungen $U \to \T M|_U, \ p \mapsto \pdifffrac[p]{}{x^{i}}$ glatte Vektorfelder (in der Karte $\overline \varphi$ sind diese genau die Abbildungen $(x,e_i)$).
  
  Ist $X$ ein glattes Vektorfeld, so gilt für jedes $u \in U$:
  \begin{align*}
    X_U = \sum \xi^i(u) \pdifffrac[U]{}{x^i},
  \end{align*}
  wobei $\xi(u) = (\xi^1(u), \ldots, \xi^m(u))$ eine glatte Abbildung $U \to \R^m$ ist. Ein Vektorfeld ist genau dann glatte, wenn für jede Karte $(\varphi, U)$ die Koeffizientenfunktionen $\xi^i(u)$ von $X_U = \sum \xi^i(u) \pdifffrac[U]{}{x^i}$ glatte Funktionen sind.
\end{bem}


%%% 
%%% 6. Vorlesung <2012-11-2 Fri>
%%% 

\begin{bsp*}
  Betrachte die $n$-Sphäre $S^n \subset \R^{n+1}$ und deren Tangentialraum $\T_pS^n = p^{\perp}$.
  Ein glattes Vektorfeld auf $S^n$ ist also eine glatte Abbildung $X \colon S^n \to \R^{n+1}$ mit $X_p \perp p$.
  
  Es sei $n=2k-1$, dann ist
  \begin{align*}
    X \colon S^n \to \R^{n+1}, (x^1,y^1, \ldots, x^k,y^k) \mapsto (-y^1,x^1, \ldots, -y^k,x^k)
  \end{align*}
  ein glattes Vektorfeld auf $S^n$ ohne eine Nullstelle.
\end{bsp*}

\begin{bem*}
  Der \CmMark{Satz vom Igel} besagt gerade: Jedes glatte Vektorfeld auf einer Sphäre gerader Dimension hat eine Nullstelle.
\end{bem*}

\begin{bem*}
  Es bezeichne $\mathcal V(M)$ die Menge aller glatten Vektorfelder auf der Mannigfaltigkeit $M$. Der sogenannte \CmMark{Nullschnitt}:
  \begin{align*}
    \sigma \colon M \to \T M, p \mapsto 0_p \in \T_pM
  \end{align*}
  ist ein glattes Vektorfeld auf $M$.
  
  \emph{"Ubungsaufgabe:} Zeige dass der Nullschnitt eine Einbettung ist.
\end{bem*}

\begin{bem*}
  Sind $X,Y \in \mathcal V(M)$ und ist $g \in \C^{\infty}(M)$, so sind die punktweise Summe $X+Y$ und das Produkt $gX$ wieder ein glatte Vektorfelder auf $M$. Damit ist $\mathcal V(M)$ ein $\R$-Vektorraum beziehungsweise $C^{\infty}(M)$-Modul.
  
  Jedes Vektorfeld $X$ ist eine Derivation von $C^{\infty}(M)$:
  \begin{align*}
    X(fg)(p) = X(f)(p)g(p) + f(p) X(g)(p) = \left(gX(f) + fX(g)\right)(p).
  \end{align*}
  Es seien $X,Y \in \mathcal V(M)$ glatte Vektorfelder. Die \CmMark{Lieklammer} $[X,Y]$ von $X$ und $Y$ ist dann durch den folgenden Ausdruck definiert:
  \begin{align*}
    [X,Y](f)(p) = X_p(Yf)-Y_p(Xf).
  \end{align*}
\end{bem*}

% Lemma 4.3
\begin{lemma}
  Die Lieklammer ist eine schiefsymmetrische $\R$-bilineare Abbildung $\mathcal V(M) \times \mathcal V(M) \to \mathcal V(M)$. Es gilt die sogenannte \CmMark{Jacobiidentität}:
  \begin{align*}
    [X,[Y,Z]] + [Y,[Z,X]] + [Z,[X,Y]] = 0.
  \end{align*}
\end{lemma}

\begin{proof}
  Es seien $X,Y \in \mathcal V(M)$, $f,g \in \C^{\infty}(M)$ und $p \in M$. Dann gilt:
  \begin{align*}
    [X,Y](fg)  = & X_p(Y(f)g + fY(g)) - Y_p(X(f)g+fX(g))\\
    = & X_p(Y(f))g(p) + \cancel{Y_p(f)X_p(g)} + \bcancel{X_p(f)Y(g)(p)} + f(p)X_p(Y(g))\\
    & - Y_p(X(f))g(p) - \bcancel{X_p(f)Y_p(g)} - \cancel{Y_p(f)X(g)(p)} - f(p)Y_p(X(g))\\
    = & (X_p(Y(f))-Y_p(X(f))g(p) + f(p)(X_p(Y(g))-Y_p(X(g))\\
    = & [X,Y]_p (f)g(p) + f(p)[X,Y]_p(g).
  \end{align*}
  \textcolor{red}{(nicht sicher ob ich die Durchstreichungen lassen oder rausnehmen soll)}
  Damit gilt $[X,Y] \in \mathcal V(M)$. Schiefsymmetrie und $\R$-Linearität gelten offensichtlich. Die Jacobiidentität sie als Übungsaufgabe überlassen (Nachrechnen!).
\end{proof}


% Lemma 4.4
\begin{lemma}
  Es seien $X,Y \in \mathcal V(M)$ glatte Vektorfelder und $(\varphi,U)$ eine Karte von $M$.
  Sind dann $X|_U = \sum \xi^i\pdifffrac{}{x^i}, \ Y|_U = \sum \eta^i \pdifffrac{}{x^i}$ und $[X,Y]|_U = \sum \zeta^i \pdifffrac{}{x^i}$ die ensprechenden lokalen Darstellungen, so gilt:
  \begin{align*}
    \zeta^j = \sum \left(\xi^i\pdifffrac{\eta^j}{x^i} - \eta^i \pdifffrac{\xi^j}{x^i} \right).
  \end{align*}
\end{lemma} 

Der Beweis ist als Übungs überlassen.


\subsection{Flüsse}

\marginnote{\textcolor{red}{[BILD]}}Was haben Vektorfelder mit Differentialgleichungen zu tun? Jedes glatte Vektorfeld $X$ definiert ein Anfangswertproblem
\begin{align*}
  \begin{cases}
    \dot \gamma(t) = X_{\gamma(t)}\\
    \gamma(0) = p
  \end{cases},
\end{align*}
oder in lokalen Koordinaten:
\begin{align*}
\begin{cases}
  \dot \gamma(t) = \xi(\tilde \gamma(t))\\
  \tilde \gamma(0) = 0, \text{ falls } \varphi(p) = 0,
\end{cases}
\end{align*}
mit $\tilde \gamma = \varphi \circ \gamma$ für eine Karte $(\varphi,U)$ um $p$ und $X|_{\textcolor{red}{U}} = \sum \xi^i \pdifffrac{}{x^i}$.

% Definition 4.5
\begin{dfn}
  Es sei $X \in \mathcal V(M)$ ein glattes Vektorfeld und $p \in M$, sowie $\calI \subset \R$ ein offenes, zusammenhängendes Intervall um $0$. Eine glatte Kurve $\gamma \colon \calI \to M$ mit
  \begin{align*}
    \dot \gamma(t) = X_{\gamma(t)}, \ \gamma(0) = p
  \end{align*}
heißt \CmMark{Integralkurve} oder \CmMark{Trajektorie} von $X$ durch $p$.
\end{dfn}

\begin{bem*}
  Eine Kurve $\gamma$ ist genau dann Integralkurve von $X$ durch $p$, wenn für jede Karte $(\varphi,U)$ die Kurve $\tilde \gamma = \varphi \circ \gamma$ eine Lösung des (autonomen) Anfangswertproblems
  \begin{align*}
    \dot{\tilde \gamma} = \xi(\gamma(t)),  \ \tilde \gamma(0) = \varphi(p)
  \end{align*}
  ist, wobei $X_U = \sum \xi^i \pdifffrac{}{x^i}$ gelte.

  Für jedes $p \in M$ ist somit (lokal) ein Anfangswertproblem gestellt. Gesucht ist eine \quot{simultane} Lösung all dieser Anfangswertprobleme, also eine Abbildung $(t,p) \mapsto \gamma(t,p) = \gamma^t(p)$ mit 
  \begin{align*}
    \begin{cases}
      \dot \gamma^t(p) = X_{\gamma^t(p)}\\
      \gamma^0(p) = p
    \end{cases}.
  \end{align*}
\end{bem*}

% Satz 4.6
\begin{satz}[Lokale Existenz und Eindeutigkeit]
  Es sei $U \subseteq \R^n$ offen und $\calI_{\varepsilon}=(-\varepsilon,\varepsilon)$ und $F \colon \calI_{\varepsilon} \times \R^n \to \R^n$ $C^k$-differenzierbar.
  Dann existiert für alle $x \in U$ eine Umgebung $V$ von $x$ in $U$ und ein $\delta > 0$, so dass
  \begin{enumerate}[label=(\roman*),leftmargin=*,widest=iii]
  \item Für alle $x \in V$ existiert eine $C^{k+1}$-Lösung $y_x\colon \calI_{\delta} \to V$, von $y_x'(t)=F(t,y(t))$ und $y_x(0) = x$.
  \item Diese Lösung ist lokal eindeutig, das hei\ss t falls $\tilde y_x$ eine weitere Lösung auf $\calI_{\tilde\delta}$ ist, so gilt
    \begin{align*}
      y_x(t) = \tilde y_x(t) \text{ für alle } |t| \leq \min\{\delta, \tilde\delta\}.
    \end{align*}
  \item Die Abbildung 
    \begin{align*}
      y\colon \calI_{\delta} \times V \to U, \ (t,x) \mapsto y_x(t)
    \end{align*}
    ist $C^k$-differenzierbar.
  \end{enumerate}
\end{satz}

\begin{proof}
  Siehe Lang: "`Differential and Riemannian Manifolds"', 3. Auflage, 1995, Chapter IV.1, P.65 \textcolor{red}{Buchverweis}
\end{proof}

% Korollar 4.7
\begin{kor}
  Es sei $X \in \mathcal V(M)$ und $p \in M$. Dann existiert eine offene Umgebung $U$ von $p$, ein $\varepsilon > 0$ und eine glatte Abbildung:
  \begin{align*}
    \gamma\colon(-\varepsilon,\varepsilon) \times U \to M
  \end{align*}
so dass $t \mapsto \gamma^t(p)$ eine Integralkurve von $X$ durch $p$ ist. (Setze dann $F(t,x) = \xi(\varphi^{-1}(x))$.)
\end{kor}

% Korollar 4.8
\begin{kor}
Sind $\gamma_1 \colon \calJ_1 \to M, \ \gamma_2 \colon \calJ_2 \to M$ Integralkurven eines Vektorfeldes $X \in \mathcal V(M)$ durch $p$, dann gilt $0 \in \calJ_1 \cap \calJ_2$ und $\gamma_1(0)= p = \gamma_2(0)$.
Nach \textcolor{red}{Satz 4.6 (ii)} % TODO: Referenz setzen
gilt dann $\gamma_1(t) = \gamma_2(t)$ für alle $t \in \calJ_1 \cap \calJ_2$. Damit ist
\begin{align*}
  \gamma \colon \calJ_1 \cap \calJ_2, \ t \mapsto 
  \begin{cases}
    \gamma_1(t) & t \in \calJ_1\\
\gamma_2(t) & t \in \calJ_2
  \end{cases}
\end{align*}
eine Integralkurve von $X$ durch $p$.
Damit existiert für jedes $p \in M$ ein maximaler Definitionsbereich $\calI_p$ für Integralkurven von $X$ durch $p$; dieser ist offen.
\end{kor}

\begin{dfn*}
  Für $X \in \mathcal V(M)$ heißt die, wie im vorigen Korollar definierte, Familie maximaler Integralkurven
  \begin{align*}
    \gamma(t,p) = \gamma^t(p),\ t \in \calI_p
  \end{align*}
der \CmMark{Fluss} des Vektorfeldes $X$.
Seinen Definitionsbereich notiert man mit:
\begin{align*}
  \mathcal D_X = \{(t,p) \in \R \times M \mid t \in \calI_p\}.
\end{align*}
\end{dfn*}

\begin{bem*}
Es gilt: $\gamma^0(\cdot) = \Id_M$. Ist $(s,p) \in \mathcal D_{X}$, so gilt
	\[ \difffrac[t=0]{}{t}(t \mapsto \gamma^{t+s}(p)) = X_{\gamma^s(p)}, \]
also ist $t \mapsto \gamma^{t+s}(p)$ eine Integralkurve von $X$ durch $q = \gamma^s(p)$. Aus der Eindeutigkeit folgt damit:
\begin{align*}
  \gamma^{t+s}(p) = \gamma^t(\gamma^s(p))
\end{align*}
f"ur alle $s, t$, $s+t \in \calI_p$, ferner gilt $\calI_q = \calI_p - s$. Kurz geschrieben: $\gamma(t+s) = \gamma^t \circ \gamma^s$. Also definiert $\gamma$ einen \quot{lokalen Gruppenhomomorphismus} von $\R$ in $M^M$.
\end{bem*}


% Satz 4.9
\begin{satz}
  Ist $X \in \mathcal V(M)$ ein glattes Vektorfeld, so ist $\mathcal D_X$ eine offene Menge und sein Fluss glatt.
\end{satz}

%%% Local Variables: 
%%% mode: latex
%%% TeX-master: "../skript-diffgeom"
%%% End: 
 \chapter{Vektorbündel}

Betrachte $\T M$ als glatte Mannigfaltigkeit durch
	\[ \T M|_U = \pi^{-1}(U) \to \varphi(U) \times \R^m \cong U \times \R^m \]
	\[ X_p = \sum \xi^i\pdifffrac[p]{}{x^i} \mapsto \left(\varphi(p),\xi\right). \]
Fasern : $\T_pM = \pi^{-1}(p)$ $m$-dimensionaler Vektorraum und $X_p = \sum \xi^i \pdifffrac[p]{}{x^i} \mapsto \xi$ ist ein linearer Isomorphismus.

% Definition 5.1
\begin{Dfn}
  Ein \CmMark{glattes reelles Vektorbündel} vom Rang $k$ über einer glatten Mannigfaltigkeit $M$ ist eine glatte Mannigfaltigkeit $E$, der sogenannte \CmMark{Totalraum} des Bündels, zusammen mit einer glatten Abbildung $\pi \colon E \to M$, der Projektion, sodass für jedes $p \in M$ gilt:
  \begin{enumerate}[label=(\roman*)]
  \item Die Faster $E_p = \pi^{-1}(p)$ trägt die Struktur eines $k$-dimensionalen reellen Vektorraumes.
  \item Es existiert eine Umgebung $U$ von $p$ in $M$ und ein Diffeomorphismus
    \begin{align*}
      \tau \colon E|_U = \pi^{-1}(U) \to U \times \R^k,
    \end{align*}
    so dass die Einschränkung
    \begin{align*}
      \tau_p \colon E_p \to \R^k \quad ( \cong \{p\} \times \R^k)
    \end{align*}
    ein linearer Isomorphismus ist.
    Ein solches $\tau$ heißt \CmMark{B"undelkarte}.
  \end{enumerate}
\end{Dfn}

\begin{bsp}
  \begin{enumerate}[label=(\arabic*)]
  \item $E = M \times \R^k$ mit $\pi \colon E \to M, (p,x) \mapsto p$.
  \item Das Tangentialbündel $\T M$ auf $M$.
  \item Ist $E \xrightarrow{\pi} M$ ein Vektorbündel über $M$ und $U \subseteq M$ offen (oder eine Untermannigfaltigkeit), so ist $E|_U = \pi^{-1}(U)$ ein Vektorbündel über $U$.
  \end{enumerate}
\end{bsp}

Ein \CmMark{Vektorbündelmorphismus} zwischen zwei Vektorbündeln $E \xrightarrow{\pi} M$ und $E' \xrightarrow{\pi'} N$ ist eine glatte Abbildung $F \colon E \to E'$, so dass eine glatte Abbildung $f$ existiert für die das folgende Diagram kommutiert

%%%
%%% TODO: Abbildung 5-1
\textcolor{red}{Abbildung 5-1: kommutierendes Diagram eines Vektorbündelmorphismus}
%%%

und ferner für $p \in M$ die Abbildung $E_p \xrightarrow{F} E'_{f(p)}$ linear ist.

Gilt $M = N$ so ist ein $M$-Vektorbündelmorphismus $F$ von $E$ nach $E'$ eine glatte Abbbildung $F \colon E \to E'$, so dass das folgende Diagram kommutiert und $F$ faserweise linear ist.

%%%
%%% TODO: Abbildung 5-2
\textcolor{red}{Abbildung 5-2: kommutierendes Diagram eines $M$-Vektorbündelmorphismus}
%%%

Die Vektorbündel $E,E'$ über $M$ heißen \CmMark{isomorph}, wenn ein $M$-Vektorbündelmorphismus
$G$ existiert mit $G F = \Id_E$ und $FG = \Id_{E'}$.
Dies ist genau dann der Fall, wenn $F$ faserweise ein Inverses besitzt.
(Der Beweis dieser Aussage sei als Übungsaufgabe überlassen.)

Ein Vektorbündel $E \xrightarrow{\pi} M$ heißt \CmMark{trivial}, wenn es einen Vektorbündelisomorphismus von $E$ auf $M \times \R^k$ gibt. Jedes
\begin{align*}
  \tau \colon E|_U \to U \times \R^k
\end{align*}
ist ein Vektorbündelisomorphismus.
Die Bündelkarten werden daher auch \CmMark{lokale Trivialisierungen} genannt.

Es sei $(\tau_\alpha,U_{\alpha})_{\alpha \in \calI}$ eine Familie lokaler Trivialisierungen von $E$ mit $M = \bigcup_{\alpha \in J}U_{\alpha}$.
Der Diffeomorphismus
\begin{align*}
  \tau_{\alpha} \circ \tau_{\beta}^{-1}\colon (U_{\alpha} \cap U_{\beta}) \times \R^k \to (U_{\alpha} \cap U_{\beta}) \times \R^k
\end{align*}
definiert die sogenannten \CmMark{Übergangsfunktionen}
\begin{align*}
  g_{\alpha\beta} \colon U_{\alpha} \cap U_{\beta} \to \gls{GL}_k(\R)
\end{align*}
durch 
\begin{align*}
  \tau_{\alpha} \circ \tau_{\beta}^{-1}(p,x) = (p,g_{\alpha\beta}(p) x).
\end{align*}
Die Übergangsfunktionen sind glatt und für alle $p \in U_{\alpha} \cap U_{\beta} \cap U_{\gamma}$ gilt:
\begin{align*}
  g_{\alpha\gamma}(p) = g_{\alpha\beta}(p) \cdot g_{\beta\gamma}(p),
\end{align*}
denn
\begin{align*}
	(p,g_{\alpha\gamma}(p)x) & = \tau_{\alpha} \circ \tau_{\gamma}^{-1}(p,x)\\
	& = \tau_{\alpha} \circ \tau_{\beta}^{-1} \circ \tau_{\beta} \circ \tau_{\gamma}^{-1}(p,x)\\
	& = (\tau_{\alpha} \circ \tau_{\beta}^{-1})(p,g_{\beta\gamma}(p)x)\\
	& = (p,g_{\alpha\beta}(p)\cdot g_{\beta\gamma}(p)x).
\end{align*}

\begin{bsp}
  Die Übergangsfunktionen von $\T M$ sind gegeben durch
  \begin{align*}
    \D(\psi \circ \varphi^{-1}) = \left(\partial_i(\psi^j \circ \varphi^{-1})\right)_{i,j \leq m}.
  \end{align*}
\end{bsp}

% Satz 5.2
\begin{Satz}
  Es sei $M$ eine glatte Mannigfaltigkeit mit einer offenen Überdeckung $\{U_{\alpha}\}_{\alpha \in \calI}$ und einer glatten Abbildung
  \begin{align*}
    g_{\alpha\beta} \colon U_{\alpha} \cap U_{\beta} \to \Gl_k(\R)
  \end{align*}
so dass für alle $\alpha,\beta,\gamma \in \calI$ und $p \in U_{\alpha} \cap U_{\beta} \cap U_{\gamma}$ gilt:
\begin{align*}
  g_{\alpha\gamma} (p) = g_{\alpha\beta}(p)g_{\beta\gamma}(p).
\end{align*}
Dann ist
	\[ E = \bigcup_{\alpha \in \calI}^{\cdot} \FakRaum{U_{\alpha} \times \R^k}{\sim}, \]
wobei $(p,x)_{\alpha} \sim (q,y)_{\beta}$ genau dann gilt, wenn $p = q$ und $x = g_{\alpha\beta}(p)y$, ein glattes Vektorbündel.
\end{Satz}

Der Beweis sei erneut als Aufgabe überlassen.

\begin{kor}
  Ist $E$ ein glattes Vektorbündel über $M$ mit Übergangsfunktionen $\{g_{\alpha\beta}\}$, so ist das oben konstruierte Vektorbündel isomorph zu $E$.
\end{kor}

Es sei $E \xrightarrow{\pi} N$ ein Vektorbündel und $\Phi \colon M \to N$ glatt.
Das längs $\Phi$ zurückgezogene Bündel ("`\CmMark{pullback}"') ist definiert durch den Totalraum
\begin{align*}
  E' = \Phi^{\ast}E = \{(p,x) \mid x \in E_{\Phi(p)}\} \subseteq M \times E,
\end{align*}
die Projektion $\pi' \colon \Phi^{\ast}E \to M, (p,x) \mapsto p$ und die folgenden Bündelkarten:
Es sei $p \in M$ und $(\tau, U)$ eine Bündelkarte von $E$ um $\Phi(p)$, sowie $(\varphi,V)$ eine Karte von $M$ um $p$ mit $\Phi(V) \subseteq U$.
Dann definiert 
\begin{align*}
  \Phi^{\ast}E|_V \to V \times \R^k, (p,x) \mapsto \left(p, \tau_{\Phi(p)}(x)\right)
\end{align*}
eine Bündelkarte.

Sind $E \xrightarrow{\pi} M, E' \xrightarrow{\pi'} N$ Vektorbündel, dann ist $E \times E' \xrightarrow{\pi \times \pi'} M \times N$ mit lokalen Trivialisierungen $\tau \times \tau'$ ebenfalls ein Vektorbündel.
Insbesondere ist im Falle $M = N$ $E \times E'$ ein Bündel über $M \X M$.
Es sei $\Delta \colon M \to M \times M$, $p \mapsto (p,p)$.

%%%
%%% TODO
\textcolor{red}{Abbildung 5-3: Diagram}

Das längs $\Delta$ zurückgezogene Bündel $E \oplus E' = \Delta^{\ast}(E \times E')$ heißt die \CmMark{Whitneysumme} von $E$ und $E'$.
Faserweise gilt
\begin{align*}
  (E \oplus E')_p = E_p \oplus E'_p.
\end{align*}
 
\emph{Überlege:} $\Hom(E,E')$, sowie $E \otimes E', \bigotimes E$ und $\Lambda^pE, \Lambda E$ sind "`vernünftige"' Bündel.


%%% Local Variables: 
%%% mode: latex
%%% TeX-master: "../skript-diffgeom"
%%% End: 
 \chapter{Riemannsche Metriken}

\begin{emptythm}[Was ist Geometrie?]
Vereinfacht ausgedr"uckt suchen wir eine M"oglichkeit um Distanzen und Winkel auszudrücken. Betrachte im Folgenden die Einheitssph"are, auf der wir den eine Reise von $x$ nach $y$ unternehmen m"ochten.
\begin{center}\begin{tikzpicture}
	%\draw[step=0.25,gray!15] (-4,-4) grid (4,4); \draw[step=0.5,gray!30] (-4,-4) grid (4,4); \fill (0,0) circle(0.1); %Hilfsgitter
	
	\draw (0,0) circle (3);
	\begin{scope}
		\clip (-3,0) rectangle (3,1.6);
		\draw[dashed] (0,0) ellipse (3 and 1);
	\end{scope}
	\begin{scope}
		\clip (-3,0) rectangle (3,-1.6);
		\draw (0,0) ellipse (3 and 1);
	\end{scope}
	
	\fill (0,3) circle(0.1) node[anchor=south west]{$N$};
	\fill (0,-3) circle(0.1) node[anchor=north west]{$S$};
	
	\coordinate (x) at (-0.5,2); \coordinate (y) at (0.75,-2);
	\fill (x) circle(0.1)node[above]{$x$};
	\fill (y) circle(0.1)node[below]{$y$};
	\draw (x) -- (y);
	\coordinate (a) at (1.25,0.75); \coordinate (b) at (1,0.25); \coordinate (c) at (1.5,-0.25);
	\draw (x) ..controls(x) and ($(a) + (-1.25,1.25)$).. (a) ..controls($(a) + (0.25,-0.25)$) and ($(b) + (0,0.25)$).. (b) ..controls($(b) + (0,-0.25)$) and ($(c) + (0,0.25)$).. (c) ..controls($(c) + (0,-0.75)$) and (y).. (y);
	
	\node at (2.75,2.75) {$S^2 \subset \R^3$};
	\draw[->] (1.5,-2)node[anchor=north]{$c$} to[out=90,in=310] (1.25,-1.25);
	
	\node at (-3.25,-2.5) {$c: [0,1] \to S^2$};
	\node at (-3.25,-3.25) {$c(0) = x,\quad c(1) = y$};
\end{tikzpicture}\end{center}
Wir definieren mit $\calL(c) = \int_0^1 \|\dot c\| \dop t$ die \CmMark[Metrik!Riemann-]{Riemann-Metrik}, also das Skalarprodukt mit allen $\T_pM$. Damit folgt dass wenn $c: [0,1] \to M$ glatt ist, dass $\calL(c) = \int_0^1 \sqrt{\langle \dot c, \dot c \rangle}$ und der Abstand auf $M$ kann ausgedr"uckt werden durch $d_M(x,y) = \inf \{\calL(c) | c$ von $x$ nach $y\}$.

Das wirft Fragen auf nach der Existenz k"urzester Abst"ande, Unterschieden zwischen lokal K"urzestem und global K"urzesten und der Eindeutigkeit.
\end{emptythm}

\begin{Dfn}
Es sei $M$ eine glatte Mannigfaltigkeit. Eine \CmMark[Metrik!Riemann-]{Riemannsche Metrik} $g$ auf $M$ ist gegeben durch ein Skalarprodukt auf jedem $\T_pM$, welches glatt von $p$ abh"angt, das hei"st $g \in \calT_2^0(M)$, so dass $g_p = \langle \cdot, \cdot \rangle_p : \T_pM \X \T_pM \ \to \R$ symmetrisch und positiv definit ist. Ist $g$ eine Riemann-Metrik auf M, so hei"st $(M,g)$ eine \CmMark[Mannigfaltigkeit!Riemannsche]{Riemannsche Mannigfaltigkeit}.
\end{Dfn}

Ist $(M,g)$ eine Riemannsche Mannigfaltigkeit, $X, Y \in \calV(M)$, $X = \sum \xi^{i} \pdifffrac{}{x^{i}}$, $Y = \sum \eta^j \pdifffrac{}{y^j}$, dann ist
\begin{align*}
	g(X, Y) &= g\left(\sum \xi^{i} \pdifffrac{}{x^{i}}, \sum \eta^j \pdifffrac{}{y^j}\right)\\
	&= \sum_{i,j}\xi^{i} \eta^j g\left(\pdifffrac{}{x^{i}}, \pdifffrac{}{y^j}\right)\\
	&= \sum_{i,j} \xi^{i} \eta^j g_{ij} \qquad (g_{ij} \text{ glatt, } g_{ij} = g_{ji})
\end{align*}

\begin{bsp}\begin{enumerate}[label=\arabic*),leftmargin=*]
\item
	$\R^m$ tr"agt eine nat"urliche Riemannsche Metrik: F"ur $x \in \R^m$ ist $\calI_x: \T_x\R^m \to \R^n$ ein (nat"urlicher) Isomorphismus. Damit definiert
		\[ g_x(\cdot,\cdot) = \langle \calI_x(\cdot,\cdot), \calI_x(\cdot,\cdot) \rangle \]
	eine Riemannsche Metrik auf $\R^m$. Bez"uglich der Karte $(\Id, \R^m)$ gilt
		\[ g_{ij} = \sum_{ij} \delta_{ij} \dop x^{i} \otimes \dop x^j = \sum_i \dop x^{i} \otimes \dop x^j \]
\item
	Betrachtet man Polarkoordinaten auf $\R^2(r, \vartheta)$:
	\begin{align*}
		\pdifffrac[(r,\vartheta)]{}{r} &= (\cos \vartheta, \sin \vartheta)\\
		\pdifffrac[(r,\vartheta)]{}{\vartheta} &= r(-\sin \vartheta, \cos \vartheta)
	\end{align*}
	\begin{align*}
		g_{rr} &= g\left( \pdifffrac{}{r}, \pdifffrac{}{r} \right) = 1\\
		g_{\vartheta\vartheta} &= g\left( \pdifffrac{}{\vartheta}, \pdifffrac{}{\vartheta} \right) = r^2\\
		g_{r\vartheta} &= g_{\vartheta r} = 0
	\end{align*}
\item
	Sei $M \subseteq \R^n$ $m$-dimensionale glatte Untermannigfaltigkeit. $M$ tr"agt eine nat"urliche Riemann-Metrik:
	\begin{center}\begin{tikzpicture}
		%\draw[step=0.25,gray!15] (-4,-4) grid (4,4); \draw[step=0.5,gray!30] (-4,-4) grid (4,4); \fill (0,0) circle(0.1); %Hilfsgitter
		
		\draw (0,0) circle(2);
		\begin{scope}
			\clip (-2,0) rectangle (2,2);
			\draw[dashed] (0,0) ellipse(2 and 0.75);
		\end{scope}
		\begin{scope}
			\clip (-2,0) rectangle (2,-2);
			\draw (0,0) ellipse(2 and 0.75);
		\end{scope}
		
		\path[name path=laenge] (0,2) to[out=335,in=33] (0,-2);
		\path[name path=breite] (0,2) ellipse(2 and 0.75);
		\path[name path=aequator] (0,0) ellipse(2 and 0.75);
		\path[name intersections={of=laenge and breite, by=p}];
		\path[name intersections={of=laenge and aequator, by={grenze1, grenze2}}];
		
		\coordinate (a) at (1,2.25); \coordinate (b) at (-1,0.5); \coordinate (c) at (1.75,-1.5); \coordinate (d) at ($(a) + (c) - (b)$);
		\path[draw,name path=raute] (a) -- (b) -- (c) -- (d) -- cycle;
		
		\begin{scope}
			\clip (0,2) rectangle ($(grenze2) + (0.5,0)$);
			\draw (0,2) to[out=335,in=33] (0,-2);
			\clip (0,0) circle(2);
			\draw (0,2) ellipse(2 and 0.75);
		\end{scope}
		\begin{scope}
			\clip (0,0) circle(2);
			\draw (0,2) ellipse(2 and 0.75);
		\end{scope}
		
		\node at (-2,2) {$S^2 \subset \R^3$};
		\fill (p) circle(0.05)node[anchor=south west,font=\scriptsize] at (p) {$p$};
		\draw[->] (p) -- ($(p) + 0.6*(1,-1.7)$);
		\draw[->] (p) -- ($(p) + 1.1*(1,0.15)$);
		
		\draw[->] (2.25,2.25)node[right,font=\scriptsize]{$\pdifffrac{}{\varphi}$} to[out=210,in=80] (1.65,1.5);
		\draw[->] (3.25,1.25)node[right,font=\scriptsize]{$\pdifffrac{}{\vartheta}$} to[out=180,in=30] (1.2, 0.75);
	\end{tikzpicture}\end{center}
	F"ur jedes $p \in M$ ist $\T_pM$ kanonisch isomorph zum von partiellen Ableitungen $\partial_1F|_p, \ldots ,\partial_mF|_p$ einer lokalen Parametrisierung $F$ aufgespannten Untervektorraum $\R^m$. Mit diesem (lokalen) isomorphismus definiert
		\[ g_{ij} = \langle \partial_i F, \partial_j F \rangle \]
	eine Riemann-Metrik auf $M$.
\end{enumerate}\end{bsp}

\begin{bem}
Sind $\varphi$ und $\psi$ Karten einer Riemannschen Mannigfaltigkeit $(M,g)$ um $p$ und sind $g = \sum g_{ij} \dop x^{i} \otimes \dop x^j$ und $h = \sum h_{ij} \dop y^{i} \otimes \dop y^j$ die lokalen Darstellungen bez"uglich $\varphi$ beziehungsweise $\psi$, so gilt
	\[ h_{kl} = g\left( \pdifffrac{}{y^k}, \pdifffrac{}{y^l} \right) = \sum_{i,j} \pdifffrac{x^{i}}{y^k} \underbrace{\pdifffrac{x^j}{y^l}}_{\mathclap{\qquad \partial_l(\varphi^{i}\circ \psi^{-1})}} g_{ij} \]
\end{bem}

Eine Riemannsche Metrik induziert eine Metrik auf dem Kotangentialb"undel: Die Isomorphismen $\T_pM \to \T_p^*M$, $X \mapsto \langle X, \cdot \rangle_p$ einen Isomorphismus von $\T M$ nach $\T^*M$. F"ur $\omega \in \T_p^*M$ sei $X(\omega) \in \T_pM$ mit $\omega = \langle X(\omega), \cdot \rangle_p$. Man definiert nun durch
	\[ \langle \omega, \tilde \omega \rangle = \langle X(\omega), X(\tilde \omega) \rangle \]
ein Skalarprodukt auf $\T_p^*M$. F"ur $\omega = \sum \omega_i \dop x^{i}$, $X(\omega) = \xi^{i} \pdifffrac{}{x^{i}}$ gilt
	\[ \omega_i = \omega \left( \pdifffrac{}{x^{i}} \right) = \left\langle X(\omega), \pdifffrac{}{x^{i}} \right\rangle = \sum_j \xi^{i} g_{ij} \]
Also $\xi^{i} = \sum g^{ij} \omega_i$, wobei $(g^{ij})$ die zu $(g_{ij})$ inverse Matrix ist. Damit gilt:
\begin{align*}
	\langle \omega, \tilde \omega \rangle &= \langle X(\omega), X(\tilde \omega) \rangle \\
	&= \sum g_{kl} \xi^k \xi^l\\
	&= \sum g_{kl} g^{ki} \omega_i g^{lj} \tilde \omega_j\\
	&= \sum \delta_l^i g^{lj} \omega_i \tilde \omega_j\\
	&= \sum g^{ij} \omega_i \tilde \omega_j\\
\end{align*}


%%%
%%% 13. Vorlesung <2012-11-27 Tue>
%%%

Für beliebige Tensoren $S, S' \in T_q^p(\T M)$ und $T, T' \in T_l^k(\T M)$ definiert man induktiv durch lineare Fortsetzung Skalarprodukte wie folgt:
\begin{align*}
  \left<S \otimes T, S' \otimes T'\right> = \left<S,S'\right>\left<T,T'\right>.
\end{align*}
Auf $\T M \otimes \T M$ hat die Metrik die folgende Gestalt:
\begin{align*}
  \left<X \otimes Y,\tilde X \otimes \tilde Y\right> = \sum g_{ij}g_{kl}\xi^i\xi^j\tilde\eta^k\tilde\eta^l.
\end{align*}

% Definition 6.2
\begin{Dfn}
  Es seien $(M, g)$ und $(N,h)$ Riemannsche Mannigfaltigkeiten.
Ein Diffeomorphismus $\Phi \colon M \to N$ heißt \CmMark{Isometrie}, falls $\Phi^{*}h = g$, i.e. für alle $p \in M$ und $X,Y \in \T_pM$ gilt:
\begin{align*}
  g_p(X,Y) = \underbrace{h_{\Phi(p)}(\Phi_{*p}X,\Phi_{*p}Y)}_{\mathclap{= \Phi^{*}h(X,Y) \; \rightsquigarrow \text{Pullback Metrik}}}
\end{align*}
Ist umgekehrt $\Phi \colon M \to N$ ein Diffeomorphismus und $h$ eine Riemannsche Metrik auf $N$, so ist $\Phi^{*}h$ eine Riemannsche Metrik auf $M$.
\end{Dfn}

% Satz 6.3
\begin{Satz}\label{Satz-6-3}
  Jede glatte Mannigfaltigkeit trägt eine Riemannsche Metrik.
\end{Satz}

Um Metriken in den Überlappungsgebieten von Karten "`verkleben"' zu können, benötigt man das folgende Hilfsmittel.

% Hilfssatz
\begin{satz}[Zerlegung der Eins]
  Es sei $M$ eine glatte Mannigfaltigkeit $\{U_i\}_{i \in J}$ eine offene Überdeckung von $M$.
  Dann existiert eine Zerlegung der Eins auf einer abzählbaren, lokal endlichen Verfeinerung von $\{U_i\}_{i \in J}$, d.h. es existiert eine abzählbare offene Überdeckung $\{V_k\}_{k\in\N}$ von $M$ und glatte Funktionen mit kompaktem Träger $\alpha_k \colon M \to \R$, so dass gilt:

  \begin{enumerate}[label=(\roman*)]
  \item $\forall k \in \N \ \exists i(k) \in J: V_k \subseteq U_{i(k)}$ (Verfeinerung),
  \item $\forall p \in M \ \exists U \ni p: \# \{k \mid V_k \cap U \neq \emptyset \} < \infty$ (lokal endlich),
  \item $\forall k \in \N: \supp (\alpha_k) \subseteq V_k$,
  \item $\forall k \in \N \ \forall p \in M: 0 \leq \alpha(p) \leq 1$,
  \item $\forall p \in M: \sum_{k\in\N}\alpha_k(p) = 1$.
  \end{enumerate}
  (Wegen (ii) und (iii) ist die Summe in (v) endlich).
\end{satz}
An dieser Stelle geht maßgeblich ein, dass die Topologie von $M$ eine abzählbare Basis besitzt. Beweis siehe Boothby, Kapitel V.4 \cite{boothby1986introduction}.

\begin{bew}(von Satz \ref{Satz-6-3})

  Es sei $M$ eine glatte, $m$-dimensionale Mannigfaltigkeit. $\{(\varphi_i,U_i)\}_{i \in J}$ ein Atlat von $M$ und $\{(V_k,\alpha_k)\}_{k \in \N}$ eine Zerlegung der Eins auf einer abzählbaren, lokal endlichen Verfeinerung von $\{U_i\}_{i \in J}$.

  Es sei $\beta$ ein Skalarprodukt auf $\R^m$.
  Für jedes $k \in \N$ ist dann
  \begin{align*}
    g_k = \left.\varphi_{i(k)}\right|_{V_k}^{*}\beta
  \end{align*}
  eine Riemannsche Metrik auf $V_k$.

  Damit ist $g = \sum g_k\alpha_k$ eine Riemannsche Metrik auf $M$.

  Die Summe ist punktweise endlich und $g$ ist als Komposition glatter Abbildungen selbst glatt.
  Symmetrie und Bilinearität folgen sofort.
  Für jedes $p \in M$ gilt $\sum_{k \in \N}\alpha_k(p) = 1$, d.h. ex existiert ein $l \in \N$ mit $\alpha_l(p) > 0$ und für $X \in T_pM$ mit $X \neq 0$ folgt:
  \begin{align*}
    g_p(X,X) & = \sum \underbrace{g_k(p)(X,X)\alpha_k(p)}_{> 0}\\
    & \geq g_l(p)(X,X)\alpha_l(p) > 0.
  \end{align*}
  Damit ist $g$ positiv definit.
\end{bew}

Für eine glatte Kurve $\gamma \colon [a,b] \to M$ heißt
\begin{align*}
  \mathcal L(\gamma) = \int_{a}^b\|\cdot \gamma\| = \int_a^b \sqrt{g_{\gamma(t)}(\cdot\gamma(t),\cdot\gamma(t))}\dop t
\end{align*}
die \CmMark[Kurvenlänge]{(Kurven-)Länge} von $\gamma$.

Ist $\tau \colon [\alpha,\beta] \to [a,b]$ glatt und monoton, so gilt
\begin{align*}
  \mathcal L(\gamma \circ \tau) & = \int_{\alpha}^{\beta}\|\cdot\gamma(\tau(s))\||\tau'(s)|\dop s\\
  & = \int_a^b\|\cdot\gamma\| = \mathcal L(\gamma).
\end{align*}
Damit ist die Kurvenlänge invariant unter Reparametrisierungen.

Ist $\gamma$ regulär, d.h. $\cdot\gamma(t) \neq 0$ für alle $t \in [a,b]$, so ist ihre sogenannte \CmMark{Bogenlänge}
\begin{align*}
  \sigma \colon [a,b] \to [0,\mathcal L(\gamma)], t \mapsto \mathcal L(\gamma|_{[a,t]}) = \int_a^b\|\cdot\gamma\|.
\end{align*}
streng monoton steigend, $\sigma'(s) = \|\cdot\gamma(s)\| > 0$.

Für $\tilde\gamma = \gamma \circ \sigma^{-1}\colon [0,\mathcal L(\gamma)] \to M$ gilt $\|\cdot\tilde\gamma\| \equiv 1$.
Die Kurve $\tilde \gamma$ heißt \CmMark{Bogenlängenparametrisierung} von $\Gamma$.
Gilt für $\gamma \colon [a,b] \to M$ gerade $\|\cdot\gamma\| \equiv \lambda$, so heißt $\gamma$ \CmMark[Parametrisierung!proportional zur Bogenlänge]{proportianal zur Bogenlänge} parametrisiert.

Sind $\gamma \colon [a,b] \to M, \tilde \gamma \colon [b,c] \to M$ glatte Kurven mit $\gamma(b) = \tilde \gamma(b)$, so sei
\begin{align*}
  \mathcal L(\gamma \cup \tilde\gamma) = \mathcal L(\gamma) + \mathcal L(\tilde \gamma).
\end{align*}
Eine Kurve $\gamma \colon [a,b] \to M$ heißt \CmMark{stückweise glatt}, wenn $t_0, \ldots, t_k$ mit $a = t_0 < t_1 < \cdots < t_k = b$ existieren, so dass $\gamma|_{[t_{i-1},t_i]}$ für alle $i \leq k$ glatt ist.

\textcolor{red}{Abbildung 13.1, stückweise glatte Wege}

% Definition 6.4
\begin{Dfn}
  Für Punkte $p, q \in M$ ist der \CmMark{Abstand} definiert durch:
  \begin{align*}
    \dop(p,q) = inf\{ \mathcal L(\gamma) \mid \gamma \colon [0,1] \to M \text{ stückweise glatt mit } \gamma(0) = p, \gamma(1) = q\}.
  \end{align*}
\end{Dfn}

% Satz 6.5
\begin{Satz}
  Es sei $(M,g)$ eine zusammenhängende Riemannsche Mannigfaltigkeit.
  Die Abstandsfunktionen bilden eine Metrik auf $M$, welche die ursprüngliche Topologie induziert.
\end{Satz}

Der Beweis sei zur Übung überlassen.

% Satz 6.6
\begin{Satz}
  Es seien $(M,g)$ und $(N,h)$ zusammenhängende Riemannsche Mannigfaltigkeiten und $\Phi \colon M \to N$ ein Diffeomorphismus.
  Dann ist $\Phi$ genau dann eine Isometrie, wenn $\mathcal L(\Phi \circ \gamma) = \mathcal L(\gamma)$ für alle glatten $\gamma \colon [0,1] \to M$ gilt. 
\end{Satz}

\begin{bew}
  Dass eine Isometrie die Kurvenlängen erhält, gilt ist offensichtlich.

  Erhält $\Phi$ die Kurvenlängen, so erhält $\Phi$ auch die Norm von Tangentialvektoren, den andernfalls gäbe es $X_p \in T_pM$ mit o.E.
  \begin{align*}
    h_{\Phi(p)}(\Phi_{*p}X,\Phi_{*p}X) > g_p(X,X)
  \end{align*}
und eine Kurve $\gamma\colon [0,1] \to M$ mit $\gamma(0) = X$ und es gälte (für hinreichend kleines $\varepsilon$):
\begin{align*}
  \mathcal L(\gamma|_{[0,\varepsilon]}) & = \int_0^{\varepsilon}\sqrt{g_{\gamma(t)}(\cdot\gamma(t),\cdot\gamma(t))}\dop t\\
& < \int_0^{\varepsilon}\sqrt{h_{\Phi(\gamma(t))}(\Phi_{*\gamma(t)},\cdot\gamma(t)\Phi_{*\gamma(t}\cdot\gamma(t)}\dop t\\
& = \int_0^{\varepsilon}\sqrt{_{\Phi(\gamma(t))}(\cdot{(\Phi \circ \gamma)}(t), \cdot{(\Phi \circ \gamma)}(t))}\dop t\\
& = \mathcal L ((\Phi \circ \gamma)|_{[0,\varepsilon]}).
\end{align*}
Mit der Polarisationsformel $\left<x,y\right> = - \frac{1}2 (\|x-y\|^{2} - \|x\|^2-\|y\|^2)$ folgt dann, dass $\Phi$ auch die Skalarprodukte erhält.
\end{bew}

% Definition 6.7
\begin{Dfn}
  Eine Kurve $\gamma \colon [a,b] \to M$ heißt \CmMark[Geodätische!minimale]{minimale Geodätische}, falls ein $\lambda \geq 0$ existiert, so dass für alle $a \leq s < t \leq b$ gilt:
  \begin{align*}
    \mathcal L(\gamma|_{[s,t]}) = \lambda(t-s) = \dop(\gamma(s),\gamma(t)).
  \end{align*}

  Eine Kurve $\gamma$ heißt \CmMark{Geodätische}, falls sie lokal minimierende Geodätische ist, d.h. für alle $t \in [a,b]$ existiert ein $\varepsilon > 0$, so dass $\gamma|_{[t-\varepsilon,t+\varepsilon]}$ minimierende Geodätische ist.
\end{Dfn}

Eine bessere Vorstellung erhält man durch Betrachtung von Geodätischen als Isometrien von Intervallen in den euklidischen Raum, denn $\dop(\gamma(s),\gamma(t)) = t-s = \dop_{\R}(t,s)$.

\textcolor{red}{Abbildung 13.2: Geodätische}


%%% Local Variables: 
%%% mode: latex
%%% TeX-master: "../skript-diffgeom"
%%% End: 

%% 
%% 14. Vorlesung <2012-11-30 Fri>, Fortsetzung
%% 

\chapter{Kovariante Ableitungen}

\paragraph{Frage:} Was ist eine \quot{gute} Differentialrechnung für Vektorfelder?

Das gewöhnliche Differential im $\R^n$ für $Y \colon \R^n \to \R^n$ ist gerade die lineare Abbildung $\D Y|_p \cdot v = \lim \frac{1}t \left(Y(p+tv) -Y(p)\right) = \difffrac[t=0]{}{t} Y(p+tr)$.
Betrachte im euklidischen Fall einen Punkt $p$, sowie einen Tangentialvektor $Y_p$.
\begin{center}\begin{tikzpicture}[font=\scriptsize]
	\coordinate (end) at (1.5,0.75); %Endrichtung
	\draw (0,0) -- ($(0,0) + 4*(end)$);
	% die Punkte
	\coordinate (p) at ($(0,0) + 0.5*(end)$); \coordinate (q) at ($(0,0) + 2.75*(end)$);
	\fill (p) circle(0.1)node[anchor=north west]{$p$}; \fill (q) circle(0.1)node[anchor=north west]{$p + tv$};
	% die Pfeile
	\draw[->] (p) --node[left]{$Y_p$} ($(p) + 1.25*(-0.75,2)$);
	\coordinate (dir) at (0.15,1);
	\def\scl{1.5}
	\draw[->] (q) --node[right]{$Y_{p+tv}$} ($(q) + \scl*(dir)$); \draw[->] (p) -- ($(p) + \scl*(dir)$);
	% Parallelverschiebung
	\draw[->,dashed] ($(q) + 0.5*\scl*(dir)$) --node[above,sloped]{Parallelverschiebung} ($(p) + 0.5*\scl*(dir)$);
\end{tikzpicture}\\
\textcolor{red}{evtl. auch noch die Idee der Parallelverschiebung erklären.}\end{center}

Nun gehe zur Betrachtung von Vektorfeldern $X \colon \R^n \to \R^n$ über und setze $\D_XY|_p = \D Y|_p\cdot X_p$. Hierfür gilt:
\begin{itemize}
\item $\D$ ist $\R$-linear in $Y$: $\D(Y + \tilde Y) = \D Y + \D \tilde Y$.
\item Es gilt die Leibnizregel: $\D(fY) = \D f \cdot Y + f\D Y$.
\item $\D$ ist $C^{\infty}(\R^n)$-linear in $X$:
  \begin{align*}
    \D_{fX}Y|_p = \D Y|_p\cdot(fX)_p = \D Y|_p \cdot f(p)X_p = f(p) \D Y|_p \cdot X_p = (f \D_XY)(p).
  \end{align*}
\end{itemize}

\emph{Erinnerung:} Die Lieableitung $\mathcal L_{(\cdot)}Y$ ist \emph{nicht} $C^{\infty}$-linear.

% Definition 7.1
\begin{Dfn}
  Es seien $M$ eine glatte Mannigfaltigkeit und $E$ ein Vektorbündel über $M$.
  Eine \CmMark{kovariante Ableitung} (oder \CmMark{Zusammenhang} ([engl.] \quot{connection}) auf $E$ ist eine Abbildung
  \begin{align*}
    \nabla \colon \mathcal V(M) \times \Gamma(E) \to \Gamma(E), \quad \nabla(X,S) = \nabla_XS
  \end{align*}
  mit den folgenden Eigenschaften:
  \begin{enumerate}[label=(\roman*),widest=iii]
  \item $\nabla S$ ist $C^{\infty}(M)$-linear, das hei"st
    \begin{align*}
      \nabla_{X+Y}S = \nabla_XS+\nabla_YS \text{ und } \nabla_{fX}S = f\nabla_XS
    \end{align*}
    f"ur alle $X, Y \in \calV(M)$ und $f \in C^{\infty}(M)$.
  \item $\nabla_X$ ist $\R$-linear, das hei"st
    \begin{align*}
      \nabla_X(\mu S + \nu T) = \mu\nabla_XS + \nu\nabla_XT.
    \end{align*}
  \item $\nabla_X$ erfüllt die folgende Leibnizregel:
    \begin{align*}
      \nabla_X(fS) = \dop f(X) \cdot S + f\cdot \nabla_XS = X(f)\cdot S + f \cdot \nabla_XS.
    \end{align*}
  \end{enumerate}
  Kurzform: $\nabla \colon \Gamma(E) \to \Gamma(\T^{*}M \otimes E), S \mapsto \nabla_{(.)}S$ ist eine $C^{\infty}(M)$-Modulderivation.
\end{Dfn}

\begin{bsp}\begin{enumerate}[label=\arabic*),leftmargin=*]
\item
	Das gewöhnliche Differential $\D$ definiert in kanonischer Weise eine kovariante Ableitung auf $\T\R^n$.
	\begin{align*}
		X \in \mathcal V(\R^n), X \colon \R^n \to \T\R^n \cong \R^n \times \R^n \text{ via } \calI\colon X_p \mapsto (p,\underbrace{\calI_p(X_p)}_{=:\overline X_p}).
	\end{align*}
	Nun ist wie folgt eine kovariante Ableitung gegeben: $(\nabla_XY)_p = \calI^{-1}(p,\D_{\overline X_p}\overline Y)$.
\item
	$E = M \times \R^n$, ein Schnitt $S$ von $E$ ist von der Form $S_p = (p,s(p))$, $s \colon M \to \R^n$ glatt.

	Hier definiert man die kovariante Ableitung:
	\begin{align*}
		& \nabla_XS = (p,\calI_{s(p)}^{-1}(s_{*p},X_p))\\
		&  s_{*p}\colon \T_{*p}M \to \T_{*p}\R^n, s_{*p}\colon X_p \in \T_{*p}\R^n \xrightarrow{\calI_{s(p)}} \R^n.
	\end{align*}
\item
	Sei $E = M \times \R^n$, ein Schnitt $S = (\Id, \sigma)$, $\sigma: M \to \R^n$. Dann ist $(\nabla_X S)_p = (p, \calI_p(\sigma_{*p}(X_p))$, $\sigma_{*p}: \T_pM \to \T_{\sigma(p)}\R^n$. Sei $ \omega = (\omega_j^k)_{j,k \le n}$ eine $(n\times n)$-Matrix von 1-Formen auf $M$, das hei"st $\omega(X)|_p \in \mathfrak M^{n\times n}(\R)$.
	Für einen Schnitt $S = (\Id, \sigma)$ und sei dann
		\[ (\nabla_XS)_p = (\Id, \calI_p(\sigma_{*p}(X_p)) + \omega(X)|_p \cdot \sigma(p). \]
	Dies definiert eine kovariante Ableitung auf $E = M \X \R^n$.
\item
	$\dop \colon \Omega^0(M) = C^{\infty}(M) = \Gamma(M\times \R) \to \Omega^1(M) = \Gamma(\T^{*}M) = \Gamma(\underbrace{\T^{*}M \otimes (M \times \R)}_{\mathclap{\text{Fasern: } \T_p^{*}M\otimes\R \cong \T_p^{*}M}})$ mit $f \mapsto [\dop f \colon X \mapsto \dop f(X) = X(f)]$.

	Dann ist
	\begin{align*}
		& \dop \colon \mathcal V(M) \times C^{\infty}(M) \to C^{\infty}(M),\\
		& \nabla_Xf = \dop (X,f) \mapsto X(f)
	\end{align*}
	eine kovariante Ableitung auf $C^{\infty}(M)$.
\item
	Es sei $M \subseteq \R^k$ eine glatte Untermannigfaltigkeit und $\nabla$ die kanonische kovariante Ableitung auf $\T \R^k$.

	Erster Ansatz für eine kovariante Ableitung:
	\begin{align*}
		\tilde \nabla_XY = \nabla_{\tilde X}\tilde Y|_M \text{ das funktioniert noch nicht.}
	\end{align*}
	Für $X,Y \in \mathcal V(M)$ seien $\tilde X, \tilde Y$ Fortsetzungen, das hei"st $\tilde X|_M = X$ und $\tilde Y|_M = Y$.
	\begin{align*}
		(\nabla_{\tilde X}\tilde Y)_p \in \T_p\R^k \supseteq \T_pM.
	\end{align*}

	Nächster Ansatz, der tasächlich eine kovariante Ableitung definiert.
	\begin{align*}
		\tilde \nabla_XY = (\nabla_{\tilde X}\tilde Y|_M)^{\text{proj}\T_pM},
	\end{align*}
	wobei $X^{\text{proj}\T_pM}$ die orthogonale Projektion von X auf den Tangentialraum $\T_pM$ bzgl. des Standardskalarproduktes ist.
\end{enumerate}\end{bsp}

Schreibt man in Beispiel 3) $\sigma = ( \sigma^1, \ldots ,\sigma^n)$, so kann man $\dop \sigma = (\dop \sigma^1,\ldots ,\dop\sigma^n)$ als 1-Form auf $M$ mit Werten in $\R^n$ auffassen:
	\begin{align*}
		\dop \sigma(X)_p &= (\dop \sigma^1(X)_p,\ldots , \dop \sigma^n(X)_p)\\
		&= (X(\sigma^1)_p,\ldots ,X(\sigma^n)_p)\\
		&= \calI_p(\sum X(\sigma^{i}) \partial_i),
	\end{align*}
	wobei $\partial_i$ das $i$-te Koordinatenfeld in der Karte $(\Id, \R^n)$ ist. Identifiziert man $E = M \X \R^n$ mit $C^{\infty}(M, \R^n)$, so gilt $\nabla_X S = \dop \sigma(X) \omega(X) \sigma$ (Kurzschreibweise f"ur die zweite Komponente von $S$). Lokal ist \emph{jede} kovariante Ableitung von dieser Form.

\begin{Lemma}
Die kovariante Ableitung $(\nabla_XS)_p$ h"angt nur von den Werten von $X$ und $S$ in einer Umgebung von $p$ ab.
\end{Lemma}

\begin{bew}
Es seien $p \in M$ und $X_1, X_2 \in \calV(M)$ sowie $S_1, S_2 \in \Gamma(E)$ und $U$ eine Umgebung von $p$ mit $X_1|_U = X_2|_U$ und $S_1|_U = S_2|_U$. W"ahle nun ein $\sigma \in C^{\infty}(M)$ mit dem Tr"ager $\supp \sigma \subseteq U$ und $\sigma|_V \equiv 1$ auf einer Umgebung $V$ von $p$. Dann gilt: $\sigma X_1 = \sigma X_2$ und $\sigma S_1 = \sigma S_2$. F"ur $q \in V$ folgt dann:
\begin{align*}
	(\nabla_{\sigma X_i} \sigma S_i)_q &= \sigma(q)(\nabla_{X_i} \sigma S_i)|_q\\
	&= \sigma(q)(\underbrace{X_i(\sigma)|_q}_{=0} S_i + \underbrace{\sigma(q)}_{=1}\nabla_{X_i} S_i|_q)\\
	&= \nabla_{X_i} S_i|_q
\end{align*}
Damit folgt $\nabla_{X_1} S_1 = \nabla_{X_2} S_2$
\end{bew}

\section{Lokale Koordinaten}

Es sei $(\varphi, U)$ eine Karte von $M$ um $p \in M$ und $E|_U \overset{\tau}{\to} U \X \R^n$. Dann ist $s_i(p) = \tau^{-1}(p, e_i)$ eine lokale Basis. Jeder Schnitt $S$ ist also lokal von der Form $S|_U = \sum_i \sigma^{i} s_i$. Somit existieren glatte Funktionen $\Gamma_{ij}^k$, die sogenannten \CmMark{Christoffelsymbole} mit 
	\[ \nabla_{\pdifffrac{}{x^{i}}} s^j = \sum_k \Gamma_{ij}^k s^k. \]
F"ur $S = \sum \sigma^j s_j$ und $X = \sum \xi^{i} \pdifffrac{}{x^{i}}$ folgt dann:
\begin{align*}
	(\nabla_XS)_p &= \sum_{i,j} \xi_p^{i} \nabla_{\pdifffrac{}{x^{i}}} \left(\sigma^j s_j\right)\\
	&= \sum_{i,j} \xi_p^{i} \left(\pdifffrac{\sigma^j}{x^{i}} \cdot s_j(p) \nabla_{\pdifffrac{}{x^{i}}} s_j|_p\right)\\
	&= \sum_{i,j} \xi_p^{i} \left(\pdifffrac[p]{\sigma^j}{x^{i}} s_j(p) + \sigma^j(p) \sum_k \Gamma_{ij}^k(p) s_k(p)\right)\\
	&= \sum_k \Bigg(\underbrace{\sum_i \xi_p^{i} \pdifffrac[p]{\sigma^k}{x^{i}}}_{\mathclap{= X(\sigma^k)|_p = \difffrac[t=0]{}{t} (\sigma^k \circ \gamma) \text{ mit } \dot\gamma(0) = X_p}} + \sum_{i,j} \xi_p^{i} \sigma^j(p) \Gamma_{ij}^k(p)\Bigg) s_k(p)
\end{align*}

\begin{bem}\begin{enumerate}[label=\arabic*),leftmargin=*]
\item
	$X \mapsto (\nabla_XS)_p$ h"angt nur von dem Wert $X_p$ von  $X$ in $p$ ab, Schreibweise $(\nabla_XS)_p = \nabla_{X_p}S$.
\item
	$S \mapsto \nabla_{X_p}S$ h"angt nur von den Werten von $S$ entlang einer Kurve $\gamma$ mit $\dot\gamma(0) = X_p$ ab. Es gilt
		\[ \nabla_XS = \sum_k X(\sigma^k)S_k + \sum_k \sum_j\left(\left(\sum_i \Gamma_{ij}^k \xi^{i}\right) \sigma^j\right) s_k. \]
	Schreibt man $\sigma = (\sigma^1,\ldots ,\sigma^n)$ und fasst $\dop \sigma = (\dop \sigma^1,\ldots ,\dop \sigma^n)$ also lokale 1-Form mit Werten in $\R^n$ auf, so ist f"ur $s=(s_1,\ldots ,s_n)$ $\dop\sigma \cdot s = \sum \dop \sigma^j s^j$ eine lokale 1-Form mit Werten in $E$. Es gilt: $\dop \sigma \cdot s(X) = \D_X \sigma \cdot s$. Analog definiert $\omega(X) = (\omega_j^k(X))_{k,j}$ eine lokale 1-Form mit Werten in den reellen $(n\X n)$-Matrizen. Dann ist 
	\[ \omega \sigma : X \mapsto \omega(X) \sigma = \left( \sum_{i,j} \Gamma_{ij}^k \xi^{i} \sigma^j \right)^k \]
eine lokale 1-Form mit Werten in $\R^n$ und $\omega\sigma \cdot s$ eine lokale 1-Form mit Werten in $E$. Damit gilt
	\[ \nabla_XS = (\dop \sigma(X) + \omega(X) \sigma) \cdot s \]
oder kurz
	\[ \nabla = \dop + \omega. \]
\end{enumerate}\end{bem}

\section{Transformationsverhalten}

asdf

\begin{bem}
asdf
\end{bem}

\begin{Prop}
asdf
\end{Prop}

\begin{Prop}
asdf
\end{Prop}


%%% Local Variables: 
%%% mode: latex
%%% TeX-master: "../skript-diffgeom"
%%% End: 
 %% 
%% Vorlesung <2012-12-14 Fri>, Fortsetzung
%%

\chapter{Geod\"atische und die Exponentialabbildung}

\begin{emptythm}[Heuristik:] Geodätische sind Minimalstellen des Energiefunktionals $\gamma \mapsto E(\gamma) = \int \|\dot\gamma\|^2$. 
Was sind kritische Punkte dieser Abbildung? Für $f \in C^{\infty}(M)$ ist $p$ kritischer Punkt, wenn alle Richtungsableitungen verschwinden, das hei"st $0 = X(f) = \difffrac[t=0]{}{t}(f(c(t)))$.
\end{emptythm}

\begin{center}\begin{tikzpicture}[font=\scriptsize]
%	\draw[step=0.25,gray!15] (-6,-4) grid (6,4); \draw[step=0.5,gray!30] (-6,-4) grid (6,4); \fill (0,0) circle(0.1); %Hilfsgitter
	
	\coordinate (p) at (-2,-1); \coordinate (q) at (2,1); \coordinate (wirbel) at (-0.5,-0.5);
	\coordinate (ctrl1) at (1,0);
	\def\left{0.75}
	\def\right{1.75}
	\fill (p) circle(0.05)node[below]{$p$}; \fill (q) circle(0.05)node[right]{$q$};
	
	\draw[name path=kurve] (p) ..controls(p) and ($(wirbel) - \left*(ctrl1)$).. (wirbel)node[below]{$\gamma(t)$} ..controls($(wirbel) + \right*(ctrl1)$) and (q)..node[below]{$\gamma$} (q);
	\fill (wirbel) circle(0.05);
	
	\coordinate (vec) at (-0.5,1);
	\foreach \shift in {0.2,0.4,...,1}{
		\coordinate (neuwirbel) at ($(wirbel) + \shift*(vec)$);
		\draw[name path=obere kurve] (p) ..controls(p) and ($(neuwirbel) - \left*(ctrl1)$).. (neuwirbel) ..controls($(neuwirbel) + \right*(ctrl1)$) and (q).. (q);
	}
	\draw[->] (wirbel) -- ($(wirbel) + 1.3*(vec)$);
	
	\path[name path=vert] (0,-1) -- (0,1);
	\path[name intersections={of={kurve and vert}}];
	\fill (intersection-1) circle(0.05);
	\draw[->] (intersection-1) -- ($(intersection-1) + (0.5,1)$);
	\path[name intersections={of={obere kurve and vert}}];
	\node[above] at (intersection-1) {$h_s$};
\end{tikzpicture}\end{center}

Eine \quot{Kurve} durch $\gamma$ ist eine sogenannte \CmMark[Variation!glatte]{glatte Variation} $h\colon[0,1]\times[0,1] \to M$, $h(s,t) = h_s(t)$ mit $h_0 = \gamma$ und $h_s(0) = p$, sowie $h_s(1) = q$ f"ur alle $s \in [0,1]$. Dann ist
\begin{align*}
  X(t) = \difffrac[s=0]{}{s}h_s(t)
\end{align*}
ein glattes Vektorfeld entlang $\gamma$.
Ferner gilt $X(0) = 0$ und $X(1) = 0$.
Nun betrachte
\begin{align*}
	0  = \difffrac[s=0]{}{s}E(h_s) &= \int_{0}^{1} \difffrac[s=0]{}{s} \left<\difffrac{}{t} h_s(t),\difffrac{}{t}h_s(t)\right>\\
	& = \int_0^1 2 \left<\nabla_s \difffrac{}{t}h_s(t), \difffrac{}{t}h_s(t)\right>\\
	& = \int_0^1 2 \left<\nabla_t\smash{\underbrace{\difffrac{}{s}h_s(t)}_{=X(t)}}, \difffrac{}{t}h_s(t)\right> \vphantom{\underbrace{\difffrac{}{s}h_s(t)}_{=X(t)}}\\
	& = \int_0^1 2 \left<\nabla_tX,\difffrac{}{t}h_s(t)\right>\\
	& = 2 \int_0^1 \difffrac{}{t}\left<X,\difffrac{}{t}h_s(t)\right> - \left<X,\nabla_t\difffrac{}{t}h_s(t)\right>\\
	& = \underbrace{2 \int_0^1 \difffrac{}{t}\left<X,\difffrac{}{t}h_s(t)\right>}_{=0} - 2 \int_0^1 \left<X,\nabla_t\difffrac{}{t}h_s(t)\right>\\
	& = -2 \int_0^1 \left<X(t),\nabla_t\dot\gamma(t)\right>\dop t
\end{align*}

% Definition 8.1
\begin{Dfn}\label{dfn-8-1}
  Eine glatte Kurve $c$ in $M$ heißt \CmMark{Geod\"atische}\footnote{Die Äquivalenz zur bereits bekannten Definition wird in Kürze gezeigt.}, wenn $\nabla_t\dot c \equiv 0$ gilt.
\end{Dfn}

Ist $c$ Geodätische, so ist $c$ proportional zur Bogenlängenparametrisierung, das hei"st $\|\dot c\| = $const, denn $\difffrac{}{t}\|\dot c(t)\|^{2} = \difffrac{}{t}\left<\dot c(t),\dot c(t) \right> = 2\left<\nabla_t\dot c(t), \dot c(t)\right> = 0$.
Mit $c$ ist auch jede affine Umparametrisierung $t \mapsto c(at + b)$ eine Geodätische.

% Proposition 8.2
\begin{Prop}
Für jedes $p \in M$ und $v \in \T_pM$ existiert genau eine Geodätische $\gamma_{p,v}\colon[0,\epsilon] \to M$ mit $\gamma_{p,v}(0) = p$ und $\dot \gamma_{p,v}(0) = v$.
Zudem hängt $\gamma_{p,v}$ glatt von $p$ und $v$ ab.
\end{Prop}

\begin{bew}\begin{enumerate}[label=(\Alph*),leftmargin=*,widest=B]
\item
	Es sei $(\phi, U)$ eine Karte um $p$, $\gamma^i(t) = \phi^i(\gamma(t))$. Dann besitzt das folgende Anfangswertproblem
	\begin{align*}
		\begin{cases}
			0 = \nabla_t\dot \gamma|_t = \sum_k\left(\ddot \gamma^k(t) + \sum_{ij}\Gamma_{ij}^k\big(\gamma(t)\big)\dot \gamma^i(t)\gamma^j(t)\right) \pdifffrac[\gamma(t)]{}{x^k}\\
			\gamma^i(0) = \phi^i(p)\\
			\dot\gamma^i(0) = \xi_p^i, \quad v = \sum \xi^i_p\pdifffrac[p]{}{x^i}
		\end{cases}
	\end{align*}
	eine eindeutige Lösung (lokal), wleche glatt von den Startwerten $p$ und $v$ abhängt.
\item
	(Alternativ) Ist $(\phi, U)$ eine Karte von $M$ um $p$, dann ist
	\[ \overline \phi \colon \left\{\begin{array}{cccl}
		\T M|_U &\to& \R^{2m}&\\
		X_p=\sum \xi_p^i\pdifffrac[p]{}{x^i} &\mapsto& \overline\phi(X_p) &= (\phi^1(p), \ldots, \phi^m(p), \xi_p^1, \ldots, \xi_p^m)\\
		&&& =: (y^1, \ldots, y^{2m})
	\end{array}\right.\]
	eine Karte von $\T M$.	
	Es sei $S$ das durch
	\[ S \colon \left\{ \begin{array}{ccc}
		\T M &\to& \T\T M\\
		X = \sum \xi^i \pdifffrac{}{x^i} &\mapsto& \sum_i^m \xi^i \pdifffrac{}{y^i} - \sum_{i,j,k=1}^{m} \Gamma_{ij}^k \xi^i\xi^j\pdifffrac{}{y^{m+k}}
	\end{array}\right.\]
	defninierte glatte Vektorfeld auf $\T M$.	
	$g^t$ ist genau dann Integralkurve von $S$ durch $X_p = \sum \xi_p^i\pdifffrac[p]{}{x^i}$, wenn
	\begin{align*}
		\difffrac{}{t}g^t = \dot g^t = S(g^t) \text{ und } g^0 = X_p.
	\end{align*}
	Setzt man $\overline \phi(g^t) = (\gamma^1(t), \ldots, \gamma^m(t),\eta^1(t), \ldots, \eta^m(t))$, so ist dies genau dann der Fall, wenn gilt:
	\begin{align*}
		& (\dot\gamma^1,\ldots, \dot\gamma^m,\dot\eta^1,\ldots, \dot\eta^m) = \left(\eta^1, \ldots, \eta^m, -\sum_{i,j}\Gamma_{ij}^1\eta^i\eta^j, \ldots, -\sum_{i,j}\Gamma_{ij}^m\eta^i\eta^j\right)\\
		& \rightsquigarrow \eta^i = \dot\gamma^i \text{ und } \ddot\gamma = -\sum_{i,j}\Gamma_{ij}^k\dot\gamma^i \dot\gamma^j
	\end{align*}
	und 
	\begin{align*}
		(\gamma^1(0), \ldots, \gamma^m(0), \eta^1(0), \ldots, \eta^m(0) = \overline\phi(X_p) = (\phi^1(p), \ldots, \phi^m(p), \xi_p^1, \ldots, \xi_p^m)
	\end{align*}
	also genau dann, wenn
	\begin{align*}
		\gamma(t) = \overline\phi^{-1}(\gamma^1(t), \ldots, \gamma^m(t))
	\end{align*}
	eine Geodätische durch $p$ mit $\dot \gamma(0) = X_p$ ist.	
	Der maximale Fluss $g^t$ von $S$ heißt \CmMark[Fluss!geod\"atischer]{geod"atischer Fluss}.
	Mit Satz \ref{satz-4-9} folgt die Aussage der Proposition.
\end{enumerate}\end{bew}

\begin{center}\begin{tikzpicture}[font=\scriptsize]
%	\draw[step=0.25,gray!15] (-6,-1) grid (6,5); \draw[step=0.5,gray!30] (-6,-1) grid (6,5); \fill (0,0) circle(0.1); %Hilfsgitter
	
	\def\breite{2.5}
	\def\hoehe{2}
	\def\shift{1}
	\def\vert{2.5}
	\draw (-\breite, \vert) -- (-\breite+\shift, \vert+\hoehe) -- (\breite+\shift, \vert+\hoehe) -- (\breite, \vert) -- cycle;
	\fill (-0.5,3.25) circle(0.05) node[left]{$0_p$};
	\draw[->] (-0.5,3.25) --node[above]{$v$} (0.75,3.75);
	\node at (3.5,3.5) {$\T_pM$};
	
	\coordinate (segel) at (-2.5,-0.75); \node at ($(segel) + (4.75,1.25)$) {$M$};
	\tikzsegel[1.5]{(segel)}
	\coordinate (pkt) at ($0.75*(-0.25,0.5)+(segel3)$);
	\coordinate (ctrl1) at (1,1); \coordinate (ctrl2) at (-1,1); \coordinate (ctrl3) at (0,1); \coordinate (ctrl4) at (-1.25,-0.25);
	\draw[dashed] (segel1) ..controls($(segel1) + 0.5*(ctrl1)$) and ($(pkt) + 0.25*(ctrl2)$).. (pkt) ..controls($(pkt) + 0.5*(ctrl3)$) and ($(segel2) + (ctrl4)$).. (segel2);
	
	\coordinate(pkt1) at ($(segel) + (1.25,0.75)$); \coordinate(pkt2) at ($(segel) + (2.5,1)$); \coordinate(pkt3) at ($(segel) + (3.75,1.75)$);
	\fill(pkt1) circle(0.05)node[anchor=north east,font=\tiny]{$p$}; \fill(pkt2) circle(0.05) node[below,font=\tiny]{$\gamma_v(1)= \exp_p(v)$};
	\coordinate (ctrl1) at (1.5,1); \coordinate (ctrl2) at (-1,-0.25);
	\draw[->](pkt1) --node[above,sloped,font=\tiny]{$\dot\gamma_v(0)=v$} ($(pkt1) + 0.75*(ctrl1)$);
	\draw (pkt1) ..controls($(pkt1) + 0.25*(ctrl1)$) and ($(pkt2) + 0.5*(ctrl2)$).. (pkt2) ..controls($(pkt2) - 0.5*(ctrl2)$) and (pkt3).. (pkt3);
\end{tikzpicture}\end{center}

Für $v \in \T_pM$ sei $\gamma_v(t) = \pi(g^t(v))$ die eindeutige Geodätische mit $\gamma_v(0) = p$ und $\dot \gamma_v(0) = v$.
Ist $\delta \in \R$ und $c(t) = \gamma_v(\delta t)$, so ist $c$ eine Geodätische durch $p$ mit $\dot c(0) = \delta v$, das hei"st $c = \gamma_{\delta v}$, beziehungsweise $\gamma_{\delta v}(t) = \gamma_v(\delta t)$.

Der Definitionsbereich $\mathcal D_S$ des geodätischen Flusses ist eine offene Menge in $\R \X \T_pM$ und somit sind sowohl $\mathcal D = \{v \in \T M \mid (1,v) \in \mathcal D_S\}$, als auch $\mathcal D_p = \mathcal D \cap T_pM$ offen für alle $p \in M$ (in $\T M$, beziehungsweise $\T_pM$). Weiterhin gilt $0_p \in \mathcal D_p$.

% Definition 8.3
\begin{Dfn}
  Die Abbildung $\exp_p\colon\mathcal D_p \to M$, $v \mapsto \gamma_v(1)$ heißt \CmMark{Exponentialabbildung}.
\end{Dfn}


%% 
%% Vorlesung <2012-12-18 Tue>
%%

Es wurde bereits gezeigt, dass $\nabla_t \dot \gamma_v \equiv 0$ ist (Geodätische Differentialgelichung).
Die Exponentialabbildung is nach Satz \ref{satz-4-6} glatt.
Es gilt $\exp_p(0_p) = p$.
Zur Berechnung des Differentiales von $\exp_p$ in $0_p$
\begin{align*}
  \exp_{p*0_p} \colon \T_{0_p}\T_pM \to \T_pM
\end{align*}
identifiziert man $\T_{0_p}\T_pM$ mit $\T_pM$.
Es gilt
\begin{align*}
  \exp_{p*0_p}(v) = \difffrac[t=0]{}{t}\exp_p(tv) = \difffrac[t=0]{}{t}\gamma_{tv}(1) = \difffrac[t=0]{}{t}\gamma_v(t) = \dot \gamma_v(0) = v,
\end{align*}
also $\exp_{p*0_p} = 1 \dop_{\T_pM}$.
Es existiert für alle $p \in M$ eine Umgebung $V$ von $0_p \in \T_pM$ und $U$ von $p$, so dass $\exp_p \colon V \to U$ ein Diffeomorphismus ist.
Wählt man eine Orthonormalbasis $e_1, \ldots, e_m$ von $\T_pM$ und setzt
\begin{align*}
  \psi \colon \T_pM \to \R^m, v = \sum_i b^ie_i \mapsto (b^1, \ldots, b^m),
\end{align*}
so ist $(\psi \circ \exp_p|_U^{-1}, U)$ eine Karte von $M$ um $p$.
Im Allgemeinen ist dies keine Isometrie!

% Definition 8.4
\begin{Dfn}
Diese Karte bezeichnet man als \CmMark[Normalkoordinaten!Riemannsche]{Riemannsche Normalkoordinaten}.
\end{Dfn}

% Proposition 8.5
\begin{Prop}
In Riemannschen Normalkoordinaten gilt für alle $i,j,k \leq m$:
\begin{enumerate}[label=(\roman*)]
\item
	$g_{ij}(0) = \delta_{ij}$
\item
	$\Gamma^k_{ij}(0) = 0$
\item
	$\partial_k g_{ij}(0) = \pdifffrac[0]{g_{ij}}{x^k} = 0$
\end{enumerate}\end{Prop}

Der Beweis sei zur "Ubung "uberlassen.

\section{Polarkoordinaten}

Es ist $\phi = (r, \theta^1, \ldots, \theta^{m-1})$ die Hintereinanderausführung von Riemannschen Normalkoordinaten des $\R^m$.

\begin{center}\begin{tikzpicture}[font=\scriptsize]
%	\draw[step=0.25,gray!15] (-3,-6) grid (9,6); \draw[step=0.5,gray!30] (-3,-6) grid (9,6); \fill (0,0) circle(0.1); %Hilfsgitter
	
	\def\breite{2.5}
	\def\hoehe{2}
	\def\shift{1}
	\def\vert{2.5}
	\draw (-\breite, \vert) -- (-\breite+\shift, \vert+\hoehe) -- (\breite+\shift, \vert+\hoehe) -- (\breite, \vert) -- cycle;
	
	\coordinate (0p) at (-1,3);
	\draw[->] (0p) --node[below]{$e_1$} +(2,0); \draw[->] (0p) --node[left]{$e_2$} +(0.5,1); \draw[->] (0p) -- +(25:2) node[below]{$v$}; \fill (0p) circle(0.05) node[anchor=north east]{$0_p$};
	%\draw[->] ($(0p) + (1.25,0)$) arc (0:25:1.25) node[right]{$\theta$};
	\draw[->] ($(0p) + (1.25,0)$) arc (0:12.5:1.25) node[right]{$\theta$} arc(12.5:25:1.25);
	
	\draw[->] (-0.5,1) to[out=120,in=240]node[right]{$\exp_p^{-1}$} +(0,1.25);
	
	\def\scl{1.5}
	\tikzsegel[\scl]{(-2.5,-1)};
	\coordinate (pkt) at ($(segel3) + 0.5*(-0.25,0.75)$);
	\draw[dashed] (segel1) ..controls($(segel1) + (ctrls6)$) and ($(pkt) + (ctrls5)$).. (pkt) ..controls($(pkt) + (ctrls4)$) and ($(segel2) + (ctrls3)$).. (segel2);
	
	\coordinate (p) at ($(segel) + (1.5,0.75)$); \coordinate (q) at ($(segel) + (3.25,0.75)$); \fill (p) circle(0.05)node[left]{$p$}; \fill (q) circle(0.05)node[above right]{$q = \exp_p(v) = \gamma_v(1)$};
	\draw[->] (p) -- +(1,0)node[below]{$\gamma_v$}; \draw (p) -- (q);
	
	\node (q) at (5,0) {$q= \exp_p(v)$}; \node (v) at (5,3.5) {$v$}; \node[align=flush left,text depth=-18pt,anchor=west] (v2) at (7.4,3.5) {$(\| v \|, \theta)$\\ $= (r,\theta)$ \\ $=\phi(q) \in (0,\epsilon) \X S^1$}; \node[anchor=west] (R) at (7.4,4) {$\R^2$};
	\draw[|->] (q) -- (v); \draw[|->] (v) -- (v2);
	\draw[->] ($(v) + (0,0.5)$) -- (R);
\end{tikzpicture}
\end{center}

Die Umkehrabbildung ist ein Diffeomorphismus
\begin{align*}
  f \colon (0, \epsilon) \times S^{m-1} \to U \subseteq M, \ 
  (t,v) \mapsto \exp_p(tv) = \gamma_v(t).
\end{align*}
Für jedes $v \in S^{m-1}$ ist $t \mapsto f(t,v) = \gamma_v(t)$ eine Geodätische in $M$. Wir bezeichnen solche Geod"atischen im Folgenden als \CmMark[Geod\"atische!radiale]{radiale Geod\"atische}.

% Lemma 8.6
\begin{Lemma}[Gauß-Lemma]
  Jede radiale Geodätische $\gamma_v$ ist orthogonal zu der geodätischen Sphäre
  \begin{align*}
    S_r = \{q \in M \mid \ \exists v \in \T_pM: \|v\| = r \text{ und } q = \exp_p(v) \}.
  \end{align*}
\end{Lemma}

\begin{bew}
Man zeigt das Folgende:
Ist $X$ ein Vektorfeld auf $S^{m-1}$ und bezeichnet man seine Fortsetzung auf $(0,\epsilon) \times S^{m-1}$ \quot{$\subseteq$} $\B_{\epsilon}(0)\setminus\{0\}$ bzw. $\B_{\epsilon}(0_p)\setminus\{0_p\} \subseteq \T_pM$ mit $X_{rv} = X_v$, so ist
\begin{align*}
	Y_q = Y_{f(r,v)} = f_{*(r,v)}(0,X_v) = \exp_{p*}(r X_v)
\end{align*}
orthogonal zu 
\begin{align*}
	\pdifffrac[q]{}{r} = \difffrac[t=r]{}{t}\exp_p(tv) = \dot \gamma_v(r)
\end{align*}

\begin{center}\begin{tikzpicture}[font=\scriptsize,normal/.style={above,sloped, inner sep=0pt,outer sep=0pt,allow upside down}]
%	\draw[step=0.25,gray!15] (-6,-1) grid (6,5); \draw[step=0.5,gray!30] (-6,-1) grid (6,5); \fill (0,0) circle(0.1); %Hilfsgitter
	
	\def\breite{2.75}
	\def\hoehe{2.5}
	\def\shift{1}
	\def\vert{2.25}
	\draw (-\breite, \vert) -- (-\breite+\shift, \vert+\hoehe) -- (\breite+\shift, \vert+\hoehe) -- (\breite, \vert) -- cycle;
	\draw[->] (-2,2.75)node[above]{$0_p$} -- (2.5,2.75); \draw[dashed, name path=strich] (-2,2.75) -- (1.5,4.75); \fill (-2,2.75) circle(0.05);
	
	\foreach \x in {-1.25, -0.25, 0.75}{
		\path[name path=senkrecht] (\x,2) -- (\x,5);
		\path[name intersections={of={strich and senkrecht}},draw];
		\draw[->] (\x,2.75) -- (intersection-1);
		\draw (\x,2.875) to[out=180,in=90] (\x-0.125,2.75); \fill ($(\x,2.75)+(135:0.125/2)$) circle(0.02);
	}
	\node at (0,3.25) {$X_v$}; \node at (1.5,3.75) {$X_{rv} = rX_v$};
	\path (-0.25,2.75)node[below]{$v$}; \path (0.75,2.75)node[below]{$rv$};
	
	\draw[->] (-0.5,2) to[out=240,in=120]node[right]{$\exp_p$} (-0.5,1);
	
	\coordinate (segel) at (-2.5,-1);
	\tikzsegel[1.5]{(segel)}
	
	\coordinate (p) at ($(segel) + (0.75,0.5)$); \coordinate (q) at ($(segel) + (2.75,-0.25)$); \fill (p) circle(0.05) node[anchor=north east]{$p$};
	\draw[->] (p) ..controls($(p) + (0.5,0.25)$) and ($(q) + (-0.5,0.75)$)..node[below]{$\gamma_v$} (q) node[pos=0.4,normal]{\tikz \draw[->] (0,0) -- ++(0,0.7);} node[pos=0.8,normal]{\tikz \draw[->] (0,0) -- ++(0,1.5);};
	\node at ($(segel) + (2,1)$) {$Y_1$}; \node at ($(segel) + (3,1.25)$) {$Y_r$};
\end{tikzpicture}\end{center}

$Y(t) = Y_{\gamma_v(t)}$ als Vektorfeld entlang $\gamma_v$.
Dann gilt:
\begin{align*}
	\difffrac[t=r]{}{t} \left<Y,\pdifffrac{}{r}\right>_{\gamma_v(t)} & = \left<\nabla_tY|_r, \dot \gamma_v(r)\right> + \left<Y(r), \smash{\underbrace{\nabla_t\dot\gamma_v|_r}_{=0}}\right> \vphantom{\underbrace{\gamma_v}_{0}}\\
	& = \left<\nabla_{Y(r)}\dot\gamma_v(r), \dot \gamma_v(r)\right> + \left< \smash{\underbrace{[\dot\gamma_v(r),Y(r)]}_{\mathclap{\begin{subarray}{l}= [f_{*}(\pdifffrac{}{r}),f_{*}(0,X_v)]\\ = f_{*}[\pdifffrac{}{r},X] = 0\end{subarray}}}}, \dot \gamma_v(r) \vphantom{\nabla_{Y(r)}} \right>  \vphantom{\underbrace{\gamma_v(r)}_{\pdifffrac{}{r}} }\\
	&= \frac{1}{2}Y(t) \|\dot\gamma_v\|^2 = 0.
\end{align*}
Ferner gilt
\begin{align*}
	\left<Y(r),\pdifffrac{}{r}\right>_{\gamma_v(r)} = \left<\exp_{p*}(rX_v),\dot\gamma_v(r)\right> \xrightarrow{r \to 0}\left<\exp_{p*}(0_p),v\right> = 0,
\end{align*}
also $\left<Y,\pdifffrac{}{r}\right> \equiv 0$.
\end{bew}

\begin{bem}
  Insbesondere gilt für alle $i \leq m-1$:
  \begin{align*}
    \left<\pdifffrac{}{r}, \pdifffrac{}{\theta^i}\right> = 0.
  \end{align*}
\end{bem}

% Satz 8.7
\begin{Satz}\label{satz-8-7}
Für jedes $p \in M$ existiert ein $\epsilon > 0$, so dass für alle $q \in \B_{\epsilon}(p)$ genau eine minimierende Geodätische von $p$ nach $q$ existiert, das hei"st eine Geodätische $\gamma$ im Sinne der Definition \ref{dfn-8-1} mit $\mathcal L(\gamma) = \dop(p,q)$.
Ist $q \notin \exp_p(\B_{\epsilon}(0_p)) = \B_{\epsilon}(p)$, so existiert ein $q' \in \partial \B_{\epsilon}(p)$ mit
\begin{align*}
	\dop(p,q) = \epsilon + \dop(q',q).
\end{align*}
Ferner, ist $\delta < \epsilon$ und $q \notin \B_\delta(p)$, so existiert ein $q' \in \B_\delta(q)$ mit
\begin{align*}
	\dop(p,q) = \delta + \dop(q',q)
\end{align*}
\end{Satz}

\begin{bew}
Es sei $\epsilon > 0$ so, dass auf $\B_{\epsilon}(p)$ Polorkoordinaten $\phi = (r,\theta^1, \ldots, \theta^{m-1})$ existieren. Sei weiter $c \colon [0,1]$ eine beliebige glatte Kurve von $p$ nach $q$ mit Koordinaten $\phi(c(t)) = (r(t), \theta^1(t), \ldots, \theta^{m-1}(t))$.
\begin{center}\begin{tikzpicture}[font=\scriptsize]
%	\draw[step=0.25,gray!15] (-6,-6) grid (6,6); \draw[step=0.5,gray!30] (-6,-6) grid (6,6); \fill (0,0) circle(0.1); %Hilfsgitter
	
	\coordinate (segel) at (-3,-1);
	\tikzsegel[2]{(segel)};
	\coordinate (p) at ($(segel) + (2.5,1.25)$); \coordinate (q) at ($(segel) + (3.25,0.75)$);
	\coordinate (pkt1) at ($(segel) + (3.25,2)$); \coordinate (pkt2) at ($(segel) + (3.75,1.75)$);
	\coordinate (ctrl1) at (1,0); \coordinate (ctrl2) at (1,1); \coordinate (ctrl3) at (0,1); \coordinate (ctrl4) at (-1,-0.25);

	\fill (p) circle(0.05)node[left]{$p$}; \fill (q) circle(0.05)node[right]{$q$};
	\draw (p) ..controls($(p) + 0.75*(ctrl1)$) and ($(pkt1) - 0.25*(ctrl2)$).. (pkt1) ..controls($(pkt1) + 0.25*(ctrl2)$) and($(pkt2) + 0.25*(ctrl3)$).. (pkt2)node[right]{$c$} ..controls($(pkt2) - 0.5*(ctrl3)$) and($(q) + (ctrl4)$).. (q);
	
	\draw[clip] ($(segel) + (2.825,1)$) ellipse(1 and 0.5);
	\draw[very thick] (p) ..controls($(p) + 0.75*(ctrl1)$) and ($(pkt1) - 0.25*(ctrl2)$).. (pkt1);
\end{tikzpicture}\\
Das Bild von $c$ ist nicht notwendig in $\B_\epsilon(0)$ enthalten
\end{center}
Für $t_0 = \inf\{t \in [0,1] \mid c(t) \notin \B_{\epsilon}(p)\}$ ist $c|_{[0,t_0]}$ eine Kurve zu $\B_{\epsilon}(p)$.
Es gilt
\begin{align*}
	\left\|\pdifffrac[t]{}{r}\right\| = \|\dot\gamma_w(t)\| = \|w\| = 1.
\end{align*}
Aus der Cauchy-Schwarz-Ungleichung folgt
\begin{align*}
	\|\dot c(t)\| & = \|\dot c(t)\|\left\|\pdifffrac[c(t)]{}{r}\right\|\\
	& \geq \left|\left< \dot c(t), \pdifffrac[c(t)]{}{r}\right>\right|\\
	& = \left|\left<\dot r(t) \pdifffrac{}{r} + \sum_{i=1}^{m-1}\dot\theta^i(t)\pdifffrac{}{\theta^i},\pdifffrac[c(t)]{}{r}\right>\right|\\
	& = \left|\left<\dot r(t)\pdifffrac[c(t)]{}{r}, \pdifffrac[c(t)]{}{r}\right>\right|\\
	& = \left|\dot r(t)\right|,
\end{align*}
wobei die Gleichheit genau dann gilt, wenn $\dot c(t)$ und $\pdifffrac[c(t)]{}{r}$ linear abhängig sind.
\begin{align*}
	\mathcal L(c) = \int_0^{t_0}\|\dot c\| + \int_{t_0}^T\|\dot c\| \geq \int_0^{t_0} \left|\left<\dot c,\pdifffrac{}{r}\right>\right| = \int_0^{t_0}|\dot r| = r(t_0)
\end{align*}
Gleichheit gilt genau dann, wenn $\theta^1(t), \ldots, \theta^{m-1}(t)$ konstant sind und $\dot r(t) \geq 0$ gilt, also genau dann, wenn $c$ eine monotone Umparametrisierung von $t \mapsto \exp_p(tv)$ f"ur $v \in S^{m-1}$ ist.

F"ur den zweiten Teil sei $\epsilon$ so, dass Polarkoordinaten $\phi=(r, \theta^1,\ldots ,\theta^{m-1})$ um $p$ existieren. Es sei $q \in \B_\delta(p)$ und $c$ sei eine glatte Kurve von $p$ nach $q$.
\marginnote{\textcolor{red}{BILD}}
F"ur $t_0 = \inf \{ t \in [0,1] | c(t) \notin \B_\delta(p) \}$ gilt dann:
\begin{align*}
	\calL(c) \ge \delta + \dop(c(t_0), q) \ge \delta + \dop(\partial \B_\delta(p), q),
\end{align*}
also $d(p,q) = \inf_c \calL(c) \ge \delta + \dop(\partial \B_\delta(p), q)$. Da $\partial \B_\delta(p)$ kompakt ist, die Abstandsfunktion $\dop(\cdot, q)$ auf $\partial \B_\delta(p)$ ihr Minimum in $q'$ an. Damit gilt
\begin{align*}
	\dop(q',q) &= \dop(\partial \B_\delta(p), q) \qquad \text{und}\\
	\dop(p,q) &= \dop(p,q') + \dop(q',q) = \delta + \dop(q',q)
\end{align*}
somit gilt dann die Behauptung.
\end{bew}

\begin{Kor}\label{kor-8-8}
F"ur alle $p \in M$ existiert ein $\rho > 0$, so dass f"ur alle $q, q' \in \B_\rho(p=$ genau eine minimierende Geod"atische von $q$ nach $q'$ existiert.
\end{Kor}

\begin{bew}
F"ur $q \in M$ existiert ein $\rho = \rho(q) > 0$, so dass $\exp$ auf $\B_\rho(q)$ ein Diffeomorphismus ist. Da $\exp: \calD \to \calD$ glatt und $\calD$ offen ist, existiert eine Umgebung $U_q$ von $q$, so dass $\exp_p: \B_{\frac{\rho}{2}}(0_q) \to \B_{\frac{\rho}{2}}(q')$ ein Diffeomorphismus ist f"ur alle $q' \in U_q$. F"ur $p \in ;$ existiert nach Satz \ref{satz-8-7} ein $\epsilon > 0$, so dass $\overline \B_\epsilon(p)$ kompakt ist. Die "Uberdeckung $\bigcup_{q \in \overline \B_\epsilon(q)} \B_{\frac{\rho(q)}{2}}(q)$ besitzt eine endliche Teil"uberdeckung. F"ur $\rho = \min_{i \le k} \{ \frac{\rho(q_i)}{4} \}$ existieren auf jedem $\B_{2\rho}(q)$, $q \in \B_\epsilon(p)$, Polarkoordinaten; insbesondere existiert f"ur $q, q' \in \B_\rho(p)$ eine eindeutige minimierende Geod"atische von $q$ nach $q'$.
\end{bew}

\begin{bem}
Die Geod"atischen im obigen Korollar h"angen stetig von ihren Endpunkten ab.
\end{bem}

\begin{Kor}\label{kor-8-9}
Es seien $p, q \in M$ und $c: [0,1] \to M$ st"uckweise glatte Kurven von $p$ nach $q$, so dass $\calL(c) = \dop(p,q)$. Damit ist $c$ eine unparametrisierte Geod"atische im Sinne von Definition \ref{dfn-8-1}.
\end{Kor}

\begin{bew}
Die Kurve ist lokal l"angenminimierend, denn ist $\overline c$ eine Kurve von $c(s)$ nach $c(t)$ mit $\calL(\overline c) < \calL(c|_{[s,t]})$, so w"are $c|_{[0,s]} \cup \overline c \cup c|_{[t,1]}$ eine Kurve k"urzer als $c$.
\begin{center}\textcolor{red}{[BILD]}\end{center}
Da $c$ kompaktes Bild hat, exisitert ein minimales $\rho > 0$ f"ur alle $c(t)$ wie in Korollar \ref{kor-8-8}. Dann findet man eine Partition $0 = t_0 < t_1 < \ldots < t_k = 1$ mit $\dop(c(t_{i-1}), c(t_{i})) < \frac{\rho}{2}$, so dass $c|_{[t_{i-1}, t_{i+1}]}$ glatt ist.
\begin{center}\textcolor{red}{[BILD]}\end{center}
Dann stimmt $c|_{[t_{i-1}, t_{i+1}]}$ f"ur jedes $i < k$ mit der nach Korollar \ref{kor-8-8} eindeutigen Geod"atischen (bis auf Umparametrisierung) "uberein.
\end{bew}

\begin{Dfn}
Eine Riemannsche Mannigfaltigkeit hei"st \CmMark[vollst\"andig!geo\"atisch]{geod"atisch vollst"andig}, wenn jede Geod"atische auf ganz $\R$ fortgesetzt werden kann.
\end{Dfn}

\begin{Satz}[Satz von Hopf-Rinow]
F"ur eine Riemannsche Mannigfaltigkeit sind die folgenden Aussagen "aquivalent:
\begin{enumerate}[label=(\roman*)]
\item
	$M$ ist ged"atisch vollst"andig
\item
	F"ur alle $p \in M$ gilt $\calD_p = \T_pM$
\item
	Es existiert ein $p \in M$ mit $\calD_p = \T_pM$
\item
	Abgeschlosse und beschr"ankte Mengen sind kompakt
\item
	$M$ ist vollst"andig (als metrischer Raum)
\end{enumerate}
Jede dieser Eigenschaften impliziert, dass je zwei Punkte in $M$ durch eine minimierende Geod"atische verbunden werden k"onnen.
\end{Satz}

\begin{bew}
Man zeigt zun"achst, dass falls (iii) f"ur $p \in M$ gilt, es zu jedem  $q \in M$ eine minimierende Geod"atische von $p$ nach $q$ gibt. Es gelte $\calD_p = \T_pM$ und es sei $q \in M$.
\begin{center}\textcolor{red}{[BILD]}\end{center}
F"ur $\epsilon > 0$ wie in Satz \ref{satz-8-7} ist um $\partial \B_{\frac{\epsilon}{2}}(p)$ kompakt; es sei $\overline q \in \partial \B_{\frac{\epsilon}{2}}(p)$ ein Punkt minimalen Abstandes zu $q$. Dann gilt $\overline q = \exp_p(\frac{\epsilon}{2}v)$ f"ur ein $v \in \T_pM$ mit $\|v\| = 1$.
\begin{description}[font=\normalfont\itshape]
\item[Behauptung:] Dann ist $\gamma_v|_{[0,R]}: t \mapsto \exp_p(tv)$ minimierende Geod"atische nach $q$ f"ur $R = \dop(p,q)$.
\end{description}
Es sei $\calI = \{t \in [0,R] | \dop(\gamma_v(t),q) = R - t\}$. Dann ist $\calI$ nichtleer und abgeschlossen, denn $t \mapsto \dop(\gamma_v(t),q) + t$ ist stetig.
\begin{center}\textcolor{red}{[BILD]}\end{center}
F"ur $t_0 \in \calI$ und $0 < \rho < \epsilon_0$ sei $q' \in \partial \B_\rho(\gamma_v(t_0))$ wie in Satz \ref{satz-8-7} angewandt auf $p_0 = \gamma_v(t_0)$. Dann gilt $\dop(p_0, q) = \rho + \dop(q',q)$ und es folgt:
\begin{align*}
	\dop(p,q') &\ge \dop(p,q) - \dop (q',q)\\
	&= \dop(p,q) - \dop(p_0,q) + \rho\\
	&= R - (R - t) + \rho = t_0 + \rho
\end{align*}
Damit ist die Verkettung von $\gamma_v|_{[0,t_v]}$ und der minimalen Geod"atischen von $p_0$ nach $q'$ nach Korollar \ref{kor-8-9} eine Geod"atische. Aus der Eindeutigkeit von kurzen Geod"atischen folgt, dass diese Zusamensetzung mit $\gamma_v$ "ubereinstimmt. Es gilt also $q' = \gamma_v(t_0 + \rho)$ und mit $\dop(\gamma_v(t_0 + p), q) = \dop(p_0,q) - \rho = R - (t_0 + \rho)$ gilt $t_0 + \rho \in \calI$.

Wir k"onnen nun die einzelnen Implikationen zeigen. Dabei gelten (i) $\Rightarrow$ (ii) $\Rightarrow$ (iii) offensichtlich.
\begin{description}[font=\normalfont]
\item[(iii) $\Rightarrow$ (iv):]
	Es gelte $\calD_p = \T_pM$ und es sei $K \subseteq M$ abgeschlossen und beschr"ankt. Dann existiert $R$ mit $K \subseteq \overline\B_R(p)$. Da $\overline\B_R(0_p)$ kompakt ist, ist auch $K$ kompakt.
\item[(iv) $\Rightarrow$ (v):]
	gilt offensichtlich
\item[(v) $\Rightarrow$ (i):]
	Es sei $c$ eine nach Bogenl"ange parametrisierte Geod"atische mit maximalem Definitionsintervall $\calI$. $\calI$ ist nichtleer und offen. Ist $(t_n)$ eine Folge in $\calI$ mit Grenzwert $t$. Dann ist $q_n = c(t_n)$ wegen
		\[ \dop(c(t_n), c(t_m)) \le |t_m - t_n| \]
	eine Cauchy-Folge und konvergiert somit gegen ein $p \in M$.
	\begin{center}\textcolor{red}{[BILD]}\end{center}
	Es sei $\rho > 0$ wie in Korollar \ref{kor-8-8}. F"ur hinreichend gro"ses $n$ gilt dann $|t_n - t| < \frac{\rho}{2}$. Die nach Bogenl"ange parametrisierte Geod"atische von $q_n = c(t_n)$ mit Startvektor $\dot c(t_n)$ existiert auf $(-\frac{\rho}{2}, \frac{\rho}{2})$, setzt also $c$ bis zum Zeitpunkt $|t_n| + \frac{\rho}{2} > |t|$ fort.
\end{description}
\end{bew}


%%% Local Variables: 
%%% mode: latex
%%% TeX-master: "../skript-diffgeom"
%%% End: 
 \chapter{Jacobifelder}

F"ur $p,q \in M$ sei $\Omega_{pq}$ der Raum aller glatten Kurven $c:[0,1] \to M$ mit $c(0)=p$ und $c(1)=q$.
\begin{center}\textcolor{red}{[BILD]}\end{center}

\begin{Dfn}
Eine \CmMark[Variation]{(glatte) Variation} einer glatten Kurve $c:[a,b] \to M$ ist eine glatte Abbildung
\begin{align*}
	h:(-\epsilon, \epsilon) \X [a,b] \to M && h_s(t) = h(s,t)
\end{align*}
mit $h_0 = c$. Gilt $h(\cdot, a) \equiv c(a)$ und $h(\cdot, b) \equiv c(b)$, so hei"st $h$ eine \CmMark[Variation!mit festen Endpunkten]{Variation mit festen Endpunkten} oder \CmMark[Variation!eigentliche]{eigentliche Variation}. Man schreibt $c_s$ f"ur eine Variation $h$ von $c$.
\end{Dfn}

Ist $c_s$ eine glatte Variation von $c$, so ist
\begin{align*}
	Y(t) &= \difffrac[s=0]{}{s} c_s(t)\\
	&= \difffrac[s=0]{}{s} h(s,t) = h_{*(0,t)}\left(\pdifffrac{}{s}\right)
\end{align*}
ein Vektorfeld entlang $c$. ist $c_s$ eigentlich, so gilt $Y(a) = 0 \in \T_{c(a)}M$ und $Y(b) = 0 \in \T_{c(b)}M$.
Tats"achlich ist jedes Vektorfeld ein solches Variationsfeld einer Variation von $c$: Ist $Y$ ein Vektorfeld entlang $c$, so definiert $h(s,t) = \exp_{c(t)}(s Y(t))$ eine Variation von $c$ und es gilt:
\begin{align*}
	\difffrac[s=0]{}{s} h(s,t) &= \exp_{c(t)*0}(Y(t))\\
	&= \id_{\T_{c(t)}M}(Y(t)) = Y(t).
\end{align*}
Falls $Y$ in den Endpunkten von $c$ verschwindet, so ist die so definierte Variation eigentlich. Bestimme $\difffrac[s=0]{}{s} E(c_s)$ und $\difffrac[s=0]{}{s}\calL(c_s)$:
\begin{align*}
	\frac{1}{2} \difffrac[s=0]{}{s} \langle \dot c_s, \dot c_s \rangle &= \langle \nabla_s \dot c(s), \dot c(s) \rangle\\
		&= \left\langle \nabla_s \difffrac{}{t} c_s, \dot c(s) \right\rangle = \left\langle \nabla_t \difffrac{}{s} c_s, \dot c_s \right\rangle \\
		&= \left\langle \difffrac[s=0]{}{s} c_s, \dot c_s \right\rangle' - \left\langle \difffrac[s=0]{}{s} c_s, \nabla_t \dot c_s \right\rangle \\
		&= \left\langle Y, \dot c \right\rangle' - \left\langle Y, \nabla_t \dot c \right\rangle\\
\end{align*}
\begin{align*}
	\difffrac[s=0]{}{t} \|\dot c_s\| &= \frac{1}{2 \|c_s\|} \difffrac[s=0]{}{s} \langle \dot c_s, \dot c_s \rangle\\
		&= \frac{\langle Y, \dot c \rangle' - \langle Y, \nabla_t \dot c \rangle}{\|\dot c\|}
\end{align*}
Damit folgt:
\begin{align*}
	\difffrac[s=0]{}{s} E(c_s) = \difffrac[s=0]{}{s} \int_a^b \frac{1}{2} \|\dot c_s\| = \left. \langle Y, \dot c \rangle \right|_a^b - \int_a^b \langle Y, \nabla_t \dot c \rangle
\end{align*}
Betrachte $E: \Omega_{pq} \to \R$. Dann ist $c \in \Omega_{pq}$ genau dann eine Geod"atische, wenn $c$ ein kritischer Punkt von $E$ ist, das hei"st $\difffrac[s=0]{}{s}E(c_s) = 0$ f"ur jede eigentliche Veriation von $c$.
Ist $c$ ein kritischer Punkt von $E$, so sei $c_s$ die von $Y = f \nabla_t \dot c$ mit $f(0) = 0 = f(1)$ erzeugte Variation.
Dann ist $c_s$ eigentlich und es gilt
\begin{align*}
	0 = \difffrac[s=0]{}{s} E(c_s) = - \int_a^b f \|\nabla_t \dot c\|^2
\end{align*}
also $\nabla_t \dot c = 0$.
Ist $c$ nach der Bogenl"ange parametrisiert, so gilt
\begin{align*}
	\difffrac[s=0]{}{s} \calL(c_s) = \difffrac[s=0]{}{s} E(c_s)
\end{align*}
Eine kurve $c \in \Omega_{pq}$ ist genau dann ein kritischer Punkt von $\calL$, wenn $c$ eine umparametrisierte Geod"atische ist.


%% 
%% Anhang
%% 

\appendix

%% 
%% Inhalte der Uebung
%% 

% Die Benennung der "section" so aendern, dass "\"Ubung 123 vom " am Anfang steht
% Der Code ist fast genau der vom Anfang der Praeambel, dort steht die Erklaerung
\renewcommand*{\othersectionlevelsformat}[3]{\ifstr{#1}{section}{\"Ubung\ #3\ vom\ }{#3\autodot\enskip}}

% Das Format der "section" in Kopfzeile der rechten Seiten
\renewcommand*{\sectionmarkformat}{\"Ubung \thesection\autodot\ vom\enskip}

\section{24. April 2012}
\setcounter{Aufg}{0} %Damit die Aufgaben jedes Mal bei Aufgabe 1 anfangen
\setcounter{Loes}{0}

\begin{dfn*}[Gra\ss mann-Manningfaltigkeiten]
Sei $k \le n$ und $\Gr_k(\R^n) = \{ V \subseteq \R^n | \ddim V = k\}$
\end{dfn*}

\begin{beh*}
$\Gr_k(\R^n)$ ist eine glatte Manningfaltigkeit.
\end{beh*}

\begin{bem*}
F"ur $k = 1$ ist $\Gr_1(\R^n) = \R \P^n$
\end{bem*}

$X_0 \in \Gr_k(\R^n) \Rightarrow \R^n = X_0 \oplus X_0^\perp$, $X_0 = \mspan\{e_1,\ldots ,e_k\}$ \marginnote{\begin{tikzpicture}
\draw[->] (-1.5, 0) -- (1.5,0) node[below]{$X_0$};
\draw[->] (0,-1.5) -- (0,1.5) node[left]{$X_0^\perp$};
\draw (-1,-1) --(1,1) node[below]{$Y$};
\end{tikzpicture}}

Definiere $U_{X_0} := \{Y \in \Gr_k(\R^n) | Y \cap X_0^\perp = \{0\}\}$. F"ur $Y \in U_{X_0}$ gilt dann: $\pr_{X_0}(Y) = X_0 \Rightarrow \pr_{X_0}$ ist ein Isomorphismus
	\[X_0 \xrightarrow{(\pr_{X_0}|_Y)^{-1}} Y \xrightarrow{\pr_{X_0^\perp}} X_0^\perp \]
Definiere
	\[ \varphi_{X_0}: \left\{\begin{array}{ccl} U_{X_0} &\to& \Hom(\underbrace{X_0, X_0^\perp}_{\cong \R^{k \cdot (n-k)}}) \\
		Y &\mapsto& \pr_{X_0^\perp} \circ (\pr_{X_0}|_Y)^{-1} \end{array}\right.\]
	\[ \varphi_{X_0}^{-1}: \left\{\begin{array}{ccl} \Hom(X_0, X_0^\perp) &\to& U_{X_0} \\
		f &\mapsto& \Graph(f) = \{x + xf | x \in X_0\} \end{array}\right.\]
\emph{Zu zeigen:}\begin{enumerate}
\item
	$\varphi_{X-0}, \varphi_{X_0}^{-1}$ sind beide stetig
\item
	$\varphi_{X-0} \circ \varphi_{X_0}^{-1}$ ist glatt
\item
	$\Gr_k(\R^n)$ ist Hausdorffsch und hat eine abz"ahlbare Basis der Topologie
\end{enumerate}

\textbf{Welche Topologie eigentlich?} Sei $V = \{ (v_1,\ldots, v_k) \in (\R^n)^k | v_1,\ldots, v_k$ linear unabh"angig$\}$ und $\pi: V \to \Gr_k(\R^n), (v_1,\ldots ,v_k) \mapsto \mspan\{v_1,\ldots ,v_k\}$. Topologie auf $\Gr_k(\R^n)$: induziert von der Quotientopologie auf $V\modulo{\sim\pi}$, also
	\[U \subset \Gr_k(\R^n) \text{ offen } \Leftrightarrow \pi^{-1}(U) \text{ offen} \]
$V$ ist offen in $(\R^n)^k$: $V = \widetilde{\ddet}^{-1}(\R^{\left(\begin{smallmatrix}n \\ k\end{smallmatrix}\right)} \setminus \{0\})$ mit $\widetilde{\ddet}(v_1,\ldots,v_k) = (\ddet(k \times k\text{-Untermatrizen}))$

\emph{Zu zeigen:} $\pi^{-1}(U_{X_0})$ offen

$\pi^{-1}(U_{X_0}) = \{(v_1,\ldots ,v_k) \in V |\ \pr_{X_0}|_{\mspan\{v_i\}} \text{ hat vollen Rang}\} = \{(v_1,\ldots ,v_k) \in V|\ \pr_{X_0}(V-i) \text{ sind linear unabh"angig} \} = (\widetilde{\ddet} \circ (\pr_{X_0},\ldots ,\pr_{X_0}))^{-1}(\R^{\left(\begin{smallmatrix}n \\ k\end{smallmatrix}\right)} \textcolor{red}{\setminus \{0\}})$

$\Rightarrow U_{Y_0}$ ist offen.
\begin{description}
\item[zu 2):]\begin{description}[font=\normalfont\itshape]
	\item[Behauptung:] f"ur alle $Y \in U_{X_0}$ gibt es genau eine Basis $(y_1,\ldots ,y_k)$ von $Y$ sodass $\pr_{X_0}(y_i) = x_i$ f"ur eine feste Orthonormalbasis $(x_1,\ldots ,x_k)$ von $X_0$. Bezeichnet $B(Y)$ diese BAsis, so ist $B: U_{X_0} \to V$ stetig
	\item[Beweis:] doajdoai
	\end{description}
\item[zu 3):]

\item[zu 4):]
\end{description}
\section{29. Oktober 2012}
\setcounter{Aufg}{0} %Damit die Aufgaben jedes Mal bei Aufgabe 1 anfangen
\setcounter{Loes}{0}

\begin{Loes}
%asdf
\textcolor{red}{[BILD]}
\begin{enumerate}[label=\alph*),leftmargin=*,widest=a,font=\normalfont]
\item
	$S^n = (S^n \setminus \{N\}) \cup (S^n\setminus \{S\}) \checkmark$
	
	$\varphi, \psi$ Hom"oomorphismen, $\Phi: \{(x^0,\ldots ,x^n) \in \R^n | x^0 < 1\} \to \R^n, x \mapsto \frac{1}{1-x^0}(x^1, \ldots ,x^n) \Rightarrow \Phi$ ist stetig $\Rightarrow  \varphi = \Phi|_{S^n \setminus \{N\}}$ ist stetig. Es ist
		\[ \varphi^{-1}(y) = \frac{1}{1+|y|^2}(\|y\|^2 - 1, 2y)\]
	also ist $\varphi^{-1}$ stetig. Analog f"ur $\psi$:
		\[\varphi \circ \psi^{-1}(y) = \frac{y}{\|y\|^2} = \psi \circ \varphi{-1}(y) \]
	f"ur $y \in \R^n \setminus \{0\}$. Also glatter Kartenwechsel.\marginnote{\textcolor{red}{[BILD]}}
		\[ \varphi_i^\pm: U_i^\pm \to B_1(0) \subset \R^n, x \mapsto (x^0,\ldots ,x^{i-1}, x^{i+1},\ldots ,x^n) \]
		\[ (\varphi_i^\pm)^{-1}: B_1(0) \to U_i^\pm, y \mapsto (y^0,\ldots ,y^{i-1}, \pm (1 - \|y\|^2), \textcolor{red}{y^i},\ldots ,y^{n+1}) \]
	$\varphi_i^\pm \circ (\varphi_j^\pm)^{-1}$ glatt
	
	$\psi \circ (\varphi_j^\pm)^{-1}$ glatt
	
	$\varphi \circ (\varphi_j^\pm)^{-1}$ glatt
	
	$\varphi_i^\pm \circ \varphi$ glatt
	
	$\varphi_i^\pm \circ \psi$ glatt
\item
	asdf
\end{enumerate}
\end{Loes}

\begin{Loes}
$\varphi: \R \to \R, x \mapsto x^3$

\emph{Behauptung:} $\varphi$ induziert eine $C^\infty$-Struktur auf $\R$, die von der Standardstruktur abweicht.

Dazu müssen wir zeigen:\begin{enumerate}[font=\normalfont,label=(\roman*)]
\item
	$\{(\varphi, \R)\}$ ist ein $C^\infty$-Atlas
\item
	$\varphi$ ist nicht vertr"aglich mit $(\Id, \R)$
\end{enumerate}
\emph{Beweis:}\begin{enumerate}[leftmargin=*,widest=ii,font=\normalfont,label=(\roman*)]
\item
	$\varphi$ ist Hom"oomorphismus, da $\varphi$ und $\varphi^{-1}: x \mapsto \sqrt[3]{x}$ stetig sind. Offensichtlich "uberdeckt $\varphi$ ganz $\R$. Der einzige Kartenwechsel $\varphi \circ \varphi^{-1} = \Id_{\R}$ ist glatt.
\item
	Betrachte
		\[ \Id_{\R} \circ \varphi^{-1} = \varphi^{-1}: x \mapsto \sqrt[3]{x} \]
	$\Id_{\R} \circ \varphi^{-1}$ ist in $0$ nicht differenzierbar $\Rightarrow$ (ii) $\checkmark$
\end{enumerate}
\begin{description}[font=\normalfont\itshape]
\item[Behauptung:]
	Die beiden $C^\infty$ Strukturen sind diffeomorph
\item[Beweis:]
	Sei
		\[\begin{array}{cccc} f:&  \overset{\text{von } \Id \text{ induziert}}{(\R, \tau_{\text{std}})} &\to& \overset{\text{von } \varphi \text{ induziert}}{(\R, \tau)} \\
			& x &\mapsto& \sqrt[3]{x} \end{array}\]
	\marginnote{\textcolor{red}{[BILD]}}
	Dann ist $f$ bijektiv. Es gilt f"ur $x \in \R$:
		\[ \varphi \circ f \circ (\Id_{\R})^{-1} (x) = (\sqrt[3]{x})^3 = x \]
	ist glatt. Betrachte nun $f^{-1}$: $\Id_{\R} \circ f^{-1} \circ \varphi^{-1}(x) = (\sqrt[3]{x})^3 = x$ ist glatt. Damit ist $f$ ein Diffeomorphismus.
\end{description}
\end{Loes}

\begin{Loes}
asdf
\end{Loes}

\begin{Loes}
asdf
\end{Loes}
\section{5. November 2012}
\setcounter{Aufg}{0} %Damit die Aufgaben jedes Mal bei Aufgabe 1 anfangen
\setcounter{Loes}{0}

\begin{Loes}
Es sei $\psi: (0, \infty) \times (0, 2\pi) \to \R^2 \setminus (\R_{\ge0} \times \{0\})$, $(r, \vartheta) \mapsto r(\cos \vartheta, \sin \vartheta)$, die Inverse $\varphi = \psi^{-1}$ ist eine Karte von $\R^2$.
\begin{align*}
	\pdifffrac[p]{}{r} &= \pdifffrac{\left(\Id^1 \circ \varphi^{-1}\right)}{r} \left(\varphi(p)\right) \pdifffrac[p]{}{x} + \pdifffrac{\left(\Id^2 \circ \varphi^{-1}\right)}{r} \left(\varphi(p)\right) \pdifffrac[p]{}{y}\\
	&= \pdifffrac{\left(\Id^1 \circ \psi\right)}{r} \left(\varphi(p)\right) \pdifffrac[p]{}{x} + \pdifffrac[p]{}{x} + \pdifffrac{\left(\Id^2 \circ \psi\right)}{r} \left(\varphi(r)\right)^2 \pdifffrac[p]{}{y}\\
	&= \cos\left(\vartheta(p)\right) \pdifffrac[p]{}{x} + \sin \left(\vartheta(p)\right) \pdifffrac[p]{}{y}\\
	&= \frac{1}{r(p)} \left(r(p) \cos\left(\vartheta(p)\right) \pdifffrac[p]{}{x} + r(p) \sin\left(\vartheta(p)\right) \pdifffrac[p]{}{y}\right)\\
	&= \frac{1}{r(p)} \left(\psi^1\left(\varphi(p)\right) \pdifffrac[p]{}{r} + \psi^2\left(\varphi(p)\right) \pdifffrac[p]{}{y}\right)\\
	&= \frac{1}{\|p\|} \left(p^1 \pdifffrac[p]{}{x} + p^2 \pdifffrac[p]{}{y}\right)
\end{align*}
Als Vektorfeld:
	\[ \pdifffrac{}{r} = \frac{1}{\|(x,y)\|} \left( x \pdifffrac{}{x} + y \pdifffrac{}{y} \right) \]
Desweiteren gilt:
\begin{align*}
	\pdifffrac[p]{}{\vartheta} &= \pdifffrac{\left(\Id^1 \circ \varphi^{-1}\right)}{\vartheta} \left( \varphi(p) \right) \pdifffrac[p]{}{x} + \pdifffrac{\left(\Id^2 \circ \varphi^{-1}\right)}{\vartheta} \left( \varphi(p) \right) \pdifffrac[p]{}{x}\\
	&= \ldots = -p_2\pdifffrac[p]{}{x} + p_1 \pdifffrac[p]{}{y}
\end{align*}
Also:
	\[ \pdifffrac{}{\vartheta} = -y \pdifffrac{}{x} + x \pdifffrac{}{y} \]
\begin{center}\begin{tikzpicture}
%\draw[step=0.25,gray!15] (-6,-3) grid (6,3); \draw[step=0.5,gray!30] (-6,-3) grid (6,3); \fill (0,0) circle(0.1); %Hilfsgitter
\draw[->] (-5,0) -- (-1,0); \draw[->] (1,0) -- (5,0); \draw[->] (-3,-2) -- (-3,2); \draw[->] (3,-2) -- (3,2);

\draw[->] (-1.5,0) -- (-1.5,1); \draw[->] (-3,1.5) -- (-4,1.5); \draw[->] (-4.5,0) -- (-4.5,-1); \draw[->] (-3,-1.5) -- (-2,-1.5);
\draw[->] (-1.875,0.875) -- ($(-1.875,0.875)+(-0.75,0.75)$); \draw[->] (-3.875, 1.125) --($(-3.875, 1.125)+(-0.75,-0.75)$);
\draw[->] (-4.125,-0.875) -- ($(-4.125,-0.875)+(0.75,-0.75)$); \draw[->] (-2.125, -1.125) -- ($(-2.125, -1.125)+(0.75,0.75)$);

\draw[->] (-2,0) -- (-2,0.5); \draw[->] (-3,1) -- (-3.5,1); \draw[->] (-4,0) -- (-4,-0.5); \draw[->] (-3,-1) -- (-2.5,-1);
\draw[->] (-2.25,0.5) -- ($(-2.25,0.5)+(-0.35,0.35)$); \draw[->] (-3.5,0.75) -- ($(-3.5,0.75)+(-0.35,-0.35)$);
\draw[->] (-3.75,-0.5) -- ($(-3.75,-0.5)+(0.35,-0.35)$); \draw[->] (-2.5,-0.75) -- ($(-2.5,-0.75)+(0.35,0.35)$);

\draw[->] (-2.5,0) -- (-2.5,0.25); \draw[->] (-3,0.5) -- (-3.25,0.5); \draw[->] (-3.5,0) -- (-3.5,-0.25); \draw[->] (-3,-0.5) -- (-2.75,-0.5);
\draw[->] (-2.625,0.25) -- ($(-2.625,0.25)+(-0.15,0.15)$); \draw[->] (-3.25,0.375) -- ($(-3.25,0.375)+(-0.15,-0.15)$);
\draw[->] (-3.375,-0.25) -- ($(-3.375,-0.25)+(0.15,-0.15)$); \draw[->] (-2.75,-0.375) -- ($(-2.75,-0.375)+(0.15,0.15)$);

\draw[->] (3.25,0.25) -- (3.75,0.75); \draw[->] (2.75,0.25) -- (2.25,0.75); \draw[->] (2.75,-0.25) -- (2.25,-0.75); \draw[->] (3.25,-0.25) -- (3.75,-0.75);
\draw[->] (2.75,0.75) -- (2.75,1.25); \draw[->] (2.25,0.25) -- (1.75,0.25);
\draw[->] (3.25,1) -- (3.5,1.5); \draw[->] (4.25,0.25) -- (4.75,0.5); \draw[->] (4,0.5) -- (4.5,0.875); \draw[->] (3.625,0.875) -- (4,1.375);
\draw[->] (3.625,1.625) -- (4.125,2.125); \draw[->] (4.25,1.5) -- (4.75,2); \draw[->] (4.75,1.25) -- (5.375,1.5);
\end{tikzpicture}\end{center}
\end{Loes}

\begin{Loes}\begin{enumerate}[label=\alph*),widest=b,leftmargin=*]
\item
	Zeige dass $\pi_i: M_1 \times M_2 \to M_i$, $(p_1,p_2) \mapsto p_i$ eine Submersion ist.
	
	Sei $(p_1,p_2) \in M_1 \times M_2$. Seien $\varphi_i$ Karten von $M_i$ um $p_i$ mit Kartengebieten $U_i$. Dann ist $\varphi_1 \times \varphi_2: U_1 \times U_2 \to \varphi_1(U_1) \times \varphi_2(U_2)$ eine Karte von $M_1 \times M_2$ um $(p_1,p_2)$. Es ist
		\[ \varphi_i \circ \pi_i \circ (\varphi_1 \times \varphi_2)^{-1} = \varphi_i \circ \pi_i \circ (\varphi_1^{-1} \times \varphi_2^{-1}). \]
	F"ur $(x_1,x_2) \in \varphi_1(U_1) \times \varphi_2(U_2)$ ist
		\[ \varphi_i \circ \pi_i \circ (\varphi_1 \times \varphi_2)^{-1} (x_1,x_2) = \varphi_i(\pi_i(\varphi_1^{-1}(x_1), \varphi_2^{-1}(x_2)))	= \varphi_i(\varphi_i^{-1}(x_i)) = x_i. \]
	Daraus folgt dass $\varphi_i \circ \pi_i \circ (\varphi_1 \times \varphi_2)^{-1}$ glatt ist. Der Rest des Beweises kann auf zwei Arten erfolgen.
	\begin{description}[font=\normalfont\itshape]
	\item[Variante 1:]
		Es folgt dass $\D(\varphi_i \circ \pi_i \circ (\varphi_1 \times \varphi_2)^{-1}) = \left( \begin{smallmatrix} 1 & & \\ & \ddots & \\ & & 1\\ & 0 &\end{smallmatrix} \right)$
		
		Dies ist die Darstellungsmatrix von $\pi_{1*}$ bez"uglich den Basen $\pdifffrac{}{\varphi_1^1}, \ldots, \pdifffrac{}{\varphi_1^d}$ und $\pdifffrac{}{(\varphi_1 \times \varphi_2)^1}, \ldots, \pdifffrac{}{(\varphi_1 \times \varphi_2)^{\ddim M_1 + \ddim M_2}}$. Also ist $\pi_{1*}$ surjektiv. Auch $\pi_1$ ist surjektiv, also ist $\pi_1$ eine Submersion. Der Beweis f"ur $\pi_2$ folgt analog.
	\item[Variante 2:]
		Sei $X = \sum_{j = 0}^{\ddim M_1} \xi^j \pdifffrac[p_1]{}{\varphi_1^j} \in \T_{p_1}M_1$. Setze
			\[ \tilde X = \sum_{j=1}^{\ddim M_1} \xi^j \pdifffrac[p]{}{(\varphi_1 \times \varphi_2)^j} \in \T_p(M_1 \times M_2). \]
		Dann ist
		\begin{align*}
			\pi_{1*_p} \tilde X &= \sum_{k=1}^{\ddim M_1} \left( \sum_j \underbrace{\partial_j \left( \varphi^k \circ \pi_1 \circ (\varphi_1 \times \varphi_2)^{-1} \right)}_{= \delta(k,j)} \left( \varphi_1(p_1), \varphi_2(p_2) \right) \xi^j \right) \pdifffrac[p_1]{}{\varphi_1^k}\\
			&= \sum_{k=1}^{\ddim M_1} \xi^k \pdifffrac[p_1]{}{\varphi_1^k} = X
		\end{align*}
		Daraus folgt dass $\pi_{1*}$ surjektiv ist.
	\end{description}
\item
	Zeige dass $f: (0,2\pi) \to \R^2$, $t \mapsto (\sin(t), \sin(2t))$ eine injektive Immersion aber keine Einbettung ist.
	\begin{description}[font=\normalfont\itshape]
	\item[$f$ ist injektiv:]
		Seien $t_1, t_2 \in (0,2\pi)$ mit $f(t_1) = f(t_2)$. Damit muss auch gelten dass $\sin(t_1) = \sin(t_2)$ und $\sin(2t_1) = \sin(2t_2)$. Aus diesen beiden Bedingungen folgt dass f"ur $t_1, t_2$ gelten muss:
		\begin{itemize}
		\item
			$t_1=t_2$ oder $\frac{\pi}{2}-t_1 = t_2-\frac{\pi}{2}$ oder $\frac{3\pi}{2}-t_1 = t_2 - \frac{3\pi}{2}$
		\item
			$2t_1=2t_2$ oder $\frac{\pi}{2}-2t_1=2t_2-\frac{\pi}{2}$ oder $\frac{3\pi}{2}-2t_1=2t_2-\frac{3\pi}{2}$
		\end{itemize}
		Aus den beiden Bedingungen folgt somit dass $t_1 = t_2$ gilt.
	\item[$f$ ist eine Immersion:]
		Es reicht zu zeigen, dass $f_{*t} = 0$ f"ur alle $t$ gilt. Es gilt $\D f(t) = (\cos(t), 2 \cos(2t))$, also
		\begin{align*}
			\D f(t) = 0 & \Leftrightarrow \cos(t) = 0 \text{ und } \cos(2t)=0\\
			& \Leftrightarrow \left(t = \frac{\pi}{2} \vee t = \frac{3\pi}{2} \right) \bigvee \left( 2t = \frac{\pi}{2} \vee 2t = \frac{3\pi}{2} \vee 2t = \frac{5\pi}{2} \vee 2t = \frac{7\pi}{2}\right)
		\end{align*}
		Das ist aber nicht m"oglich, also ist $\D f(t) \ne 0$. $\D f(t)$ ist die Darstellungsmatrix, also ist auch $f_{*_t} \ne 0$.
	\item[$f$ ist keine Einbettung:]
		\marginnote{\emph{Skizze:}\\
			\begin{tikzpicture}[scale=0.25,baseline=0]
				%\draw[step=0.25,gray!15] (-6,-3) grid (6,3); \draw[step=0.5,gray!30] (-6,-3) grid (6,3); \fill (0,0) circle(0.1); %Hilfsgitter
				\draw[->] (-3,0) -- (3,0); \draw[->] (0,-3) -- (0,3);
				\def\streck{4}		
				\draw (0,0) ..controls(0,0) and (2.5,\streck).. (2.5,2.5) -- (2.5,-2.5) ..controls(2.5,-\streck) and (0,0).. (0,0) ..controls(0,0) and (-2.5,\streck)..  (-2.5,2.5) -- (-2.5,-2.5) ..controls(-2.5,-\streck) and (0,0).. (0,0) -- cycle;
			\end{tikzpicture}}
		Es gilt dass $\left( \frac{1}{k} \right)_{k \in \N}$ nicht in $(0,2\pi)$ konvergiert, aber es ist $f\left( \frac{1}{k} \right) \to (0,0) = f(\pi) \in \Bild f$. Damit ist $f$ kein Hom"oomorphismus auf das Bild.
	\end{description}
\end{enumerate}\end{Loes}

\setcounter{Loes}{3}
\begin{Loes}
asdf
\end{Loes}
%%
%% Skript Differentialgeometrie im Wintersemester 12/13
%% Zur Vorlesung von Dr. Grensing am KIT Karlsruhe
%%
%% Uebung 3
%%

\section{12. November 2012}
\setcounter{Aufg}{0} %Damit die Aufgaben jedes Mal bei Aufgabe 1 anfangen
\setcounter{Loes}{0}

\begin{Aufg}
Es sei $M$ eine glatte Mannigfaltigkeit. Zeigen Sie:
\begin{enumerate}[label=\alph*),leftmargin=*,widest=b]
\item
	Die kanonische Projektion $\pi:\mathrm{T}M \to M$ ist ein Submersion.
\item
	Der Nullschnitt $\sigma:M\to \mathrm{T}M$, $p\mapsto 0 \in \mathrm{T}_p M$ ist eine Einbettung.
\item
	Ist $N$ eine weitere glatte Mannigfaltigkeit und $\Phi:M \to N$ glatt, so ist $\Phi_*:\mathrm{T}M \to \mathrm{T}N$ glatt.
\end{enumerate}\end{Aufg}

\begin{Aufg}
Es sei $M$ eine glatte $n$-dimensionale Mannigfaltigkeit und $X,Y \in \mathcal{V}(M)$.
\begin{enumerate}[label=\alph*),leftmargin=*,widest=b]
\item
	Zeigen Sie, dass die Lieklammer im Allgemeinen nicht $C^{\infty}(M)$-bilinear ist.
\item
	Zeigen Sie, dass $XY$ mit $XY(p)(f):=X_p(Y(f))$ für $p \in M$ und $f\in C^\infty(M)$ im Allgemeinen kein Vektorfeld ist.
\end{enumerate}
Es sei ferner $(\phi,U)$  eine Karte von $M$ und $X|_U=\sum\limits_{i=1}^n \xi^i \pdifffrac{}{x^i}$, $Y|_U=\sum\limits_{i=1}^n \eta^i \pdifffrac{}{x^i}$, sowie $[X,Y]|_U=\sum\limits_{i=1}^n \zeta^i \pdifffrac{}{x^i}$ die lokalen Darstellungen von $X$, $Y$ und $[X,Y]$ bezüglich $\phi$.
\begin{enumerate}[label=\alph*),leftmargin=*,widest=b]
\item[c)]
	Zeigen Sie, dass gilt: 
		\[\zeta^j=\sum_{i=1}^n \left(\xi^i \frac{\partial \eta^j}{\partial x^i} - \eta^i \frac{\partial \xi^j}{\partial x^i}\right).\]
\end{enumerate}\end{Aufg}

\begin{Aufg}\begin{enumerate}[label=\alph*),leftmargin=*,widest=b]
\item
	Es seien auf $\R^2$ die beiden Vektorfelder $X=-y \pdifffrac{}{x}+x \pdifffrac{}{y}$ und $Y=-2 y \pdifffrac{}{x}+\tfrac{1}{2} x \pdifffrac{}{y}$ gegeben. Skizzieren Sie die Vektorfelder und bestimmen Sie die Flüsse von $X$ und $Y$.
\item
	Auf dem Torus $\mathrm{T}^2=\mathrm{S}^1 \times \mathrm{S}^1=\{(e^{i \theta^1}, e^{i \theta^2})\in \C^2 | \theta^1, \theta^2 \in \R\}$ betrachten wir für $k \in \N$  das Vektorfeld $X_k=\pdifffrac{}{\theta^1}+ \tfrac{1}{k}\pdifffrac{}{\theta^2}$. Bestimmen Sie die Integralkurve von $X_k$ durch den Punkt $(1,1) \in \mathrm{T}^2$.
\end{enumerate}\end{Aufg}

\begin{Loes}
Sei $M$ eine glatte Mannigfaltigkeit. Sei desweiteren f"ur alle drei Teilaufgaben $p \in M$, $\phi$ eine Karte um $p$ mit Kartengebiet $U$ und
	\[ \overline \phi: \left\{ \begin{array}{ccc}  \T M|_U &\to& \R^{2n} \\
		\sum_i \xi^i\pdifffrac[q]{}{x^i} &\mapsto& (\phi(q), \xi) \end{array} \right.\]
eine Karte von $\T M$. Alle diese Karten bilden dann einen Atlas von $\T M$.
\begin{enumerate}[label=\alph*),widest=a,leftmargin=*]
\item
	\emph{Zeige:} $\pi: \T M \to M$, $\T_p M \ni x \mapsto p$ ist eine Submersion.
	
	Es ist
	\begin{align*} \phi \circ \pi \circ \overline \phi^{-1} \underbrace{(y, \xi)}_{\mathclap{\in \phi(U) \X \R^n}} &= \phi\left(\pi\left(\sum_i \xi^i \pdifffrac[\phi^{-1}(y)]{}{x^i}\right)\right)\\
		&= \phi(\phi^{-1}(y)) = y
	\end{align*}
	also ist $\pi$ glatt. Desweiteren ist
		\[ \D(\phi \circ \pi \circ \overline \phi^{-1})|_{(y, \xi)} = \left(\begin{smallmatrix}
        1 &  & \\
        & \ddots & & \\
         & & 1
      \end{smallmatrix} 0\right) \]
     surjektiv und damit auch $\pi_{*\overline\phi^{-1}(y, \xi)}$. Offensichtlich ist $\pi$ surjektiv und damit eine Submersion.
\item
	\emph{Zeige:} $\sigma: M \to \T M$, $p \mapsto 0_{\T_pM}$ ist eine Einbettung.
	
	Es gilt
		\[\overline \phi \circ \sigma \circ \phi^{-1} (y) = \overline \phi(0_{\T_{\phi^{-1}(y)}M}) = (\phi(\phi^{-1}(y)),0) = (y,0) \]
	Daraus folgt folgt dass $D(\overline \phi \circ \sigma \circ \phi^{-1})|_y = \left( \begin{smallmatrix} 1 & &  \\ & \ddots & \\ & & 1 \\ & 0 & \end{smallmatrix} \right)$ injektiv ist und damit auch $\sigma_{* \phi^{-1}(y)}$. $\sigma$ ist injektiv und stetig, $\pi \circ \sigma = \Id_m$, also ist $\sigma^{-1} = \pi|_{\Bild(\sigma)}$ stetig.
\item
	\emph{Zeige:} Ist $\Phi: M \to N$ eine glatte Abbildung, so auch $\Phi_*: \T M \to \T N$.
	
	Sei $\psi$ eine Karte um $\Phi(p)$ und $\overline \psi$ die zugeh"orige Karte von $\T N$.
	\begin{align*}
		(\overline \psi \circ \Phi_* \circ \overline \phi^{-1})\underbrace{(y, \xi)}_{\mathclap{\in \phi(U) \X \R^n}} &= (\overline \psi \circ \Phi_*)\left(\sum \xi^{i} \pdifffrac[\phi^{-1}(y)]{}{x^{i}}\right)\\
		&= \overline \psi\left(\sum_i\left(\sum_j \pdifffrac[\phi^{-1}(y)]{\Phi^{i}}{x^j} \xi^j\right) \pdifffrac[\Phi(\phi^{-1}(y))]{}{\psi^{i}}\right)\\
		&= \left( \underbrace{(\psi \circ \Phi \circ \phi^{-1})}_{\text{glatt}}(y), \underbrace{\left( \sum_j \underbrace{\pdifffrac[_\phi^{-1}(y)]{\Phi^{i}}{x^j}}_{\text{glatt in }y} \overbrace{\xi^j}^{\substack{\text{glatt}\\ \text{in }\xi}} \right)_{i = 1,\ldots ,n}}_{\text{glatt}} \right)
	\end{align*}
	Daraus folgt dass $\Phi_*$ glatt ist.
\end{enumerate}\end{Loes}

\begin{Loes}\begin{enumerate}[label=\alph*),leftmargin=*,widest=a]
\item
	\emph{Zu zeigen:} $[\cdot,\cdot]$ ist im Allgemeinen nicht $C^{\infty}(M)$-bilinear.
	
	$M = \R$, $\pdifffrac{}{x} = X = Y$, $f = \Id \quot{= x}$
		\[\begin{array}{rcl} [X,\underbrace{fY}_{=x \pdifffrac{}{x}}] &\overset{\text{c)}}{=}& \left( 1 \underbrace{\pdifffrac{x}{x}}_{=1} - x \underbrace{\pdifffrac{1}{x}}_{=0} \right) \pdifffrac{}{x} = \pdifffrac{}{x} \\
			f[X,Y] &\overset{\text{c)}}{=}& f \left( 1 \underbrace{\pdifffrac{1}{x}}_{=0} - 1 \underbrace{\pdifffrac{1}{x}}_{=0} \right) \pdifffrac{}{x} = 0 \end{array}\]
\item
	\emph{Zu zeigen:} f"ur $X, Y \in \calV(M)$ ist $XY$ mit $(XY)|_p(f) = X_p(Y(f))$ im Allgemeinen keine Derivation.
	\begin{align*}
		(XY)_p(fg) &= X_p(Y(fg))\\
		&= X_p(q \mapsto Y_q(fg))\\
		&= X_p(q \mapsto f(q) Y_q(g) + g(q) Y_q(f))\\
		&= X_p(fY(g) + gY(f))\\
		&= X_p(fY(g)) + X_p(gY(f))\\
		&= f(p) \cdot X_p(Y(g)) + Y_p(g) \cdot X_p(f) + g(p) \cdot X_p(Y(f)) + Y_p(f)X_p(g)\\
		&= f(p) \cdot (XY)|_p(g) + g(p)(XY)|_p(f) + \underbrace{Y_p(g)X_p(f) + Y_p(f)X_p(g)}_{\ne 0}
	\end{align*}
	$M = \R$, $X = Y = \pdifffrac{}{x}$, $f = g = \Id$ $\Rightarrow $ Leibnitz-Regel gilt nicht.
\item
	\emph{Bemerkung:} Ist $X|_U = \sum_{i=1}^{n}\xi^{i}\pdifffrac{}{x^{i}}$ lokale Darstellung bez"uglich $\phi$ von $X \in \calV(M)$, so ist
		\[ \xi^{i} = X(\phi^{i}) \]
	Seien $X|_U = \sum_i \xi^i \pdifffrac{}{x^{i}}$, $Y|_U = \sum_i \eta^{i} \pdifffrac{}{x^{i}}$. Damit gilt dann
	\begin{align*}
		[X,Y](x^j) &= (XY - YX)(x^j)\\
		&= X(Y(x^j)) - Y(X(x^j))\\
		&= X\left(\sum_i \eta^{i}\underbrace{\pdifffrac{}{x^{i}}(x^j)}_{\delta_{ij}}\right) - Y\left(\sum_i \xi^{i}\underbrace{\pdifffrac{}{x^{i}}(x^j)}_{\delta_{ij}}\right)\\
		&= X(\eta^j) - Y(\xi^j)\\
		&= \sum_i \left( \xi^{i} \pdifffrac{}{x^{i}}(\eta^j) - \eta^{i} \pdifffrac{}{x^{i}}(\xi^j) \right)
	\end{align*}
\end{enumerate}\end{Loes}

\begin{Loes}\begin{enumerate}[label=\alph*),widest=a,leftmargin=*]
\item
	Es seien $X = -y\pdifffrac{}{x} + x\pdifffrac{}{y}, Y = -2y\pdifffrac{}{x} + \frac{1}{2}x\pdifffrac{}{y} \in \calV(\R)$, bestimme $\gamma_x^t$ und $\gamma_y^t$.\begin{description}[font=\normalfont\itshape]
	\item[F"ur $X$:]
		$t \mapsto \gamma_*^t(p)$ ist Integralkurve von $X$ mit $\gamma_x^0(p) = p$. \emph{Gesucht:} Kurve mit $\gamma(0) = p$, $\gamma_{*t}\pdifffrac{}{t} = X(\gamma(t)) \Leftrightarrow \gamma_{*t}\pdifffrac{}{t}(x) = X(\gamma(t))(x)$ und $\gamma_{*t}\pdifffrac{}{t}(y) = X(\gamma(t))(y) \Leftrightarrow \gamma_1'(t) = - \gamma_2(t)$ und $\gamma_2'(t) = \gamma_1(t)$. Das Anfangswertproblem
			\[ \begin{pmatrix} \gamma_1 \\ \gamma_2 \end{pmatrix}' (t) = \begin{pmatrix} 0 & -1 \\ 1 & 0 \end{pmatrix} \begin{pmatrix} \gamma_1(t) \\ \gamma_2(t) \end{pmatrix} \text{ und } \gamma(0) = p \]
		hat als L"osung $t \mapsto \exp(t \left( \begin{smallmatrix} 0 & -1 \\ 1 & 0 \end{smallmatrix} \right) ) \cdot p$. Es gilt:
			\begin{align*}
				\begin{pmatrix} 0 & -1 \\ 1 & 0 \end{pmatrix}^{2n} = \begin{pmatrix} (-1)^n & 0 \\ 0 & (-1)^n \end{pmatrix} &&\text{und}&&
				\begin{pmatrix} 0 & -1 \\ 1 & 0 \end{pmatrix}^{2n+1} = (-1)^n \begin{pmatrix} 0 & -1 \\ 1 & 0 \end{pmatrix}
			\end{align*}
		Damit gilt:
		\begin{align*}
			\exp\left(t \begin{pmatrix} 0 & -1 \\ 1 & 0 \end{pmatrix} \right) &= \sum_{k=0}^{\infty} \frac{1}{k!} t^k (\ldots)^k\\
			&= \begin{pmatrix} \sum\limits_{k=0}^{\infty} \frac{(-1)^k}{2k!} t^{2k} & -\sum\limits_{k=0}^{\infty} \frac{t^{2k+1}}{(2k+1)!} (-1)^{k} \\
				-\sum\limits_{k=0}^{\infty} \frac{t^{2k+1}}{(2k+1)!} (-1)^{k} & \sum\limits_{k=0}^{\infty} \frac{(-1)^k}{2k!} t^{2k} \end{pmatrix} \\
			&= \begin{pmatrix} \cos(t) & -\sin(t) \\ \sin(t) & \cos(t) \end{pmatrix}
		\end{align*}
	Daraus folgt $\gamma(t) = \left( \begin{smallmatrix} \cos(t) & -\sin(t) \\ \sin(t) & \cos(t) \end{smallmatrix} \right) \cdot p = \gamma_x^t(p)$
	\item[F"ur $Y$:]
		$\gamma_{*t}\pdifffrac{}{t} = Y(\gamma(t))$, $\gamma(0) = p \xLeftrightarrow{\text{analog}} \gamma'(t) = \left( \begin{smallmatrix} 0 & -2 \\ \frac{1}{2} & 0 \end{smallmatrix} \right) \gamma(t)$
			\[ \begin{pmatrix} 0 & -2 \\ \frac{1}{2} & 0 \end{pmatrix}^{2n} = \begin{pmatrix} (-1)^n & 0 \\ 0 & (-1)^n \end{pmatrix} \]
		Daraus folgt $\underbrace{\gamma(t)}_{\mathclap{= \gamma_x^t(p)}} = \exp(t \left( \begin{smallmatrix} 0 & -2 \\ \frac{1}{2} & 0 \end{smallmatrix} \right) )\cdot p = \left( \begin{smallmatrix} \cos(t) & -2\sin(t) \\ \frac{1}{2}\sin(t) & \cos(t) \end{smallmatrix} \right) \cdot p$
	\end{description}
\item
	asdf
\end{enumerate}\end{Loes}
%%
%% Skript Differentialgeometrie im Wintersemester 12/13
%% Zur Vorlesung von Dr. Grensing am KIT Karlsruhe
%%
%% Uebung 4
%%

\section{19. November 2012}
\setcounter{Aufg}{0} %Damit die Aufgaben jedes Mal bei Aufgabe 1 anfangen
\setcounter{Loes}{0}

\begin{Loes}
asdf
\end{Loes}

\begin{Loes}
Sei $\{U_\alpha | \alpha \in I\}$ eine offene "Uberdeckung vom $M$, $g_{\alpha\beta}: U_\alpha \cap U_\beta \to \GL(k,\R)$, $g_{\alpha\gamma}(p) = g_{\alpha\beta}(p) \cdot g_{\beta\gamma}(p)$ f"ur alle $ \in U_\alpha \cap U_\beta \cap U_\gamma$. Sei $E := \dot\bigcup_{\alpha \in I} (U_\alpha \X \R^k)_{/\sim}$, wobei f"ur $(p,v)_\alpha \in U_\alpha \X \R^k$, $(q,w)_\beta \in U_\beta \X \R^k$ gilt $(p,v)_\alpha \sim (q,w)_\beta \Leftrightarrow p=q$ und $v = g_{\alpha\beta}(p) \cdot w$.

\emph{Behauptung:} $\pi: E \to M$, $[p,v] \mapsto p$ ist ein Vektorb"undel.

\begin{description}[leftmargin=*]
\item[\quot{$\bm{\sim}$} ist "Aquivalenzrelation:]\begin{itemize}[leftmargin=*]
	\item
		$(p,v)_\alpha \sim (p,v)_\alpha$ gilt: $g_{\alpha\alpha}(p) = \Id v$ ($g_{\alpha\alpha}(p) = \underbrace{g_{\alpha\alpha}(p)}_{\mathclap{\in \GL(k,\R)}} \cdot g_{\alpha\alpha}(p)$)
	\item
		$(p,v)_\alpha \sim (q,w)_\beta \Rightarrow (q,w)_\beta \sim (p,v)_\alpha$ gilt: $p=q$, $v=g_{\alpha\beta}(p)w$ $\Rightarrow w = (g_{\alpha\beta}(p))^{-1} v = g_{\beta\alpha} v$ ($g_{\alpha\alpha}(p) = g_{\alpha\beta}(p) g_{\beta\alpha}(p)$)
	\item
		Transitivit"at folgt aus $g_{\alpha\gamma} = g_{\alpha\beta} g_{\beta\gamma}$
	\end{itemize}
\item[$\bm{E_p}$ ist $\bm{k}$-dimensionaler Vektorraum:]
	\[ [(p,v)_\alpha] + \lambda[(p,w)_\alpha] := [(p, v + \lambda w)_\alpha] \]
	\begin{description}[font=\normalfont\itshape,leftmargin=*]
	\item[unabh"angig von $\alpha$:]
		\begin{align*}
			[(p,v)_\beta] + \lambda[(p,w)_\beta] &= [(p,g_{\alpha\beta}(p)v)_\alpha] + \lambda[(p,g_{\alpha\beta}(p)w)_\alpha] \\
				&= [(p,g_{\alpha\beta}(p)v + \lambda g_{\alpha\beta}(p)w)_\alpha] \\
				&= [(p,g_{\alpha\beta}(p) \cdot(v + \lambda w))_\alpha] = [(p,v + \lambda w)_\beta]
		\end{align*}
	\item[$k$-dimensional:]
		$q|_{\{p\} \X \R^k} : \{p\} \X \R^k \to E_p$ ist Vektorraum-Isomorphismus (wobei $q: \dot \bigcup_{\alpha \in I} (U_\alpha \X \R^k) \to E$)
	\end{description}
\item[B"undelkarten (glatt):]
	$\Phi_\alpha: U_\alpha \X \R^k \to E|_{U_\alpha}$, $(p,v) \mapsto [(p,v)_\alpha]$ ist Hom"oomorphismus, da $\sim|_{(U_\alpha \X \R^k) \X (U_\alpha \X \R^k)}$ die triviale "Aquivalenzrelation ist.
	\begin{align*}
		\Phi_\alpha \circ \Phi_\beta^{-1}(p,v) &= \Phi_\alpha([(p,v)_\beta])\\
		&= \Phi_\alpha([(p,g_{\alpha\beta}(p)v)_\alpha])\\
		&= (p, g_{\alpha\beta}(p)v)
	\end{align*}
	$\Rightarrow \Phi_\alpha \circ \Phi_\beta^{-1}$ ist glatt. $\Phi_\alpha|_{E_p}$ ist Vektorraum-Isomorphismus.
\item[\quot{normale} Karten:]
	Sei $\phi$ Karte von $M$ mit Kartengebiet $U \subset U_\alpha$ $\leadsto$ $\overline\phi_\alpha: E|_U \to \phi(U) \X \R^k$, $e \mapsto (\phi(\pi(e)), (\Phi_\alpha)^2(e))$.
	
	Glatte Kartenwechsel $\checkmark$
\item[$\bm{E}$ Hausdorffsch:]
	$[(p,v)_\alpha] \ne [(q,w)_\beta] \in E$
	\begin{description}[font=\normalfont,leftmargin=*]
	\item[$p\ne q$:]
		Die Urbilder in $M$ trennender Umgebungen von $p$ und $q$ unter $\pi$ trennen die Punkte in $E$.
	\item[$p=q$:]
		$v \ne g_{\alpha\beta}(p) w$ $\leadsto$ trennen im $\R^k$ und "uber $\Phi_\alpha$ zur"uckziehen.
	\end{description}
\item[abz"ahlbare basis der Topologie (f"ur $\bm{U_{\alpha}} \bm{\X} \R^{\bm{k}} \bm{\checkmark}$):]
	Es gibt ein $I ' \subseteq I$ mit $I$ abz"ahlbar und $M = \bigcup_{\alpha \in I'} U_\alpha$. Sei $\{V_j | j \in J'\}$ abz"ahlbare Basis der Topologie von $M$. Dann ist mit $J = \{j \in J' | V_j \subset U_\alpha \text{ f"ur ein } \alpha \in I\}$, $\{V_j | j \in J\}$ auch abz"ahlbare Basis der Topologie von $M$, denn $U \underset{\mathclap{\text{offen}}}{\subset} M$ $\Rightarrow$
	\begin{align*}
		U &= \bigcup\limits_{\alpha \in I} (U_\alpha \cap U) \\
		&= \bigcup\limits_{\alpha \in I} \bigcup\limits_{\substack{j \in J' \\ U_j \subset U_\alpha \cap U}} V_j \tag{$U_j \subset U_\alpha \cap U \Rightarrow j \in J$} \\
		&= \bigcup\limits_{\alpha \in I} \bigcup\limits_{\substack{j \in J \\ V_j \subset U_\alpha \cap M}} V_j
	\end{align*}
	F"ur $j \in J$ sei $\alpha(j) \in I$, sodass $V_j \subset U_{\alpha(j)}$. Setze $I' := \{ \alpha(j) | j \in J\}$.
		\[ \bigcup_{\alpha \in I'} U_\alpha = \bigcup_{j \in J} U_{\alpha(j)} \supseteq \bigcup_{j \in J} V_j = M \]
\end{description}\end{Loes}

\begin{Loes}
Es seien $E, E'$ Vektorb"undel "uber $M$ und $F: E \to E'$ sei ein B"undelmorphismus mit dem Isomorphismus $F_p: E_p \to E_p'$ f"ur alle $p \in M$.

\emph{Behauptung:} $F$ ist ein B"undelisomorphismus

$F$ ist surjektiv, denn f"ur $e \in E'$ ist $F_{\pi'(e)}: E_{\pi'(e)} \to E'_{\pi'(e)}$ bereits ein Isomorphismus, also existiert ein Urbild $\tilde e \in E_{\pi'(e)} \subset E$ mit $F(\tilde e) = F_{\pi'(e)}(\tilde e) = e$. Dass $F$ injektiv ist folgt analog, da $\pi' \circ F = \pi$.

Damit existiert ein $G: E' \to E$ mit $G \circ F = \Id$, $F \circ G = \Id$ und $\pi \circ G = \pi'$. Damit gilt auch
	\[ G(e') = (F_{\pi'(e)})^{-1}(e') \]
Es sei nun ein offenes $U  \subseteq M$ mit den Trivialisierungen $E|_U$ und $E'|_U$ gegeben.
\begin{center}\begin{tikzpicture}
	%\draw[step=0.25,gray!15] (-4,-4) grid (4,4); \draw[step=0.5,gray!30] (-4,-4) grid (4,4); \fill (0,0) circle(0.1); %Hilfsgitter
	
	\def\hor{2.5}
	\def\vert{1.25}
	\def\angle{30}
	\node (1) at (-\hor,\vert) {$E|_U$}; \node (2) at (\hor,\vert) {$U \X \R^k$}; \node (3) at (-\hor,-\vert) {$E'|_U$}; \node (4) at (\hor,-\vert) {$U \X \R^k$};
	
	\draw[->] (1) --node[font=\scriptsize,above]{$\Phi$}node[font=\scriptsize,below]{$\cong$} (2);
	\draw[->] (3) --node[font=\scriptsize,above]{$\Phi'$}node[font=\scriptsize,below]{$\cong$} (4);
	
	\draw[->] (1) to[out=270-\angle,in=90+\angle]node[left,font=\scriptsize]{$F$} (3);
	\draw[->] (3) to[out=90-\angle,in=270+\angle]node[right,font=\scriptsize]{$G$} (1);
	
	\draw[->] (2) to[out=270-\angle,in=90+\angle]node[left,font=\scriptsize]{$\Phi' \circ F \circ \Phi^{-1} = \tilde F$} (4);
	\draw[->] (4) to[out=90-\angle,in=270+\angle]node[right,font=\scriptsize]{$\tilde G = \Phi \circ G \circ \Phi'^{-1}$} (2);
\end{tikzpicture}\end{center}
Da $\pi' \circ F = F$ ist, existiert eine Abbildung $f: U \to \GL(k, \R)$ sodass $\tilde F(p,v) = (p, f(p) \cdot v)$ ist. Daraus folgt dass $\tilde G(p,w) = (p, (f(p))^{-1}w)$ glatt ist, da $\cdot^{-1}: \GL(k,\R) \to \GL(k, \R)$ glatt ist (denn $A^{-1} = \frac{1}{\det A}((-1)^{i+j} \det A[i,j])_{i,j}^T$). Damit ist $G$ glatt und somit auch ein B"undelmorphismus.
\end{Loes}

\begin{Loes}\begin{enumerate}[label=(\alph*),leftmargin=*,widest=a]
\item
	\emph{Behauptung:} Es sei $E$ ein Vektorb"undel vom Rang $k$ "uber $M$ auf dem $k$ punktweise linear unabh"angige Schnitte existieren. Dann ist $E$ trivial.
	
	Es seien $\sigma_1, \ldots, \sigma_k: M \to E$ Schnitte, die punktweise linear unabh"angig sind. Dann hat $E_p$ als Basis $\sigma_1(p),\ldots ,\sigma_k(p)$. Definiere nun $F: E \to M \X \R^k$, $\sum_{i=k}^k a_i \sigma_i(p) \mapsto (p, a_1,\ldots ,a_k)$. Es gilt $\pi_1 \circ F = \pi$ und $F_p$ ist ein Isomorphismus. $F$ ist glatt, denn f"ur eine B"undelkarte $\Phi$ ist $\tilde \sigma_i = \Phi^2 \circ \sigma_i$\marginnote{\scriptsize{$\Phi^2$ ist die zweite Komponente}}, das hei"st $\Phi(\sigma(p)) = (p, \tilde\sigma(p))$. Mit $A(p) = (\tilde\sigma_1(p),\ldots ,\tilde\sigma_k(p))^{-1}$ gilt:
	\[ F \circ \Phi^{-1}(p,v) = (p, A(p) \cdot V) \]
Daher ist $F$ glatt und mit Aufgabe 3 folgt, dass $F$ ein B"undelmorphismus ist.
\item
	\emph{Zu zeigen:} $\T S^3 \cong S^3 \X \R^3$
	
	Es gilt:
		\[ \T_pS^3 \cong p^\perp \ni \underbrace{\begin{pmatrix} -p_2 \\ p_1 \\ -p_4 \\ p_3 \end{pmatrix}}_{=:\sigma_1(p)} \cdot \underbrace{\begin{pmatrix} p_3 \\ -p_4 \\ -p_1 \\ p_2 \end{pmatrix}}_{\sigma_2(p)} \cdot \underbrace{\begin{pmatrix} -p_4 \\ -p_3 \\ p_2 \\ p_1 \end{pmatrix}}_{\sigma_3(p)} \]
	Dadurch sehen wir dass $\sigma_i(p) \perp \sigma_j(p)$ f"ur $ i \ne j$, woraus folgt dass $\sigma_1, \ldots ,\sigma_3$ punktweise linear unabh"agige Schnitte sind. Mit (a) folgt dann die Behauptung.
\end{enumerate}\end{Loes}

\begin{emptythm}[Anmerkung]
Der Raum der Schnitte $\Gamma(M,E)$ ist ein $\R$-Vektorraum, also ist lineare Unabh"angigkeit f"ur Schnitte definiert. Linear unabh"angige Schnitte sind im Allgemeinen \emph{nicht} punktweise linear unabh"angig. Betrachte beispielsweise
	\[ \pi_1: \underset{\mathclap{\text{\textcolor{gray}{Mf.}}}}{\R} \X \underset{\mathclap{\text{\textcolor{gray}{VR}}}}{\R} \to \R \]
Dann sind $\sigma_1(t) = (t,1)$ und $\sigma_2(t) = (t,t)$ lineare unabh"angig, aber in jedem Punkt linear abh"angig.
\end{emptythm}
%%
%% Skript Differentialgeometrie im Wintersemester 12/13
%% Zur Vorlesung von Dr. Grensing am KIT Karlsruhe
%%
%% Uebung 5
%%

\section{26. November 2012}
\setcounter{Aufg}{0} %Damit die Aufgaben jedes Mal bei Aufgabe 1 anfangen
\setcounter{Loes}{0}

\begin{emptythm}[Einschub]
F"ur ein Tensorprodukt $V \otimes W$ und ein Element $v_1 \otimes w_1 + v_2 \otimes w_2 \in V \otimes W$ gilt im Allgemeinen
	\[v_1 \otimes w_1 + v_2 \otimes w_2 \ne v_3 \otimes w_3 \]
\emph{Beispiel:} $\left( \begin{smallmatrix} 1 \\ 0 \end{smallmatrix} \right) \otimes \left( \begin{smallmatrix} 1 \\ 0 \end{smallmatrix} \right) + \left( \begin{smallmatrix} 0 \\ 1 \end{smallmatrix} \right) \otimes \left( \begin{smallmatrix} 0 \\ 1 \end{smallmatrix} \right)$
\end{emptythm}

\begin{Loes}\label{exc:ueb5-a1}
Wir zeigen dass die $(r,s)$-Tensorfelder den $C^\infty(M)$-multilinearen Abbildungen entsprechen.
	\[ \underbrace{\calV^*(M) \X \ldots \X \calV^*(M)}_{r\text{-mal}} \X \underbrace{\calV(M) \X \ldots \X \calV(M)}_{s\text{-mal}} \]
Zun"achste zeigen wir die Behauptung punktweise. Sei dazu $p \in M$ und die Abbildung
	\[ F_p: \T_pM \otimes \ldots \otimes \T_pM \otimes \T_p^*M \otimes \ldots \otimes \T_p^*M \to \text{Multilin}_{\R}(\T_p^*M \otimes \ldots \otimes \T_pM) \]
definiert durch
\begin{align*}
	F_p(\sum_i a_i \overbrace{X_1^{i}}^{\mathclap{\in \T_pM}} \otimes \ldots \otimes X_r^{i} \otimes \overbrace{\omega_1^{i}}^{\mathclap{\in \T_p^*M}} \otimes \ldots \otimes \omega_s^{i})(\eta_1,\ldots ,\eta_r, Y_1, \ldots ,Y_s)\\
	\qquad := \sum_i a_i \eta_1(X_1^{i}) \cdot \ldots \cdot \eta_n(X_r^{i}) \cdot \omega_1^{i}(Y_1) \cdot \ldots \cdot \omega_s^{i}(Y_s)
\end{align*}
\begin{description}
\item[$\bm{F_p}$ ist wohldefiniert:]
	\begin{itemize}[leftmargin=*]
		\item
			$F_p(\ldots)$ ist $\R$-multilinear $\checkmark$
		\item
			Sei $Z_1,\ldots ,Z_r$ Basis von $\T_pM$, $\mu_1,\ldots ,\mu_s$ die dazu duale Basis von $\T_p^*M$. Damit ist $\{Z_{i_1} \otimes \ldots \otimes Z_{i_r} \otimes \mu_{j_1} \otimes \ldots \otimes \mu_{j_s} | i_1, \ldots i_r, j_1, \ldots , j_s \in \{1,\ldots ,n\}\}$ eine Basis von $\T_pM \otimes \ldots \otimes \T_p^*M$. Sei $X_k^{i} = \sum_\alpha \chi_{k, \alpha}^{i} Z_\alpha$, $\omega_l^{i} = \sum_\beta w_{l, \beta}^{i} \mu_\beta$, dann folgt
			\begin{align*}
				& \sum_i a_i X_1^{i} \otimes \ldots \otimes \omega_s^{i}\\
				=& \sum_i a_i (\sum_{\alpha_1} \chi_{k, \alpha_1}^{i} Z_{\alpha_1}) \otimes \ldots  \otimes (\sum_{\beta_s} w_{s, \beta_s}^{i} \mu_{\beta_s})\\
				=& \sum_{\mathclap{\substack{\alpha_1,\ldots ,\alpha_r \\ \beta_1,\ldots ,\beta_s}}} \Big( \underbrace{\sum_i a_i \chi_{1,\alpha_1}^{i} \cdot \ldots \cdot \chi_{r,\alpha_r}^{i} \cdot w_{1, \beta_1}^{i} \cdot \ldots w_{s, \beta_s}^{i} }_{= A_{\alpha_1,\ldots , \alpha_r, \beta_1, \ldots ,\beta_s}} \Big) Z_{\alpha_1} \otimes \ldots \otimes Z_{\alpha_r} \otimes \mu_{\beta_1} \otimes \ldots \otimes \mu_{\beta_s}
			\end{align*}
			Damit folgt insgesamt
			\begin{align*}
				F_p\left(\sum a_i X_1^{i} \otimes \ldots \right) &= \sum a_i \eta_1 (X_1^{i}) \cdot \ldots \cdot \omega_s^{i} (Y_s) \\
					& = \ldots \\
					& = \sum_{\mathclap{\alpha_1,\ldots ,\beta_s}} A_{\alpha_1,\ldots ,\beta_s} \eta_1(Z_{\alpha_1}) \ldots \mu_{\beta_s}(Y_s)\\
					& = F_p \left(\sum A_{\alpha_1,\ldots ,\beta_s} Z_{\alpha_1} \otimes \ldots \otimes \mu_{\beta_s}\right)
			\end{align*}
	\end{itemize}
\item[$\bm{F_p}$ ist $\R$-linear]
\item[$\bm{F_p}$ ist surjektiv:]
	Sei $g: \T_p^*M \X \ldots \T_pM \to \R$ eine $\R$-multilineare Abbildung, dann ist $g$ eindeutig bestimmt durch
		\[ \overset{\R \ni}{A_{\alpha_1, \ldots , \alpha_r, \beta_1, \ldots , \beta_s}} = g(\mu_{\alpha_1}, \ldots , \mu_{\alpha_r}, \beta_1, \ldots , \beta_s) \text{, mit } \alpha_1, \ldots , \beta_s \in \{1,\ldots ,n\} \]
	Damit ist dann
	\begin{align*}
		F_p\left(\sum A_{\alpha_1,\ldots ,\beta_s} Z_{\alpha_1} \otimes \ldots \otimes \mu_{\beta_s}\right) (\mu_{\tilde\alpha_1},\ldots ,Z_{\tilde\beta_s}) &= \sum A_{\alpha_1,\ldots ,\beta_s} \underbrace{\mu_{\tilde \alpha_1}(Z_{\alpha_1})}_{\delta_{\tilde \alpha_1 \alpha_1}} \ldots \underbrace{\mu_{\beta_s} (Z_{\tilde \beta_s})}_{\delta_{\beta_s\tilde\beta_s}}\\
		&= A_{\tilde\alpha_1,\ldots ,\tilde\beta_s}\\
		&= g(\mu_{\tilde\alpha_1},\ldots ,Z_{\tilde\beta_s})
	\end{align*}
	Insgesamt folgt
		\[ g = F_p\left(\sum A_{\alpha_1,\ldots ,\beta_s} Z_{\alpha_1} \otimes \ldots  \otimes Z_{\beta_s}\right) \]
\item[$\bm{F_p}$ ist injektiv:]
	Ist $0 = F_p(\sum A_{\alpha_1,\ldots ,\beta_s} Z_{\alpha_1} \otimes \ldots \otimes \mu_{\beta_s})$, so folgt
	\begin{align*}
		0 &= F_p() (\mu_{\tilde\alpha_1},\ldots ,\mu_{\tilde\alpha_r}, Z_{\tilde\beta_1}, \ldots , Z_{\tilde\beta_s})\\
		&= A_{\tilde\alpha_1,\ldots ,\tilde\beta_s} \text{ f"ur alle } \tilde\alpha_1, \ldots , \tilde\beta_s \in \{1,\ldots ,n\}
	\end{align*}
	Daraus folgt $\sum A_{\alpha_1,\ldots ,\beta_s} Z_{\alpha_1} \otimes \ldots \otimes \mu_{\beta_s} = 0$
\end{description}
Insgesamt folgt damit dass $F_p$ ein Isomorphismus von $\R$-Vektorr"aumen ist. Wir definieren nun
	\[ F: \calT_s^r(M) \to \text{Multilin}_{C^\infty(M)}(\underbrace{\calV^*(M) \X \ldots \X \calV^*(M)}_{r\text{-mal}} \X \underbrace{\calV(M) \X \ldots \calV(M)}_{s\text{-mal}}, C^\infty(M)) \]
durch
	\[ F(S)(\overset{\in \calV^*(M) \ni}{\omega_1,\ldots ,\omega_r}, \overset{\in \calV(M) \ni}{X_1,\ldots ,X_s})(p) := F_p(S_p)(\omega_1|_p,\ldots ,\omega_r|_p, X_1|_p,\ldots ,X_s|_p) \]
\begin{description}
\item[$\bm{F(S)(\omega_1,\ldots ,X_s) \in C^\infty(M)}$:]
	lokale Koordinaten $\leadsto \pdifffrac{}{x^{i}} x^{i}$, Koeffizienten von $\omega_1, \ldots ,X_s$ glatt $\Rightarrow F(S)(\omega_1,\ldots ,X_s)$ glatt
\item[$\bm{F(S)}$ ist $\bm{C^\infty(M)}$-multilinear:]
	Seien $f \in C^\infty(M)$ und $\tilde\omega_i \in \calV^*(M)$, damit ist dann:
	\begin{align*}
		F(S)(\omega_1,\ldots ,\omega_i + f \tilde\omega_i, \ldots ,X_s)(p) =& F_p(S_p)(\omega_1|_p,\ldots ,\omega_i|_p + f(p)\tilde\omega_i|_p,\ldots ,X_s|_p)\\
		=& F_p(S_p)(\omega_1|_p,\ldots ,\omega_i|_p,\ldots ,X_s|_p)\\
		 & + f(p)F_p(S_p)(\omega_1|_p,\ldots ,\tilde\omega_i|_p,\ldots ,X_s|_p)\\
		=& F(S)(\omega_1,\ldots ,\omega_i,\ldots ,X_s)(p)\\
		 & + f(p)F(S)(\omega_1,\ldots ,\tilde\omega_i,\ldots ,X_s)(p)
	\end{align*}
\item[$\bm{F}$ ist $\bm{C^\infty(M)}$-linear] $\checkmark$
\item[$\bm{F}$ ist injektiv:]
	$F(S) = 0$, also ist $F_p(S_p) = 0$ f"ur alle $p \in M$. Da $F_p$ injektiv ist, ist $S_p = 0$ f"ur alle $p \in M$ und damit $S = 0$.
\item[$\bm{F}$ ist surjektiv:]
	Sei $g: \calV^*(M) \X \ldots \calV^*(M) \X \calV(M) \X \ldots \X \calV(M) \to C^\infty(M)$ eine $C^\infty(M)$-multilineare Abbildung. Seien weiter $p \in M$, $\phi$ eine Karte um $p$ und $\chi$ eine glatte cut-off Funktion mit $\supp \chi \subset$\marginnote{\scriptsize{$\supp$ \quot{Tr"ager}}} Kartengebiet von $\phi$ und $\chi \equiv 1$ auf einer Umgebung $V$ von $p$. F"ur $q \in V$ ist
	\begin{align*}
		g(\omega_1, \ldots ,X_s)(q) =&\, g(\chi\omega_1 + (1-\chi)\omega_1,\ldots ,\chi X_s + (1 - \chi) X_s)(q)\\
		=&\, g(\chi \omega_1, \chi \omega_2 + (1 - \chi) \omega_2, \ldots \chi X_s + (1-\chi) X_s)(q)\\
		 &\, + \underbrace{(1-\chi)(q)}_{=0} g(\omega_1, \chi \omega_2 + (1-\chi) \omega_2,\ldots )\\
		=&\, g(\chi\omega_1, \chi\omega_2 + (1-\chi)\omega_2,\ldots ,\chi X_s+ (1-\chi) X_s)(q)\\
		=&\, \ldots = g(\chi \omega_1, \ldots ,\chi X_s)(q) \qquad (*)
	\end{align*}
	Sei $S_p := \sum A_{\alpha_1, \ldots, \beta_s}(p) \pdifffrac[p]{}{x^{\alpha_1}} \otimes \ldots \otimes \dop x^{\beta_s}|_p$ mit
		\[ A_{\alpha_1, \ldots, \beta_s}(p) = g(\chi \dop x^{\alpha_1},\ldots ,\chi \pdifffrac{}{x^{\beta_s}}) \]
	Bachrechnen ergibt dass andere Karten das gleiche $S_p$ liefern. Daher ist $S$ auf ganz $M$ definiert. Wegen (*) und der lokalen Darstellung gilt $F(S) = g$.
\end{description}
\end{Loes}

\begin{Loes}\begin{enumerate}[label=\alph*), leftmargin=*]
\item
	$\omega_1 = yz \dop x + xz \dop y + xy \dop z$ ist geschlossen und exakt
		\[ \textcolor{gray}{\left(\dop f = \pdifffrac{f}{x} \dop x + \pdifffrac{f}{y} \dop y + \pdifffrac{f}{z} \dop z \right)} \]
	$\dop(xyz) = (yz) \dop x + (xz) \dop y + (xy) \dop z \Rightarrow 0 = \dop \circ \dop(xyz) = \dop \omega_1$
\item
	$\omega_2 = y^2 \dop x + x^3yz \dop y + x^2y \dop z$ ist weder geschlossen noch exakt.
	
	$\dop \omega_2 \ne 0$ (nachrechnen); angenommen $\exists \eta : \dop \eta = \omega_2 \Rightarrow 0 = \dop^2 \eta = \dop \omega_2 \ne 0 \lightning \Rightarrow \omega_2$ nicht exakt
\item
	$\dop \omega_3 \ne 0$
\item
	$\omega_4$ ist exakt
\end{enumerate}\end{Loes}

\begin{Loes}
Skizze:
\begin{center}\begin{tikzpicture}
	\draw[name path=kreis] (0,0) circle(1);
	\draw[->] (1,0) -- (1,1); \draw[->] (-1,0) -- (-1,-1); \draw[->] (0,1) -- (-1,1); \draw[->] (0,-1) -- (1,-1);
	\path[name path=a] (-1,-1) -- (1,1);
	\path[name path=b] (-1,1) -- (1,-1);
	\path[name intersections={of= kreis and a}];
	\draw[->] (intersection-1) -- ($(intersection-1) + sqrt(0.5)*(-1,1)$); \draw[->] (intersection-2) -- ($(intersection-2) + sqrt(0.5)*(1,-1)$);
	\path[name intersections={of= kreis and b}];
	\draw[->] (intersection-1) -- ($(intersection-1) + sqrt(0.5)*(-1,-1)$); \draw[->] (intersection-2) -- ($(intersection-2) + sqrt(0.5)*(1,1)$);
\end{tikzpicture}\end{center}
\end{Loes}
%%
%% Skript Differentialgeometrie im Wintersemester 12/13
%% Zur Vorlesung von Dr. Grensing am KIT Karlsruhe
%%
%% Uebung 6
%%

\section{3. Dezember 2012}
\setcounter{Aufg}{0} %Damit die Aufgaben jedes Mal bei Aufgabe 1 anfangen
\setcounter{Loes}{0}

\begin{Aufg}
Es sei $(M,g)$ eine zusammenhängende Riemannsche Mannigfaltigkeit. Zeigen Sie:
\begin{enumerate}[label=\alph*),leftmargin=*,widest=b]
\item
	Für je zwei Punkte in $M$ existiert eine stückweise glatte Kurve, die diese verbindet.
\item
	Die Abstandsfunktion 
		\[d(p,q)=\inf\{\mathcal{L}(c)|\text{ $c:[0,1] \to M$ ist stückweise glatt, } c(0)=p, c(1)=q\}\]
	ist eine Metrik, welche die ursprüngliche Topologie erzeugt.
\end{enumerate}\end{Aufg}

\begin{Aufg}
Es sei für $x,y\in \R^{n+1}$ 
	\[\langle x,y \rangle:= - x^0 y^0 + x^1 y^1 +\dots + x^n y^n,\]
sowie 
	\[\mathbb{H}^n:= \{x\in \R^{n+1}| \langle x,x \rangle =-1, x^0 > 0\}.\]
Zeigen Sie, dass $\mathbb{H}^n$ eine glatte Mannigfaltigkeit ist, und $\langle. \,, . \rangle$ für alle $p\in \mathbb{H}^n$ ein Skalarprodukt auf $\T_p \mathbb{H}^n \subset \T_p \R^{n+1}$   definiert und die Gesamtheit dieser Skalarprodukte eine Riemannsche Metrik $g$ auf $\mathbb{H}^n$ ist.

Die Riemannsche Mannigfaltigkeit $(\mathbb{H}^n, g)$ heißt \emph{$n$-dimensionaler hyperbolischer Raum}.
\end{Aufg}

\begin{Aufg}
Es sei $s=(-1,0,\dots,0) \in \R^{n+1}$. Zeigen Sie:
\begin{enumerate}[label=\alph*),leftmargin=*,widest=b]
\item
	Die Abbildung $\phi$ mit 
		\[\phi(x):=s-\frac{2 (x-s)}{\langle x-s,x-s\rangle}, \quad x\in \mathbb{H}^n, \]
	ist ein Diffeomorphismus von $\mathbb{H}^n$ auf $\{\xi \in \R^n \cong \{0\} \times \R^n\subset \R^{n+1}\;|\; \| \xi\|<1\}$.
\item
	In der Karte $\phi$ hat die Riemannsche Metrik auf $\mathbb{H}^n$ die Form \[ \frac{4}{(1-\|\xi \|^2)^2} \sum_{i=1}^nd\xi^i \tensor d\xi^i.\] 
\end{enumerate}\end{Aufg}

\begin{Loes}\begin{enumerate}[label=\alph*),leftmargin=*,widest=b]
\item
	asdf
\item
	asdf
\end{enumerate}\end{Loes}

\begin{Loes}
asdf
\end{Loes}

\begin{Loes}\begin{enumerate}[label=\alph*),leftmargin=*,widest=b]
\item
	asdf
\item
	asdf
\end{enumerate}\end{Loes}
\section{10. Dezember 2012}
\setcounter{Aufg}{0} %Damit die Aufgaben jedes Mal bei Aufgabe 1 anfangen
\setcounter{Loes}{0}

\begin{Loes}
$c: [0,1] \to (S^2, g_{\text{std}})$ k"urzeste $C^1$-Kurve zwischen $c(0) = N = (0,0,1)$ und $c(1)$. \emph{Behauptung:} $\Bild c$ ist im Gro"skreis enthalten.

W"ahle ein geeignetes Intervall der L"ange $2 \pi$, sodass
	\[\begin{array}{cccc} f: & I \X \left( -\frac{\pi}{2}, \frac{\pi}{2} \right) &\to& f \left( I \X \left( -\frac{\pi}{2}, \frac{\pi}{2} \right) \right)\\
		& (\varphi, \vartheta) &\mapsto& (\cos\vartheta \cos\varphi, \cos\vartheta\sin\varphi, \sin\vartheta)\end{array}\]
bijektiv ist und $c(1) \in U$ (falls $c(1) \ne N,S$). Daraus folgt dass $f^{-1}$ eine Karte von $S^2$ ist. Sei nun ohne Einschr"ankung $c$ in keiner Umgebung von $0$ konstant. Weiter sei $\gamma := f^{-1} \circ c$ (eventuell nur in einer Umgebung von $0$ ohne $\{0\}$ definiert). Es bleibt nun zu zeigen, dass $\gamma_1$ konstant ist.

Bestimme $g_{ij}$ bez"uglich $f^{-1}$. Es ist
\begin{align*}
	\partial_1 f = & \pdifffrac{f}{\varphi} = \left( -\cos\vartheta \sin\varphi, \cos\vartheta \cos\varphi, 0 \right)\\
	\partial_2 f = & \pdifffrac{f}{\vartheta} = \left( -\sin\vartheta \cos\varphi, -\sin\vartheta \sin\varphi, \cos\vartheta \right)\\
\end{align*}
Daraus folgt
\begin{align*}
	g_{11} = & \left\langle \partial_1 f, \partial_1 f \right\rangle = \cos^2 \vartheta\\
	g_{22} = & \left\langle \partial_2 f, \partial_2 f \right\rangle = 1\\
	g_{12} = & \left\langle \partial_1 f, \partial_2 f \right\rangle = 0 = g_{21}
\end{align*}
Damit ist dann
\begin{align*}
	L(c|_{(0, \varepsilon)}) &= \int_0^\varepsilon \sqrt{g(\dot c, \dot c)} = \int_0^\varepsilon \sqrt{\vphantom{g(\dot c, \dot c)} \smash{\underbrace{g_{11}(c(t)) \dot\gamma_1(t)^2 + g_{22}(c(t)) \dot\gamma_2(t)^2}_{= (\dot\gamma_1, \dot\gamma_2) \left(\begin{smallmatrix} g_{11} & g_{12} \\ g_{21} & g_{22} \end{smallmatrix}\right) \left(\begin{smallmatrix} \dot\gamma_1 \\ \dot\gamma_2 \end{smallmatrix}\right)}}} \dop t\\
	&= \int_0^\varepsilon \sqrt{ \smash{\underbrace{\cos^2(\gamma_2(t))}_{\ge 0}} \dot\gamma_1(t)^2 + \dot\gamma_2(t)^2} \dop t \rule{0pt}{33pt}\\ % \smash macht dass die Groesse nicht von \sqrt beruecksichtig wird
	&\ge \int_0^\varepsilon \left| \dot\gamma_2(t) \right| \dop t =L(\tilde c) \rule{0pt}{25pt}% \rule fuegt ein wenig vertikalen Abstand nach oben ein
\end{align*}
f"ur $\tilde c(t) = f(\gamma_1(\varepsilon) \gamma_2(t))$. Dann ist $c$ die Kürzeste. Daraus folgt $L(c|_{(0,\varepsilon]}) = L(\tilde c|_{(0,\varepsilon]}$ und somit ist f"ur alle $t \in (0,\varepsilon]$ stets $\cos^2(\gamma_2(t)) \dot\gamma_1(t) = 0$. Da $\cos^2(\gamma_2(t)) > 0$ muss $\dot\gamma_1(t) = 0$ gelten. Da sich $c$ nicht aus $\Bild f$ heraus bewegt, au"ser eventuell f"ur $c(1)$, ist $\dot\gamma_1(t) = 0$ f"ur alle $t \in (0,1)$.
\end{Loes}

\begin{Loes}
asdf
\end{Loes}

\begin{Loes}\begin{enumerate}[label=\alph*),leftmargin=*,widest=b]
\item
	asdf
\item
	asdf
\end{enumerate}\end{Loes}
\section{3. Dezember 2012}
\setcounter{Aufg}{0} %Damit die Aufgaben jedes Mal bei Aufgabe 1 anfangen
\setcounter{Loes}{0}

\begin{description}[font=\normalfont\itshape]
\item[Behauptung:]
	F"ur eine glatte Kurve $c$ und Vektorfelder $X, Y$ l"angs $c$ gilt:
		\[ \difffrac{}{t}g_{c(t)} \left( X(t), Y(t) \right) = g_{c(t)} \left( (\nabla_tX)(t), Y(t) \right) + g_{c(t)} \left( X(t), (\nabla_tY)(t) \right) \]
\item[Beweis:]
	Mit dem Levi-Civita Zusammenhang gilt
		\[ \nabla_t(g \circ c) = \nabla_{c_*\pdifffrac{}{t}} g = 0. \]
	Daraus folgt dann mit $g = \sum g_{ij} \dop x^{i} \otimes \dop x^j$:
	\begin{align*}
		0 =& \left( \nabla_t(g \circ c) \right) \left( X(t), Y(t) \right)\\
		=& \left( \nabla_t \left( (\sum g_{ij} \dop x^{i} \otimes \dop x^j) \circ c \right) \right) \left(X(t), Y(t)\right)\\
		=& \left( \sum \difffrac{}{t} (g_{ij} \circ c) \cdot \dop x^{i}|_{c(t)} \otimes \dop x^j|_{c(t)} + \sum g_{ij} \circ c \cdot \smash{\underbrace{\nabla_t(\dop x^{i}|_{c(t)} \otimes \dop x^j|_{c(t)}}_{\mathclap{\substack{= \nabla_t(\dop x^{i}|_{c(t)} \otimes \dop x^j|_{c(t)} \\ + \dop x^{i}|_{c(t)} \otimes (\nabla_t(\dop x^j|_{c(t)}))}}}} \right) \left( X(t), Y(t) \right) \vphantom{\underbrace{\nabla_t}_{\substack{\nabla_t \\ \nabla_t}}}\\
		=& \sum \difffrac{}{t} \left( g_{ij} \circ c \right) \dop x^{i}|_{c(t)} \otimes \dop x^{j}|_{c(t)} \left( X(t), Y(t) \right)\\
		 & + \sum \left( g_{ij} \circ c \right) \left( (\nabla_t \dop x^{i}|_{c(t)})(X(t)) \cdot \dop x^j|_{c(t)} (Y(t)) + \dop x^{i}|_{c(t)}(X(t)) \cdot (\nabla_t(\dop x^j|_{c(t)}))(Y(t)) \right)\\
		=& \sum \difffrac{}{t} (g_{ij} \circ c) \cdot \dop x^{i}|_{c(t)}(X(t)) \cdot \dop x^j|_{c(t)} (Y(t))\\
		 & + \sum (g_{ij} \circ c) \left( \left( \difffrac{}{t} \dop x^{i}|_{c(t)}(X(t)) - \dop x^{i}|_{c(t)} ( \nabla_t X(t)) \right) \cdot \dop x^j|_{c(t)} Y(t) \right.\\
		 & \hphantom{+ \sum (g_{ij} \circ c) ()} \left. + \dop x^{i}|_{c(t)}(X(t)) \cdot \left( \difffrac{}{t}(\dop x^j|_{c(t)} (Y(t)) - \dop x^j|_{c(t)}(\nabla_t Y(t))) \right) \right)\\
		=& \sum \difffrac{}{t} \left( (g_{ij} \circ c) \cdot \dop x^{i}|_{c(t)} (X(t)) \cdot \dop x^j|_{c(t)} (Y(t)) \right)\\
		 & - \sum (g_{ij} \circ c) \dop x^{i}|_{c(t)} ( \nabla_t X(t)) \cdot \dop x^j|_{c(t)} (Y(t))\\
		 & - \sum (g_{ij} \circ c) \dop x^{i}|_{c(t)} (X(t)) \cdot \dop x^j|_{c(t)} (\nabla_tY(t))\\
		=& \difffrac{}{t} \left( \sum g_{ij}(c(t)) \cdot (\dop x^{i}|_{c(t)} \otimes \dop x^j|_{c(t)}) (X(t), Y(y)) \right)\\
		 & - \sum g_{ij}(c(t)) \cdot \dop x^{i}|_{c(t)} \otimes \dop x^{j}|_{c(t)} (\nabla_tX(t), Y(t))\\
		 & - \sum g_{ij}(c(t)) \cdot \dop x^{i}|_{c(t)} \otimes \dop x^{j}|_{c(t)} (X(t), \nabla_tY(t))\\
		=& \difffrac{}{t}\left( g_{c(t)}\big(X(t), Y(t)\big) \right) - g_{c(t)}\big(\nabla_t X(t), Y(t)\big) - g_{c(t)}\big(X(t), \nabla_t Y(t)\big)
	\end{align*}
	Wir verwenden dabei f"ur $\nabla$ das $\nabla^*$ aus Aufgabe 3 a) von Blatt 7.
\end{description}

\begin{Loes}
asdf
\end{Loes}

\begin{Loes}\begin{enumerate}[label=\alph*),leftmargin=*,widest=b]
\item
	asdf
\item
	asdf
\item
	asdf
\end{enumerate}\end{Loes}

\begin{Loes}
\emph{Behauptung:} F"ur den Levi-Civita Zusammenhang ist die Parallelverschiebung eine Isometrie.

Es sei $c: I \to M$ und $X_{c(0)}, Y_{c(0)} \in \T_{c(0)}M$ mit $X(t) = P_{0,t}^c X_{c(0)}$ und $Y(t) = P_{0,t}^c Y_{c(0)}$. Daraus folgt
	\[ \difffrac{}{t} g\left(X(t), Y(t)\right) = g\left( \smash{\underbrace{\nabla_tX(t)}_{=0}}, Y(t) \right) + g \left( X(t), \smash{\underbrace{\nabla_t Y(t)}_{=0}} \right) = 0\]
und damit gilt dann
	\[ g \left( P_{0,t}^c X_{c(0)}, P_{0,t}^c Y_{c(0)} \right) = g \left( X_{c(0)}, Y_{c(0)} \right) \]
\end{Loes}
%%
%% Skript Differentialgeometrie im Wintersemester 12/13
%% Zur Vorlesung von Dr. Grensing am KIT Karlsruhe
%%
%% Uebung 9
%%

\section{7. Januar 2012}
\setcounter{Aufg}{0} %Damit die Aufgaben jedes Mal bei Aufgabe 1 anfangen
\setcounter{Loes}{0}

\begin{Aufg}
Es sei $(M,g)$ eine Riemannsche Mannigfaltigkeit und $p\in M$. Berechnen Sie $g_{ij}(p)$, $\frac{\partial g_{ij}}{\partial x^k}(p)$  und $\Gamma_{ij}^k(p)$ in Riemannschen Normalkoordinaten um $p$.

{\footnotesize \textbf{Hinweis:} Welche Form haben die Geodätischen durch $p$ in dieser Karte?}
\end{Aufg}

\begin{Aufg}
Bestimmen Sie die Schnittkrümmungen der Riemannschen Mannigfaltigkeiten $(\R^n,g_{\mathrm{eukl}})$ und $\mathbb{H}^n$ (siehe Blatt 6 Aufgaben 2 und 3).
\end{Aufg}

\begin{Aufg}
Bestimmen Sie die Schnittkrümmungen der $n$-Sphäre vom Radius $r>0$, also von $S^n$ mit der von $S^n=\{x \in \R^{n+1} | \|x\|=r\}$ induzierten Riemannschen Metrik.

{\footnotesize \textbf{Hinweis:} Benutzen Sie, dass der Levi-Civita Zusammenhang auf $S^n$  durch $(\nabla_X Y)_p=((\mathrm{D} Y)_p \cdot X_p)^{\T_p S^n}$ gegeben ist.}
\end{Aufg}

\begin{Loes}
Nach der Vorlesung gibt es einen Diffeomorphismus
\begin{align*}
	\exp_p: \underset{\subset \T_pM}{U(0)} \to \underset{\subset M}{U(p)}
\end{align*}
Seien $e_1,\ldots , e_n \in \T_pM$ eine Orthonormalbasis von $\T_pM$ und sei $\phi^{-1}(x_1,\ldots ,x_n) := \exp_p(x_1e_1,\ldots ,x_ne_n)$ definiert.

\textbf{Behauptung:}\begin{enumerate}[label=(\roman*)]
\item
	$g_{ij}(p) = \delta_{ij}$
\item
	$\difffrac{}{x^2}(g_{ij})(p) = 0$
\item
	$\Gamma_{ij}^k(p) = 0$
\end{enumerate}
\textbf{Beweis:}\begin{enumerate}[label=(\roman*),leftmargin=*,widest=iii]
\item
	\begin{align*}
		g_p \left( \smash{ \underset{\substack{=\exp_{p*}(e_i) \\ \equiv \id}}{\pdifffrac[p]{}{x^{i}}}, \underset{= \exp_{p*}(e_j)}{\pdifffrac[p]{}{x^{j}}} } \vphantom{\pdifffrac{}{}} \right) = g_p(e_i, e_j) = \delta_{ij} \vphantom{\underset{\substack{a\\a}}{\pdifffrac{}{}}}
	\end{align*}
\end{enumerate}
$\gamma_v(t) := \exp_p(tv)$ ist die Geod"atische die in $p$ in Richtung $v$ startet.

\emph{Beweis:} Es sei $c_v$ die Geod"atische mit $c_v(0) = p$, $\dot c_v(0) = v$, sowie $c_{\lambda v}$ die Ged"atische mit $c_{\lambda v}(0) = p$, $\dot c_{\lambda v}(0) = \lambda v$, wobei $\lambda \in \R$.
Definiere nun $\gamma(t) := c_v(\lambda t)$. Dann ist $\gamma(t) = c_{\lambda v}(t)$, denn:\begin{itemize}
\item
	$\gamma(0) = c_v(0) = p$
\item
	$\dot \gamma(0) = \lambda \cdot \dot c_v(0) = \lambda v$
\item
	$\ddot \gamma^k(t) + \sum_{i,j=1}^n \Gamma_{ij}^k(\gamma(t)) \dot\gamma^{i}(t) \dot\gamma^{j}(t)$\\
	$= \lambda^2 \left( \ddot c_v^k(\lambda t) + \sum_{ij} \Gamma_{ij}^k(c_v(\lambda t)) \dot c_v^{i}(\lambda t) \dot c_v^{j}(\lambda t) \right)$\\
	$=0$
\end{itemize}
Betrachte also
\begin{align*}
	c_v(t) = c_v(t \cdot 1) = c_{tv}(1) = \exp_p(t \cdot v)
\end{align*}
Die Geod"atischen durch $p$ werden also von den Normalkoordinaten auf die Ursprungsgeraden abgebildet. Setze $\gamma(t) = t \cdot e_{i_0}$, dann ist $\exp_p \circ \gamma$ eine Geod"atische und damit gilt f"ur alle $k$:
\begin{align*}
	0 = \ddot\gamma^k + \sum_{ij} \Gamma_{ij}^k \dot\gamma^{i} \dot\gamma^{j} = 0 + \sum_{ij} \Gamma_{ij}^k \circ \gamma \delta_{ii_0} \delta_{ji_0} = \Gamma_{i_0i_0}^k \circ \gamma
\end{align*}
\textbf{Beweis:}\begin{enumerate}[label=(\roman*),leftmargin=*,widest=iii]
\item[(iii)]
	Nun sei $i_0 \ne j_0$ und $\tilde\gamma(t) = t(e_{i_0} + e_{j_0})$, damit ist $\exp_p \circ \tilde\gamma$ eine Geod"atische und f"ur alle $k$ gilt:
	\begin{align*}
		0 &= \ddot{\tilde\gamma}^k + \sum_{ij} \Gamma_{ij}^k \dot{\tilde\gamma}^{i} \dot{\tilde\gamma}^{j} = 0 + \sum_{ij} \Gamma_{ij}^k (\delta_{ii_0} + \delta{ij_0}) (\delta_{ji_0} + \delta_{jj_0})\\
		&= \left( \Gamma_{i_0i_0}^k + \Gamma_{j_0j_0}^k + \Gamma_{i_0j_0}^k +\Gamma_{j_0i_0}^k \right) \circ \gamma
	\end{align*}
	In $0$ gilt $\gamma(0) = \tilde\gamma(0) = p$, also $\Gamma_{ij}^k(\gamma(0)) = 0$. Daraus folgt dann:
	\begin{align*}
		0 = \Gamma_{i_0j_0}^k (\tilde\gamma(0)) + \Gamma_{j_0i_0}^k (\tilde\gamma(0)) = 2 \Gamma_{i_0j_0}^2(p)
	\end{align*}
	Damit folgt schlie"slich $\Gamma_{ij}^k(p) = 0$ f"ur alle $i, j, k$.
\item[(ii)]
	\begin{align*}
		\pdifffrac[p]{g_{ij}}{x^k} ={}& \pdifffrac[p]{}{x^k} \left( g \left( \pdifffrac{}{x^{i}}, \pdifffrac{}{x^j} \right) \right)\\
		={}& g \left( \nabla_{\pdifffrac[p]{}{x^k}} \pdifffrac{}{x^{i}}, \pdifffrac[p]{}{x^{j}} \right) + g \left( \pdifffrac[p]{}{x^{i}}, \nabla_{\pdifffrac[p]{}{x^j}} \pdifffrac[p]{}{x^{j}} \right)\\
		={}& g \left( \sum_l \Gamma_{ki}^l(p) \cdot \pdifffrac[p]{}{x^l}, \pdifffrac[p]{}{x^j} \right) + g \left( \pdifffrac[p]{}{x^{i}}, \sum_l \smash{\underbrace{\Gamma_{kj}^l(p)}_{=0}} \pdifffrac[p]{}{x^l} \right)\\
		={}& 0
	\end{align*}
\end{enumerate}
\end{Loes}

\begin{Loes}\begin{enumerate}[label=\alph*), widest=b, leftmargin=*]\item
\emph{Behauptung:} $\sec(\R^n, g_{\eukl}) \equiv 0$

Es gilt $\nabla_XY = \D Y \cdot X$ und, wegen Torsionsfreiheit, $[X,Y] = \nabla_XY - \nabla_YX = \D Y \cdot X - \D X \cdot Y$. Nun gilt
\begin{align*}
	R(X,Y)Z &= \nabla_X(\nabla_YZ) - \nabla_Y(\nabla_XZ) - \nabla_{[X,Y]}Z
\end{align*}
Bevor wir fortfahren ben"otigen wir noch eine Nebenrechnung:
\begin{align*}
	&\left( \D(\D Z \cdot Y) \cdot X - \D(\D Z \cdot X) \cdot Y \right)_i\\
	&= \sum_l \left( \D(\D Z \cdot Y) \right)_{il} \cdot X_l - \sum_l \left( \D(\D Z \cdot X) \right)_{il} \cdot Y_l\\
	&= \sum_l \partial_l \left( (\D Z \cdot Y)_i \right) \cdot X_l - \sum_l \partial_l(\D Z \cdot X)_i \cdot Y_l\\
	&= \sum_l \partial_l \left( \sum_m \smash{\underbrace{(\D Z)_{im}}_{=\partial_mZ_i}} \cdot Y_m \right) \cdot X_l - \sum_l \partial_l \left( \sum_m (\D Z)_{im} \cdot X_m \right) \cdot Y_l \vphantom{\underbrace{A}_{A_A}}\\
	&= \sum_{lm} \left( \partial_l (\partial_m Z_i \cdot Y_m) \cdot X_l - \partial_l (\partial_m Z_i \cdot X_m) \cdot Y_l \right)\\
	&= \sum_{lm} ( \partial_l\partial_m Z_i Y_m X_l + \partial_m Z_i \partial_l Y_m X_l - \underbrace{\partial_l\partial_m Z_i}_{= \partial_m\partial_l Z_i} X_m Y_l + \partial_m Z_i \partial_l X_m Y_l )\\
	&= \sum_m \partial_m Z_i \cdot (\D Y \cdot X)_m - \sum_m \partial_m Z_i ( \D X \cdot Y )_m\\
	&= (\D Z \cdot (\D Y \cdot X) - \D Z \cdot \D X \cdot Y)_i\\
	&= (\D Z \cdot (\D Y \cdot X - \D X \cdot Y))_i
\end{align*}
Mit dieser Nebenrechnung folgern wir schlie"slich $R(X,Y)Z = 0$ und daraus folgt letztendlich
\begin{align*}
	\sec(\mspan\{X,Y\}) = \frac{g(R(X,Y)Y,X)}{\|X\|^2\{Y\|^2 - \langle X,Y \rangle^2} = 0
\end{align*}
\item
	Wir berechnen die Komponenten $R_{ijkl}=R(\pdifffrac{}{\xi^i},\pdifffrac{}{\xi^j},\pdifffrac{}{\xi^k},\pdifffrac{}{\xi^l})=g(R(\pdifffrac{}{\xi^i},\pdifffrac{}{\xi^j})\pdifffrac{}{\xi^k},\pdifffrac{}{\xi^l})$ in der Karte $\phi$ aus Aufgabe 6.3. In dieser Karte gilt $g=\frac{4}{(1-\|\xi\|^2)^2}\sum\limits_i d\xi^i\tensor d\xi^i$, also $g_{ij}=\delta_{ij} \frac{4}{(1-\|\xi\|^2)^2}$.
	
	Für die Ableitungen der Metrik gilt dann:
		\[g_{ij,k}=\frac{\partial g_{ij}}{\partial \xi^k}=16\delta_{ij} \frac{\xi^k}{(1-\|\xi\|^2)^3}.\]
	Die Koeffizienten der zu $(g_{ij})$ inversen Matrix sind $g^{kl}=\delta_{kl} \frac{(1-\|\xi\|^2)^2}{4}$. Damit gilt für die Christoffelsymbole
		\[\Gamma_{ij}^k=\frac{1}{2}\sum_l g^{kl}(g_{jl,i}-g_{ij,l} + g_{li,j})=\frac{2}{1-\|\xi\|^2}(\delta_{jk}\xi^i-\delta_{ij} \xi^k + \delta_{ki} \xi^j).\]
	Für die Ableitungen gilt also:
	\begin{align*}
		\pdifffrac{}{\xi^l}(\Gamma_{ij}^k)&=\pdifffrac{}{\xi^l} \left(\frac{2}{1-\|\xi\|^2}\right)(\delta_{jk}\xi^i-\delta_{ij} \xi^k + \delta_{ki} \xi^j)+ \frac{2}{1-\|\xi\|^2} \pdifffrac{}{\xi^l}(\delta_{jk}\xi^i-\delta_{ij} \xi^k + \delta_{ki} \xi^j)\\
		&=\frac{4\xi^l}{(1-\|\xi\|^2)^2}(\delta_{jk}\xi^i-\delta_{ij} \xi^k + \delta_{ki} \xi^j)+ \frac{2}{1-\|\xi\|^2}(\delta_{jk}\delta_{li}-\delta_{ij} \delta_{lk} + \delta_{ki} \delta_{lj}).
	\end{align*}
	
	Nun können wir die Koeffizienten des Krümmungstensors berechnen. Es gilt :
	\begin{align*}
		R(\pdifffrac{}{\xi^i}, \pdifffrac{}{\xi^j})\pdifffrac{}{\xi^k}&=\nabla_{\pdifffrac{}{\xi^i}}\nabla_{\pdifffrac{}{\xi^j}}\pdifffrac{}{\xi^k}-\nabla_{\pdifffrac{}{\xi^j}}\nabla_{\pdifffrac{}{\xi^i}}\pdifffrac{}{\xi^k}-\nabla_{[\pdifffrac{}{\xi^i},\pdifffrac{}{\xi^j}]=0}\pdifffrac{}{\xi^k}\\ 
		&=\nabla_{\pdifffrac{}{\xi^i}}\Big(\sum_l \Gamma_{jk}^l\pdifffrac{}{\xi^l}\Big)-\nabla_{\pdifffrac{}{\xi^j}}\Big(\sum_l \Gamma_{ik}^l\pdifffrac{}{\xi^l}\Big)\\
		&=\sum_l\left( \pdifffrac{}{\xi^i}(\Gamma_{jk}^l) \pdifffrac{}{\xi^l} + \Gamma_{jk}^l \nabla_{\pdifffrac{}{\xi^i}}\pdifffrac{}{\xi^l} - \pdifffrac{}{\xi^j}(\Gamma_{ik}^l) \pdifffrac{}{\xi^l} - \Gamma_{ik}^l \nabla_{\pdifffrac{}{\xi^j}}\pdifffrac{}{\xi^l}\right)\\
		&=\sum_l\left( \pdifffrac{}{\xi^i}(\Gamma_{jk}^l) \pdifffrac{}{\xi^l}  - \pdifffrac{}{\xi^j}(\Gamma_{ik}^l) \pdifffrac{}{\xi^l}\right)+\sum_{l,m}\left(\Gamma_{jk}^l\Gamma_{il}^m \pdifffrac{}{\xi^m}-\Gamma_{ik}^l \Gamma_{jl}^m \pdifffrac{} {\xi^m} \right)\\
		&=\sum_l \left(\pdifffrac{}{\xi^i}(\Gamma_{jk}^l)  - \pdifffrac{}{\xi^j}(\Gamma_{ik}^l)+\sum_{\alpha}\left(\Gamma_{jk}^\alpha\Gamma_{i\alpha}^l-\Gamma_{ik}^\alpha \Gamma_{j\alpha}^l \right)\right)  \pdifffrac{} {\xi^l}=: \sum_l R_{ijk}^{\phantom{ijk}l} \;\pdifffrac{}{\xi^l}
	\end{align*}
	und
	\begin{align*}
		R_{ijk}^{\phantom{ijk}l}&=\tfrac{4}{(1-\|\xi\|^2)^2}\Big(\xi^i(\delta_{kl} \xi^j- \delta_{jk} \xi^l + \delta_{lj}\xi^k)-\xi^j(\delta_{kl} \xi^i- \delta_{ik} \xi^l + \delta_{li}\xi^k)\\
		&\qquad\qquad + \tfrac{1-\|\xi\|^2}{2}\big((\delta_{kl} \delta_{ij} - \delta_{jk} \delta_{il}+\delta_{lj}\delta_{ik})-(\delta_{kl} \delta_{ij}-\delta_{ik}\delta_{jl}+\delta_{li}\delta_{jk})\big)\\
		&\qquad \qquad + \sum_\alpha \big( ( \delta_{k\alpha} \xi^j-\delta_{jk}\xi^{\alpha}+\delta_{\alpha j} \xi^k)(\delta_{\alpha l} \xi^i-\delta_{i \alpha}\xi^l+\delta_{li}\xi^\alpha)\\[-0.8em]
		&\qquad \qquad \qquad -  ( \delta_{k\alpha} \xi^i-\delta_{ik}\xi^{\alpha}+\delta_{\alpha i} \xi^k)(\delta_{\alpha l} \xi^j-\delta_{j \alpha}\xi^l+\delta_{lj}\xi^\alpha)\big)\Big)\\
		&= \tfrac{4}{(1-\|\xi\|^2)^2}\Big(- \delta_{jk}\xi^i \xi^l + \delta_{lj}\xi^i\xi^k+ \delta_{ik}\xi^j \xi^l - \delta_{li}\xi^j\xi^k\\
		&\qquad\qquad + \tfrac{1-\|\xi\|^2}{2}\big( - \delta_{jk} \delta_{il}+\delta_{lj}\delta_{ik}+\delta_{ik}\delta_{jl}-\delta_{li}\delta_{jk}\big)\\
		&\qquad \qquad + \sum_\alpha \big( ( \delta_{k\alpha} \delta_{\alpha l} \xi^i \xi^j-\delta_{k\alpha} \delta_{i \alpha}  \xi^j\xi^l+\delta_{k\alpha} \delta_{li} \xi^j \xi^\alpha\\
		&\qquad \qquad \qquad -\delta_{jk}\delta_{\alpha l} \xi^i\xi^{\alpha}+\delta_{jk}\delta_{i \alpha}\xi^l\xi^{\alpha} -\delta_{jk}\delta_{li}(\xi^{\alpha})^2\\
		&\qquad \qquad \qquad+ \delta_{\alpha j}\delta_{\alpha l} \xi^i \xi^k - \delta_{\alpha j}\delta_{i \alpha}\xi^k \xi^l + \delta_{\alpha j} \delta_{li}   \xi^k \xi^\alpha)\\
		& \qquad \qquad - (  \delta_{k\alpha} \delta_{\alpha l} \xi^i \xi^j-\delta_{k\alpha} \delta_{j \alpha}  \xi^i\xi^l+\delta_{k\alpha} \delta_{lj} \xi^i \xi^\alpha\\
		&\qquad \qquad \qquad -\delta_{ik}\delta_{\alpha l} \xi^j\xi^{\alpha}+\delta_{ik}\delta_{j \alpha}\xi^l\xi^{\alpha} -\delta_{ik}\delta_{lj}(\xi^{\alpha})^2\\
		&\qquad \qquad \qquad+ \delta_{\alpha i}\delta_{\alpha l} \xi^j \xi^k - \delta_{\alpha i}\delta_{j \alpha}\xi^k \xi^l + \delta_{\alpha i} \delta_{lj}  \xi^k \xi^\alpha)\big)\Big)\\
	\end{align*}
	\begin{align*}
		&= \tfrac{4}{(1-\|\xi\|^2)^2}\Big(- \delta_{jk}\xi^i \xi^l + \delta_{lj}\xi^i\xi^k+ \delta_{ik}\xi^j \xi^l - \delta_{li}\xi^j\xi^k + (1-\|\xi\|^2)\big(\delta_{ik}\delta_{jl}-\delta_{li}\delta_{jk}\big)\\
		&\qquad \qquad + \sum_\alpha \big( ( \delta_{k l} \xi^i \xi^j- \delta_{i k}  \xi^j\xi^l+ \delta_{li} \xi^j \xi^k -\delta_{jk} \xi^i\xi^{l}+\delta_{jk}\xi^l\xi^{i} -\delta_{jk}\delta_{li}\|\xi\|^2+ \delta_{j l} \xi^i \xi^k - \delta_{i j}\xi^k \xi^l + \delta_{li} \xi^k \xi^j)\\
		&\qquad \qquad \qquad - (  \delta_{k l} \xi^i \xi^j- \delta_{j k}  \xi^i\xi^l+ \delta_{lj} \xi^i \xi^k -\delta_{ik} \xi^j\xi^{l}+\delta_{ik}\xi^l\xi^{j} -\delta_{ik}\delta_{lj}\|\xi\|^2 + \delta_{i l} \xi^j \xi^k - \delta_{ij}\xi^k \xi^l + \delta_{lj} \xi^k \xi^i)\big)\Big)\\
		&=  \tfrac{4}{(1-\|\xi\|^2)^2}\Big( (1-\|\xi\|^2)\big(\delta_{ik}\delta_{jl}-\delta_{li}\delta_{jk}\big) +  \|\xi\|^2 \big(\delta_{ik}\delta_{jl}-\delta_{li}\delta_{jk}\big)\\
		&=  \tfrac{4}{(1-\|\xi\|^2)^2}\big(\delta_{ik}\delta_{jl}-\delta_{li}\delta_{jk}\big)
	\end{align*}
	Und somit 
		\[R_{ijkl}=g(\sum_m R_{ijk}^{\phantom{ijk}m} \pdifffrac{}{\xi^m},\pdifffrac{}{\xi^l})=\sum_m  R_{ijk}^{\phantom{ijk}m} g(\pdifffrac{}{\xi^m},\pdifffrac{}{\xi^l})=\frac{4}{(1-\|\xi\|^2)^2} R_{ijk}^{\phantom{ijk}l}=\frac{16}{(1-\|\xi\|^2)^4}\big(\delta_{ik}\delta_{jl}-\delta_{li}\delta_{jk}\big).\]
	Für linear unabhängige $X=\sum X_i \pdifffrac{}{\xi^i}$, $Y=\sum Y_j \pdifffrac{}{\xi^j} \in \T_p\mathbb{H}^n$ gilt dann
	\begin{align*}
		R(X,Y,Y,X)&= \sum_{i,j,k,l} X_iX_lY_j Y_k \;R(\pdifffrac{}{\xi^i},\pdifffrac{}{\xi^j},\pdifffrac{}{\xi^k},\pdifffrac{}{\xi^l}) \\
		{}=& \sum_{i,j,k,l} X_iX_lY_j Y_k \;\tfrac{16}{(1-\|\xi\|^2)^4}\big(\delta_{ik}\delta_{jl}-\delta_{li}\delta_{jk}\big)\\
		{}=& \sum_{i,j} X_iX_jY_jY_i \; \left(\tfrac{4}{(1-\|\xi\|^2)^2}\right)^2- \sum_{i,j} X_i^2 Y_j^2 \; \left(\tfrac{4}{(1-\|\xi\|^2)^2}\right)^2\\
		{}=& \sum_{i,j} X_iX_jY_jY_i g(\pdifffrac{}{\xi^i},\pdifffrac{}{\xi^i})g(\pdifffrac{}{\xi^j},\pdifffrac{}{\xi^j})-  \sum_{i,j} X_i^2 Y_j^2 \;g(\pdifffrac{}{\xi^i},\pdifffrac{}{\xi^i})g(\pdifffrac{}{\xi^j},\pdifffrac{}{\xi^j})\\
		{}=& \Big(\sum_ig(X_i\pdifffrac{}{\xi^i}, Y_i\pdifffrac{}{\xi^i})\Big)\Big(\sum_jg(X_j\pdifffrac{}{\xi^j}, Y_j\pdifffrac{}{\xi^j})\Big) \\
		&- \Big(\sum_ig(X_i\pdifffrac{}{\xi^i}, X_i\pdifffrac{}{\xi^i})\Big)\Big(\sum_jg(Y_j\pdifffrac{}{\xi^j}, Y_j\pdifffrac{}{\xi^j})\Big)\\
		{}=& g(X,Y)^2 - \|X\|^2 \|Y\|^2=-(\|X\|^2 \|Y\|^2 - g(X,Y)^2).
	\end{align*}
	Hieraus folgt
		\[\sec(\mathrm{span}\{X,Y\})=\frac{R(X,Y,Y,X)}{\|X\|^2 \|Y\|^2 - g(X,Y)^2}=-1.\]
\end{enumerate}\end{Loes}

\begin{Loes}
Sei $S^n(r) := \{x \in \R^{n+1} | \|x\| = r\}$, \emph{Behauptung:} $\sec_{S^n(r)} \equiv \frac{1}{r^2}$

Sei $\nabla = \nabla S^n(r)$ der Levi-Civita Zusammenhang von $(S^n(r), g_{\text{ind}})$. Daraus folgt:
\begin{align*}
	(\nabla_X Y)_p &= \left( (\nabla_X^{\R^{n+1}} Y)_p \right) \T_p S^n(r) \\
	&= (\nabla_X^{\R^{n+1}} Y)_p - \langle (\nabla_X^{\R^{n+1}} Y)_p, N(p) \rangle \cdot N(p)
\end{align*}
wobei $N(p) = \frac{1}{r} \cdot p$ das Normaleneinheitsvektorfeld an $S^n(r)$ ist. Betrachte nun
\begin{align*}
	\langle (\nabla_X^{\R^{n+1}} Y)_p, N(p) \rangle &= \underbrace{X_p ( \overbrace{\langle Y, N \rangle}^{\equiv 0} )}_{=0} - \langle Y_p, (\nabla_X^{\R^{n+1}} N)_p \rangle\\
	&= - \langle Y_p, \underbrace{(\D N)_p}_{\frac{1}{r}\cdot\id} \cdot X_p \rangle\\
	&= - \frac{1}{r} \langle Y_p, X_p \rangle
\end{align*}
Daraus folgt dann $\nabla_XY = \nabla_X^{R^{n+1}} Y + \frac{1}{r} \langle X, Y \rangle N$. Als n"achstes betrachten wir nun:
\begin{align*}
R^{S^n(r)}(X,Y)Z ={}& \nabla_X \nabla_Y Z - \nabla_Y \nabla_X - \nabla_{[X,Y]}Z\\
={}& \nabla_X^{\R^{n+1}} (\nabla_Y Z) + \frac{1}{r} \langle X, \nabla_Y Z \rangle \cdot N\\
   & - \nabla_Y^{\R^{n+1}} (\nabla_X Z) - \frac{1}{r} \langle Y, \nabla_X Z \rangle \cdot N\\
   & - \nabla_{[X,Y]}^{\R^{n+1}} Z - \frac{1}{r} \langle [X,Y], Z \rangle \cdot N\\
={}& \overbrace{R^{\R^{n+1}} (X,Y)Z}^{=0} + \nabla_X^{\R^{n+1}} (\frac{1}{r} \langle Y, Z \rangle \cdot N) + \frac{1}{r} \langle X, \nabla_Y^{\R^{n+1}} Z + \frac{1}{r} \langle Y, Z \rangle \cdot \overset{\textcolor{gray}{\perp X}}{N} \rangle N\\
   & - \nabla_Y^{\R^{n+1}} ( \frac{1}{r} \langle X, Y \rangle N) - \frac{1}{r} \langle Y, \nabla_X^{\R^{n+1}} Z + \frac{1}{r} \langle Y, Z \rangle \cdot \overset{\textcolor{gray}{\perp Y}}{N} \rangle \cdot N\\
   & - \frac{1}{r} \langle \underbrace{[X,Y]}_{\mathclap{=\D Y \cdot X - \D X \cdot Y}} Z \rangle \cdot N\\
={}& X(\frac{1}{r} \langle Y, Z \rangle) N + \frac{1}{r} \langle Y, Z \rangle \nabla_X^{\R^{n+1}} N + \frac{1}{r} \langle X, \D Z \cdot Y \rangle \cdot N\\
   & - Y ( \frac{1}{r} \langle X, Z \rangle ) \cdot N - \frac{1}{r} \langle X, Z \rangle \nabla_Y^{\R^{n+1}} N - \frac{1}{r} \langle Y, \D Z \cdot X \rangle \cdot N - \frac{1}{r} \langle \D Y \cdot X\\
   & - \D X \cdot Y, Z \rangle \cdot N\\
={}& \frac{1}{r} ( \langle \D Y \cdot X, Z \rangle + \langle Y, \D Z \cdot X \rangle ) \cdot N + \frac{1}{r} \langle Y, Z \rangle \overbrace{\D N}^{= \frac{1}{r} \cdot \id} \cdot X + \langle X, \D Z \cdot Y \rangle \cdot N\\
   & - \frac{1}{r} ( \langle \D X \cdot Y, Z \rangle + \langle X, \D Z \cdot Y \rangle ) \cdot N - \frac{1}{r^2} \langle X Z \rangle \cdot Y + \langle Y, \D Z \cdot X \rangle \cdot N\\
   & - \frac{1}{r} \langle \D Y \cdot X - \D X \cdot Y, Z \rangle \cdot N\\
={}& \frac{1}{r^2} ( \langle Y, Z \rangle \cdot X - \langle X, Z \rangle \cdot Y)
\end{align*}
Daraus folgt dann
\begin{align*}
	\langle \D(X,Y) Y, X \rangle = \frac{1}{r^2} ( \langle Y, Y, \rangle \langle X, X \rangle - \langle X, Y \rangle^2)
\end{align*}
und damit folgt dann schlie"slich
\begin{align*}
\sec_{S^n(r)}(\span\{X,Y\}) = \frac{1}{r^2}
\end{align*}
\end{Loes}
%%
%% Skript Differentialgeometrie im Wintersemester 12/13
%% Zur Vorlesung von Dr. Grensing am KIT Karlsruhe
%%
%% Uebung 10
%%

\section{14. Januar 2012}
\setcounter{Aufg}{0} %Damit die Aufgaben jedes Mal bei Aufgabe 1 anfangen
\setcounter{Loes}{0}

"Ubungsblatt 10 enthielt keine Aufgaben, deshalb befassen wir uns hier mit den Geod"atischen von $\H^2$.
Wir betrachten zun"achst die folgenden drei bisherigen Modelle \begin{description}[leftmargin=*]
\item[Hyperboloid:]
	$\{ x \in \R^{2+1} \mid \overbrace{-x_0^2 + x_1^2 + x_2^2}^{= \langle x,x \rangle} = -1, x_0 > 0 \}$ und $\langle \cdot, \cdot \rangle|_{\T_p\H^2 \X \T_p\H^2}$ ist das Skalarprodukt.
\item[Poincare Kreisscheibenmodell:]
	$D := \{ \xi \in \R^2 \mid \|\xi\| < 1 \}$, $g_D = \frac{4}{(1 - \|\xi||)^2} \sum \dop \xi^{i} \otimes \dop \xi^{i}$
\item[Poincare obere Halbebene Modell:]
	$H := \{ x + iy \in \C \mid y > 0 \}$, $g_H = \frac{1}{y^2}(\dop x \otimes \dop x + \dop y \otimes \dop y)$
\end{description}

\begin{emptythm}[Isometrie zwischen $D$ und $H$:]
Betrachte $D$ als Teilmenge von $\C$ mittels $\psi: H \to D$, $z \mapsto \frac{z-i}{z+i}$
\begin{align*}
	|\psi(z)|^2 &= \frac{z-i}{z+i} \cdot \frac{\overline z+i}{\overline z-i} = \frac{z \overline z + i\overbrace{(z - \overline z)}^{=2i\Im(z)} + 1}{z \overline z - i(z - \overline z) + 1} = \frac{|z|^2 - 2 \Im(z) + 1}{|z|^2 + 2 \Im(z) + 1} \overset{\Im(z) > 0}{<} 1
\end{align*}
Definiere weiter $\phi: D \to H$ mit $\xi = \xi_1 + i \xi_2 \mapsto -i \frac{\xi + 1}{\xi - 1}$. Dann folgt
\begin{align*}
	\Im(\phi(\xi)) = \frac{\overbrace{1 - |\xi|^2}^{>0}}{\underbrace{|\xi|^2 - 2 \Re(\xi) + 1}_{\ge (\Re(\xi) - 1)^2 > 0}} > 0
\end{align*}
Durch Nachrechnen ergibt sich $\phi \circ \psi = \id$ und $\psi \circ \phi = \id$.
Da $\phi$ und $\psi$ holomorph sind, folgt dass sie auf $C^\infty$ glatt sind.
Wir zeigen schlie"slich dass $\psi$ eine Isometrie ist. Dazu fassen wir $\psi$ als reelle Funktion auf:
\begin{align*}
	\psi(x, y) = \begin{pmatrix}\psi_1(x,y) \\ \psi_2(x,y)\end{pmatrix}
\end{align*}
Da $\psi$ holomorph ist erf"ullt $\psi$ auch die Cauchy-Riemannschen Differentialgleichungen:
\begin{align*}
	\pdifffrac{\psi_1}{x} = \pdifffrac{\psi_2}{y}& &\pdifffrac{\psi_1}{y} = \pdifffrac{\psi_2}{x}
\end{align*}
Es sei $p = (x, y) \in H$ und sei $ye_1, ye_2$ eine Orthonormalbasis von $\T_pH$. Dann gilt:
\begin{align*}
	g_D(\psi_{*p}(y e_1), \psi_{*p}(ye_2)) &= y^2 g_D \left(\pdifffrac{\psi}{x}, \pdifffrac{\psi}{y} \right) \\
	&= \frac{4y^2}{(1 - \|\psi\|^2)^2} \cdot \left( \pdifffrac{\psi_1}{x} \cdot \pdifffrac{\psi_1}{y} + \pdifffrac{\psi_2}{x} \cdot \pdifffrac{\psi_2}{y} \right) \\
	&= 0 \\
	g_D(\psi_{*p}(y e_1), \psi_{*p}(ye_1)) &= \frac{4y^2}{(1 - \|\psi\|^2)^2} \cdot \left( \left( \pdifffrac{\psi_1}{x} \right)^2 + \left( \pdifffrac{\psi_2}{x} \right)^2 \right) \\
	&= \ldots = 1 \\
	g_D(\psi_{*p}(y e_2), \psi_{*p}(ye_2)) &= \ldots = 1
\end{align*}
Daraus folgt dass $\psi_*$ eine Orthonormalbasis auf eine Orthonormalbasis abbildet und daher ist $\psi$ eine Isometrie.
\end{emptythm}

\begin{emptythm}[Isometrien von $H$]
Zu $a, b, c, d \in \R$ mit $ad - bc > 0$ betrachte 
\begin{align*}
	h(z) = \frac{az + b}{bz + d} \tag{(spezielle) M"obiustransformation}
\end{align*}
Es gilt:
\begin{align*}
	\Im(h(z)) &= \Im \left( \frac{(az + b)(c\overline{z} + d)}{|cz + d|^2} \right) \\
	&= \Im \left( \frac{acz\overline{z} + bd + adz + bc\overline{z}}{|cz + d|^2} \right) \\
	&= \frac{(ad - bc) \Im(z)}{|cz + d|^2} > 0 \tag{f"ur $z \in H$}
\end{align*}
Es gilt somit $h: H \to H$, sowie
\begin{align*}
	h^{-1}(z) = \frac{1}{ad-bc} \cdot \frac{dz-b}{-cz+a} \tag{nachrechnen}
\end{align*}
Daraus folgt dass $h$ ein Diffeomorphismus ist.
F"ur $v \in \T_zH$ und $v = \left( \begin{smallmatrix} v_1 \\ v_2 \end{smallmatrix} \right)$ gilt:
\begin{align*}
	h_{*z}V = Dh|_z V &= \begin{pmatrix} v_i \pdifffrac{\Re(h)}{x} + v_2 \pdifffrac{\Re(h)}{y} \\ v_2 \pdifffrac{\Im(h)}{x} + v_2 \pdifffrac{\Im(h)}{y} \end{pmatrix} \overset{\text{C-R}}{\underset{\text{DGL}}{=}} \begin{pmatrix} v_1 \lambda - v_2 \mu \\ v_2 \mu + v_2 \lambda \end{pmatrix} & \begin{matrix} \pdifffrac{\Re(h)}{x} =: \lambda \\ \pdifffrac{\Im(h)}{x} =: \mu \end{matrix}\\
	&= \begin{pmatrix} \Re((\lambda + i \mu)(v_1 + i v_2)) \\ \Im((\lambda + i \mu)(v_1 + i v_2)) \end{pmatrix} \\
	&= \begin{pmatrix} \Re(h'(z) \cdot (v_1 + i v_2)) \\ \Im(h'(z) \cdot (v_1 + i v_2)) \end{pmatrix}
\end{align*}
F"ur $v, w \in \T_zH$ gilt:
\begin{align*}
	g_H(v, w) &= \frac{1}{\Im(z)^2} (v_1 w_1 + v_2 w_2) \\
	&= \frac{1}{\Im(z)^2} \Re(\underbrace{(v_1 + iv_2)}_{=: \tilde{v}} \overline{\underbrace{(w_1 + i w_2)}_{=:\overline{\tilde{w}}}})
\end{align*}
Es ist
\begin{align*}
	h'(x) = \frac{a(cz + d) - c(az + b)}{(cz + d)^2} = \frac{ad - bc}{(cz + d)^2}
\end{align*}
Daraus folgt dann
\begin{align*}
	g_H|_{h(z)}(h_{*z}v, h_{*z}w) &= \frac{1}{(\Im(h(z)))^2} \cdot \Re(h'(z) \tilde{v} \overline{h'(z) \tilde{w}}) \\
	&= \frac{|h'(z)|^2}{\Im(h(z))^2} \cdot \Re(\tilde{v} \overline{\tilde{w}}) = \frac{|h'(z)|^2}{\Im(h(z))^2} \cdot g_h|_z(v,w)
\end{align*}
Wobei $\Im(h(z)) = \frac{ad - bc}{|cz + d|^2} \cdot \Im(z) = |h'(z)| \cdot \Im(z)$ gilt, daraus folgt dass $h$ eine Isometrie ist.
\end{emptythm}

\begin{bsp}\begin{enumerate}[label=(\arabic*),leftmargin=*]
\item
	F"ur $w \in H$ ist
	\begin{align*}
		h_w(z) = \frac{\Im(w) \cdot z + \Re(w)}{0 \cdot z + 1} = \Im(w) \cdot z + \Re(w)
	\end{align*}
	eine Isometrie von $H$, da $\Im(w) > 0$ und $h_w(i) = \Im(w) i + \Re(w) = w$. Daraus folgt dass es gen"ugt die Geod"atischen durch $i$ zu betrachten.
\item
	F"ur $\vartheta \in \R$ ist
	\begin{align*}
		h_\vartheta = \frac{\cos(\vartheta)z - \sin(\vartheta)}{\sin(\vartheta)z + \cos(\vartheta)}
	\end{align*}
	Eine Isometrie von $H$ mit
	\begin{align*}
		h_\vartheta(i) = i \frac{\cos(\vartheta) - \frac{1}{i} \sin(\vartheta)}{\sin(\vartheta) i + \cos(\vartheta)} = i
	\end{align*}
	und
	\begin{align*}
		h_{\vartheta}'(i) = \frac{1}{(\sin(\vartheta) i + \cos(\vartheta))^2} = e^{-2i\vartheta}
	\end{align*}
	F"ur $v \in \T_iH$ mit $\|v\| = 1$ k"onnen wir also $\vartheta \in \R$ w"ahlen mit $h_{\vartheta *}e_2 = v$.
	Daraus folgt dass es gen"ugt die Geod"atischen $\gamma$ mit $\gamma(0) = i$ und $\dot\gamma(0) = e_2$ zu bestimmen.
	\begin{align*}
		g_{ij,1} &= \pdifffrac{g_{ij}}{x} = \delta_{ij} \cdot \pdifffrac{}{x} \left( \frac{1}{y^2} \right) = 0 \\
		g_{ij,2} &= \pdifffrac{g_{ij}}{y} = \delta_{ij} \cdot \frac{-2}{y^3} \\
		g_{kl} &= y^2 \cdot \delta_{kl} \\
		\Gamma_{ij}^k &= \frac{1}{2} \sum_{l=1}^2 g^{kl} (g_{jl,i} - g_{ij,l} + g_{il,j})
	\end{align*}
	Daraus folgt
	\begin{align*}
		\Gamma_{12}^1 &= \Gamma_{21}^1 = - \frac{1}{y} \\
		\Gamma_{11}^2 &= \frac{1}{y}, \Gamma_{22}^2 = - \frac{1}{y}
	\end{align*}
	alle Anderen sind \quot{$=0$}.
	Die geod"atische Differentialgleichung lautet $\ddot\gamma^k + \sum_{ij} (\Gamma_{ij}^k \circ \gamma) \cdot \dot\gamma^{i} \cdot \dot\gamma^j = 0$. Daraus folgt
	\begin{align*}
		\ddot\gamma^1 - \frac{2}{\gamma^2} \dot\gamma^1 \cdot \dot\gamma^2 &= 0 \\
		\ddot\gamma^2 - \frac{1}{\gamma^2} ((\dot\gamma^2)^2 - (\dot\gamma^1)^2) &= 0
	\end{align*}
	\begin{description}[font=\normalfont\itshape,leftmargin=*]
	\item[Ansatz:]
		$\gamma^1 \equiv 0$ (erf"ullt die erste Gleichung) $\leadsto \ddot\gamma^2 = \frac{1}{\gamma^2}(\dot\gamma^2)^2$
	\item[L"osung:]
		$\gamma^2(t) = e^t$
	\end{description}
	Damit ist $\gamma(t) = ie^t$ die Geod"atische durch $i$ mit der Startrichtung $e_2$.
	Die anderen Geod"atischen, die in $i$ starten sind von der Form
	\begin{align*}
		h_\vartheta(\gamma(t)) = \frac{\cos(\vartheta) i e^t - \sin(\vartheta)}{\sin(\vartheta) i e^t + \cos(\vartheta)}
	\end{align*}
	M"obiustransformationen bilden Geraden und Kreise auf Geraden und Kreise ab.
	Damit ist $\gamma_\vartheta = h_\vartheta \circ \gamma$ eine Gerade oder ein Kreis.
	\begin{align*}
		\gamma_\vartheta(t) \to \begin{cases} -\frac{\sin \vartheta}{\cos \vartheta} & \text{ f"ur } t \to -\infty \\ \frac{\cos \vartheta}{\sin \vartheta} & \text{f"ur } t \to \infty \end{cases}
	\end{align*}
	F"ur $\sin \vartheta, \cos \vartheta \ne 0$ ergibt sich:
	\begin{center}\begin{tikzpicture}[font=\scriptsize]
		\def\Radius{1.5}
		\def\radius{1}
		\draw[name path=achse] (-2,0) -- (4,0);
		\draw (0,-0.5) --node[left,pos=0.8]{$\gamma$} (0,2);
		
		\draw (2*\Radius,0) arc[start angle=0,end angle=180,radius=\Radius];
		\draw (2*\Radius,0) -- ++(0,1) (2*\Radius + 0.25,0) arc[start angle=0,end angle=90,radius=0.25];
		\fill ($(2*\Radius,0) + (45:0.125)$) circle(0.03);
		
		\draw (\Radius,0.25) -- +(300:1.25) node[right]{?};
		
		\clip (0,0) rectangle (2*\Radius,2*\Radius);
		\draw[name path=kreis] (\Radius+\radius,-0.5) arc[start angle=0,end angle=180,radius=\radius];
		
		\path[name intersections={of=kreis and achse}];
		\coordinate (pkt) at (intersection-1);
		\draw ($(pkt)!-1!0:(\Radius+\radius,-0.5)$) coordinate (endpkt) -- ($(pkt)!1!0:(\Radius+\radius,-0.5)$); % Tangente
		\clip (pkt) --(endpkt) -- (4,0) -- cycle;
		\draw (pkt) circle(0.25);
	\end{tikzpicture}\end{center}
	Es ist
	\begin{align*}
		\frac{\dot\gamma_\vartheta(t)}{|\dot\gamma_\vartheta(t)|} = \ldots = \frac{i \cos \vartheta + e^t \sin \vartheta}{\cos \vartheta + i e^t \sin \vartheta} \to \begin{cases} i & \text{f"ur } t \to -\infty \\ -i & \text{f"ur } t \to \infty \end{cases}
	\end{align*}
	Also schneidet der Kreis die $\R$-Achse im rechten Winkel und damit liegt der Mittelpunkt in $\R$: $\frac{1}{2} \left( \frac{\cos \vartheta}{\sin \vartheta} - \frac{\sin \vartheta}{\cos \vartheta} \right)$.
	Wegen $h_w(\R) \subset \R$ gilt das Gleiche f"ur alle Geod"atischen.
	\begin{center}\begin{tikzpicture}[font=\scriptsize,remember picture]
		\def\Radius{2.5}
		\tikzstyle{reverseclip}=[insert path={(current page.north east) -- (current page.south east) -- (current page.south west) -- (current page.north west) -- (current page.north east)}]
		
		\draw[name path=Kreis] (0,0) circle(\Radius);
		\clip (0,0) circle(\Radius);
		
		\def\angleA{250}
		\def\angleB{\angleA - 180}
		\coordinate (a) at (\angleA:\Radius); \coordinate (b) at ($-1*(a)$);
		\draw (a) -- (b);
		\begin{scope}
			\clip (a) arc[start angle=\angleA,end angle=430,radius=\Radius] -- cycle;
			\draw (a) circle(0.25);
			\fill ($(a) + (45 - 270 + \angleA:0.125)$) circle(0.03);
		\end{scope}\begin{scope}
			\clip (b) arc[start angle=\angleB,end angle=\angleA,radius=\Radius] -- cycle;
			\draw (b) circle(0.25);
			\fill ($(b) + (225 - 90 + \angleB:0.125)$) circle(0.03);
		\end{scope}
		
		\begin{pgfinterruptboundingbox}		
			\draw[name path=kreis] (\angleA:1.5*\Radius) circle(1.5*\Radius);
			\path[clip] (\angleA:1.5*\Radius) circle(1.5*\Radius) [reverseclip];
			\path[name intersections={of=kreis and Kreis}];
			\foreach \i in {1,2}{
				\draw (intersection-\i) circle (0.25);
				\draw ($(intersection-\i)!-0.25!90:(\angleA:1.5*\Radius)$) -- ($(intersection-\i)!0.25!90:(\angleA:1.5*\Radius)$); % Tangente
			}
			\fill ($(intersection-1) + (85:0.125)$) circle(0.03) ($(intersection-2) + (55:0.125)$) circle(0.03);
		\end{pgfinterruptboundingbox}
	\end{tikzpicture}\end{center}
\end{enumerate}\end{bsp}
%%
%% Skript Differentialgeometrie im Wintersemester 12/13
%% Zur Vorlesung von Dr. Grensing am KIT Karlsruhe
%%
%% Uebung 11
%%
\section{21. Januar 2012}
\setcounter{Aufg}{0} %Damit die Aufgaben jedes Mal bei Aufgabe 1 anfangen
\setcounter{Loes}{0}

Allgemein sei $\kappa \in \R$
\begin{Loes}
\emph{Erinnerung:} Definition von Jacobifeldern
\begin{align*}
	\ddot{\calJ}(t) + (R(\calJ, \dot\gamma) \dot\gamma) (t) = 0 \tag{Jacobi-Gleichung}
\end{align*}
(Vektorfelder von Variationen durch Geod"atische)
\begin{align*}
	\langle \ddot\calJ, \calJ \rangle & = -R(\calJ, \dot\gamma, \ddot\gamma, \calJ) \\
	& = -\sec(\span\{\calJ, \dot\gamma\}) \cdot (\|\calJ\|^2 \underbrace{\|\dot\gamma\|}_{=0}^2 - \langle\calJ, j \rangle^2) \\
	& = - \kappa \langle \calJ, \calJ \rangle
\end{align*}
Wir m"ochten nun zeigen dass $\ddot\calJ = -\kappa \calJ$ gilt. Es sei $e_1, \ldots ,e_{n-1}, e_n = \dot\gamma(t)$ eine Orthonormalbasis von $\T_{\gamma(t)}M$.
Daraus folgt $0 = R(\calJ, \dot\gamma, \dot\gamma, e_n)$ und $0 = \langle \calJ, e_n \rangle$. F"ur $i < n$ gilt:
\begin{align*}
	R(\calJ + e_i, \dot\gamma, \dot\gamma, \calJ + e_i) = \kappa \cdot ( \| \calJ + e_i \|^2 \cdot \underbrace{\| \gamma \|^2}_{=1} - \underbrace{\langle \calJ + e_i, \dot\gamma \rangle^2}_{=0}) = \kappa \cdot \| \calJ + e_i \|^2
\end{align*}
und
\begin{align*}
	R(\calJ + e_i, \dot\gamma, \dot\gamma, \calJ + e_i) &= R(\calJ, \dot\gamma, \dot\gamma, \calJ) + R(e_i, \dot\gamma, \dot\gamma, e_i) + 2R(\calJ, \dot\gamma, \dot\gamma, e_i) \\
	&= \kappa \cdot ( ||\calJ\|^2 + \|e_i\|^2) + 2R(\calJ, \dot\gamma, \dot\gamma, e_i)
\end{align*}
Daraus folgt
\begin{align*}
	R(\calJ, \dot\gamma, \dot\gamma, e_i) = \frac{1}{2}\kappa ( \|\calJ + e_i\|^2 - \|\calJ\|^2 - \|e_i\|^2 ) = \kappa \langle \calJ, e_i \rangle
\end{align*}
und damit gilt $R(\calJ, \dot\gamma, \dot\gamma) = \kappa \cdot \calJ$ und damit wird die Jacobi-Gleichung zu $\ddot\calJ = -\kappa \calJ$.
Setze $A(0) = \calJ(0)$ und $B(0) = \dot\calJ(0)$ parallel fort zu $A(t)$ und $B(t)$ und definiere $\tilde\calJ(t) = C_\kappa(t) \cdot A(t) + S_\kappa(t) \cdot B(t)$.
\marginnote{$\frac{\D}{\dop t} \hat{=} \nabla_t$}Betrachte nun:
\begin{align*}
	\frac{\D}{\dop t} \tilde\calJ &= C_\kappa' A + C_\kappa \underbrace{\frac{\D}{\dop t} A}_{=0} + S_\kappa' B + S_\kappa \underbrace{\frac{\D}{\dop t} B}_{=0} = C_\kappa' A + S_\kappa' B \\
	\ddot{\tilde{\calJ}} &= C_\kappa'' A + S_\kappa'' B
\end{align*}
Es gilt:
\begin{align*}
	C_\kappa'' = -\kappa C_\kappa && S_\kappa'' = -\kappa S_\kappa
\end{align*}
Daraus folgt $\ddot{\tilde{\calJ}} = -\kappa \tilde\calJ$ und damit $\tilde\calJ = \calJ$ (eindeutige L"osung zu gegebenen $\calJ(0)$, $\dot{\calJ}(0)$).
Der Beweis zeigt, dass parallele $A, B \perp \dot\gamma$ ein Jacobifeld definieren.
\end{Loes}

\begin{Loes}\begin{description}[leftmargin=*]
\item[$\bm{k > 0}$:]
	Sei $v \in \T_pM$ mit $\|v|| = 1$ und $\gamma(t) = \exp_p(tv)$.
	Da $M$ vollst"andig ist, ist $\gamma$ auf ganz $\R$ definiert.
	Es sei $B(0) \in v^\perp$ und $B$ die parallele Fortsetzung l"angs $\gamma$.
	Setze $\calJ(t) = S_\kappa(t) \cdot B(t)$.
	Daraus folg dass $\calJ$ ein Jacobifeld ist mit $\calJ(0) = \underbrace{S_\kappa(0)}_{=0} B(0) = 0$ und
	\begin{align*}
		\calJ \left( \frac{\pi}{\sqrt\kappa} \right) = \frac{1}{\kappa} \sin(\pi) \cdot B \left( \frac{\pi}{\sqrt\kappa} \right) = 0
	\end{align*}
	Daraus folgt dass $p$ und $\gamma(\frac{\pi}{\sqrt\kappa})$ konjugiert sind.
\item[$\bm{k \le 0}$:]
	Angenommen $p$ und $q$ sind konjugiert l"angs $\gamma$ (mit $\|\dot\gamma\| = 1$).
	Es sei $\gamma(0) = p$ und $\gamma(t_0) = q$ mit $t_0 \ge 0$.
	Dann gibt es ein Jacobifeld $\calJ \ne 0$ l"angs $\gamma$ mit $\calJ(0) = 0 = \calJ(t_0)$ und damit ist $\calJ$ orthogonal.
	Nach Aufgabe 1 gilt $\calJ = C_\kappa \cdot A + S_\kappa \cdot B$. Es gilt:
	\begin{align*}
		0 = \calJ(0) = C_\kappa(0) \cdot A(0) + S_\kappa(0) \cdot B(0) = A(0)
	\end{align*}
	Da $A$ parallel ist gilt $A \equiv 0$ und daraus folgt
	\begin{align*}
		0 = \calJ(t_0) = \underbrace{S_\kappa(t_0)}_{\mathclap{= \begin{cases} \scriptstyle{t_0} & \scriptsize{\text{falls }} \scriptstyle{k = 0} \\ \scriptstyle{\frac{1}{\sqrt{|\kappa|}} \sinh\left(\sqrt{|\kappa|}t_0\right)} & \scriptsize{\text{falls }} \scriptstyle{k < 0} \end{cases}}} \cdot B(t_0) > 0
	\end{align*}
	Daraus folt $B(t_0) = 0$ und, da $B$ parallel ist, $B \equiv 0$.
	Also ist $\calJ \equiv 0$, was einen Widerspruch darstellt. $\lightning$
\end{description}\end{Loes}

\begin{bem}
Der Beweis von Aufgabe 2 zeigt:
Eine vollst"andige Riemannsche Mannigfaltigkeit mit $\sec \equiv \kappa > 0$ hat Durchmesser $\le \frac{\pi}{\sqrt\kappa} = \diam(S^n(\sqrt\kappa) = \{ x \in \R^{n+1} \mid \|x\| = \sqrt\kappa\})$
\end{bem}

\begin{bsp}\begin{enumerate}[label=(\arabic*),leftmargin=*]
\item
	$S^n, \sec \equiv 1$; $p$ und $-p$ sind konjugiert:
	\begin{center}\begin{tikzpicture}[font=\scriptsize]
		%\tikzgitter{(-3,-3)}{(3,3)}
		\def\radius{1.5}
		\coordinate (p) at (220:\radius); \coordinate (q) at (80:\radius);
		
		\draw (0,0) circle (\radius);
		\begin{scope}
			\clip (-\radius,0) rectangle (\radius,\radius);
			\draw[dashed] (0,0) ellipse[x radius=\radius, y radius=0.5];
		\end{scope}\begin{scope}
			\clip (-\radius,0) rectangle (\radius,-\radius);
			\draw (0,0) ellipse[x radius=\radius, y radius=0.5];
		\end{scope}
		
		\def\angleLow{90}
		\def\angleHigh{210}
		
		\foreach \angle in {20,-30,-50,-70}{
			\draw[gray] (p) ..controls($(p) + (\angleLow+\angle:1)$) and ($(q) + (\angleHigh-\angle:1)$).. (q);
		}
		\draw[very thick] (p) ..controls($(p) + (\angleLow:1)$) and ($(q) + (\angleHigh:1)$)..coordinate[pos=0.7] (pkt) (q);
		
		\draw[->] (-2,1.5) node[left]{$\gamma$} to[out=0,in=140] (pkt);
		
		\fill (p) circle(0.05) node[below left]{$p$} (q) circle(0.05) node[above right]{$-p = \gamma(\pi)$};
		\node at (\radius,-1.25) {$S^{n-1}$};
	\end{tikzpicture}\end{center}
	und $p$ ist zudem zu sich selbst konjugiert ($\calJ(2\pi) = 0$), und dies sind alle zu $p$ konjugierten Punkte.
\item
	$\underline{\R\P^n} = \FakRaum{S^n}{q \sim (-q)}$ mit der von $S^n$ induzierten Metrik $\leadsto [p] = [-p]$. Daraus folgt dass $[p]$ der einzige zu $p$ konjugierte Punkt ist.
	\begin{center}\begin{tikzpicture}[font=\scriptsize,scale=0.8]
		%\tikzgitter{(-4,-4)}{(4,4)}
		\def\radius{2.5}
		\coordinate (vec) at (-1.25,-0.25);

		\begin{scope}
			\clip (-\radius-1,0) rectangle (\radius+1,\radius+1);
			\draw (0,0) circle(\radius);
		\end{scope}\begin{scope}
			\clip (-\radius,0) rectangle (\radius,\radius);
			\draw[dashed] (0,0) ellipse[x radius=\radius,y radius=0.75];
		\end{scope}\begin{scope}
			\clip (-\radius,0) rectangle (\radius,-\radius);
			\draw (0,0) ellipse[x radius=\radius,y radius=0.75];
		\end{scope}
		
		\coordinate (p) at (0,\radius); \coordinate (a) at (240:0.7*\radius); \coordinate (b) at ($-1*(a)$); \coordinate (pktA) at (235:2.5 and 0.75); \coordinate (pktB) at (70:2.5 and 0.75);
		\fill (p) circle(0.05) node[above]{$p$} (a) circle(0.05) (b) circle(0.05);
		
		\draw[decoration={markings,mark=at position 0.5 with{\arrow{>}}},postaction={decorate}] (a) -- (b);
		\draw (pktA) ..controls(pktA) and ($(p) + (vec)$).. (p) ..controls($(p) - 0.5*(vec)$) and (b).. (b);
		\draw[dashed] (a) -- (pktA) (b) -- (pktB);
	\end{tikzpicture}\end{center}
 \item
	$(\R^n, g_{\text{eukl}})$ parallele Vektorfelder $=$ konstante Vektorfelder.
	Damit haben die Jacobifelder die Form $\calJ(t) = A + t B$. (mit $A, B \in \R^n$)
\item
	$\H^2$
	\begin{center}\begin{tikzpicture}[font=\scriptsize]
		\fill[gray!20] (-2.5,0) rectangle (2.5,1.75);
		\draw[->] (-3,0) -- (3,0);
		\draw[->,decoration={markings,mark=at position 0.75 with{\arrow{>}}},postaction={decorate}] (0,0) --node[right,pos=0.75]{$\gamma$} (0,2);
		\fill (0,1) circle(0.05) node[right]{$i$};
	\end{tikzpicture}\end{center}
	\begin{align*}
		\gamma(t) &= i e^t \text{ Geod"atische (letzte "Ubung)}\\
		\dot\gamma(t) &= e^t \pdifffrac{}{y} = e^t \begin{pmatrix}0 \\ 1 \end{pmatrix}
	\end{align*}
	Es sei $A$ eine paralleles Vektorfeld l"angs $\gamma$. Dann gilt:
	\begin{align*}
		0 &\overset{!}{=} \nabla_t A = \nabla_t \begin{pmatrix} A_1 \\ A_2 \end{pmatrix} = \nabla_t \begin{pmatrix} A_1 \\ 0 \end{pmatrix} + \nabla_t \begin{pmatrix} 0 \\ A_2 \end{pmatrix} \\
		&= A_1' \begin{pmatrix} 1 \\ 0 \end{pmatrix} + A_1 \cdot \nabla_t \begin{pmatrix} 1 \\ 0 \end{pmatrix} + A_2' \begin{pmatrix} 0 \\ 1 \end{pmatrix} + A_2 \cdot \nabla_t \begin{pmatrix} 1 \\ 0 \end{pmatrix} \\
		&= \begin{pmatrix} A_1' \\ A_2' \end{pmatrix} + A_1 \cdot \nabla_{\dot\gamma} \begin{pmatrix} 1 \\ 0 \end{pmatrix} + A_2 \cdot \nabla_{\dot\gamma} \begin{pmatrix} 0 \\ 1 \end{pmatrix} \qquad \text{wobei } \nabla_{\dot\gamma}= e^t \begin{pmatrix}0\\1\end{pmatrix} \\
		&= \begin{pmatrix} A_1' \\ A_2' \end{pmatrix} + e^tA_1 \begin{pmatrix} \overbrace{\Gamma_{21}^1(\gamma(t))}^{= - \frac{1}{\gamma^2(t)} = \frac{-1}{e^t}} \\ \underbrace{\Gamma_{21}^2(\gamma(t))}_{=0} \end{pmatrix} + e^t A_2 \begin{pmatrix} \overbrace{\Gamma_{22}^1(\gamma(t))}^{=0} \\ \underbrace{\Gamma_{22}^2(\gamma(t))}_{= \frac{1}{e^t}} \end{pmatrix} \\
		&= \begin{pmatrix} A_1' - A_1 \\ A_2' + A_2 \end{pmatrix}
	\end{align*}
	Daraus folgt $A_1(t) = A_1(0) \cdot e^t$ und $A_2(t) = A_2(0) \cdot e^{-t}$.
\end{enumerate}\end{bsp}
%%
%% Skript Differentialgeometrie im Wintersemester 12/13
%% Zur Vorlesung von Dr. Grensing am KIT Karlsruhe
%%
%% Uebung 12
%%

\section{28. Januar 2012}
\setcounter{Aufg}{0} %Damit die Aufgaben jedes Mal bei Aufgabe 1 anfangen
\setcounter{Loes}{0}

\begin{Loes}
asdf
\end{Loes}
%%
%% Skript Differentialgeometrie im Wintersemester 12/13
%% Zur Vorlesung von Dr. Grensing am KIT Karlsruhe
%%
%% Uebung 12
%%

\section{4. Februar 2012}
\setcounter{Aufg}{0} %Damit die Aufgaben jedes Mal bei Aufgabe 1 anfangen
\setcounter{Loes}{0}

\begin{Aufg}
Es seien $2\leq p \in \N$ und $1\leq q_1,\dots,q_k<p$ zu $p$ teilerfremde natürliche Zahlen. Zeigen Sie, dass die Gruppe der $p$-ten Einheitswurzeln $E_p=\{z \in \C\;|\;z^p=1\}$ durch
	\[z.(z_1,\dots, z_k):=(z^{q_1}z_1,\dots z^{q_k} z_k)\]
frei und eigentlich diskontinuierlich auf $\mathrm{S}^{2k-1}=\{(z_1, \dots, z_k) \in \C^k \;|\;\sum_{i=1}^k |z_i|^2 =1\}$ operiert.

Die Quotientenmannigfaltigkeit $L(p,q_1,\dots, q_k)=\mathrm{S}^{2k-1}/E_p$ nach dieser Wirkung wird \emph{Linsenraum} vom Typ $(p,q_1,\dots,q_k)$ genannt.
\end{Aufg}

\begin{Aufg}
Zeigen Sie, dass es auf $\T^n$ keine Riemannsche Metrik mit positiver Schnittkrümmung gibt.
\end{Aufg}

\begin{Aufg}
Es seien $M, M_1,M_2$ zusammenhängende Riemannsche Mannigfaltigkeiten und $\pi_i:M \to M_i$ Riemannsche universelle Überlagerungen mit Decktransformationsgruppen $\Gamma_i$ ($i=1,2$). Zeigen Sie, dass $M_1$ und $M_2$ genau dann isometrisch sind, wenn es eine Isometrie $\hat{\phi}:M \to M$ gibt, so dass $\Gamma_1=\hat{\phi} \Gamma_2 \hat{\phi}^{-1}$.
\end{Aufg}

\begin{Loes}\begin{description}[leftmargin=*]
\item[Operation:]
	F"ur alle $z \in E_p$ ist $S^{2k-1} \to S^{2k-1}$, $x \mapsto z.x$ stetig und f"ur alle $z, \tilde z \in E_p$ und alle $x \in S^{2k-1}$ gilt $z.(\tilde z.x) = (z \cdot \tilde z).x$.
\item[Die Operation ist frei:]
	Es ist zu zeigen dass f"ur alle $x \in S^{2k-1}$ gilt $(E_p)_x = \{1\}$, beziehungsweise f"ur alle $z \in E_p \setminus \{1\}$ und alle $x \in S^{2k-1}$ gilt $z.x \ne x$, beziehungsweise dass f"ur alle $z \in E_p$ gilt dass wenn es ein $x \in S^{2k-1}$ mit $z.x$ gibt $z=1$ gelten muss.
	Es seien $E_p$ und $(z_1,\ldots ,z_k) \in S^{2k-1}$ mit
	\begin{align*}
		(z_1,\ldots , z_k) = z.(z_1,\ldots ,z_k) = (z^{q_1} z_1, \ldots , z^{q_k} z_k)
	\end{align*}
	Da $(z_1, \ldots , z_k) \in S^{2k-1}$ ist, existiert ein $j \in \{1, \ldots ,k\}$ mit $z_j \ne 0$, und daraus folgt $z^{q_j} = 1$.
	Es seien $a,b \in \Z$ mit $1 = a q_j + b p$. Es gilt dann
	\begin{align*}
		1 = 1^a \cdot 1^b = (z^{q_j})^a (z^p)^b = z^{a q_j + b p} = z^1 = z
	\end{align*}
\item[Die Operation ist eigentlich kontinuierlich]
	Die Gruppe ist endlich und alle endlichen Gruppen operieren eigentlich diskontinuierlich, dann bleibt zu zeigen dass f"ur alle $K \in S^{2k-1}$ die Menge $\{ z \in E_p \mid z.K \cap K \ne \emptyset \}$ endlich ist.
\end{description}\end{Loes}

Der Quotient $L(p, q_1, \ldots ,q_k)$ nach dieser Operation ist also eine Mannigfaltigkeit.
F"ur $z = e^{2 \pi i \frac{l}{p}}$ ist die induzierte Abbildung eine Drehung um den Winkel $\frac{2 \pi l}{p} q_j$ in der $x_{2j-1}$-$x_{2j}$-Ebene.
Damit ist die Operation bez"uglich der Standardmetrik isometrisch und daraus folgt dass $L(p, q_1, \ldots , q_k)$ eine $\sec > 0$-Metrik besitzt.
Im Fall $p = 2$ gilt $q_1 = \ldots = q_k = 1$ und es ist $L(2, 1, \ldots ,1) = \R\P^{2k-1}$.
Laut Vorlesung gilt: F"ur $k \ge 2$ ist $S^{2k-1} \to L(p, q_1, \ldots  q_k)$ die universelle "Uberlagerung.
Dann sind die Einheitswurzeln gerade die Decktransformationsgruppe, also folgt
\begin{align*}
	\pi_1(L(p, q_1, \ldots ,q_k)) \cong E_p \cong \Z_p \cong \FakRaum{\Z}{\Z_p}
\end{align*}

\begin{kor}
Jede endlich erzeugte abelsche Gruppe ist isomorph zur Fundamentalgruppe einer kompakten Riemannschen Mannigfaltigkeit mit nichtnegativer Schnittkr"ummung.
\end{kor}

\begin{bew}
Sei $G$ eine endlich erzeugte abelsche Gruppe.
Nach dem Hauptsatz "uber endlich erzeugte abelsche Gruppen gilt:
\begin{enumerate}[label=(\arabic*)]
\item
	$G \cong \Z^l \oplus \Z_{p_1} \oplus \ldots \oplus \Z_{p_n}$
\item
	$(M, g_M)$ und $(N, g_N)$ haben die Schnittkr"ummung $\sec \ge 0$ und damit hat $(M \X N, g_M \oplus g_N)$ auch $\sec \ge 0$ (wobei $(g_M \oplus g_N)(X_M + X_N, Y_M + Y_N) = g_M(X_M + Y_M) + g_N(X_N, Y_N)$)
\end{enumerate}
Daraus folgt:
\begin{align*}
	&\pi_1(T^n \X L(p_1, 1, \ldots ,1) \X \ldots \X L(p_n, 1, \ldots , 1)) \\
	&\cong \pi_1(T^n) \oplus \pi_1(L(p_1, 1, \ldots ,1)) \oplus \ldots \oplus \pi_1(L(p_1, 1, \ldots ,1)) \\
	&\cong \Z^n \oplus \Z_{p_1} \oplus \ldots \oplus \Z_{p_n} \cong G
\end{align*}
\end{bew}

Die Mannigfaltigkeiten $L(p, q_1, \ldots ,q_k)$ und $L(p, q_1', \ldots ,q_k')$ sind:\begin{itemize}
\item
	homotopie"aquivalent genau dann wenn $q_1 \cdot \ldots \cdot q_k \equiv \pm l^k q_1' \ldots q_k' (\mmod p)$ f"ur ein $l \in \Z_p$ \cite{olum}
\item
	hom"oomorph genau dann wenn es ein $l \in \Z_p$ und ein $\sigma \in S_k$ gibt sodass f"ur alle $i$ gilt $q_i \equiv \pm l q' \sigma(i) (\mmod p)$ \cite{brody1960}
\end{itemize}
Also handelt es sich um homotopie"aquivalente, aber nicht hom"oomorphe Mannigfaltigkeiten.

\begin{Loes}
Nach dem Korollar von Bonnet-Myers ist die Fundamentalgruppe eine vollst"andige Mannigfaltigkeit mit $\ric \ge (n-1) \kappa > 0$ endlich.
$\pi_1(T^n) = \Z^n$ nicht.
\end{Loes}

\begin{Loes}
Wir k"urzen mit (*) die rechte Seite der Behauptung ab, also
\begin{align*}
	\text{Es gibt eine Isometrie } \hat\phi: M \to M \text{, sodass } \Gamma_1 = \hat\phi \Gamma_2 \hat\phi^{-1} \tag{*}
\end{align*}
\begin{description}[leftmargin=*]
\item[\quot{$\bm{\Leftarrow}$}:]
	Es sei $\hat\phi: M \to M$ eine Isometrie mit (*).
	\marginnote{\begin{tikzpicture}[font=\scriptsize]
		\node (1) at (-1,0.75) {$M$};
		\node (2) at (1,0.75) {$M$};
		\node (3) at (-1,-0.75) {$M_2$};
		\node (4) at (1,-0.75) {$M_1$};
		\draw[->] (1) --node[above]{$\hat\phi$} (2);
		\draw[->,dashed] (3) --node[below]{$\phi$} (4);
		\draw[->] (1) --node[left]{$\pi_2$} (3);
		\draw[->] (2) --node[right]{$\pi_1$} (4);
		\draw[->] (180:0.25) arc (180:-90:0.25);
		\node at (0.4,0) {?};
	\end{tikzpicture}}
	Definiere die Abbildung
	\begin{align*}
		\phi: M_2 \to M_1 && \phi(\overset{\mathclap{\text{Bahn}}}\Gamma_2 x) = \Gamma_1 \hat\phi(x)
	\end{align*}
	\begin{description}[leftmargin=*,font=\normalfont\itshape]
	\item[Behauptung:]
		$\phi$ ist wohldefiniert
	\item[Beweis:]
		Es ist zu zeigen dass f"ur alle $\gamma_2 \in \Gamma_2$ es ein $\gamma_1 \in \Gamma_1$ gibt mit $\hat\phi(\gamma_2 x) = \gamma_1 \hat\phi(x)$.
		F"ur $\gamma_2 \in \Gamma_2$ ist $\hat\phi \circ \gamma_2 \circ \hat\phi^{-1} =: \gamma_1 \in \Gamma_1$. Dann folgt $\gamma_1(\hat\phi(x)) = \hat\phi(\gamma_2(x))$.
	\end{description}
	Die gleiche Konstruktion f"ur $\hat\phi^{-1}$ liefert $\phi^{-1}$.
	Da $\pi_i$ ein lokaler Diffeomorphismus ist folgt dass $\phi$ und $\phi^{-1}$ glatt sind.
	Da $\phi$ und $\pi_i$ lokale Isometrien sind und $\pi_2$ surjektiv ist, ist $\phi$ eine lokale Isometrie, und weil $\phi$ ein Diffeomorphismus ist folgt dass $\phi$ eine Isometrie ist.
\item[\quot{$\bm{\Rightarrow}$}:]
	Es sei $\phi: M_1 \to M_2$ eine Isometrie.
	\marginnote{\begin{tikzpicture}[font=\scriptsize]
		\node (1) at (-1,0.75) {$M$};
		\node (2) at (1,0.75) {$M$};
		\node (3) at (-1,-0.75) {$M_2$};
		\node (4) at (1,-0.75) {$M_1$};
		\draw[->,dashed] (1) -- (2);
		\draw[->] (3) --node[below]{$\phi$} (4);
		\draw[->] (1) --node[left]{$\pi_2$} (3);
		\draw[->] (2) --node[right]{$\pi_1$} (4);
	\end{tikzpicture}}
	Dann ist $\phi \circ \pi_2$ eine lokale Isometrie, da $\phi$ ein Diffeomorphismus ist und $\pi_2$ eine "Uberlagerung, und damit ist $\phi \circ \pi_2$ eine Riemannsche "Uberlagerung.
	Aus $\pi_1(M) = \{0\}$ folgt dass $\phi \circ \pi_1$ die universelle "Uberlagerung ist; diese ist nach der Vorlesung bis auf Isomorphie eindeutig.
	Das bedeutet es existiert eine Isometrie $\hat\phi: M \to M$ mit $\pi_1 \circ \hat\phi = \phi \circ \pi_2$.
	\begin{description}[leftmargin=*,font=\normalfont\itshape]
	\item[Behauptung:] $\Gamma_1 = \hat\phi \Gamma_2 \hat\phi^{-1}$
	\item[Beweis:] \begin{description}[leftmargin=*,font=\normalfont]
		\item[\quot{$\subseteq$}:]
			\begin{align*}
				\pi_1 \circ \hat\phi \circ \gamma_2 \circ \hat\phi^{-1} = \phi \circ \pi_2 \circ \gamma_2 \circ \hat\phi^{-1} \overset{\gamma_1 \in \Gamma_2}{=} \phi \circ \pi_2 \circ \hat\phi^{-1} = \pi_1 \circ \hat\phi \circ \hat\phi^{-1} = \pi_1
			\end{align*}
		\item[\quot{$\supseteq$}:]
			\begin{align*}
				\pi_2 \circ \hat\phi^{-1} \circ \gamma_1 \circ \hat\phi = \phi^{-1} \circ \pi_1 \circ \gamma_1 \circ \hat\phi = \phi^{-1} \circ \pi_1 \circ \hat\phi = \pi_2 \circ \hat\phi^{-1} \circ \hat\phi = \pi_2
			\end{align*}
			Daraus folgt
			\begin{align*}
				\gamma_2 = \hat\phi^{-1} \circ \gamma_1 \circ \hat\phi \in \Gamma_2 && \gamma_1 = \hat\phi \circ \gamma_2 \circ \hat\phi^{-1} \in \hat\phi \Gamma_2 \hat\phi^{-1}
			\end{align*}
		\end{description}
	\end{description}
\end{description}\end{Loes}

%% 
%% Stichwortverzeichnis
%% 
\printindex

%% 
%% Glossar
%% 
\printglossaries

%% 
%% Bibliographie
%% 
\bibliographystyle{plain}
\bibliography{literature}

\end{document}