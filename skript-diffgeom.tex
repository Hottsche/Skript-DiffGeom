%%
%% Skript Differentialgeometrie im Wintersemester 12/13
%% Zur Vorlesung von Dr. Grensing am KIT Karlsruhe
%%
%% Mitschrieb und Textsatz von Jan-Bernhard Kordaß.
%%

\documentclass[a4paper, twoside, 11pt]{scrartcl}

\usepackage[utf8x]{inputenc}
\usepackage[T1]{fontenc}
\usepackage{lmodern}

\usepackage[ngerman]{babel}
\usepackage[a4paper, top=2.5cm, bottom=3cm, left=2.5cm, right=5cm]{geometry}

\usepackage{fancyhdr}

\usepackage{marginnote}

\usepackage{canonicalsync}
\CsUsePackage[/home/JB/Projects/tex-package-canonical-sync/]{canonicalsync}
\CsUsePackageWithOptions[/home/JB/Projects/tex-package-canonical-math/]{canonicalmath}{marginthm}

\usepackage{graphicx}
\usepackage{float}
\usepackage{xcolor}
\usepackage{transparent}
\usepackage{wrapfig}

\graphicspath{{img/}}

\parindent0pt

\setlist[enumerate]{label=(\arabic*), itemsep=0cm, leftmargin=1cm}

\begin{document}

\CmHeadline{Vorlesung Differentialgeometrie}{Gehalten von Dr. Grensing}{Wintersemester 2012/13}

\vspace{0.5cm}

Version \textbf{0.01} \quad Build: \today

\paragraph{Wichtiger Hinweis:}
Dies ist eine Zusammenfassung der Vorlesung "`Differentialgeometrie"' von Dr. Grensing im Wintersemester 2012/13 am KIT und dient lediglich dazu die Inhalte für meine eigene Verwendung besser zusammenzufassen und aufzubereiten. Es besteht weder eine Garantie über Vollständigkeit, noch Korrektheit der enthaltenen Aussagen.\\

Bei Anmerkungen bzw. beim Auffinden von Fehlern schicken Sie bitte eine E-Mail an
\begin{center}
  jan-bernhard.kordass@student.kit.edu
\end{center}

%%
%% 1. Vorlesung <2012-10-16 Tue>
%% 
%% Skript Differentialgeometrie im Wintersemester 12/13
%% Zur Vorlesung von Dr. Grensing am KIT Karlsruhe
%%
%% Mitschrieb und Textsatz von Jan-Bernhard Kordaß.
%%

\section*{"Ubersicht}

\begin{itemize}
\item Mannigfaltigkeiten, Tangentialvektoren
\item Kovariante Ableitung
\item Riemannsche Metriken
\item Krümmung
\item Jacobifelder
\item Satz von Bonnet
\end{itemize}

\chapter{Differenzierbare Mannigfaltigkeiten}

\begin{dfn}
  Eine $n$-dimensionale \CmMark[Mannigfaltigkeit!topologische]{topologische Mannigfaltigkeit} $M$ ist ein topologischer \gls{Hausdorff-Raum} mit einer abzählbaren Basis der \gls{Topologie} in dem zu jedem Punkt $p \in M$ eine offene Menge $U$ mit $p \in U$ existiert und ein \gls{Homoeomorphismus} $\phi \colon U \to V$ auf eine offene Menge $V \subset \R^{n}$.

% Abbildung 1-1
%\CmPutSvg{1-1-topologische-mf}{8.5cm}
\begin{center}\begin{tikzpicture}[font=\scriptsize]
	\draw[->] (-1.5,0) to[out=20, in=160]node[above,font=\scriptsize]{$\phi' \circ \phi^{-1}$} (1.5,0);
	
	\draw[->] (-4,-0.5) -- (-2,-0.5); \draw[->] (-3.75,-0.75) -- (-3.75, 1.25); \node[font=\scriptsize] at (-4, 1.25) {$\R^n$};
	\draw[->] (2,-0.5) -- (4,-0.5); \draw[->] (2.25,-0.75) -- (2.25, 1.25); \node[font=\scriptsize] at (2, 1.25) {$\R^m$};
	
	\node[font=\scriptsize] at (0,2) {$U \cap U' \ne 0$};
	
	\draw (-4.25, 1.75) to[out=70,in=180] (-1.75,3) to[out=300,in=90] (-1.25, 1.25) to[out=180,in=340] (-4.25, 1.75) -- cycle; \node at (-1.25,3) {$M$};
	\filldraw[fill=gray!20] (-2.75,2) circle(0.4); \node[font=\scriptsize] at (-3.25,2.25) {$U$};
	\filldraw[fill=gray!20] (-3,0.25) circle (0.5); \node at (-2.25, 0.5) {$V$};
	\draw[->] (-2.75,1.5) to[out=280,in=80] node[right]{$\phi$} (-2.75,0.75);
			
	\draw (1.75, 1.75) to[out=70,in=180] (4.25,3) to[out=300,in=90] (4.75, 1.25) to[out=180,in=340] (1.75, 1.75) -- cycle; \node at (4.75,3) {$M$};
	\filldraw[fill=gray!20] (3.55,2.25) circle(0.6); \node[font=\scriptsize] at (2.75,2.25) {$U'$};
		
	\coordinate (ctrl0up) at ($(2.5,-0.25) + 0.2*(0.5,2)$); \coordinate (ctrl0down) at ($(2.5,-0.25) + 0.2*(0,-1.5)$);
	\coordinate (ctrl1down) at ($(3,0.25) - 0.1*(0.5,1)$); \coordinate (ctrl1up) at ($(3,0.25) + 0.1*(0.5,1)$);
	\coordinate (ctrl2down) at ($(3,0.7) - 0.1*(0.5,1)$); \coordinate (ctrl2up) at ($(3,0.7) + 0.1*(0.5,1)$);
	\coordinate (ctrl3down) at ($(4,0.5) + 0.3*(-0.5,1)$); \coordinate (ctrl3up) at ($(4,0.5) - 0.3*(-0.25,1)$);
	\coordinate (ctrl4down) at ($(3.75,-0.3) + 0.2*(0.8,1)$); \coordinate (ctrl4up) at ($(3.75,-0.3) - 0.2*(0.7,0.75)$);
	\begin{scope}
		\fill[gray!20] (2.5,-0.25) ..controls(ctrl0up) and (ctrl1down).. (3,0.25) ..controls(ctrl1up) and (ctrl2down).. (3,0.7) ..controls(ctrl2up) and (ctrl3down).. (4,0.5) ..controls(ctrl3up) and (ctrl4down).. (3.75,-0.3) ..controls(ctrl4up) and (ctrl0down).. (2.5,-0.25); \node at (4.25, 0.5) {$V'$};
		\clip(2.5,-0.25) ..controls(ctrl0up) and (ctrl1down).. (3,0.25) ..controls(ctrl1up) and (ctrl2down).. (3,0.7) ..controls(ctrl2up) and (ctrl3down).. (4,0.5) ..controls(ctrl3up) and (ctrl4down).. (3.75,-0.3) ..controls(ctrl4up) and (ctrl0down).. (2.5,-0.25); \node at (4.25, 0.5) {$V'$};
		\fill[gray] (2,0) circle (1);
		 (2.5,-0.25) ..controls(ctrl0up) and (ctrl1down).. (3,0.25) ..controls(ctrl1up) and (ctrl2down).. (3,0.7) ..controls(ctrl2up) and (ctrl3down).. (4,0.5) ..controls(ctrl3up) and (ctrl4down).. (3.75,-0.3) ..controls(ctrl4up) and (ctrl0down).. (2.5,-0.25); \node at (4.25, 0.5) {$V'$};
	\end{scope}
	\draw  (2.5,-0.25) ..controls(ctrl0up) and (ctrl1down).. (3,0.25) ..controls(ctrl1up) and (ctrl2down).. (3,0.7) ..controls(ctrl2up) and (ctrl3down).. (4,0.5) ..controls(ctrl3up) and (ctrl4down).. (3.75,-0.3) ..controls(ctrl4up) and (ctrl0down).. (2.5,-0.25) -- cycle; \node at (4.25, 0.5) {$V'$};
	\draw[->] (3.5,1.5) to[out=280,in=80] node[right]{$\phi'$} (3.5,0.75);
\end{tikzpicture}\end{center}

  $\phi' \circ \phi^{-1}$ ist ein Hom"oomorphismus offener Mengen des $\R^n$ bzw. $\R^m$. Nach dem \href{http://de.wikipedia.org/wiki/Fixpunktsatz_von_Brouwer}{Satz von Brouwer} (1912) gilt dann $m = n$. Damit ist die Dimension einer zusammenh"angenden topologischen Mannigfaltigkeit eindeutig definiert.\\

  Die Abbildung $\phi \colon U \to V \subset \R^n$ hei\ss t \CmMark{Karte} von $M$ um $p$, die Menge $U$ hei\ss t \CmMark{Kartengebiet}.\\

  Eine Menge von Karten $\mathcal A = \{(\phi_{\alpha}, U_{\alpha}) \mid \alpha \in J \}$ hei\ss t \CmMark{Atlas} von $M$, falls $\bigcup_{\alpha \in J}U_{\alpha} = M$.\\

  Ein Atlas $\mathcal A$ von $M$ hei\ss t $C^k$-Atlas, wenn für alle $\alpha, \beta \in J$ mit $U_{\alpha} \cap U_{\beta} \neq \emptyset$ der sogenannte \CmMark{Kartenwechsel}:
  \begin{align*}
    \phi_{\beta} \circ \phi_{\alpha}^{-1}\colon \phi_{\alpha}(U_{\alpha} \cap U_{\beta}) \to \phi_{\beta}(U_{\alpha} \cap U_{\beta})
  \end{align*}
  ein $C^k$-\gls{Diffeomorphismus} ist.

  \begin{center}\begin{tikzpicture}[font=\scriptsize]
  	\draw[->] (-1.5,0) to[out=20, in=160]node[above,font=\scriptsize]{$\phi_\beta \circ \phi^{-1}_\alpha$} (1.5,0);
	
	\draw[->,thick] (-4,-0.5) -- (-2,-0.5); \draw[->,thick] (-3.75,-0.75) --node[left]{$\R^n$} (-3.75, 1.25);
	\draw[->,thick] (2,-0.5) -- (4,-0.5); \draw[->,thick] (2.25,-0.75) -- (2.25, 1.25);
	
	\draw[thick]  (-0.25, 3) to[out=0,in = 150] (2,2.5) -- (1.75, 1.5) to[out=190,in=350] (-1.75, 1.5) to[out=90,in=180] (-0.25, 3) -- cycle; \node at (2.25,2.75) {$M$};
	
	\begin{scope}
		\clip (0.25,2.25) circle(0.5);
		\clip (-0.25,2) circle(0.5);
		\fill[gray!20] (0,2) circle(1);
	\end{scope}
	\draw (0.25,2.25) circle(0.5) (-0.25,2) circle(0.5); \node at (-1, 2.25) {$U_\alpha$}; \node at (1,2.5) {$U_\beta$};
	
	\draw[->] (-0.5,2) to[out=180,in=75] node[left]{$\phi_\alpha$} (-3,0.25);
	\draw[->] (0.5,2.25) to[out=0,in=105] node[right]{$\phi_\beta$} (3,0.25);
  \end{tikzpicture}\end{center}


  Eine Karte $\psi \colon U \to V$ von $M$ hei\ss t \CmMark[Karte!vertr{\"a}gliche]{verträglich} mit einem $C^k$-Atlas $\mathcal A = \{(\phi_{\alpha},U_{\alpha}) \mid \alpha \in J\}$ wenn jeder Kartenwechsel
  \begin{align*}
    \phi_{\alpha} \circ \psi^{-1}: \psi(U \cap U_{\alpha}) \to \phi_{\alpha}(U \cap U_{\alpha})
  \end{align*}
  ein $C^k$-Diffeomorphismus ist, also $\mathcal A' = \mathcal A \cup \{(\psi, U)\}$ ebenfalls ein $C^k$-Atlas ist. Die Menge aller mit $\mathcal A$ verträglichen Karten ist ein \CmMark[Atlas!maximaler]{maximaler $C^k$-Atlas}. Jeder maximale Atlas enthält alle mit ihm verträglichen Karten. Ein maximaler $C^k$-Atlas hei\ss t auch \CmMark{$C^k$-differenzierbare Struktur}.

\end{dfn}

\begin{Dfn}[differenzierbare Mannigfaltigkeit]
  Eine \CmMark[Mannigfaltigkeit!differenzierbare]{differenzierbare Mannigfaltigkeit} der Klasse $C^k$ ist eine topologische Mannigfaltigkeit zusammen mit einer $C^{k}$-differenzierbaren Struktur.\\
\end{Dfn}

\begin{bsp}
  Einige Beispiele f"ur glatte Mannigfaltigkeiten:
  \begin{enumerate}[leftmargin=*,label=\arabic*)]
  \item $M = \R^n, \mathcal A = \{(\Id_{\R^n},\R^n)\}$
  \item $M \subset \R^n$ offen, $\mathcal A = \{(\imath_{M},M)\}$
  \item $S^1 \subset \R^2$ ist eine eindimensionale $C^{\infty}$-Mannigfaltigkeit:
    \begin{align*}
      U = \{(\sin t, \cos t) \mid t \in (0,2\pi)\}
    \end{align*}

    % Abbildung 1-3
    \marginnote{\begin{center}\begin{tikzpicture}[font=\footnotesize]
    		%\draw[step=0.25,gray!15] (-1,-1) grid (1,1); \draw[step=0.5,gray!30] (-1,-1) grid (1,1); \fill (0,0) circle(0.1); %Hilfsgitter
		\draw (0,0) circle (1); \draw[dashed] (0,0) circle (1.1); \draw[dotted] (0,0) circle (0.9); \node at (1,1) {$S^1$};
		\filldraw[fill=white] (-1,0) circle (0.1) (1,0) circle (0.1);
    \end{tikzpicture}\\
    \textcolor{gray}{$S^1$ Einheitskreis}
    \end{center}}[-2cm]
    % \CmMarginSvg[-2cm]{1-3-karten-der-s1}{3cm}

    ist offen in $S^1$ und die Kartenabbildung
    \begin{align*}
      \phi \colon (\sin t, \cos t) \mapsto t
    \end{align*}
    ist ein Hom"oomorphismus.
    \begin{align*}
      \phi' \colon U' = \{(\sin t, \cos t) \mid t \in (-\pi,\pi)\} \to (-\pi,\pi)
    \end{align*}
    ebenfalls. $\mathcal A = \{(\phi, U), (\phi',U')\}$ ist ein Atlas von $S^1$, denn $U \cup U' = S^1$.
    \begin{align*}
      & \phi' \circ \phi^{-1} \colon \phi(U \cap U') \to \phi'(U \cap U')\\
      & (0,\pi)\cup(\pi,2\pi) \to (-\pi,0)\cup(0,\pi), t \mapsto \begin{cases}
        t & 0 < t < \pi\\
        t-2\pi & \pi < t < 2\pi
      \end{cases}
    \end{align*}

  \item Jeder reelle Vektorraum endlicher Dimension ist in kanonischer Weise eine $C^{\infty}$-Mannigfaltigkeit.\\
    W"ahle eine Basis $\{v_1, \ldots, v_n\}$ von $V$. Diese definiert mit
    \begin{align*}
      \phi\left(\sum\lambda_iv_i\right) = (\lambda_1, \ldots, \lambda_n)
    \end{align*}
    eine Bijektion auf $\R^n$. Damit erhält man eine globale Karte von $V$.
    Der zugehörige Atlas h"angt nicht von der Wahl der Basis ab, denn ist $\{w_1, \ldots, w_n\}$ eine weitere Basis von $V$ und $\psi(\sum \lambda_iw_i) = (\lambda_1, \ldots, \lambda_n)$ eine weitere Karte, so ist $\phi \circ \psi^{-1}$ als \gls{Endomorphismus} des $\R^n$ schon $C^{\infty}$.

  \item $S^n = \{(x^0, x^1, \ldots, x^n) \mid \sum_{i = 0}^n(x^{i})^2 = 1\}$.\\

    % Abbildung 1.4
    %\CmMarginSvg{1-4-s3-sphaere}{3.5cm}
    \marginnote{\begin{center}\begin{tikzpicture}[font=\scriptsize]
    		%\draw[step=0.25,gray!15] (-1,-1) grid (1,1); \draw[step=0.5,gray!30] (-1,-1) grid (1,1); \fill (0,0) circle(0.1); %Hilfsgitter
		% Koordinatenachsen mit Beschriftung
		\draw[->] (0,-1.25) -- (0,1.5) node[left]{$x^0$}; \draw[->] (-1.25,0) -- (1.5,0) node[below]{$x^1$}; \draw[->] (1,1) -- (-1.25,-1.25) node[right]{$x^2$}; \node at (1.25, 1.5) {$S^2 \subset \R^3$};
		% Kreis, Ellipse und Gerade (verwende Namen um Schnittpunkt bestimmen zu koennen)
		\path[draw, thick, name path=kreis] (0,0) circle (1) ellipse(1 and 0.45); \path[draw,name path=gerade] (0,1) -- (1,-1.25);
		% Punkte N und p
		\filldraw[fill=white] (0,1) circle (0.05) node[anchor=south west,xshift=-2,yshift=-1.5]{$N$} ($(0,1)+0.35*(1,-1.25)-0.35*(0,1)$) circle (0.05) node[right]{$p$};
		% Punkt phi(p) bei Schnittpunkt von Gerade und Kreis
		\path [name intersections={of=kreis and gerade}]; \filldraw[fill=white] (intersection-2) circle(0.05) node[right]{$\phi(p)$};
	\end{tikzpicture}\end{center}}%[3.5cm]
    
    Betrachte den Nordpol $N = (1,0,\ldots,0)$ und den S"udpol $S = (-1,0,\ldots,0)$ und die Abbildung
    \begin{align*}
      & \phi \colon U = S^{n}\setminus\{N\} \to \R^n, x \mapsto \left(\frac{x^1}{1-x^0}, \ldots, \frac{x^{n}}{1-x^0}\right),\\
      & \psi \colon U' = S^{n} \setminus \{S\} \to \R^n, x \mapsto \left(\frac{x^1}{1+x^0}, \ldots, \frac{x^n}{1+x^0}\right)
    \end{align*}

    Aufgabe: Zeige, dass $(\phi, U), (\psi, U')$ einen $C^{\infty}$-Atlas auf $S^n$ definiert.

  \end{enumerate}
\end{bsp}

\begin{Dfn}[Differenzierbare Abbildungen]
Eine stetige Abbildung $f \colon M \to N$ zwischen glatten Mannigfaltigkeiten $M$ und $N$ hei\ss t \CmMark{glatt} ($C^{\infty}$-differenzierbar), wenn es zu jedem $p \in M$ Karten $(\phi, U)$ in $M$ um $p$ und geeignete $(\phi', U')$ in $N$ um $f(p)$ gibt, so dass $\phi' \circ f\circ\phi^{-1}$ glatt ist.
% Abbildung 1-5
%\CmPutSvg{1-5-glatte-abb}{9cm}
\begin{center}\begin{tikzpicture}[font=\scriptsize]
	%\draw[step=0.25,gray!15] (-5,-1) grid (5,5); \draw[step=0.5,gray!30] (-5,-1) grid (5,5); \fill (0,0) circle(0.1); %Hilfsgitter
	
	% Die Abbildungspfeile
	\draw[->] (-1.5,0) to[out=20, in=160]node[above]{$\phi' \circ f \circ \phi^{-1}$} (1.5,0);
	\draw[->] (-1,2) --node[above]{$f$} (1.5,2);
	
	% Die Achsen
	\draw[->,thick] (-4.5,-0.5) -- (-2,-0.5); \draw[->,thick] (-4.25,-0.75) --node[left]{$\R^n$} (-4.25, 1.25);
	\draw[->,thick] (2,-0.5) -- (4.5,-0.5); \draw[->,thick] (2.25,-0.75) --node[left]{$\R^m$} (2.25, 1.25);
	
	% Die Blasen
	\draw[thick] (-4.25, 1.75) to[out=70,in=180] (-1.75,3) to[out=300,in=90] (-1.25, 1.25) to[out=180,in=340] (-4.25, 1.75) -- cycle; \node[font=\normalfont] at (-1.25,3) {$M$};
	\draw[thick] (1.75, 1.75) to[out=70,in=180] (4.25,3) to[out=300,in=90] (4.75, 1.25) to[out=180,in=340] (1.75, 1.75) -- cycle; \node[font=\normalfont] at (4.75,3) {$N$};
	
	% Die linke Kartoffel (zuerst werden die Punkte definiert, dann die Richtungsvektoren der Splines, dann die Kartoffel selbst)
	\coordinate (kartoffel0l) at (-3.25,1.75); \coordinate (kartoffel1l) at (-3.25,2.5); \coordinate (kartoffel2l) at (-2.25,2.25); \coordinate (kartoffel3l) at (-2.5,1.75);
	\coordinate (ctrlk0l) at (-0.25,0.5); \coordinate (ctrlk1l) at (0.5,0.25); \coordinate (ctrlk2l) at (-0.25,1); \coordinate (ctrlk3l) at (2,0.25);
	\draw (kartoffel0l) ..controls($(kartoffel0l)+0.5*(ctrlk0l)$) and ($(kartoffel1l)-0.3*(ctrlk1l)$).. (kartoffel1l) ..controls($(kartoffel1l)+0.6*(ctrlk1l)$) and($(kartoffel2l)+0.45*(ctrlk2l)$).. (kartoffel2l) ..controls($(kartoffel2l)-0.25*(ctrlk2l)$) and ($(kartoffel3l)+0.15*(ctrlk3l)$).. (kartoffel3l)  ..controls($(kartoffel3l)-0.1*(ctrlk3l)$) and ($(kartoffel0l)-0.9*(ctrlk0l)$).. (kartoffel0l); \node at (-3.5,2.25) {$U$};
	% Der Punkt in der Kartoffel, der Pfeils raus und der Kreis
	\draw[->] (-2.75,2) node[right]{$p$} to[out=280,in=80] node[right]{$\phi$} (-2.75,0); \fill (-2.75,2) circle (0.05);
	\draw (-3,0.25) circle(0.5); \node at (-3.5,0.75) {$V$};
	
	% Die rechte Kartoffel
	\coordinate (kartoffel0r) at (3.25,1.75); \coordinate (kartoffel1r) at (3.5,2.5); \coordinate (kartoffel2r) at (4.5,2.25); \coordinate (kartoffel3r) at (4.25,1.5);
	\coordinate (ctrlk0r) at (-0.25,0.5); \coordinate (ctrlk1r) at (-0.25,0.25); \coordinate (ctrlk2r) at (-0.25,0.5); \coordinate (ctrlk3r) at (0.25,0);
	\draw (kartoffel0r) ..controls($(kartoffel0r)+0.5*(ctrlk0r)$) and ($(kartoffel1r)-(ctrlk1r)$).. (kartoffel1r) ..controls($(kartoffel1r)+(ctrlk1r)$) and ($(kartoffel2r)+(ctrlk2r)$).. (kartoffel2r) ..controls($(kartoffel2r)-(ctrlk2r)$) and ($(kartoffel3r)+(ctrlk3r)$).. (kartoffel3r) ..controls($(kartoffel3r)-(ctrlk3r)$) and ($(kartoffel0r)-(ctrlk0r)$).. (kartoffel0r); \node at (3.25,2.25) {$U'$};
	
	\draw[->] (3.75,2) node[right]{$f(p)$} to[out=280,in=80] node[right]{$\phi'$} (3.75,0); \fill (3.75,2) circle (0.05);
	\draw (3.5,0.25) circle(0.5); \node at (3,0.75) {$V'$};
\end{tikzpicture}

% Korrektur
% \textcolor{red}{Sollte das in der Zeichnung beim unteren Pfeil nicht $\phi'\circ f \circ \phi^{-1}$ hei\ss en?}
% Ja! sollte es! (JB, <2012-11-9 Fri>)

\end{center}

Die Menge aller glatten Abbildungen von $M$ nach $N$ wird $C^{\infty}(M,N)$ genannt.
\end{Dfn}

\begin{emptythm}[Konvention:]
Ab jetzt seien zunächst alle Mannigfaltigkeiten, wie auch alle Abbildungen als glatt vorrausgesetzt.
\end{emptythm}

\begin{bem}
  Da Kartenwechsel $C^{\infty}$ sind, gilt obige Bedingung automatisch für alle Karten von $M$ und $N$ (evtl. nach Einschränkung).
\end{bem}

\begin{bsp}
  Es folgen zwei Beispiele für differenzierbare Abbildungen:
  \begin{enumerate}
  \item $(\phi,U) \in \mathcal A \Rightarrow \phi \in C^{\infty}(U,\R^n)$, denn
    \begin{align*}
      \Id_{\R^n}\circ \phi \circ \phi^{-1} = \phi \circ \phi^{-1} \in C^{\infty}.
    \end{align*}
  \item $f \in C^{\infty}(M,N), \ g \in C^{\infty}(N,P) \Rightarrow g \circ f \in C^{\infty}(M,P)$, denn
    \begin{align*}
      \phi_p \circ g \circ f \circ \phi^{-1}_m = (\phi_p \circ g \circ \phi_n^{-1}) \circ (\phi_n \circ f \circ \phi_m^{-1}) \in C^{\infty}.
    \end{align*}
  \end{enumerate}
\end{bsp}

\begin{Dfn}[Diffeomorphismus]
Eine Abbildung $f \colon M \to N$ hei\ss t \CmMark{Diffeomorphismus}, wenn $f$ bijektiv ist und $f$, sowie $f^{-1}$ $C^{\infty}$-Abbildungen von $M$ nach $N$ sind. Insbesondere haben $M$ und $N$ in diesem Fall dieselbe Dimension. Die Menge der Diffeomorphismen von $M$ nach $M$ wird mit $\Diff(M)$ bezeichnet. $(\Diff(M), \circ)$ ist bez"uglich der Hintereinanderausf"uhrung eine Gruppe.
\end{Dfn}


% 2. Vorlesung <2012-10-19 Fri>

\section{Produkte von Mannigfaltigkeiten}

Es seien $M$ und $N$ glatte Mannigfaltigkeiten der Dimensionen $m$ und $n$. Dann hat $M \times N$ versehen mit der \gls{Produkttopologie}, die Struktur einer Mannigfaltigkeit. Da $M$ und $N$ hausdorffsch sind und abzählbare Basen ihrer Topologie besitzen gilt dies auch für $M \times N$.
Sind $(\phi, U)$ und $(\psi, V)$ Karten von $M$ bzw. $N$, so ist $\phi \times \psi$ ein Homöomorphismus von $U \times V$ auf sein offenes Bild in $\R^m \times \R^n \cong \R^{m + n}$.

Seien $\mathcal A = \{(\phi_{\alpha}, U_{\alpha}) \mid \alpha \in \calI \}$ und $\mathcal A' = \{(\psi_{\beta}, V_{\beta}) \mid \beta \in \calJ\}$ $C^{\infty}$-Atlanten von $M$ und $N$. Dann ist $\mathcal B = \{(\phi_{\alpha}\times\psi_{\beta},U_{\alpha}\times V_{\beta}) \mid (\alpha, \beta) \in \calI \times \calJ\}$ ein $C^{\infty}$-Atlas von $M\times N$, denn 
\begin{align*}
  (\phi_{\alpha} \times \psi_{\beta}) \circ (\phi_{\mu} \times \psi_{\nu})^{-1} = (\phi_{\alpha} \circ \phi_{\mu}^{-1}) \times (\psi_{\beta} \circ \psi_{\nu}^{-1})
\end{align*}
ist ein $C^{\infty}$-Diffeomorphismus. Damit ist $M\times N$ in kanonischer Weise eine glatte $(m+n)$-dimensionale Mannigfaltigkeit. Die kanonischen Projektionen $\pi_M\colon M\times N \to M, \ \pi_N\colon M \times N \to N$ und die Abbildung $\tau \colon M \times N \to N \times M, (p,q) \mapsto (q,p)$ sind glatte Abbildungen.

\begin{bsp}\marginnote{\begin{tikzpicture}[scale=0.9,font=\scriptsize]
	%\draw[step=0.25,gray!15] (0,-2) grid (3,2); \draw[step=0.5,gray!30] (0,-2) grid (3,2); \fill (0,0) circle(0.1); %Hilfsgitter
	\tikztorus{(0,0)};
	\draw (0,0) ellipse (1.5 and 0.7*\torushoehe);
	\begin{scope}
		\clip (torusUntenLoch) rectangle ($(torusUnten)-(0.25,0)$);
		\draw ($0.5*(torusUntenLoch) + 0.5*(torusUnten)$) ellipse (0.2 and 0.5*\torusdicke);
	\end{scope}\begin{scope}
		\clip (torusUntenLoch) rectangle ($(torusUnten)+(0.25,0)$);
		\draw[dashed] ($0.5*(torusUntenLoch) + 0.5*(torusUnten)$) ellipse (0.2 and 0.5*\torusdicke);
	\end{scope}
	\node[below,font=\scriptsize] at (torusUnten) {$S^1$}; \node[font=\normalsize] at (-1.5,1) {$T^2$};
	\draw[->] (1.75,1) node[right]{$S^1$} to[out=180,in=450] (1.25,0.5);
\end{tikzpicture}}
Es folgen einige Beispiele für Produkt-Mannigfaltigkeiten:
\begin{enumerate}[label=\arabic*)]
\item
	Zylinder $\R \times S^1$
\item
	$T^n = \bigtimes_{i=1}^nS^1$
  
    $\iota: \R^m \hookrightarrow \R^n$, $(x^1,\ldots ,x^m) \mapsto (x^1,\ldots ,x^m,0, 0,\ldots)$
\end{enumerate}
\end{bsp}


\section{Untermannigfaltigkeiten}

\begin{Dfn}[Untermannigfaltigkeit]\label{def-1-4}
  Es sei $N$ eine glatte Mannigfaltigkeit. Eine Teilmenge $M \subseteq N$ heißt \CmMark{Untermannigfaltigkeit} von $N$, wenn für alle $p \in M$ eine Karte $(\phi, U)$ von $N$ um $p$ existiert, so dass
  \begin{align*}
    \phi(U \cap M) = \phi(U) \cap \underbrace{(\R^m \times \{0\})}_{\mathclap{\{(x^1,\ldots,x^m,0,\ldots,0) \in \R^m\times\R^{n-m} \cong \R^{n} \}}}
  \end{align*}
gilt. Eine solche Karte heißt an $M$ \CmMark[Karte!adaptierte]{adaptierte Karte}. Die Zahl $n-m$ heißt \CmMark{Kodimension} von $M$ in $N$.

% Abbildung 1-7
%\CmPutSvg{1-7-untermf}{8cm}
\begin{center}\begin{tikzpicture}[font=\scriptsize,scale=0.9]
	%\draw[step=0.25,gray!15] (-6,-1) grid (6,5); \draw[step=0.5,gray!30] (-6,-1) grid (6,5); \fill (0,0) circle(0.1); %Hilfsgitter
	
	\draw[->] (1.5,-0.5) -- (5.5,-0.5)node[below]{$\R^m$}; \draw[->] (1.5,-0.5) -- (1.5,2.5)node[left]{$\R^{n-m}$}; \draw[ultra thick] (1.5,-0.5) --node[above]{$\phi(U\cap M)$} (4,-0.5); \node at (4.5,2) {$\R^n$};
	
	\draw[->] (-3,0.75) to[out=330,in=180]node[below]{$\phi$} (1,0);
	
	\coordinate(1) at (-0.25,2.75); \coordinate(2) at (-5.25,1.25); \coordinate(3) at (-4.25,4.25);
	\coordinate(ctrl1) at (0,-1.25); \coordinate(ctrl2) at (1.75,-1); \coordinate(ctrl3) at (-0.5,-0.25);
	\draw (1) ..controls($(1)+(ctrl1)$) and($(2)+(ctrl2)$).. (2) ..controls($(2)-(ctrl2)$) and($(3)+(ctrl3)$).. (3);
	
	\coordinate (a) at (-1.25, 2.75); \coordinate (b) at (-3.5, 1.75); \coordinate (c) at (-4.25, 1.75); \coordinate (d) at (-4.5, 2.25); \coordinate (e) at (-5, 2.5); \coordinate (f) at (-4.25, 3.5);
	\coordinate (ctrlb) at (1,0); \coordinate (ctrlc) at (0.52,-0.25); \coordinate (ctrle) at (0,-0.25);
	\draw[very thick] (a) ..controls(a) and ($(b) + 1.75*(ctrlb)$).. (b) ..controls($(b) - 0.5*(ctrlb)$) and ($(c) + 0.5*(ctrlc)$).. (c) ..controls($(c) - 0.5*(ctrlc)$) and ($(d) + 0.5*(ctrlc)$).. (d) ..controls($(d) - 0.5*(ctrlc)$) and ($(e) + 0.5*(ctrle)$).. (e) ..controls($(e) - 2*(ctrle)$) and(f).. (f);
	
	\filldraw[fill=white] (b) circle(0.05) node[anchor=north west]{$p$}; \draw (b) circle(0.5); \node at (-4,1.25) {$U$}; \node at (-1.75,2) {$M$}; \node at (-5.5,3.5) {$N$};
	
	\coordinate (cntr) at (-3,3);
	\begin{scope}
		\clip ($(cntr)-(1,0.6)$) rectangle ($(cntr)+(1,-0.1)$);
		\draw[name path=l] (cntr) ellipse(1 and 0.5);
	\end{scope}
	\path[name path=u] ($(cntr) - (0,0.5)$) ellipse(0.75 and 0.5);
	\path[name intersections={of=u and l}];
	\begin{scope}
		\clip (intersection-1) rectangle ($(intersection-2)+(0,0.5)$);
		\draw ($(cntr) - (0,0.5)$) ellipse(0.75 and 0.5);
	\end{scope}
\end{tikzpicture}\end{center}

\end{Dfn}

\begin{Lemma}\label{lemma-1-5}
  Es seien $N$ eine $n$-dimensionale glatte Mannigfaltigkeit und $M \subseteq N$ eine $m$-dimensionale Untermannigfaltigkeit von $N$. Bezeichnet $\mathcal A$ einen $C^{\infty}$-Atlas von $N$ und $\pi \colon \R^n \to \R^m, (x^1, \ldots, x^m,\ldots,x^n) \mapsto (x^1, \ldots, x^m)$, so ist
  \begin{align*}
    \mathcal B = \{(\pi \circ \phi|_{U \cap M},U\cap M) \mid (\phi, U) \in \mathcal A \text{ an } M \text{ adaptierte Karte}\}
  \end{align*}
  ein $C^{\infty}$-Atlas von $M$.
\end{Lemma}

\begin{bew}
  Die Hausdorff-Eigenschaft und die Abzählbarkeit der Topologie werden von $N$ auf $M$ vererbt.\\
  Ist $p \in N$, so existiert eine adaptierte Karte $(\phi,U)$ von $N$ um $p$ und $\pi \circ \phi|_{U \cap M}$ ist ein Homöomorphismus von $U \cap M$ auf eine offene Teilmenge des $\R^m$. Jeder Kartenwechsel
  \begin{align*}
    (\pi \circ \phi|_{U \cap M}) \circ (\pi \circ \psi|_{V \cap M})^{-1} = (\pi \circ \phi) \circ (\psi^{-1} \circ \imath) = \pi \circ (\phi \circ \psi^{-1}) \circ \imath
  \end{align*}
  ist ein $C^{\infty}$-Diffeomorphismus.
\end{bew}

\begin{bem}
  Erinnerung: $M \subseteq \R^n$ heißt glatte $n$-dimensionale Untermannigfaltigkeit des $\R^n$, wenn für alle $p \in M$ eine offene Umgebung $U$ und eine Abbildung $\phi \colon U \to \R^n$  mit folgenden Eigenschaften existiert:
  \begin{enumerate}[label=(\roman*),widest=ii]
  \item $\phi \colon U \to \phi(U)$ ist ein Diffeomorphismus auf sein offenes Bild im $\R^{n}$.
  \item $\phi(U \cap M) = \phi(U) \cap (\R^{m} \times \{0\})$.
  \end{enumerate}
Jedes solche $M$ ist eine Untermannigfaltigkeit im Sinne von Definition \ref{def-1-4}, denn jedes $\phi$ wie oben ist wegen (i) eine Karte von $\R^n$ (im Sinne glatter Mannigfaltigkeiten) und wegen (ii) eine an $M$ adaptierte Karte. Also sind mit Lemma \ref{lemma-1-5} glatte Untermannigfaltigkeiten des $\R^n$ glatte Mannigfaltigkeiten (im allgemeineren Sinne).
\end{bem}

%%% Local Variables: 
%%% mode: latex
%%% TeX-master: "../skript-diffgeom"
%%% End: 


\printindex

\end{document}
