%%
%% Skript Differentialgeometrie im Wintersemester 12/13
%% Zur Vorlesung von Dr. Grensing am KIT Karlsruhe
%%
%% Mitschrieb und Textsatz von Jan-Bernhard Kordaß.
%%

\documentclass[paper=A4, twoside, chapterprefix=true, bibliography=totoc, headsepline]{scrbook}

% Abstand zwischen zwei Textbloecken
\setlength\parskip{\smallskipamount}

% Nummerierung der Paragraphen anpassen (sonst kommt etwas wie "Definition 2.9.1" heraus)
\renewcommand{\thesection}{\arabic{section}}

%Aendert die Kapitelbeschriftung in der Kopfzeile der linken Seiten
\renewcommand*{\chaptermarkformat}{\chapappifchapterprefix{\ }\thechapter:\enskip}
\renewcommand*{\sectionmarkformat}{\thesection\autodot\enskip}

% Einheitliche Schriftart (KOMA Script verwendet fuer einige Ueberschriften eine serifenlose Schrift, mischt also
% Schriftarten. Ich habe mir die Argumente dafuer durchgelsen und war nicht ueberzeugt. Wenn jemand, der mehr als
% ich von der Materie versteht, anderer Meinung ist kann er diese Zeilen hier einfach auskommentieren)
\setkomafont{chapter}{\Huge\bfseries\rmfamily}
\setkomafont{chapterentry}{\bfseries\rmfamily}
\setkomafont{disposition}{\bfseries\rmfamily}
\setkomafont{descriptionlabel}{\bfseries\rmfamily}

\usepackage[utf8x]{inputenc}
\usepackage[T1]{fontenc}
\usepackage{lmodern}

\usepackage[ngerman]{babel}
\usepackage[top=2.5cm, bottom=3cm, left=2.5cm, right=4.5cm]{geometry}

%\usepackage[nobottomtitles]{titlesec}

\usepackage{fancyhdr} % erlaubt mehr Optionen in Kopf- und Fusszeile

% header configuration
%\pagestyle{fancy}
%\fancyhf{
%\lhead[\thepage]{\rightmark}}
%\rhead[\nouppercase{\leftmark}]{\thepage}		

\usepackage{xcolor} % Farben
\usepackage{marginnote} % Randnotizen
\usepackage{enumitem} % Fuer mehr Einstellungmoeglichkeiten bei Aufzaehlungen
\usepackage{xifthen} % Erlaubt die Verwendung von if-then-else Befehlen im Code
\usepackage{index} % Index erzeugen
\newindex{default}{idx}{ind}{Stichwortverzeichnis}
\usepackage{xspace} % intelligende Leerzeichen bei Macros
\usepackage[normalem]{ulem} % unterstreichen von Text
\usepackage{cancel} % schraeg durchstreichen von Text

\renewcommand{\CancelColor}{\color{gray}} % Farbe zum schraegen Druchstreichen in grau

\definecolor{rltred}{rgb}{0.75,0,0}
\definecolor{rltgreen}{rgb}{0,0.5,0}
\definecolor{rltblue}{rgb}{0,0,0.75}

%sichere Fraben, die sich auch bei einem SW-Druck unterscheiden lassen (Platzhalter momentan)
\definecolor{color1}{cmyk}{1,0,0,0} %cyan
\definecolor{color2}{rgb}{0,1,0} %green

\usepackage[hyperindex=true]{hyperref} % Verweise als Hyperlinks
\hypersetup{
  pdftitle={Differentialgeometrie Dr. Grensing},
  pdfsubject={Differentialgeometrie Geometrie},
  pdfkeywords={Differentialgeometrie Grensing},
  pdfproducer={pdfLaTeX},
  pdfpagemode={UseOutlines},
  colorlinks=true,
  bookmarksopen=true,
  bookmarksnumbered=true,
  urlcolor=rltblue,
  filecolor=rltgreen,
  linkcolor=rltblue,
  backref=true,
  pagebackref=true,
  pdfpagemode=None,
  citecolor=rltblue
}

% vertausche die Theta, Phi, Rho und Epsilon mit ihrer "var" Version
%\newcommand{\swapcmd}[2]{
%	\let\temp\#1
%	\left\#1\#2
%	\let\#2\temp
%}
\let\temp\phi
\let\phi\varphi
\let\varphi\temp

\let\temp\theta
\let\theta\vartheta
\let\vartheta\temp

\let\temp\epsilon
\let\epsilon\varepsilon
\let\varepsilon\temp

\let\temp\rho
\let\rho\varrho
\let\varrho\temp

\usepackage{tikz} % Fuer Zeichnungen in TikZ
\usetikzlibrary{matrix,arrows,calc,intersections, positioning, patterns, decorations.text, decorations.pathmorphing}

% neue Befehle fuer haeufig benutzte TikZ Formen; erstes Argument steht fuer die Position, Zweites fuer die Groesse
\newcommand{\tikzrichtung}[3][1]{ % zeichnet eine rote Linie von einem Punkt in eine Richtung mit rotem Knoten am Ende
	\draw[red] #2 -- ($#2 + #1*#3$) circle(0.05);
}
\newcommand{\tikzschnuller}[2][1]{
	% definiere die Knoten relativ zum ersten Knoten skaliert mit dem Faktor
	\coordinate (schnuller1) at #2; \coordinate (schnuller2) at ($(schnuller1)+#1*(-1.75,-0.75)$); \coordinate (schnuller3) at ($(schnuller1)+#1*(-2.5,-2.25)$); \coordinate (schnuller4) at ($(schnuller1)+#1*(0,-2)$); \coordinate (schnuller5) at ($(schnuller1)+#1*(1.75,-0.25)$);
    %\fill (schnuller1) circle (0.05) (schnuller2) circle (0.05) (schnuller3) circle (0.05) (schnuller4) circle (0.05) (schnuller5) circle (0.05);
    
    % die Richtungsvektoren der Bezier Tangenten fuer die einzelnen Knoten (der Erste und der letzte haben keine Tangente)
    \coordinate (ctrls1) at ($#1*(1.25,0.25)$); \coordinate (ctrls2) at ($-0.5*(ctrls1)$); \coordinate (ctrls4) at ($#1*(1,-1)$); \coordinate (ctrls3) at ($-0.5*(ctrls4)$); \coordinate (ctrls6) at ($#1*(1,1.5)$); \coordinate (ctrls5) at ($-0		.33*(ctrls6)$);
	% die eigentlichen Tangenten
    \coordinate (tang1) at ($(schnuller2)+(ctrls1)$); \coordinate (tang2) at ($(schnuller2)+(ctrls2)$); \coordinate (tang3) at ($(schnuller3)+(ctrls3)$); \coordinate (tang4) at ($(schnuller3)+(ctrls4)$); \coordinate (tang5) at ($(schnuller4)+(ctrls5)$); \coordinate (tang6) at ($(schnuller4)+(ctrls6)$);
    %\fill[red] (tang1) circle (0.05); \fill[red] (tang2) circle (0.05); \fill[red] (tang3) circle (0.05); \fill[red] (tang4) circle (0.05); \fill[red] (tang5) circle (0.05); \fill[red] (tang6) circle (0.05);
    %\draw[red] (tang1) -- (tang2); \draw[red] (tang3) -- (tang4); \draw[red] (tang5) -- (tang6);
	
	\draw (schnuller1) ..controls(schnuller1) and (tang1).. (schnuller2) ..controls(tang2) and (tang3).. (schnuller3) ..controls(tang4) and (tang5).. (schnuller4) ..controls(tang6) and (schnuller5).. (schnuller5);
	
	% zeichne nun das Loch in der Mitte
	\def\angle{20} % Rotationswinkel
	\coordinate (c) at ($#2+#1*(-1.25,-1.25)$); % Mittelpunkt der Ellipse die den unteren Bogen bildet
	\begin{scope}
		\clip[rotate=\angle] ($(c)-#1*(1,0.6)$) rectangle ($(c)+#1*(1,-0.1)$);
		\path[draw,rotate=\angle,name path=l] (c) ellipse(#1*1 and #1*0.5);
	\end{scope}
	\path[name path=u,rotate=\angle] ($(c)-#1*(0,0.5)$) ellipse(#1*0.75 and #1*0.5);
	\path[name intersections={of=u and l}];
	\begin{scope}
		\clip[rotate=\angle] (intersection-1) rectangle ($(intersection-2)+#1*(0,0.5)$);
		\draw[rotate=\angle] ($(c)-#1*(0,0.5)$) ellipse(#1*0.75 and #1*0.5);
	\end{scope}		
}
%\newcommand{\tikzkartoffel}[2][1]{}		
\newcommand{\tikzsegel}[2][1]{
	% definiere die Knoten relativ zum ersten Knoten skaliert mit dem Faktor
	\coordinate (segel1) at #2; \coordinate (segel2) at ($(segel1)+#1*(4,1.5)$); \coordinate (segel3) at ($(segel1)+#1*(2,-0.5)$);
	%\fill (segel1) circle (0.05) (segel2) circle (0.05) (segel3) circle (0.05);
	
	% die Richtungsvektoren der Bezier Tangenten fuer die einzelnen Knoten (der Erste und der letzte haben keine Tangente)
	\coordinate (ctrls1) at ($#1*(0.75,1.5)$); \coordinate (ctrls2) at ($#1*(-0.75,0.25)$); \coordinate (ctrls3) at ($#1*(-0.5,-0.25)$); \coordinate (ctrls4) at ($#1*(0.25,1)$); \coordinate (ctrls5) at ($#1*(-0.375,0.375)$); \coordinate (ctrls6) at ($#1*(0.75,0.125)$);
	% die eigentlichen Tangenten
	\coordinate (tang1) at ($(segel1)+(ctrls1)$); \coordinate (tang2) at ($(segel2)+(ctrls2)$); \coordinate (tang3) at ($(segel2)+(ctrls3)$); \coordinate (tang4) at ($(segel3)+(ctrls4)$); \coordinate (tang5) at ($(segel3)+(ctrls5)$); \coordinate (tang6) at ($(segel1)+(ctrls6)$);
%	\fill[red] (tang1) circle (0.05); \fill[red] (tang2) circle (0.05); \fill[red] (tang3) circle (0.05); \fill[red] (tang4) circle (0.05); \fill[red] (tang5) circle (0.05); \fill[red] (tang6) circle (0.05);
 %   \draw[red] (tang1) -- (segel1) -- (tang6); \draw[red] (tang2) -- (segel2) -- (tang3); \draw[red] (tang4) -- (segel3) -- (tang5);
	
	\draw (segel1) ..controls(tang1) and (tang2).. (segel2) ..controls(tang3) and (tang4).. (segel3) ..controls(tang5) and (tang6).. (segel1) --cycle;
}
\newcommand{\tikztorus}[2][1]{
	% \draw[step=0.25,gray!15] (-6,-1) grid (6,5); \draw[step=0.5,gray!30] (-6,-1) grid (6,5); \fill (0,0) circle(0.1); %Hilfsgitter
	% zuerst die aeussere Ellips
	\draw[] #2  ellipse (#1*2 and #1*1);
	
	% dann das Loch
	\begin{scope}
      \clip ($#2 - #1*(1, 0.5)$) rectangle ($#2 + #1*(1, 1)$);
      \path[draw,name path=gkreis] ($#2 + #1*(0,0.75)$) ellipse (#1*1.25 and #1*1);
    \end{scope}
    \path[name path=kkreis] ($#2 - #1*(0,0.5)$) ellipse (#1*1 and #1*0.75);
    \path[name intersections={of=gkreis and kkreis}];
    \begin{scope}
      \clip (intersection-1) rectangle ($(intersection-2)+(0,0.5)$);
      \draw ($#2 - #1*(0,0.5)$) ellipse (#1*1 and #1*0.75);
    \end{scope}
    
    % definiere Werte auf die wir in der restlichen Zeichnung zurueckgreifen koennen
	\def\torusbreite{#1*2}
	\def\torushoehe{#1*1}
	\def\torusdicke{#1*0.75}
	\coordinate (torusUntenLoch) at ($#2 - #1*(0,0.25)$);
	\coordinate (torusUnten) at ($#2 - #1*(0,1)$);
}

\usepackage[toc]{glossaries} % Symbolverzeichnis
\glossarystyle{treehypergroup}
\makeglossaries

% Mathe Pakete
\usepackage{amsmath}
\usepackage{amssymb}
\usepackage{stmaryrd}
\usepackage{bm} % fette Mathe Zeichen
%\usepackage{amsthm}
\usepackage[hyperref,amsmath,thmmarks,thref]{ntheorem}

% common mathematical operators and sets
\DeclareMathOperator{\aff}{aff} % affine Huelle
\DeclareMathOperator{\ddet}{det} % Determinante
\DeclareMathOperator{\diam}{diam} % diameter
\DeclareMathOperator{\dist}{dist} % distance
\DeclareMathOperator{\ddim}{dim} % dimension
\DeclareMathOperator{\dR}{dR} % deRahm
\DeclareMathOperator{\ggT}{ggT} % goesster gemeinsamer Teiler
\DeclareMathOperator{\id}{id} % identity
\DeclareMathOperator{\inh}{inh} % Inhalt
\DeclareMathOperator{\grad}{grad} % Gradient
\DeclareMathOperator{\kgV}{kgV} % kleinstes gemeinsames Vielfaches
\DeclareMathOperator{\mspan}{span} % Lineare Huelle
\DeclareMathOperator{\n}{n} % Umlaufzahl
\DeclareMathOperator{\offen}{offen}
\DeclareMathOperator{\pr}{pr}
\DeclareMathOperator{\res}{res} % Residuum
\DeclareMathOperator{\rg}{rg} % rank (i)
\DeclareMathOperator{\scal}{scal} % Skalarkruemmung
\DeclareMathOperator{\sgn}{sgn} % Signum
\DeclareMathOperator{\spur}{spur} % Spur
\DeclareMathOperator{\supp}{supp} % support
\DeclareMathOperator{\sternf}{sternf}
\DeclareMathOperator{\tr}{tr} % Spur

\DeclareMathOperator{\Abb}{Abb} % maps
\DeclareMathOperator{\Aut}{Aut} % automorphisms
\DeclareMathOperator{\Bild}{Bild}
\DeclareMathOperator{\Charakteristik}{char}
\DeclareMathOperator{\Charakt}{char}
\DeclareMathOperator{\D}{D} % Jacobi matrix or derivative
\DeclareMathOperator{\Diff}{Diff}
\DeclareMathOperator{\End}{End} % endomorphisms
\DeclareMathOperator{\Gl}{GL} % general linear group
\DeclareMathOperator{\GL}{GL} % general linear group
\DeclareMathOperator{\Gr}{Gr}
\DeclareMathOperator{\Graph}{Graph}
\DeclareMathOperator{\Hom}{Hom} % homomorphisms
\DeclareMathOperator{\Id}{id} % identity
\DeclareMathOperator{\Inn}{Inn} % Untergruppe der inneren Automorphismen
\DeclareMathOperator{\Kern}{Kern}
\DeclareMathOperator{\Oo}{O} % Matrizen sie mit ihrer Transponierten multipiziert die Einheitsmatrix ergeben
\DeclareMathOperator{\Relation}{\scriptstyle\mathrm{R}} % custom Relation
\DeclareMathOperator{\Rang}{Rang} % rank (ii)
\DeclareMathOperator{\SL}{SL} % Matrizen mit Deteminante 1
\DeclareMathOperator{\Stab}{Stab} % Stabilisator
\DeclareMathOperator{\Sym}{Sym} % symmetric group
\DeclareMathOperator{\T}{T} % tangent bundle

\DeclareMathOperator{\ric}{ric} % Ricci Tensor
\DeclareMathOperator{\Ric}{Ric} % Ricci Tensor field

\newcommand{\Zentrum}[1]{\ensuremath{\mathrm Z(#1)}} % Zentrum einer Gruppe
\newcommand{\Ordnung}[1][]{ % Ordnung einer Gruppe
  \ifthenelse{\isempty{#1}}{
    \#
  }{
    \left|#1\right|
  }
}

% X als Malzeichen
\newcommand{\X}{\times}

% ein schoener aussehender Faktorraum anstatt einfach nur A/B
\newcommand{\FakRaum}[2]{
	\raisebox{0.7ex}{\ensuremath{#1}}
	\ensuremath{\mkern-3mu}\big/\ensuremath{\mkern-3mu}
	\raisebox{-0.6ex}{\ensuremath{#2}}}
\newcommand{\smallFakRaum}[2]{
	\scriptsize{\raisebox{0.7ex}{\ensuremath{#1}}
	\ensuremath{\mkern-3mu}\ / \ensuremath{\mkern-3mu}
	\raisebox{-0.6ex}{\ensuremath{#2}}}}

%Realteil und Imaginaerteil
\renewcommand{\Re}{\ensuremath{\operatorname{Re}}} % <-- sollte man da nicht besser \DeclareMathOperator verwenden?
% [kann man nicht, weil \Re und \Im schon deklariert sind
% einen "\ReDeclareMathOperator" Befehl gibt es nicht. JB]
\renewcommand{\Im}{\ensuremath{\operatorname{Im}}}

% \DeclareMathOperator{\Real}{Re} % real part
% \DeclareMathOperator{\Imag}{Im} % imaginary part

% canonic sets
\DeclareMathOperator{\C}{\mathbb{C}}
\DeclareMathOperator{\F}{\mathbb{F}}
\DeclareMathOperator{\K}{\mathbb{K}}
\DeclareMathOperator{\N}{\mathbb{N}}
\DeclareMathOperator{\Q}{\mathbb{Q}}
\DeclareMathOperator{\R}{\mathbb{R}}
\DeclareMathOperator{\RP}{\mathbb{RP}} % real projection plane
\DeclareMathOperator{\Tor}{\mathbb{T}} % torus
\DeclareMathOperator{\Z}{\mathbb{Z}}
\DeclareMathOperator{\B}{\mathbb{B}} % unit ball

%  geschwungene Buchstaben
\DeclareMathOperator{\calD}{\mathcal{D}}
\DeclareMathOperator{\calI}{\mathcal{I}}
\DeclareMathOperator{\calJ}{\mathcal{J}}
\DeclareMathOperator{\calL}{\mathcal{L}}
\DeclareMathOperator{\calT}{\mathcal{T}}
\DeclareMathOperator{\calV}{\mathcal{V}}

% Redeclare \P (Prim or Propability) and put the old, reversed "breakline P" in \BreakLineP
\let\BreakLineP\P
\renewcommand{\P}{\ensuremath{\mathbb{P}}}

% Differentialoperatoren als Brüche
\newcommand{\dop}{\mathrm{d}}	
\newcommand{\difffrac}[3][]{\ifthenelse{\isempty{#1}}{\frac{\dop #2}{\dop #3}}{\left. \frac{\dop #2}{\dop #3} \right|_{#1}}}
\newcommand{\pdifffrac}[3][]{\ifthenelse{\isempty{#1}}{\frac{\partial #2}{\partial #3}}{\left. \frac{\partial #2}{\partial #3} \right|_{#1}}}

% stellt einen großen vertikalen Strich an einen Term, nuetzlich in Bruechen
\newcommand{\bigvert}[1]{\left. #1 \right|}

% quotient space or group
\newcommand{\modulo}[1]{\ensuremath{/_{\displaystyle #1}}}

% declaring Index for group theory
\newcommand{\Index}[2]{\ensuremath{(#1 \SlimDdot #2)}}


% canonic environments
\newcounter{thmglobal}
%\swapnumbers
\theoremstyle{plain}

%%%%%%%%%%%%%%%%%%%%%%%%%%%%%%%%%%%%%%%%%%%%%%%%%%%%%%%%%%%%%%%%%%%%%%%%%%%%%%%%%%%%%%%%%%%%%%%%%%%%%%%%%%%%%%%%%%%%%%%%%%%%%%%%%%%%%%%%%%%%%%%%%%%%%%%%%%%%%%%%%%%%%%%%

\makeatletter

% Options
\newboolean{enableDeepNumbering}
\setboolean{enableDeepNumbering}{false}

\DeclareOption{deepnum}{
  \setboolean{enableDeepNumbering}{true}
}

\newboolean{enableMarginThm}
\setboolean{enableMarginThm}{false}

\DeclareOption{marginthm}{
  \setboolean{enableMarginThm}{true}
}

\ProcessOptions\relax

% call makeindex for an index register
%\makeindex

% Name language settings

% theorem names, ngerman
\newcommand{\cmLangThmSatz}{Satz\xspace}
\newcommand{\cmLangThmLemma}{Lemma\xspace}
\newcommand{\cmLangThmKor}{Korollar\xspace}
\newcommand{\cmLangThmProp}{Proposition\xspace}

\newcommand{\cmLangThmDfn}{Definition\xspace}
\newcommand{\cmLangThmBsp}{Beispiel\xspace}

\newcommand{\cmLangThmBem}{Bemerkung\xspace}

% short forms, ngerman
\newcommand{\cmLangThmShortSatz}{Satz\xspace}
\newcommand{\cmLangThmShortLemma}{Lemma\xspace}
\newcommand{\cmLangThmShortKor}{Kor\xspace}
\newcommand{\cmLangThmShortProp}{Prop\xspace}

\newcommand{\cmLangThmShortDfn}{Def\xspace}
\newcommand{\cmLangThmShortBsp}{Bsp\xspace}

\newcommand{\cmLangThmShortBem}{Bem\xspace}


\theoremstyle{plain}
\newtheorem{Dfn}{\cmLangThmDfn}[chapter]
\newtheorem{Satz}[Dfn]{Satz}
\newtheorem{Lemma}[Dfn]{\cmLangThmLemma}
\newtheorem{Kor}[Dfn]{\cmLangThmKor}
\newtheorem{Prop}[Dfn]{\cmLangThmProp}
\theorembodyfont{\normalfont}
\newtheorem{Bsp}[Dfn]{\cmLangThmBsp}
\newtheorem{Bem}[Dfn]{\cmLangThmBem}
\newtheorem{Aufg}{Aufgabe}
\newtheorem{Loes}{L\"osung}

\theoremstyle{nonumberplain}
\newtheorem{dfn}{\cmLangThmDfn}
\newtheorem{satz}{Satz}
\newtheorem{lemma}{\cmLangThmLemma}
\newtheorem{kor}{\cmLangThmKor}
\newtheorem{prop}{\cmLangThmProp}

\newtheorem{bsp}{\cmLangThmBsp}
\newtheorem{bem}{\cmLangThmBem}

\theoremsymbol{\ensuremath{\Box}}
\theorembodyfont{\normalfont}
\newtheorem{bew}{Beweis}

\theoremsymbol{}
\theoremstyle{empty}
\newtheorem{emptythm}{}% druckt nur den optionalen Namen aus

\theoremstyle{break}

% Add unnumbered Theorems, use amsthm style in both style modes
\theoremstyle{plain}
%\newtheorem*{satz*}{\cmLangThmSatz}
%\newtheorem*{lemma}{\cmLangThmLemma}
%\newtheorem*{kor*}{\cmLangThmKor}
%\newtheorem*{prop*}{\cmLangThmProp}

\theoremstyle{definition}
%\newtheorem*{dfn}{\cmLangThmDfn}
%\newtheorem*{bsp*}{\cmLangThmBsp}

\theoremstyle{remark}
%\newtheorem*{bem*}{\cmLangThmBem}
\newtheorem*{beh*}{Behauptung}


% 2-level numbering$
\numberwithin{thmglobal}{section}

% check if 3-level numbering is enabled
\ifthenelse{\boolean{enableDeepNumbering}}{
  \numberwithin{thmglobal}{subsection}
}{}


% some other customisations

% changing enumerations
\setlist[enumerate]{label=(\arabic*), itemsep=0cm, leftmargin=2cm}
\setlist[itemize]{itemsep=0cm} %\setlist[itemize]{itemsep=0cm, leftmargin=2cm}

% replace the slim emptyset symbol
\let\emptyset\varnothing

% set line distances
\linespread{1.1}

% Add a ':' for mathmode with tiny whitespaces around
\newcommand{\SlimDdot}{\ensuremath{\mathrm{:}}}


% headline and cover generation commands

% generate a simple headline
% usage: \CmHeadline[date]{title}{topic}{author}
\newcommand{\CmHeadline}[4][]{
  \begin{minipage}[t]{\textwidth}
    \huge{\textbf{#2}}\\
    \large{#3, #4}\relax
    \ifthenelse{\isempty{#1}}{}{\relax\large{, #1}}
  \end{minipage}
}

% generate a simple cover page
% usage: \CmCover[type(,skript)]{title}{subtitle}{date}
\newcommand{\CmCover}[4][]{
  \thispagestyle{empty}
  \begin{titlepage}
    \begin{center}
      \begin{minipage}[b]{0.8\textwidth}
	\vspace*{5cm}
        \ifthenelse{\isempty{#1}}{
          % Default cover arrangement
          \Huge{\textbf{#2}}\\[0.5cm]
          \huge{#3}\\[0.8cm]
          \Large{#4}
        }{
          \ifthenelse{\equal{#1}{skript}}{
            % Cover for lecture scripts
            \huge{#3}\\[0.5cm]
            \Huge{\textbf{#2}}\\[0.5cm]
            \Large{#4}
          }{}
        }
      \end{minipage}
    \end{center}
  \end{titlepage}
  \pagebreak
}


% indexing support

% Print and index given text
% usage: \CmIndex{[(optionally put another text for the index in here)]{(text to print and add to index)}
\newcommand{\CmIndex}[2][]{\ifthenelse{\isempty{#1}}{\index{#2}}{\index{#1}}#2}

% Highlight(bold) and index the given text
% usage: \CmMark[(optionally put another text for the index in here)]{(text to highlight and add to index)}
\newcommand{\CmMark}[2][]{\textbf{\CmIndex[#1]{#2}}}


% sectioning support

% Prints a description for a section in italic, bold. Most likely to use right under \section.
% usage: \CmSectionDescription{(short description of section contents)}
\newcommand{\CmSectionDescription}[1]{
  \vspace{-0.3cm}
  \hangindent=0.4cm
  \hangafter=0
  \begin{itshape}
    \textbf{#1}
  \end{itshape}
  \vspace{0.3cm}
}

% Starts a new paragraph inside of a theorem environment (as defined above)
% usage \CmSubThm[(paragraph title)]
\newenvironment{CmSubThm}[1]{
  \begin{itemize}[leftmargin=0.5cm,label=]
  \item
    \ifthenelse{\isempty{#1}}{}{
      \hspace{-0.5cm}(\textit{#1})\\[0.2cm]
    }
  }{
  \end{itemize}
}


% svg updater
% needs shell escape option

% Checks if the given image file has been modified and a custom command (if possible) via command line to generate something new.
\newcommand{\CmExecuteIfFileNewer}[3]{
  \ifnum\pdfstrcmp{\pdffilemoddate{#1}}
  {\pdffilemoddate{#2}}>0
  {\immediate\write18{#3}}\fi
}

% Tries to include an image file and checks if the given one has been modified. If so it calls inkscape (if possible) via command line to generate new pdf und pdf_tex files from the corresponding svg.
\newcommand{\CmIncludeSvg}[1]{
  \def\svg@filepath{}
  
  % check if corrosponding svg file exists
  \IfFileExists{#1.svg}
  {
    \def\svg@filepath{#1}
  }{
    % if it does not, search in the graphicspath for it
    \expandafter\@tfor\expandafter\currentsvgpath\expandafter:\expandafter=\Ginput@path\do{
      \IfFileExists{\currentsvgpath#1.svg}{
        \edef\svg@filepath{\currentsvgpath #1}
      }{}
    }
  }
  % if something was found, include the graphic, TODO: Make it work correctly
  \ifthenelse{\isundefined{\svg@filepath} \OR \isempty{\svg@filepath}}{
    \PackageError{canonicalmath}{Image file not found!}
  }{
    \PackageWarning{FilePath}{|\svg@filepath|}
    \CmExecuteIfFileNewer{\svg@filepath.svg}{\svg@filepath.pdf}{inkscape -z -D --file=\svg@filepath.svg --export-pdf=\svg@filepath.pdf --export-latex}
    \input{\svg@filepath.pdf_tex}
  }
}

% Tries to include an image file on the center of the margin at the current position.
\newcommand{\CmMarginSvg}[3][0cm]{
  \marginnote{
    \centering
    \def\svgwidth{#3}
    \CmIncludeSvg{#2}
  }[#1]
}

% Tries to include an image file centering it at the current position
\newcommand{\CmPutSvg}[3][0cm]{
  \begin{figure}[h!]
    \vspace{#1}
    \centering
    \def\svgwidth{#3}
    \CmIncludeSvg{#2}
  \end{figure}
}
\makeatother


%%%%%%%%%%%%%%%%%%%%%%%%%%%%%%%%%%%%%%%%%%%%%%%%%%%%%%%%%%%%%%%%%%%%%%%%%%%%%%%%%%%%%%%%%%%%%%%%%%%%%%%%%%%%%%%%%%%%%%%%%%%%%%%%%%%%%%%%%%%%%%%%%%%%%%%%%%%%%%%%%%%%%%%%

%\usepackage{canonicalsync}
%\CsUsePackage[/home/JB/Projects/tex-package-canonical-sync/]{canonicalsync}
%\CsUsePackageWithOptions[/home/JB/Projects/tex-package-canonical-math/]{canonicalmath}{marginthm}

\usepackage{mathtools}

\usepackage{graphicx}
\usepackage{float}
\usepackage{transparent}
\usepackage{wrapfig}

\graphicspath{{img/}}

\parindent0pt

% Befehl fuer Anfuerungszeichen unten und oben
\newcommand{\quot}[1]{\textrm{\glqq}{#1}\textrm{\grqq}}



\setlist[enumerate]{label=(\arabic*), itemsep=0cm, leftmargin=1cm}

%--------------------------------------------------------------------
%-------------------- Eintraege fuer das Glossar --------------------
%--------------------------------------------------------------------

\newglossaryentry{Dualraum}{
	name=Dualraum,
	description={Die zu einem Vektorraum \ensuremath{V} "uber einem K"orper \ensuremath{K} geh"orende Menge aller linearen Abbildungen von \ensuremath{V} nach \ensuremath{K}. Der Dualraum selbst ist ebenfalls ein Vektorraum mit Skalarmultiplikation mit Elementen aus \ensuremath{K}},
	text={Dualraum}
}

\newglossaryentry{GL}{
	name={GL\ensuremath{_n}},
	description={Allgemeine lineare Gruppe, Gruppe aller regul"aren $n \X n$-Matrizen mit Koeffizienten aus einem K"orper $K$},text={\ensuremath{\Gl}},
	symbol={\ensuremath{\Gl}},
	sort=GL
}

\newglossaryentry{Hausdorff-Raum}{
	name=Hausdorff-Raum,
	description={Ein topologischer Raum \ensuremath{M}, in dem es f"ur alle \ensuremath{x, y \in M, x \ne y} disjunkte offene Umgebungen \ensuremath{U(x)} und \ensuremath{U(y)} gibt, es werden also alle paarweise verschiedenen Punkte \ensuremath{x, y} durch Umgebungen getrennt},
	text={Hausdorff-Raum}
}

\newglossaryentry{Homoeomorphismus}{
	name={Hom"oomorphismus},
	description={Eine bijektive, stetig differenzierbare Abbildung zwischen zwei Objekten, deren Umkehrabbildung ebenfalls stetig differenzierbar ist},
	text={Hom{\"o}omorphsimus},
	sort={Homoomorphismus}
}

\newglossaryentry{topologischer Raum}{
	name=Topologischer Raum,
	description={Eine Menge \ensuremath{X} zusammen mit einer Topologie \ensuremath{T}, das hei\ss t einem Mengensystem das offene Teilmengen von \ensuremath{X} definiert, wobei die leere Menge, die Grundmenge, der Durchschnitt endlich vieler offener Mengen und die Vereinigung beliebig vieler offener Mengen offen sind},
	text={topologischer Raum}
}

\newglossaryentry{Topologie}{
	name=Topologie,
	description={Ein Mengensystem das Teilmengen einer Grundmenge als offene Mengen definiert, wobei die leere Menge und die Grundmenge selbst offen sind und der Durchschnitt endlich vieler offener Mengen und die Vereinigung beliebig vieler offener Mengen wieder offen sind}
}

\newglossaryentry{spur}{
	name={Spur},
	description={asdf},text={\ensuremath{\spur}},
	symbol={\ensuremath{\spur} oder \ensuremath{\tr}},
	sort=spur
}

\newglossaryentry{supp}{
	name={Tr\"ager},
	description={asdf},text={\ensuremath{\supp}},
	symbol={\ensuremath{\supp}},
	sort=supp
}

\newglossaryentry{Diffeomorphismus}{
	name={Diffeomorphismus},
	description={Eine bijektive, stetige Abbildung zwischen zwei Objekten, deren Umkehrabbildung ebenfalls stetig ist}
}

\newglossaryentry{Homomorphismus}{
	name={Hompmorphismus},
	description={asdf}
}

\newglossaryentry{Endomorphismus}{
	name={Endomorphismus},
	description={Ein Homomorphismus einer Struktur in sich selbst}
}

\newglossaryentry{Isomorphismus}{
	name={Isomorphismus},
	description={Ein Homomorphismus einer Struktur in sich selbst}
}

\newglossaryentry{Produkttopologie}{
	name={Produkttopologie},
	description={asdf}
}

\newglossaryentry{Kurve}{
	name={Kurve},
	description={asdf}
}


%-----------------------------------------------------------------------------------
%-------------------- Allgemeine Informationen ueber das Skript --------------------
%-----------------------------------------------------------------------------------

%\titlehead{inoffizielles Vorlesungsskript}
\subject{inoffizielles Skript}
\title{Differentialgeometrie}
\subtitle{Gehalten von Dr. S. Grensing im Wintersemester 2012/13}
\author{getippt von Aleksandar Sandic\thanks{\href{mailto:aleksandar.sandic@student.kit.edu}{Aleksandar.Sandic@student.kit.edu}} und Jan-Bernhard Korda\ss\thanks{\href{mailto:jan-bernhard.kordass@student.kit.edu}{Jan-Bernhard.Kordass@student.kit.edu}}}

%--------------------------------------------------------------------
%-------------------- Hier beginnt das Skript -----------------------
%--------------------------------------------------------------------

\begin{document}

%\CmCover{Vorlesung Differentialgeometrie}{Gehalten von Dr. Grensing}{Wintersemester 2012/13}

\maketitle

% Inhaltsverzeichnis
\pdfbookmark[1]{Inhaltsverzeichnis}{contents}
\setlength\parskip{0.6pt}
\tableofcontents

\setlength\parskip{\smallskipamount}

\vspace{0.5cm}

Version \textbf{0.01} \quad Build: \today

\paragraph{Wichtiger Hinweis:}
Dies ist eine Zusammenfassung der Vorlesung "`Differentialgeometrie"' von Dr. Grensing im Wintersemester 2012/13 am KIT und dient lediglich dazu die Inhalte für meine eigene Verwendung besser zusammenzufassen und aufzubereiten. Es besteht weder eine Garantie über Vollständigkeit, noch Korrektheit der enthaltenen Aussagen.\\

Bei Anmerkungen bzw. beim Auffinden von Fehlern schickt bitte eine E-Mail an
\begin{center}
  \href{mailto:jan-bernhard.kordass@student.kit.edu}{jan-bernhard.kordass@student.kit.edu}\\
  \href{mailto:aleksandar.sandic@student.kit.edu}{aleksandar.sandic@student.kit.edu}
\end{center}

\paragraph{\textcolor{red}{Hinweise f"ur Autoren (dieser Teil wird nach der Fertigstellung entfernt)}}
Hier ist eine Liste an Konventionen f"ur das Skript. Diese Regeln sind nicht in Stein gemei\ss elt und wenn jemand anderer Meinung ist können wir uns absprechen und die Regeln "andern, aber ansonsten sollte jeder der am Skript mitarbeitet sich nach M"oglichkeit an sie halten. Selbst wenn wir unsere Meinung anschließend "andern ist es viel leichter die "Anderungen vorzunehmen wenn alles einheitlich ist.

\begin{description}[font=\normalfont\itshape]
\item[Anf"uhrungszeichen:]
	sie können nicht direkt im Code gesetzt werden, man muss entweder den genauen Befehl kennen oder man nimmt einfach unseren eigenen \verb|\quot{}| Befehl.
\item[Abs"atze:]
	benutzt nicht \verb|\\| f"ur neue Abs"atze, dieser Befehl ist dazu da um einen Zeilenumbruch zu erzwingen. F"ur einen neuen Absatz fügt eine leere Zeile ein (also zweimal die Enter Taste dr"ucken), der Unterschied ist dass dadurch ein etwas gr"o\ss erer Abstand zwischen Abs"atzen eingef"ugt wird der die Unterscheidung leichter macht (der Abstand wird durch \verb|parskip| bestimmt). Es ist in der Literatur "ublich die erste Zeile eines Absatzes einzur"ucken (\verb|parindent|), allerdings sind unsere Abs"atze dafür zu kurz, es w"urde h"asslich aussehen alle paar Zeilen eine Einr"uckung zu haben. Nach einem Theorem wird automatisch ein noch gr"o"serer Abstand eingef"ugt, es gibt also keinen Grund an der Stelle etwas zu "andern.
\item[Theoreme:]
	das \verb|ntheorem| Paket stellt theoremartige Umgebungen zur Verf"ugung. Unsere Konvention besagt dass die Befehle aller nummerierten Theoreme mit Gro"sbuchstaben anfangen sollen (zum Beispiel \verb|Dfn| f"ur Definitionen). F"ur alle nummerierten Theoreme gibt es auch eine nicht nummerierte Version mit Kleinbuchstaben (also \verb|dfn|). F"ur Beweise sollte \verb|bew| anstelle von \verb|proof| verwendet werden. Bei jedem Theorem kann man als optionalen Parameter einen speziellen Namen vergeben, wie \verb|\begin{Satz}[Satz des Pythagoras]|. Wenn in der Vorlesung keine Umgebung verwendet wurde dann benutzt auch keine im Skript, wenn wir "uberall Bemerkung auf Bemerkung haben bläht es das Skript auf und st"ort den Lesefluss. Bitte nummeriert nur Theoreme die auch in der Vorlesung nummeriert wurden.
\item[Spezielle Theoreme:]
	wenn eine spezielle Theoremumgebung benötig wird, wie zum Beispiel \quot{Anwendung}, f"ur die es sich nicht lohnen w"urde eine eigene Theoremumgebung zu definieren, kann die \verb|\emptythm| Umgebung verwendet werden. Gebt dabei den gew"unschten Namen als optionales Argument ein, also \verb|\emptythm[Anwendung]|. Der Name wird dabei ganz normal gesetzt, also ohne Klammern wie sonst bei optionalen Argumenten "ublich.
\item[Listen und Aufz"ahlungen:]
	die Umgebungen \verb|itemize|, \verb|enumerate| und \verb|description| k"onnen dank des \verb|enumitem| Pakets stark beeinflusst werden. Im Allgemeinen sind die Standardeinstellungen in Ordnung, allerdings m"u"sen wir manchmal ein Ausnahme machen und Listen "andern. Die h"aufigste Option wird \verb|label=| sein um die Nummerierung zu "andern. Haltet euch an den Standard der Vorlesung. Eintr"age sind standardm"a\ss ig leicht einger"uckt, das ist gut f"ur die Lesbarkeit, aber bei mehr als einer oder zwei Zeilen pro Eintrag wird es h"asslich. Mit der Option \verb|leftmargin=*| wird der Listeineintrag ganz an den Rand geschoben (beziehungsweise soweit es geht bei Unterlisten), das spart Platz und sieht besser aus. Die Schriftart von Labels bei der \verb|description| Umgebung kann mit der Option \verb|font=| eingestellt werden.
\item[Zeichungen:]
	sollten vorzugsweise in TikZ gschrieben werden anstatt ein externes Programm zu verwenden. TikZ liefert saubere Ergebnisse, passt nahtlos in das Design, erzeugt keine weiteren Abh"angigkeiten und die Zeichnungen können jederzeit von jedem ge"andert oder korrigiert werden. F"ur h"aufig verwendete Formen wurden eigene Befehle geschrieben, mehr dazu findet man im Quellcode.
\item[Umlaute und "s:]
	je nach Kodierung kann man diese Zeichen entweder direkt im Code benutzen oder man schreibt \verb|"a|, \verb|"o|, \verb|"u| und \verb|"s|. Wir k"onnen f"ur's Erste bei normalen Umlauten bleiben, ich würde aber trotzdem am Ende einmal Suchen\&Ersetzen durchlaufen lassen.
\item[Farben:]
	Blau ist f"ur Hyperlinks reserviert, rot f"ur Fehler und Unsicherheiten, grau f"ur Anmerkungen und Notizen f"ur den Leser. Der Vorteil von blau ist dass es bei einem Schwarz-Wei"s Ausdruck nicht von schwarz unterscheidbar ist.
\item[Randnotizen:]
	können mit \verb|\marginnote{}| gesetzt werden (beachtet dass keine Leerzeilen im Befehl erlaubt sind). Randnotizen sind gut f"ur Anmerkungen oder kleine Zeichnungen. Wir k"onnen bei Bedarf auch einstellen dass bei Definitionen der definierte Begriff im Rand daneben steht. Das \verb|ntheorem| Paket erlaubt es auch die Nummer von Theormen in den Rand zu stellen, das w"are auch eine "Uberlegung wert.
\item[Glossareintr"age:]
	das \verb|glossaries| Paket erlaubt es ein Glossar zu erstellen. Hier sollten alle Begriffe stehen die \emph{nicht} in dieser Vorlesung definiert wurden aber verwendet werden. Die Idee ist dass der Leser manchmal nicht die genaue Definition wei"s und es ist praktisch wenn man eine kleine Ged"achtnisst"utze im Skript hat.
\end{description}

%%
%% 1. Vorlesung 16.10.12
%% 
%% Skript Differentialgeometrie im Wintersemester 12/13
%% Zur Vorlesung von Dr. Grensing am KIT Karlsruhe
%%
%% Mitschrieb und Textsatz von Jan-Bernhard Kordaß.
%%

\section*{"Ubersicht}

\begin{itemize}
\item Mannigfaltigkeiten, Tangentialvektoren
\item Kovariante Ableitung
\item Riemannsche Metriken
\item Krümmung
\item Jacobifelder
\item Satz von Bonnet
\end{itemize}

\section{Differenzierbare Mannigfaltigkeiten}

\begin{dfn*}
  Eine $n$-dimensionale \CmMark{topologische Mannigfaltigkeit} $M$ ist ein topologischer Hausdorff-Raum mit einer abzählbaren Basis der Topologie in dem zu jedem Punkt $p \in M$ eine offene Menge $U$ mit $p \in U$ existiert und ein Hom"oomorphismus $\phi \colon U \to V$ auf eine offene Menge $V \subset \R^{n}$.

% Abbildung 1-1
%\CmPutSvg{1-1-topologische-mf}{8.5cm}
\begin{center}\begin{tikzpicture}[font=\scriptsize]
	\draw[->] (-1.5,0) to[out=20, in=160]node[above,font=\scriptsize]{$\varphi' \circ \varphi^{-1}$} (1.5,0);
	
	\draw[->] (-4,-0.5) -- (-2,-0.5); \draw[->] (-3.75,-0.75) -- (-3.75, 1.25); \node[font=\scriptsize] at (-4, 1.25) {$\R^n$};
	\draw[->] (2,-0.5) -- (4,-0.5); \draw[->] (2.25,-0.75) -- (2.25, 1.25); \node[font=\scriptsize] at (2, 1.25) {$\R^m$};
	
	\node[font=\scriptsize] at (0,2) {$U \cap U' \ne 0$};
	
	\draw (-4.25, 1.75) to[out=70,in=180] (-1.75,3) to[out=300,in=90] (-1.25, 1.25) to[out=180,in=340] (-4.25, 1.75) -- cycle; \node at (-1.25,3) {$M$};
	\filldraw[fill=gray!20] (-2.75,2) circle(0.4); \node[font=\scriptsize] at (-3.25,2.25) {$U$};
	\filldraw[fill=gray!20] (-3,0.25) circle (0.5); \node at (-2.25, 0.5) {$V$};
	\draw[->] (-2.75,1.5) to[out=280,in=80] node[right]{$\varphi$} (-2.75,0.75);
			
	\draw (1.75, 1.75) to[out=70,in=180] (4.25,3) to[out=300,in=90] (4.75, 1.25) to[out=180,in=340] (1.75, 1.75) -- cycle; \node at (4.75,3) {$M$};
	\filldraw[fill=gray!20] (3.55,2.25) circle(0.6); \node[font=\scriptsize] at (2.75,2.25) {$U'$};
		
	\coordinate (ctrl0up) at ($(2.5,-0.25) + 0.2*(0.5,2)$); \coordinate (ctrl0down) at ($(2.5,-0.25) + 0.2*(0,-1.5)$);
	\coordinate (ctrl1down) at ($(3,0.25) - 0.1*(0.5,1)$); \coordinate (ctrl1up) at ($(3,0.25) + 0.1*(0.5,1)$);
	\coordinate (ctrl2down) at ($(3,0.7) - 0.1*(0.5,1)$); \coordinate (ctrl2up) at ($(3,0.7) + 0.1*(0.5,1)$);
	\coordinate (ctrl3down) at ($(4,0.5) + 0.3*(-0.5,1)$); \coordinate (ctrl3up) at ($(4,0.5) - 0.3*(-0.25,1)$);
	\coordinate (ctrl4down) at ($(3.75,-0.3) + 0.2*(0.8,1)$); \coordinate (ctrl4up) at ($(3.75,-0.3) - 0.2*(0.7,0.75)$);
	\begin{scope}
		\fill[gray!20] (2.5,-0.25) ..controls(ctrl0up) and (ctrl1down).. (3,0.25) ..controls(ctrl1up) and (ctrl2down).. (3,0.7) ..controls(ctrl2up) and (ctrl3down).. (4,0.5) ..controls(ctrl3up) and (ctrl4down).. (3.75,-0.3) ..controls(ctrl4up) and (ctrl0down).. (2.5,-0.25); \node at (4.25, 0.5) {$V'$};
		\clip(2.5,-0.25) ..controls(ctrl0up) and (ctrl1down).. (3,0.25) ..controls(ctrl1up) and (ctrl2down).. (3,0.7) ..controls(ctrl2up) and (ctrl3down).. (4,0.5) ..controls(ctrl3up) and (ctrl4down).. (3.75,-0.3) ..controls(ctrl4up) and (ctrl0down).. (2.5,-0.25); \node at (4.25, 0.5) {$V'$};
		\fill[gray] (2,0) circle (1);
		 (2.5,-0.25) ..controls(ctrl0up) and (ctrl1down).. (3,0.25) ..controls(ctrl1up) and (ctrl2down).. (3,0.7) ..controls(ctrl2up) and (ctrl3down).. (4,0.5) ..controls(ctrl3up) and (ctrl4down).. (3.75,-0.3) ..controls(ctrl4up) and (ctrl0down).. (2.5,-0.25); \node at (4.25, 0.5) {$V'$};
	\end{scope}
	\draw  (2.5,-0.25) ..controls(ctrl0up) and (ctrl1down).. (3,0.25) ..controls(ctrl1up) and (ctrl2down).. (3,0.7) ..controls(ctrl2up) and (ctrl3down).. (4,0.5) ..controls(ctrl3up) and (ctrl4down).. (3.75,-0.3) ..controls(ctrl4up) and (ctrl0down).. (2.5,-0.25) -- cycle; \node at (4.25, 0.5) {$V'$};
	\draw[->] (3.5,1.5) to[out=280,in=80] node[right]{$\varphi'$} (3.5,0.75);
\end{tikzpicture}\end{center}

  $\varphi' \circ \varphi^{-1}$ ist ein Hom"oomorphismus offener Mengen des $\R^n$ bzw. $\R^m$. Nach dem Satz von Brouwer (1912) gilt dann $m = n$. Damit ist die Dimension einer zusammenh"angenden topologischen Mannigfaltigkeit eindeutig definiert.\\

  Die Abbildung $\varphi \colon U \to V \subset \R^n$ hei\ss t \CmMark{Karte} von $M$ um $p$, die Menge $U$ hei\ss t \CmMark{Kartengebiet}.\\

  Eine Menge von Karten $\mathcal A = \{(\varphi_{\alpha}, U_{\alpha}) \mid \alpha \in J \}$ hei\ss t \CmMark{Atlas} von $M$, falls $\bigcup_{\alpha \in J}U_{\alpha} = M$.\\

  Ein Atlas $\mathcal A$ von $M$ hei\ss t $C^k$-Atlas, wenn für alle $\alpha, \beta \in J$ mit $U_{\alpha} \cap U_{\beta} \neq \emptyset$ der sogenannte \CmMark{Kartenwechsel}:
  \begin{align*}
    \varphi_{\beta} \circ \varphi_{\alpha}^{-1}\colon \varphi_{\alpha}(U_{\alpha} \cap U_{\beta}) \to \varphi_{\beta}(U_{\alpha} \cap U_{\beta})
  \end{align*}
  ein $C^k$-Diffeomorphismus ist.\\

  % Abbildung 1-2
  %\CmPutSvg{1-2-kartenwechsel}{8cm}
  \begin{center}\begin{tikzpicture}[font=\scriptsize]
  	\draw[->] (-1.5,0) to[out=20, in=160]node[above,font=\scriptsize]{$\varphi_\beta \circ \varphi^{-1}_\alpha$} (1.5,0);
	
	\draw[->] (-4,-0.5) -- (-2,-0.5); \draw[->] (-3.75,-0.75) --node[left]{$\R^n$} (-3.75, 1.25);
	\draw[->] (2,-0.5) -- (4,-0.5); \draw[->] (2.25,-0.75) -- (2.25, 1.25);
	
	\draw[thick]  (-0.25, 3) to[out=0,in = 150] (2,2.5) -- (1.75, 1.5) to[out=190,in=350] (-1.75, 1.5) to[out=90,in=180] (-0.25, 3) -- cycle; \node at (2.25,2.75) {$M$};
	
	\begin{scope}
		\clip (0.25,2.25) circle(0.5);
		\clip (-0.25,2) circle(0.5);
		\fill[gray!20] (0,2) circle(1);
	\end{scope}
	\draw (0.25,2.25) circle(0.5) (-0.25,2) circle(0.5); \node at (-1, 2.25) {$U_\alpha$}; \node at (1,2.5) {$U_\beta$};
	
	\draw[->] (-0.5,2) to[out=180,in=75] node[left]{$\varphi_\alpha$} (-3,0.25);
	\draw[->] (0.5,2.25) to[out=0,in=105] node[right]{$\varphi_\beta$} (3,0.25);
  \end{tikzpicture}\end{center}


  Eine Karte $\psi \colon U \to V$ von $M$ hei\ss t \CmMark{verträglich} mit einem $C^k$-Atlas $\mathcal A = \{(\varphi_{\alpha},U_{\alpha}) \mid \alpha \in J\}$ wenn jeder Kartenwechsel
  \begin{align*}
    \varphi_{\alpha} \circ \psi(U \cap U_{\alpha}) \to \varphi_{\alpha}(U \cap U_{\alpha})
  \end{align*}
  ein $C^k$-Diffeomorphismus ist, i.e. $\mathcal A' = \mathcal A \cup \{(\psi, U)\}$ ist ebenfalls ein $C^k$-Atlas.\\

  Die Menge aller mit $\mathcal A$ verträglichen Karten ist ein \CmMark{maxmaler $C^k$-Atlas}. Jeder maximale Atlas enthält alle mit ihm verträglichen Karten. Ein maximaler $C^k$-Atlas hei\ss t auch \CmMark{$C^k$-differenzierbare Struktur}.

\end{dfn*}

\begin{dfn}[differenzierbare Mannigfaltigkeit]
  Eine \CmMark{differenzierbare Mannigfaltigkeit} der Klasse $C^k$ ist eine topologische Mannigfaltigkeit zusammen mit einer $C^{k}$-differenzierbaren Struktur.\\
\end{dfn}

\begin{bsp}
  Einige Beispiele f"ur glatte Mannigfaltigkeiten:
  \begin{enumerate}%[1)]
  \item $M = \R^n, \mathcal A = \{(\Id_{\R^n},\R^n)\}$
  \item $M \subset \R^n$ offen, $\mathcal A = \{(\imath_{M},M)\}$
  \item $S^1 \subset \R^2$ ist eine eindimensionale $C^{\infty}$-Mannigfaltigkeit:
    \begin{align*}
      U = \{(\sin t, \cos t) \mid t \in (0,2\pi)\}
    \end{align*}

    % Abbildung 1-3
    \marginnote{\begin{center}\begin{tikzpicture}[font=\footnotesize]
    		%\draw[step=0.25,gray!15] (-1,-1) grid (1,1); \draw[step=0.5,gray!30] (-1,-1) grid (1,1); \fill (0,0) circle(0.1); %Hilfsgitter
		\draw (0,0) circle (1); \draw[dashed] (0,0) circle (1.1); \draw[dotted] (0,0) circle (0.9); \node at (1,1) {$S^1$};
		\filldraw[fill=white] (-1,0) circle (0.1) (1,0) circle (0.1);
    \end{tikzpicture}\\
    \textcolor{gray}{$S^1$ Einheitskreis}
    \end{center}}[-2cm]
    % \CmMarginSvg[-2cm]{1-3-karten-der-s1}{3cm}

    ist offen in $S^1$ und die Kartenabbildung
    \begin{align*}
      \varphi \colon (\sin t, \cos t) \mapsto t
    \end{align*}
    ist ein Hom"oomorphismus.
    \begin{align*}
      \varphi' \colon U' = \{(\sin t, \cos t) \mid t \in (-\pi,\pi)\} \to (-\pi,\pi)
    \end{align*}
    ebenfalls. $\mathcal A = \{(\varphi, U), (\varphi',U')\}$ ist ein Atlas von $S^1$, denn $U \cup U' = S^1$.
    \begin{align*}
      & \varphi' \circ \varphi^{-1} \colon \varphi(U \cap U') \to \varphi'(U \cap U')\\
      & (0,\pi)\cup(\pi,2\pi) \to (-\pi,0)\cup(0,\pi), t \mapsto \begin{cases}
        t & 0 < t < \pi\\
        t-2\pi & \pi < t < 2\pi
      \end{cases}
    \end{align*}

  \item Jeder reelle Vektorraum endlicher Dimension ist in kanonischer Weise eine $C^{\infty}$-Mannigfaltigkeit.\\
    W"ahle eine Basis $\{v_1, \ldots, v_n\}$ von $V$. Diese definiert mit
    \begin{align*}
      \varphi\left(\sum\lambda_iv_i\right) = (\lambda_1, \ldots, \lambda_n)
    \end{align*}
    eine Bijektion auf $\R^n$. Damit erhält man eine globale Karte von $V$.
    Der zugehörige Atlas h"angt nicht von der Wahl der Basis ab, denn ist $\{w_1, \ldots, w_n\}$ eine weitere Basis von $V$ und $\psi(\sum \lambda_iw_i) = (\lambda_1, \ldots, \lambda_n)$ eine weitere Karte, so ist $\varphi \circ \psi^{-1}$ als Endomorphismus des $\R^n$ schon $C^{\infty}$.

  \item $S^n = \{(x^0, x^1, \ldots, x^n) \mid \sum_{i = 0}^n(x^{i})^2 = 1\}$.\\

    % Abbildung 1.4
    %\CmMarginSvg{1-4-s3-sphaere}{3.5cm}
    \marginnote{\begin{center}\begin{tikzpicture}[font=\scriptsize]
    		%\draw[step=0.25,gray!15] (-1,-1) grid (1,1); \draw[step=0.5,gray!30] (-1,-1) grid (1,1); \fill (0,0) circle(0.1); %Hilfsgitter
		% Koordinatenachsen mit Beschriftung
		\draw[->] (0,-1.25) -- (0,1.5) node[left]{$x^0$}; \draw[->] (-1.25,0) -- (1.5,0) node[below]{$x^1$}; \draw[->] (1,1) -- (-1.25,-1.25) node[right]{$x^2$}; \node at (1.25, 1.5) {$S^2 \subset \R^3$};
		% Kreis, Ellipse und Gerade (verwende Namen um Schnittpunkt bestimmen zu koennen)
		\path[draw, thick, name path=kreis] (0,0) circle (1) ellipse(1 and 0.5); \path[draw,name path=gerade] (0,1) -- (1,-1.25);
		% Punkte N und p
		\filldraw[fill=white] (0,1) circle (0.05) node[anchor=south west,xshift=-2,yshift=-1.5]{$N$} ($(0,1)+0.35*(1,-1.25)-0.35*(0,1)$) circle (0.05) node[right]{$p$};
		% Punkt phi(p) bei Schnittpunkt von Gerade und Kreis
		\path [name intersections={of=kreis and gerade}]; \filldraw[fill=white] (intersection-2) circle(0.05) node[right]{$\varphi(p)$};
	\end{tikzpicture}\end{center}}%[3.5cm]
    
    Betrachte den Nordpol $N = (1,0,\ldots,0)$ und den S"udpol $S = (-1,0,\ldots,0)$ und die Abbildung
    \begin{align*}
      & \varphi \colon U = S^{n}\setminus\{N\} \to \R^n, x \mapsto \left(\frac{x^1}{1-x^0}, \ldots, \frac{x^{n}}{1-x^0}\right),\\
      & \psi \colon U' = S^{n} \setminus \{S\} \to \R^n, x \mapsto \left(\frac{x^1}{1+x^0}, \ldots, \frac{x^n}{1+x^0}\right)
    \end{align*}

    Aufgabe: Zeige, dass $(\varphi, U), (\psi, U')$ einen $C^{\infty}$-Atlas auf $S^n$ definiert.

  \end{enumerate}
\end{bsp}

\begin{dfn}[Differenzierbare Abbildungen]
Eine stetige Abbildung $f \colon M \to N$ zwischen glatten Mannigfaltigkeiten $M$ und $N$ hei\ss t \CmMark{glatt} ($C^{\infty}$-differenzierbar), wenn es zu jedem $p \in M$ Karten $(\varphi, U)$ in $M$ um $p$ und geeignete $(\varphi', U')$ in $N$ um $f(p)$ gibt, so dass $\varphi' \circ f\circ\varphi^{-1}$ glatt ist.
% Abbildung 1-5
%\CmPutSvg{1-5-glatte-abb}{9cm}
\begin{center}\begin{tikzpicture}[font=\scriptsize]
	%\draw[step=0.25,gray!15] (-5,-1) grid (5,5); \draw[step=0.5,gray!30] (-5,-1) grid (5,5); \fill (0,0) circle(0.1); %Hilfsgitter
	
	% Die Abbildungspfeile
	\draw[->] (-1.5,0) to[out=20, in=160]node[above]{$\varphi' \circ f \circ \varphi$} (1.5,0);
	\draw[->] (-1,2) --node[above]{$f$} (1.5,2);
	
	% Die Achsen
	\draw[->] (-4.5,-0.5) -- (-2,-0.5); \draw[->] (-4.25,-0.75) --node[left]{$\R^n$} (-4.25, 1.25);
	\draw[->] (2,-0.5) -- (4.5,-0.5); \draw[->] (2.25,-0.75) --node[left]{$\R^m$} (2.25, 1.25);
	
	% Die Blasen
	\draw[thick] (-4.25, 1.75) to[out=70,in=180] (-1.75,3) to[out=300,in=90] (-1.25, 1.25) to[out=180,in=340] (-4.25, 1.75) -- cycle; \node[font=\normalfont] at (-1.25,3) {$M$};
	\draw[thick] (1.75, 1.75) to[out=70,in=180] (4.25,3) to[out=300,in=90] (4.75, 1.25) to[out=180,in=340] (1.75, 1.75) -- cycle; \node[font=\normalfont] at (4.75,3) {$N$};
	
	% Die linke Kartoffel (zuerst werden die Punkte definiert, dann die Richtungsvektoren der Splines, dann die Kartoffel selbst)
	\coordinate (kartoffel0l) at (-3.25,1.75); \coordinate (kartoffel1l) at (-3.25,2.5); \coordinate (kartoffel2l) at (-2.25,2.25); \coordinate (kartoffel3l) at (-2.5,1.75);
	\coordinate (ctrlk0l) at (-0.25,0.5); \coordinate (ctrlk1l) at (0.5,0.25); \coordinate (ctrlk2l) at (-0.25,1); \coordinate (ctrlk3l) at (2,0.25);
	\draw (kartoffel0l) ..controls($(kartoffel0l)+0.5*(ctrlk0l)$) and ($(kartoffel1l)-0.3*(ctrlk1l)$).. (kartoffel1l) ..controls($(kartoffel1l)+0.6*(ctrlk1l)$) and($(kartoffel2l)+0.45*(ctrlk2l)$).. (kartoffel2l) ..controls($(kartoffel2l)-0.25*(ctrlk2l)$) and ($(kartoffel3l)+0.15*(ctrlk3l)$).. (kartoffel3l)  ..controls($(kartoffel3l)-0.1*(ctrlk3l)$) and ($(kartoffel0l)-0.9*(ctrlk0l)$).. (kartoffel0l); \node at (-3.5,2.25) {$U$};
	% Der Punkt in der Kartoffel, der Pfeils raus und der Kreis
	\draw[->] (-2.75,2) node[right]{$p$} to[out=280,in=80] node[right]{$\varphi$} (-2.75,0); \fill (-2.75,2) circle (0.05);
	\draw (-3,0.25) circle(0.5); \node at (-3.5,0.75) {$V$};
	
	% Die rechte Kartoffel
	\coordinate (kartoffel0r) at (3.25,1.75); \coordinate (kartoffel1r) at (3.5,2.5); \coordinate (kartoffel2r) at (4.5,2.25); \coordinate (kartoffel3r) at (4.25,1.5);
	\coordinate (ctrlk0r) at (-0.25,0.5); \coordinate (ctrlk1r) at (-0.25,0.25); \coordinate (ctrlk2r) at (-0.25,0.5); \coordinate (ctrlk3r) at (0.25,0);
	\draw (kartoffel0r) ..controls($(kartoffel0r)+0.5*(ctrlk0r)$) and ($(kartoffel1r)-(ctrlk1r)$).. (kartoffel1r) ..controls($(kartoffel1r)+(ctrlk1r)$) and ($(kartoffel2r)+(ctrlk2r)$).. (kartoffel2r) ..controls($(kartoffel2r)-(ctrlk2r)$) and ($(kartoffel3r)+(ctrlk3r)$).. (kartoffel3r) ..controls($(kartoffel3r)-(ctrlk3r)$) and ($(kartoffel0r)-(ctrlk0r)$).. (kartoffel0r); \node at (3.25,2.25) {$U'$};
	
	\draw[->] (3.75,2) node[right]{$f(p)$} to[out=280,in=80] node[right]{$\varphi'$} (3.75,0); \fill (3.75,2) circle (0.05);
	\draw (3.5,0.25) circle(0.5); \node at (3,0.75) {$V'$};
\end{tikzpicture}

\textcolor{red}{Sollte das in der Zeichnung beim unteren Pfeil nicht $\varphi'\circ f \circ \varphi^{-1}$ hei\ss en?}\end{center}
Die Menge aller glatten Abbildungen von $M$ nach $N$ wird $C^{\infty}(M,N)$ genannt.

\end{dfn}

\textbf{Konvention}: Ab jetzt seien zunächst alle Mannigfaltigkeiten, wie auch alle Abbildungen als glatt vorrausgesetzt.

\begin{bem}
  Da Kartenwechsel $C^{\infty}$ sind, gilt obige Bedingung automatisch für alle Karten von $M$ und $N$ (evtl. nach Einschränkung).
\end{bem}

\begin{bsp}
  Es folgen zwei Beispiele für diffenrenzierbare Abbildungen:
  \begin{enumerate}
  \item $(\varphi,U) \in \mathcal A \Rightarrow \varphi \in C^{\infty}(U,\R^n)$, denn
    \begin{align*}
      \Id_{\R^n}\circ \varphi \circ \varphi^{-1} = \varphi \circ \varphi^{-1} \in C^{\infty}.
    \end{align*}
  \item $f \in C^{\infty}(M,N), \ g \in C^{\infty}(N,P) \Rightarrow g \circ f \in C^{\infty}(M,P)$, denn
    \begin{align*}
      \varphi_p \circ g \circ f \circ \varphi^{-1}_m = (\varphi_p \circ g \circ \varphi_n^{-1}) \circ (\varphi_n \circ f \circ \varphi_m^{-1}) \in C^{\infty}.
    \end{align*}
  \end{enumerate}
\end{bsp}

\begin{dfn}[Diffeomorphismus]
  Eine Abbildung $f \colon M \to N$ hei\ss t \CmMark{Diffeomorphismus}, wenn $f$ bijektiv ist und $f$, sowie $f^{-1}$ $C^{\infty}$-Abbildungen von $M$ nach $N$ sind. Insbesondere haben $M$ und $N$ in diesem Fall dieselbe Dimension.\\

Die Menge der Diffeomorphismen von $M$ nach $N$ wird mit $\Diff(M,N)$ bezeichnet. Die Menge der Diffeomorphismen von $M$ nach $M$ wird mit $\Diff(M)$ bezeichnet. $(\Diff(M), \circ)$ ist eine Gruppe.

\end{dfn}

%%% Local Variables: 
%%% mode: latex
%%% TeX-master: "../skript-diffgeom"
%%% End: 

%% 
%% 2. Vorlesung <2012-10-19 Fri>, Fortsetung
%% 
%% Skript Differentialgeometrie im Wintersemester 12/13
%% Zur Vorlesung von Dr. Grensing am KIT Karlsruhe
%% 
%% Mitschrieb und Textsatz von Jan-Bernhard Kordaß.
%% 

\chapter{Tangentialvektoren und Tangentialräume}

% Abbildung 2-1
\CmMarginSvg{2-1-tangentialvektoren-motivation}{3.5cm}

Betrachte in der nebenstehenden Abbildung eine differenzierbare \gls{Kurve} $c \colon (-\varepsilon,\varepsilon) \to S^2$ mit $c(0) = p$. Dann gilt:
\begin{align*}
  0 = \difffrac[t=0]{}{t} \left<c(t),c(t)\right> = 2\left<\dot c(0),c(0)\right> = 2 \left<\dot c(0),p\right> 
  \Rightarrow \dot c(0) \in p^{\perp}.
\end{align*}

% Bemerke $1 = \left<c(t),c(t)\right>$

\textcolor{red}{(Ich habe Punkte "uber einige der $c$ gesetzt, bitte "uberpr"ufen)}Es sei $M$ eine glatte Mannigfaltigkeit und es seien glatte Kurven $c_i\colon (-\varepsilon_i,\varepsilon_i) \to M$ mit $c_1(0) = c_2(0) = p \in M$ gegeben.

Die Kurven heißen \CmMark{äquivalent}, wenn es eine Karte $(\varphi,U)$ von $M$ und $p$ gibt, so dass gilt
\begin{align*}
  \difffrac[t=0]{}{t}(\varphi \circ c_1) = \difffrac[t=0]{}{t}(\varphi \circ c_2)
\end{align*}

\begin{Lemma}
  Der oben definierte Begriff der Äquivalenz ist unabhängig von der Wahl der Karte.
\end{Lemma}

\begin{proof}
  Es sei $(\psi,V)$ eine weitere Karte von $M$ um $p$. Dann gilt:
  \begin{align*}
    \difffrac[t=0]{}{t}(\psi\circ c_1) & = \difffrac[t=0]{}{t}(\psi\circ\varphi^{-1}\circ\varphi \circ c_1) = \D (\psi \circ \varphi^{-1})|_{\varphi(p)} \cdot \difffrac[t=0]{}{t}(\varphi \circ c_1)\\
    & = \D(\psi \circ \varphi^{-1})|_{\varphi(p)} \cdot \difffrac[t=0]{}{t}(\varphi \circ c_2) = \ldots = \difffrac[t=0]{}{t}(\psi \circ c_2).
  \end{align*}
\end{proof}

\begin{Dfn}[Geometrische Definition des Tangentialraums]
  Es sei $M$ eine glatte Mannigfaltigkeit und $p \in M$. Ein (geometrischer) \CmMark{Tangentialvektor} an $M$ in $p$ ist eine Äquivalenzklasse von Kurven $c$ mit $c(0) = p$. Die Menge
  \begin{align*}
    \T_{p}^{\text{geo}}M = \{ [c] \mid c \colon (-\varepsilon,\varepsilon) \to M \text{ glatt}, c(0) = p\}
  \end{align*}
  heißt (geometrischer) \CmMark{Tangentialraum} an $M$ in $p$.
\end{Dfn}

\begin{Bem}
  Mit den Bezeichnungen wie oben ist die folgende Abbildung bijektiv:
  \begin{align*}
    A \colon \T_p^{\text{geo}}M \to \R^n, [c] \mapsto \difffrac[t=0]{}{t}(\varphi \circ c).
  \end{align*}
\end{Bem}

\begin{proof}
  Zu jedem Vektor $v \in \R^n$ sei $B(v) = [t \mapsto \varphi^{-1}(\varphi(p) + tv)]$ die Äquivalenzklasse der abgebildeten Kurve auf der Mannigfaltigkeit.

  % Abbildung 2-2
  \CmPutSvg{2-2-beweis-bijektivitaet-tpm-rn}{10cm}

  \[ A B(v) = \difffrac[t=0]{}{t}(\varphi \circ B(v)) = \difffrac[t=0]{}{t}(\varphi \circ (\varphi^{-1}(\varphi(p) + tv)) = \difffrac[t=0]{}{t}(\varphi(p) + tv) = v. \]
  \[ B A (\underbrace{[c]}_{\ni c}) = B(v_c) = [t \mapsto \varphi^{-1}(\varphi(p) + tv_c)] \text{ wobei } v_c = \difffrac[t=0]{}{t}(\varphi \circ c). \]
  Die Kurven $c$ und $t \mapsto \varphi^{-1}(\varphi(p) + tv_c)$ sind äquivalent, also ist $B A[c] = [c]$ und somit $A$ bijektiv.
\end{proof}

Damit erhält $\T_p^{\text{geo}}M$ die Struktur eines reellen Vektorraumes vermöge der folgenden Verknüpfung:
\begin{align*}
  \lambda[c_1] + \mu[c_2] = A^{-1}(\lambda A[c_1]+ \mu A[c_2]).
\end{align*}
Dabei gilt $\lambda[c_1]+\mu[c_2] = [c]$ für $c(t) = \varphi^{-1}(\varphi(p) + t(\lambda v_1 + \mu v_2))$ mit $v_i = \difffrac[t=0]{}{t}(\varphi \circ c_i)$.

\begin{Lemma}
  Die oben definierte Lineare Struktur ist unabhängig von der Wahl der Karte.
\end{Lemma}

\begin{proof}
  Es sei $(\psi, V)$ eine Karte von $M$ um $p$ und $A'[c] = \difffrac[t]{}{t}(\psi \circ c)$. Dann gilt:
  \begin{align*}
    A A'^{-1}(v) & = \difffrac[t=0]{}{t}(\varphi \circ (\psi^{-1} (\psi(p) + tv)))\\
    & = \D(\varphi \circ \psi^{-1})|_{\psi(p)} \cdot \difffrac[t=0]{}{t}(\psi \circ \psi^{-1}(\varphi(p) + tv)) = \D (\varphi \circ \varphi^{-1}) \cdot v.
  \end{align*}
  Also ist $A A'^{-1}$ linear,
  \begin{align*}
    A'^{-1}(\lambda A'[c_1] + \mu A'[c_2]) & = A^{-1}(A A'^{-1}(\lambda A'[c_1] + \mu A'[c_2]))\\
    & = A^{-1} (\lambda A A'^{-1}[c_1] + \mu A A'^{-1} [c_2])\\
    & = A^{-1}(\lambda A [c_1] + \mu A [c_2]).
  \end{align*}
\end{proof}


% 3. Vorlesung <2012-10-23 Tue>

\paragraph{Motivation: Richtungsableitungen im $\R^n$}\hfill
\begin{Bem}
  
  Für $f,g \in C^{\infty}(\R^n), \ x,y \in \R^n$ ist die \CmMark{Richtungsableitung} wie folgt definiert:
  \begin{align*}
    \partial_vf(x) = \D f|x \cdot v = \difffrac[t=0]{}{t}f(x+tv).
  \end{align*}
  Diese erfüllt die Leibniz-Regel:
  \begin{align*}
    \partial_v(fg)(x) = \partial_vf(x)\cdot g(x) + f(x) \cdot \partial_v(g)(x).
  \end{align*}
\end{Bem}

\begin{Dfn}[Algebraische Definition des Tangentialraumes]
  Es sei $M$ eine glatte Mannigfaltigkeit und $p\in M$. Ein (algebraischer) \CmMark{Tangentialvektor} an $M$ in $p$ ist eine Lineare Abbildung $X_p \colon C^{\infty}(M) \to \R$, welche die Leibniz-Regel erfüllt:
  \begin{align*}
    X_p(fg) = X_p(f) \cdot g(p) + f(p) \cdot X_p(g).
  \end{align*}

  Die algebraischen Tangentialvektoren bilden einen reellen Vektorraum $\T_p^{\text{alg}}M$, den Tangentialraum an $M$ in $p$.
\end{Dfn}

\begin{Lemma}
  Es sei U eine Umgebung von $p \in M$. Dann existiert eine Umgebung $V \subset U$ von $p$ und eine glatte reellwertige Funktion $\sigma \in C^{\infty}(M)$ mit den Eigenschaften $\sigma|_V = 1$ und $\supp(\sigma) \subset U$.
\end{Lemma}

%%%
%%% Abbildung 2-3
%%% 
\textcolor{red}{Abbildung 2-3}


\begin{proof}
  Man kann o.E. annehmen, dass $U$ Kartengebiet einer Karte $\varphi$ von $M$ um $p$ ist und $\varphi(p) = 0 \in \R^n$.\\

  Es sei $\varepsilon > 0$ so, dass $\overline B_c(0) \subset \varphi(U)$. \\

  %%% 
  %%% Abbildung 2-4
  %%% 
  \textcolor{red}{Abbildung 2-4}

  Ist dann $\eta$ eine glatte Funktion auf $\R$ mit $\eta \equiv 1$ auf $\left[\frac{-\varepsilon^{2}}{2},\frac{\varepsilon^2}{2}\right]$ und $\eta \equiv 0$ auf $\R \setminus (-\varepsilon^2,\varepsilon^2)$, so hat für $U_1 = \varphi^{-1}(B_{\frac{\varepsilon}{2}}(0))$ die Funktion
  \begin{align*}
    \sigma(q) =
    \begin{cases}
      \eta(\|\varphi(q)\|^2) & \text{ für } q \in U_1\\
      0 & \text{ sonst }
    \end{cases}.
  \end{align*}
  die gewünschten Eigenschaften.
\end{proof}

% Lemma 2.
\begin{Lemma}
Für alle $X_p\in\T_p^{\text{alg}}M$ gilt:
\begin{enumerate}[label=(\roman*),widest=ii]
\item $X_p(f) = 0$ falls $f$ in einer Umgebung von $p$ konstant ist.
\item $X_p(f) = X_p(g)$ falls $f$ und $g$ auf einer Umgebung übereinstimmen.
\end{enumerate}
\end{Lemma}

\begin{proof}\begin{enumerate}[label=(\roman*),widest=ii,leftmargin=*]
\item[(ii)]
	Es sei $U$ eine Umgebung von $p$ mit $f|_U = g|_U$. Ist dann $\sigma$ wie in \textcolor{red}{Lemma 2.5}, so gilt $\sigma f = \sigma g$ und aus
	\begin{align*}
		X_p(\sigma)f(p)+\sigma(p)X_p(f) = X_{p}(\sigma f) = X_p(\sigma g) = X_p(\sigma) g(p) + \sigma(p) X_p(g)
	\end{align*}
	folgt $X_p(f) = X_p(g)$.\\
\item[(i)]
	Wegen der $\R$-Linearität und (ii) genügt es $f \equiv 1$ zu betrachten. Es gilt
	\begin{align*}
		X_p(1) = X_p(1 \cdot 1) = X_p(1) \cdot 1 + 1 \cdot X_p(1) = 2 \cdot X_p(1),
	\end{align*}
	also $X_p(1) = 0$.
\end{enumerate}\end{proof}

\begin{Bem}
  Also gilt für $f \in C^{\infty}(M)$ und $g \in C^{\infty}(U)$ direkt:
  \begin{align*}
    & \sigma g =
    \begin{cases}
      \sigma g|_U & \textcolor{red}{\sigma g \in C^{\infty}(M)}\\
      0 & \text{ sonst }
    \end{cases},\\
    & \sigma g \in C^{\infty}(M) 
    \Rightarrow X_p(g) = X_p(\sigma g).
  \end{align*}
  Für eine Karte $\varphi \colon U \to V$ von $M$ und $p$ seien algebraische Tangentialvektoren definiert:
  \begin{align*}
    \pdifffrac[p]{}{x^i} \in \T_p^{\text{alg}}M, \pdifffrac[p]{}{x^i}(f) = \partial_i(f \circ \varphi^{-1})(\varphi(p)) = \D(f \circ \varphi^{-1})|_{\varphi(p)}e_i.
  \end{align*}
\end{Bem}

% Satz 2.7
\begin{Satz}
  Die Vektoren $\pdifffrac[p]{}{x^1},\ldots,\pdifffrac[p]{}{x^n}$ bilden eine Basis von $T_p^{\text{alg}}M$.
\end{Satz}

% Lemma 2.8
\begin{Lemma}
  Es sei $x_0 \in \R^n$ und $g \in C^{\infty}(B_{\rho}(x_0))$.
  Dann existieren glatte Funktionen $h_i \in C^{\infty}(B_{\rho}(x_0))$ mit $h_i(x_0) = \partial_ig(x_0)$ und 
  \begin{align*}
    g(x) = g(x_0) + \sum_{i=1}^n(x^i-x_0^i)h_i(x).
  \end{align*}
\end{Lemma}

\begin{bew}[Beweis des Satzes]
Die $j$-te Komponente $\varphi^j$ der Karte ist glatt und es gilt:
\begin{align*}
	\pdifffrac[p]{}{x^i}(\varphi^j) = \partial_i(\varphi^j \circ \varphi^i)(\varphi(p)) = \partial_ix^j(\varphi(p)) = \delta_i^j.
\end{align*}
Damit sind die Vektoren linear unabhängig.

Es sei $X_p\in \T_p^{\text{alg}}M$ und $f \in C^{\infty}(M)$.
Für $x_0=\varphi(p) \in \R^n, \ B_{\rho}(x_0) \subset \varphi(U)$ und für $g = f \circ \varphi^{-1}|_{B_{\rho}(x_0)}$ gilt mit den Bezeichnungen wie im letzten Lemma:
\begin{align*}
	X_p(f) & = X_p(g \circ \varphi) = X_p(g(\varphi(p)) + \sum \left(\varphi^i - \varphi(p)^i)(h_i \circ \varphi) \right)\\
	& = \underbrace{X_p(g(\varphi(p)))}_{\mathclap{=0}} + \sum X_p((\varphi^i-\varphi(p)^i)(h_i \circ \varphi))\\
	& = \sum X_p(\varphi^i)(h_i\circ\varphi)(p) - X_p(\varphi(p)^i)(h_i\circ \varphi)(p) + \sum (q^i-\varphi(p)^i)(p) X_p(h_i \circ \varphi)\\
	& = \sum_{i=1}^n X_p(\varphi^i)\underbrace{(h_i \circ \varphi)(p)}_{\mathclap{=h_i(\varphi(p) = h_i(x_0) = \partial_ig(x_0) = \partial_i(f\circ \varphi^{-1})(\varphi(p)) = \pdifffrac[p]{}{x^i}(f)}}\\
	& = \sum_{i=1}^nX_p(\varphi^i)\pdifffrac[p]{}{x^i}(f).
\end{align*}
\end{bew}

\begin{Bem}
  Ist $X_p=\sum \xi^i\pdifffrac[p]{}{x^i}$, so gilt $\xi^i = X_p(\varphi^i)$.
\end{Bem}

\begin{bew}[Beweis des Lemmas]
Es gilt:
\begin{align*}
	g(x) - g(x_0) = \int_0^1\difffrac{}{t}g(tx + (1-t)x_0)dt = \sum_{i=1}^n(x^i-x_0^i)\underbrace{\int_0^1\partial_ig(tx + (1-t)x_0) dt}_{=: h_i(x)}.
\end{align*}
\end{bew}

% Satz 2.9
\begin{Satz}[Äquivalenz der Tangentialraumbegriffe]
  Die Abbildung
  \begin{align*}
    J_p \colon \T_p^{\text{geo}}M \to \T_p^{\text{alg}}M, \ J_{p}[c](f) = \difffrac[t=0]{}{t}(f\circ c)
  \end{align*}
  ist ein linearer \gls{Isomorphismus} \quot{$c(0)(f)$}.
\end{Satz}

\begin{bew}
  Wegen
  \begin{align*}
    J_p[c](f)& = \difffrac[t=0]{}{t}(f\circ c) = \difffrac[t=0]{}{t}(f \circ \varphi^{-1} \circ \varphi \circ c)\\
    &  = \D(f \circ \varphi^{-1})|_{\varphi(p)} \difffrac[t=0]{}{t} (\varphi \circ c) = \D (f\circ \varphi^{-1})|_{\varphi(p)}A[c]
  \end{align*}
  ist $J_p = \D(\cdot)\circ A$ linear.

  Ist $[c] \in \Kern J_p$, so folgt aus $0 = J_p[c](\varphi^i) = \difffrac[t=0]{}{t}(\varphi^i \circ c)$, dass $\difffrac[t=0]{}{t}(\varphi \circ c) = 0$ gilt, also $[c] = 0$. Damit ist $J_p$ injektiv, also ein Isomorphismus.
\end{bew}

\begin{Bem}
  \begin{enumerate}[label=\arabic*)]
  \item Ist $X_p = \sum \xi^i\pdifffrac[p]{}{x^i}$, so gilt $X_p = \dot c(0)$ für $c(t) = \varphi^{-1}(\varphi(p) + t\xi)$.
\item Für jede glatte Kurve $c$ durch $p$ ist $\difffrac[t=0]{}{t}(\varphi \circ c)$ der Koeffizientenvektor von $\dot c(0)$ in der Basis $\pdifffrac[p]{}{x^i}$.
  \end{enumerate}
\end{Bem}


% Satz 2.10
\begin{Satz}[Transformationsverhalten bei Kartenwechsel]
  Es seien $\varphi$ und $\psi$ Karten in $M$ um $p$ und es bezeichnen $\pdifffrac[p]{}{x^i}$ und $\pdifffrac[p]{}{y^i}$ die damit assoziierten Basen von $\T_pM$. Dann gilt
  \begin{align*}
    \pdifffrac[p]{}{x^i} = \sum_j \partial_i(\psi^j \circ \varphi^{-1})(\varphi(p)) \pdifffrac[p]{}{y^j}.
  \end{align*}
Es sei $X_p = \sum \xi^i \pdifffrac[p]{}{x^i} = \sum \eta^i\pdifffrac[p]{}{y^i}$. Dann gilt:
\begin{align*}
  \eta^j = \sum \partial_i(\psi^j \circ \varphi^{-1})(\varphi(p))\xi^i \text{ bzw. }
  \eta = \D(\psi \circ \varphi^{-1})(\varphi(p))\xi.
\end{align*}
\end{Satz}

\begin{proof}
  Es gelte $\pdifffrac[p]{}{x^i} = \sum \alpha_i^j\pdifffrac[p]{}{y^j}$ und nach obiger Bemerkung zum vorletzten Satz gilt:
  \begin{align*}
    \alpha_i^j = \pdifffrac[p]{}{x^i}(\psi^j) = \partial_i(\psi^j \circ \varphi^{-1})(\varphi(p))
  \end{align*}
\end{proof}

\textcolor{red}{Kapitel 2 hat drei Nummerierungen zu viel, bitte die Nummerierung korrigieren}
%%% Local Variables: 
%%% mode: latex
%%% TeX-master: "../skript-diffgeom"
%%% End: 


%% 
%% 4. Vorlesung <2012-10-26 Fri>
%% 
%% Skript Differentialgeometrie im Wintersemester 12/13
%% Zur Vorlesung von Dr. Grensing am KIT Karlsruhe
%% 
%% Mitschrieb und Textsatz von Jan-Bernhard Kordaß.
%% 

\section{Differentiale}

% Abb 4/1

Es seien $M$ und $N$ Mannigfaltigkeiten und $\Phi \colon M \to N$ eine glatte Abbildung.
Sind $p \in M$ und $X_p \in \T_pM$ , so ist 
\begin{align*}
  \Phi_{*p}X_p \colon C^{\infty}(N) \to \R, f \mapsto X_p(\underbrace{f \circ \Phi}_{\in C^{\infty}(N)}).
\end{align*}
ein Tangentialvektor an $N$ in $\Phi(p)$:
\begin{align*}
  \Phi_{*p}X_p(fg) & = X_p((f \circ \Phi)(g \circ \Phi)) = X_p(f \circ \Phi)(g \circ \Phi)(p) + (f \circ \Phi)(p)X_p(g \circ \Phi)\\
  & = \Phi_{*p}X_p(f)q(\Phi(p)) + f(\Phi(p)) \Phi_{*p}X_p(g).
\end{align*}
\begin{center}\begin{tikzpicture}[font=\scriptsize]
	%\draw[step=0.25,gray!15] (-6,-1) grid (6,5); \draw[step=0.5,gray!30] (-6,-1) grid (6,5); \fill (0,0) circle(0.1); %Hilfsgitter
	
	% Die Abbildungspfeile
	\draw[->] (-1.5,0) to[out=20, in=160]node[above]{$\psi' \circ \Phi \circ \varphi^{-1}$}node[below]{diff'bar} (1.5,0);
	\draw[->] (-1.5,3) to[out=20, in=160]node[above]{$\Phi$} (1.5,3);
	
	% Die Achsen
	\draw[->,thick] (-5.5,-0.5) -- (-2,-0.5); \draw[->,thick] (-5.25,-0.75) -- (-5.25, 1.25); \node[font=\normalfont] at (-2,1.25) {$\R^m$};
	\draw[->,thick] (2,-0.5) -- (5.5,-0.5); \draw[->,thick] (2.25,-0.75) -- (2.25, 1.25); \node[font=\normalfont] at (5.5,1.25) {$\R^n$};
	
	% das linke Ding
	\coordinate (ding0) at (-3.5,4.5); \coordinate (ding1) at (-4.25,3.5); \coordinate (ding2) at (-4.5,2); \coordinate (ding3) at (-2.75,2.5); \coordinate (ding4) at (-1.5,4);
	\coordinate (ctrld0) at (0.5,-0.25); \coordinate (ctrld1) at (0.75,0.5); \coordinate (ctrld2) at (-0.5,0.5); \coordinate (ctrld3) at (0.25,0.5); \coordinate (ctrld4) at (-0.75,0); 
	\draw[thick] (ding0) ..controls($(ding0)+(ctrld0)$) and ($(ding1)+(0.75,0.5)$).. (ding1) ..controls($(ding1)-(ctrld1)$) and($(ding2)+(ctrld2)$).. (ding2) ..controls($(ding2)-(ctrld2)$) and ($(ding3)-2*(ctrld3)$).. (ding3) ..controls($(ding3)+(ctrld3)$) and ($(ding4)+(ctrld4)$).. (ding4);
	% das Loch in der Mitte nicht vergessen, es besteht aus zwei geclipten Kreisen
	\begin{scope}
		\clip (-4.25,2.5) rectangle (-2.75,3);
		%\draw[thick] (-4.25,3) to[out=330,in=180] (-3.5,2.75) to[out=0,in=210] (-2.75,3);
		\path[draw,thick,name path=gkreis] (-3.5,4.25) circle (1.5);
	\end{scope}
	\path[name path=kkreis] (-3.5,2) circle(1);
	\path[name intersections={of=gkreis and kkreis}];
	\begin{scope}
		\clip (intersection-1) rectangle ($(intersection-2)+(0,0.5)$);
		\draw[thick]  (-3.5,2) circle(1);
	\end{scope}
	
	% das rechte Ding
	\draw[thick] (4, 3)  ellipse (2 and 1);
	% und das Loch
	\begin{scope}
		\clip (3, 2.75) rectangle (5, 4);
		\path[draw,thick,name path=gkreis] (4,3.75) ellipse (1.25 and 1);
	\end{scope}
	\path[name path=kkreis] (4,2.5) ellipse (1 and 0.75);
	\path[name intersections={of=gkreis and kkreis}];
	\begin{scope}
		\clip (intersection-1) rectangle ($(intersection-2)+(0,0.5)$);
		\draw[thick] (4,2.5) ellipse (1 and 0.75);
	\end{scope}
	
	% die linke Kartoffel
	\coordinate (kartoffel0) at (-4.25,2.5); \coordinate (kartoffel1) at (-4.25,2); \coordinate (kartoffel2) at (-4,2.125); \coordinate (kartoffel3) at (-3.75,2); \coordinate (kartoffel4) at (-3.25,2.5); \coordinate (kartoffel5) at (-3.75,2.5); 
	\coordinate (ctrlk0) at (0.25,0.25); \coordinate (ctrlk1235) at (0.25,0); \coordinate (ctrlk4) at (0,0.25);
	\draw (kartoffel0) ..controls($(kartoffel0)-(ctrlk0)$) and ($(kartoffel1)-0.5*(ctrlk1235)$).. (kartoffel1) ..controls($(kartoffel1)+0.5*(ctrlk1235)$) and ($(kartoffel2)-(ctrlk1235)$).. (kartoffel2) ..controls($(kartoffel2)+(ctrlk1235)$) and ($(kartoffel3)-0.5*(ctrlk1235)$).. (kartoffel3) ..controls($(kartoffel3)+2*(ctrlk1235)$) and ($(kartoffel4)-(ctrlk4)$).. (kartoffel4) ..controls($(kartoffel4)+(ctrlk4)$) and ($(kartoffel5)+(ctrlk1235)$).. (kartoffel5) ..controls($(kartoffel5)-0.5*(ctrlk1235)$) and ($(kartoffel0)+(ctrlk0)$).. (kartoffel0) -- cycle;
	\fill (-3.75,2.25) circle (0.05) node[right]{$p$};
	
	% die rechte Kartoffel (Kreis)
	\draw (4,2.5) ellipse (0.5 and 0.25); \fill (4,2.5) circle (0.05) node[right]{$q$};
	
	% die beiden Umgebungen unten
	\draw (-3.75,0.25) circle (0.5); \fill (-3.75,0.25) circle (0.05) node[right]{$x$};
	\draw (4,0.25) circle (0.5); \fill (4,0.25) circle (0.05) node[right]{$y$};
	
	% Abbildungspfeile
	\draw[->] (-4, 2.25) to[out=250,in=110] node[left]{$\varphi$} (-4,0.75);
	\draw[->] (3.75, 2.5) to[out=250,in=110] node[left]{$\psi$} (3.75,0.75);
\end{tikzpicture}\end{center}

% Definition 3.1
\begin{dfn}
  Die lineare Abbildung $\Phi_{*p} \colon \T_pM \to \T_{\Phi(p)}N$ heißt das \CmMark{Differential} von $\Phi$ in $p$. Der Rang von $\Phi_{*p}$ bezeichnet man als den Rang von $\Phi$ in $p$.
\end{dfn}

% Lemma 3.2
\begin{lemma}[Differentiale in lokalen Koordinaten]
  Sind $\varphi$ und $\psi$ Karten von $M$ und $N$ um $p$ und $\Phi(p) = q$, sowie $\pdifffrac[p]{}{x^i}$ und $\pdifffrac[q]{}{y^i}$ die Standardbasen von $\T_pM$ und $\T_qN$ bezüglich der Karten $\varphi$ und $\psi$, so gilt:
  \begin{align*}
    \Phi_{*p}\pdifffrac[p]{}{x^i} = \sum \partial_i(\psi^j \circ \Phi \circ \varphi^{-1})(\varphi(p))\pdifffrac[q]{}{y^j}.
  \end{align*}
  Die partielle Ableitung $\partial_i(\psi^j \circ \Phi \circ \varphi^{-1})(\varphi(p))$ bezeichnet man auch kurz $\frac{\partial \Phi^j}{\partial x^i}(p)$.
\end{lemma}

\begin{bem}
  Aus der Linearität von $\Phi_{*p}$ folgt, dass für $X_p = \sum \xi^i\pdifffrac[p]{}{x^i} \in \T_pM$ und $\Phi_{*p}X_p = \sum \eta^j\pdifffrac[q]{}{y^j}$ gilt:
  \begin{align*}
    \eta^j = \sum \frac{\partial \Phi^j}{\partial x^i}\xi^i \text{, beziehungsweise } \eta = \D(\psi \circ \Phi \circ \varphi^{-1})\xi.
  \end{align*}
\end{bem}

\begin{proof}
  \begin{align*}
    \underbrace{\left(\Phi_{*p}\pdifffrac[p]{}{x^i}\right)}_{\textcolor{red}{\in \T_qN}}(\psi^j) = \pdifffrac[p]{}{x^i}(\psi^j \circ \Phi) = \partial_i (\psi^j \circ \Phi \circ \varphi^{-1})(\varphi(p)) = \frac{\partial \Phi^j}{\partial y^i}(p).
  \end{align*}
\end{proof}

\begin{bem}[Charakterisierung durch Kurven]
  Ist $[c] \in \T_p\textcolor{red}{M}$, so gilt für $f \in C^{\infty}(M)$:
  \begin{align*}
    \Phi_{*p}[c](f) = [c](f \circ \Phi) = \difffrac[t=0]{}{t}(\underbrace{f \circ \Phi \circ c)}_{\substack{\text{glatte Kurve}\\ \text{auf }N}} = [\Phi \circ c](f)
  \end{align*}
  also $\Phi_{*p}[c] = [\Phi \circ c]$.
\end{bem}

\begin{bem}[Tangentialräume an Untermannigfaltigkeiten der $\R^n$]
  Ist $U$ eine Untermannigfaltigkeit in $\R^n$ mit den Eigenschaften
  \begin{itemize}
  \item $F \colon U \to M \cap F(U)$ ein Homöomorphismus,
  \item $\D F|_x\colon \R^m \to \R^{m+k}$ injektiv für alle $x \in U$.
  \end{itemize}
  Dann ist $\psi = F^{-1}$ eine Karte von $M$. Es bezeichnen $\pdifffrac[p]{}{y^i}$ die Standardbasis bezüglich $\psi$ und $\pdifffrac[x]{}{x^i}$ die Standardbasis bezüglich der kanonischen Karte $\Id_{\R^m}$ des $\R^m$.\\
  Dann gilt für $g \in C^{\infty}(M)$ beliebig:
  \begin{align*}
    & \pdifffrac[p]{}{y^i}(g) = \partial_i(g \circ \psi^{-1})(\underbrace{\psi(p)}_{=x}) = \partial_i(g \circ F)(x) = F_{*x}\left(\pdifffrac[p]{}{x_i}\right)(f).\\
    & F_{*x}\left(\pdifffrac[p]{}{x^i}\right) = F_{*x}[t \mapsto x + te_i] = [t \mapsto F(x+te_i)] \sim \difffrac[t=0]{}{t}F(x+te_i) = \D F|_x(e_i) = \partial_iF|_x.\\
    & \T_pM \quot{=} \left<\partial_1F|_x, \ldots, \partial_m F|_x\right).
\end{align*}
%$\pdifffrac[a]{b}{c}, \pdifffrac[a]{b}{}, \pdifffrac[a]{}{c}, \pdifffrac{b}{c}, \pdifffrac{}{c}, \pdifffrac{b}{}$
\end{bem}

% Abb 4/2

\begin{bem*}[Eigenschaften des Differentials]\hfill
  \begin{itemize}
  \item (Kettenregel) Sind $\Phi \colon M \to N$ und $\Psi \colon N \to P$ glatt, so gilt:
    \begin{align*}
      (\Psi \circ \Phi)_{*p} = \Psi_{*\Phi(p)} \circ \Phi_{*p}.
    \end{align*}
  \item Ist $\textcolor{red}{\Phi} \colon M \to N$ ein Diffeomorphismus, so ist $\Phi_{*p}$ ein Vektorraumisomorphismus. % Verwendet die Kettenregel
  \item (Satz von der Umkehrabbildung) Ist $\Phi \colon M \to N$ glatt und $\Phi_{*p}$ bijektiv, so existieren Umgebungen $U$ von $p$ und $V$ von $\Phi(p)$, so dass $\Phi|_{U} \colon U \to V$ ein Diffeomorphismus ist.
  \end{itemize}
\end{bem*}

% Definition 2.3
\begin{dfn}[Reguläre Punkte, Submersion, Immersion]
  Es sei $\Phi \colon M \to N$ glatt.
  \begin{itemize}
  \item Es Punkt $p \in M$ heißt \CmMark{regulärer Punkt} von $\Phi$, wenn $\Phi_{*p}$ surjektiv ist. Ein Punkt $q \in N$ heißt regulärer Wert, wenn jeder Punkt $p \in \Phi^{-1}(q)$ regulär ist.
  \item Die Abbildung $\Phi$ heißt \CmMark{Submersion}, wenn $\Phi$ surjektiv ist und alle $p \in M$ reguläre Punkte sind.
  \item Die Abbildung $\Phi$ heißt \CmMark{Immersion}, wenn für alle $p \in M$ $\Phi_{*p}$ injektiv ist.
  \item Die Abbildung $\Phi$ heißt \CmMark{Einbettung}, wenn $\Phi$ Immersion und Homöomorphismus auf sein Bild ist.
  \end{itemize}
\end{dfn}

\begin{bsp}
  \begin{itemize}
  \item Betrachte eine Abbildung $\Phi$

    % Abb 4/3

    Immersion: $\difffrac{}{t}$ Basis von $\T_x\R$, $\Phi_{*x}(\difffrac{}{t}) \quot{=} \difffrac{}{t}\Phi$
  \item $\R \to \R^2 \cong \C, t \mapsto e^{it}$ ist eine Immersion aber ebenfalls nicht injektiv.
  \item $\R \to S^1 \subset \C, t \mapsto e^{it}$ ist Immersion und Submersion.
  \item $\R \to S^1 \times \R, t \mapsto (e^{it},t)$ ist eine Einbettung.

    % Abb 4/4

  \item Ist $M \subset N$ Untermannigfaltigkeit, so ist $\imath \colon M \hookrightarrow N$ eine Einbettung.% INKLUSIONSSPFEIL
  \end{itemize}
\end{bsp}

% Satz 3.4
\begin{satz}
  Es seien $M$ und $N$ glatte Mannigfaltigkeiten, $\Phi \colon M \to N$ eine glatte Abbildung und $p \in M$, sowie $q = \Phi(p)$. Es bezeichnen $m$ und $n$ die Dimensionen von $M$ und $N$ und $r$ den Rang von $\Phi$ in $p$. Dann gelten folgende Aussagen:
  \begin{itemize}
  \item Zu jeder Karte $\psi$ von $N$ um $q$ mit $\psi(q) = 0$ existiert eine Karte $\alpha$ von $M$ um $p$ mit $\alpha(p) = 0$ und glatte Funktionen $f^{r+1},\ldots,f^n$ mit
    \begin{align*}
      \left(\psi \circ \Phi \circ \alpha^{-1}\right)\left(x^1,\ldots, x^m\right) = \left(x^1, \ldots, x^{r}, f^{r+1}(x), \ldots, f^n(x)\right).
    \end{align*}
  \item Falls der Rang von $\Phi$ auf einer Umgebung von $p$ konstant $r$ ist, so existieren Karten $\alpha$ um $p$ mit $\alpha(p) = 0$ und $\beta$ um $q$ mit $\beta(q) = 0$, so dass
    \begin{align*}
      \left(\beta \circ \Phi \circ \alpha^{-1}\right)\left(x^1, \ldots, x^m\right) = \left(x^1, \ldots, x^r, 0, \ldots, 0\right).
    \end{align*}
  \end{itemize}
\end{satz} 

% Korollar 3.5
\begin{kor}
  \begin{enumerate}[label=(\roman*)]
  \item Falls $\Phi$ auf einer offenen Umgebung von $P = \Phi^{-1}(q)$ konstanten Rang $r$ hat, so ist $P$ eine Untermannigfaltigkeit der Kodimension $r$.
  \item Ist $q$ ein regulärer Wert von $\Phi$, so ist $P = \Phi^{-1}(q)$ eine Untermannigfaltigkeit von $M$ der Kodimension $n$.\\
    Beispiel: $\|\cdot\|^{-1}(1) = S^n \supset \R^{n+1} \to \R, x \mapsto \|n\|$.
  \item Ist $\Phi_{*p}$ injektiv, so existiert eine Umgebung $U$ von $p$, so dass $\Phi(U) = Q \subset N$ eine Untermannigfaltigkeit von $N$ ist.
  \item Ist $\Phi$ eine Einbettung, so ist $Q = \Phi(M)$ eine $m$-dimensionale Untermannigfaltigkeit von $M$ und $\Phi \colon M \to Q$ ist ein Diffeomorphismus.
  \end{enumerate}
\end{kor}

\begin{proof}
  ad (i): Ist $p \in P = \Phi^{-1}(q)$. Nach der zweiten Aussage des vorrangegangenen Satzes existieren Karten $(\alpha,U), (\beta, V)$ mit 
  \begin{align*}
    (\beta \circ \Phi \circ \alpha^{-1})(x^1,\ldots,x^n) = (x^1, \ldots,x^r, 0, \ldots, 0)
  \end{align*}
  und es gilt:
  \begin{align*}
    \alpha(P \cap U) & = (\alpha \circ \Phi^{-1} \circ \beta^{-1})(0) \\
    & = \{x \in \alpha (U) \mid x^1 = \ldots = x^r = 0\} = \alpha(U) \cap \{0\} \times \R^{m-r}.
  \end{align*} 
  ad (ii): Ist $q$ ein regulärer Wert von $\Phi$, so existieren nach dem ersten Teil des vorigen Satzes Karten $\psi,\alpha$ mit 
  \begin{align*}
    (\psi \circ \Phi \circ \alpha^{-1})(x^1, \ldots, x^m) = (x^1, \ldots, x^n) \ m \geq n = r \quad \forall x \in \alpha(U).
  \end{align*}
  Es gilt also für alle $u \in U$:
  \begin{align*}
    \Rang \Phi_{*u} = \Rang \D(\psi \circ \Phi \circ \alpha^{-1})|_x = \Rang
    \left(\begin{array}{ccc|c}
      1 &  & 0 & \\
      & \ddots & & 0 \\
      0 & & 1 & 
    \end{array}\right)
    = \textcolor{red}{n}
  \end{align*}
  Damit folgt die Behauptung aus (i).\\
  ad (iii): $\Phi_{*p}$ ist injektiv $\Rightarrow r = m \leq n$. Nach Wahl von Karten wie in (ii):
  \begin{align*}
    & (\psi \circ \Phi \circ \alpha^{-1})(x^1, \ldots, x^m) = (x^1, \ldots, x^m,f^{m+1}(x), \ldots, f^n(x))\\
    & \Rang \Phi_{*u} = \Rang 
    \left( \begin{array}{ccc}
      1 & & 0 \\
        & \ddots &  \\
      0 & & 1 \\
      \hline
        & 0      & 
      \end{array} \right)
    = m % Matrix 4/5
  \end{align*}
  Nach der ersten Aussage des letzten Satzes erhalten wir spezielle Karten:
  \begin{align*}
    (\beta \circ \Phi \circ \alpha^{-1})(x^1, \ldots, x^m) = (x^1, \ldots, x^m, 0, \ldots, 0) \in \R^{m} \times \{0\},
  \end{align*}
  wobei $\beta$ eine adaptierte Karte für $\Phi(U) = Q$ ist.\\
  (iv) folgt aus (iii).
\end{proof}

%%% Local Variables: 
%%% mode: latex
%%% TeX-master: "../skript-diffgeom"
%%% End: 

%% 
%% 5. Vorlesung <2012-10-30 Tue>, Fortsetzung
%% 
%% Skript Differentialgeometrie im Wintersemester 12/13
%% Zur Vorlesung von Dr. Grensing am KIT Karlsruhe
%% 
%% Mitschrieb und Textsatz von Jan-Bernhard Kordaß.
%% 

\chapter{Tangentialbündel und Vektorfelder}

\begin{Dfn}[Tangentialbündel]
  Es sei $M$ eine glatte Mannigfaltigkeit. Die Menge $\T M = \dot \bigcup_{p \in M}\T_pM$ zusammen mit der sogenannten kanonischen Projektion $\pi \colon \T M \to M, \T_pM \ni X_p \mapsto p$ heißt das \CmMark[Tangential!-b\"undel]{Tangentialb"undel} von $M$.
\end{Dfn}

\section{Das Tangentialbündel als glatte Mannigfaltigkeit}

Es sei $(\varphi, U)$ eine Karte von $M$. Setzt man $\T M|_U = \pi^{-1}(U) = \dot \bigcup_{p\in U}\T_pM$, so ist nach Satz \ref{satz-2-9}
die Abbildung
\begin{align*}
  & \overline \varphi \colon \T M|_U \to \varphi(U) \times \R^m \subset \R^{2m}\\
  &\T_p M \ni X_p = \sum \xi^i\pdifffrac[p]{}{x^i} \mapsto (\varphi(p),\xi)
\end{align*}
bijektiv.
Es sei eine Topologie auf $\T M$ dadurch erklärt, dass eine Menge $V \subset \T M$ genau dann offen ist, wenn für alle Karten $(\varphi, U)$ die Menge $\overline \varphi(V \cap \T M|_U)$ offen in $\R^{2m}$ ist. Diese Topologie ist hausdorffsch und besitzt eine abzählbare Basis, da dies für $M$ und $\R^m$ gilt. Nach Konstruktion sind alle $\overline \varphi$ Homöomorphismen. Ist $\mathcal A = \{(\varphi, U\}$ ein $C^{\infty}$-Atlas von $M$, so definiert
\begin{align*}
  \overline A = \{(\overline \varphi, \T M|_U) \mid (\varphi, U) \in \mathcal A\}
\end{align*}
eine glatte Struktur auf $\T M$. Für Karten $(\varphi, U), (\psi, V)$ von $M$ ist der Kartenwechsel
\[\begin{array}{ccc}
  \overline \psi \circ \overline \varphi^{-1} \colon \varphi(U \cap V) \times \R^m &\to& \psi(U \cap V) \times \R^m \\
  (x, \xi) &\mapsto& (\psi \circ \varphi^{-1}(x),\D(\psi \circ \varphi^{-1}|_x\xi),
\end{array}\]
$(X_p = \sum \xi^i \pdifffrac[p]{}{x^i})$ ist glatt. Damit trägt $\T M$ in kanonischer Weise eine glatte Struktur.
Darüber hinaus ist die kanonische Projektion $\pi \colon \T M \to M$ bezüglich dieser glatten Struktur eine Submersion. (Beweis: Übungsaufgabe)

Ist $N$ eine weitere glatte Mannigfaltigkeit und $\Phi \colon M \to N$ glatt, so ist $\Phi_{*} \colon \T M \to \T N, \ X_p \mapsto \Phi_{*p}X_p$ eine glatte Abbildung (ebensfalls Übungsaufgabe).

\begin{Dfn}
  Eine stetige Abbildung $X \colon M \to \T M$ mit $\pi \circ X = \Id_M$ heißt \CmMark{Vektorfeld} auf $M$.
  Ist $X$ glatt (als Abbildung zwischen glatten Mannigfaltigkeiten), so heißt $X$ ein \CmMark[Vektorfeld!glattes]{glattes Vektorfeld}.
\end{Dfn}

\begin{bem}
Ist $(\varphi, U)$ eine Karte von $M$, so sind die Abbildungen $U \to \T M|_U, \ p \mapsto \pdifffrac[p]{}{x^{i}}$ glatte Vektorfelder (in der Karte $\overline \varphi$ sind diese genau die Abbildungen $(x,e_i)$).

Ist $X$ ein glattes Vektorfeld, so gilt für jedes $u \in U$:
\begin{align*}
	X_U = \sum \xi^i(u) \pdifffrac[U]{}{x^i},
\end{align*}
wobei $\xi(u) = (\xi^1(u), \ldots, \xi^m(u))$ eine glatte Abbildung $U \to \R^m$ ist. Ein Vektorfeld ist genau dann glatte, wenn für jede Karte $(\varphi, U)$ die Koeffizientenfunktionen $\xi^i(u)$ von $X_U = \sum \xi^i(u) \pdifffrac[U]{}{x^i}$ glatte Funktionen sind.
\end{bem}


%%% 
%%% 6. Vorlesung <2012-11-2 Fri>
%%% 

\begin{bsp}
  Betrachte die $n$-Sphäre $S^n \subset \R^{n+1}$ und deren Tangentialraum $\T_pS^n = p^{\perp}$.
  Ein glattes Vektorfeld auf $S^n$ ist also eine glatte Abbildung $X \colon S^n \to \R^{n+1}$ mit $X_p \perp p$.
  
  Es sei $n=2k-1$, dann ist
  \begin{align*}
    X \colon S^n \to \R^{n+1}, (x^1,y^1, \ldots, x^k,y^k) \mapsto (-y^1,x^1, \ldots, -y^k,x^k)
  \end{align*}
  ein glattes Vektorfeld auf $S^n$ ohne eine Nullstelle.
\end{bsp}

\begin{bem}
  Der \CmMark{Satz vom Igel} besagt gerade: Jedes glatte Vektorfeld auf einer Sphäre gerader Dimension hat eine Nullstelle.
\end{bem}

\begin{bem}
  Es bezeichne $\mathcal V(M)$ die Menge aller glatten Vektorfelder auf der Mannigfaltigkeit $M$. Der sogenannte \CmMark{Nullschnitt}:
  \begin{align*}
    \sigma \colon M \to \T M, p \mapsto 0_p \in \T_pM
  \end{align*}
  ist ein glattes Vektorfeld auf $M$.
  
  \emph{"Ubungsaufgabe:} Zeige dass der Nullschnitt eine Einbettung ist.
\end{bem}

\begin{bem}
  Sind $X,Y \in \mathcal V(M)$ und ist $g \in \C^{\infty}(M)$, so sind die punktweise Summe $X+Y$ und das Produkt $gX$ wieder ein glatte Vektorfelder auf $M$. Damit ist $\mathcal V(M)$ ein $\R$-Vektorraum beziehungsweise $C^{\infty}(M)$-Modul.
  
  Jedes Vektorfeld $X$ ist eine Derivation von $C^{\infty}(M)$:
  \begin{align*}
    X(fg)(p) = X(f)(p)g(p) + f(p) X(g)(p) = \left(gX(f) + fX(g)\right)(p).
  \end{align*}
  Es seien $X,Y \in \mathcal V(M)$ glatte Vektorfelder. Die \CmMark{Lieklammer} $[X,Y]$ von $X$ und $Y$ ist dann durch den folgenden Ausdruck definiert:
  \begin{align*}
    [X,Y](f)(p) = X_p(Yf)-Y_p(Xf).
  \end{align*}
\end{bem}

% Lemma 4.3
\begin{Lemma}
  Die Lieklammer ist eine schiefsymmetrische $\R$-bilineare Abbildung $\mathcal V(M) \times \mathcal V(M) \to \mathcal V(M)$. Es gilt die sogenannte \CmMark{Jacobiidentität}:
  \begin{align*}
    [X,[Y,Z]] + [Y,[Z,X]] + [Z,[X,Y]] = 0.
  \end{align*}
\end{Lemma}

\begin{bew}
  Es seien $X,Y \in \mathcal V(M)$, $f,g \in \C^{\infty}(M)$ und $p \in M$. Dann gilt:
  \begin{align*}
    [X,Y](fg)  = & X_p(Y(f)g + fY(g)) - Y_p(X(f)g+fX(g))\\
    = & X_p(Y(f))g(p) + \cancel{Y_p(f)X_p(g)} + \bcancel{X_p(f)Y(g)(p)} + f(p)X_p(Y(g))\\
    & - Y_p(X(f))g(p) - \bcancel{X_p(f)Y_p(g)} - \cancel{Y_p(f)X(g)(p)} - f(p)Y_p(X(g))\\
    = & (X_p(Y(f))-Y_p(X(f))g(p) + f(p)(X_p(Y(g))-Y_p(X(g))\\
    = & [X,Y]_p (f)g(p) + f(p)[X,Y]_p(g).
  \end{align*}
  \textcolor{red}{(nicht sicher ob ich die Durchstreichungen lassen oder rausnehmen soll)}
  Damit gilt $[X,Y] \in \mathcal V(M)$. Schiefsymmetrie und $\R$-Linearität gelten offensichtlich. Die Jacobiidentität sie als Übungsaufgabe überlassen (Nachrechnen!).
\end{bew}


% Lemma 4.4
\begin{Lemma}
  Es seien $X,Y \in \mathcal V(M)$ glatte Vektorfelder und $(\varphi,U)$ eine Karte von $M$.
  Sind dann $X|_U = \sum \xi^i\pdifffrac{}{x^i}, \ Y|_U = \sum \eta^i \pdifffrac{}{x^i}$ und $[X,Y]|_U = \sum \zeta^i \pdifffrac{}{x^i}$ die ensprechenden lokalen Darstellungen, so gilt:
  \begin{align*}
    \zeta^j = \sum \left(\xi^i\pdifffrac{\eta^j}{x^i} - \eta^i \pdifffrac{\xi^j}{x^i} \right).
  \end{align*}
\end{Lemma} 

Der Beweis ist als Übungs überlassen.


\section{Flüsse}

\marginnote{\begin{tikzpicture}[scale=0.75]
    % \draw[step=0.25,gray!15] (-3,-1) grid (2,5); \draw[step=0.5,gray!30] (-3,-1) grid (2,5); \fill (0,0) circle(0.1); %Hilfsgitter
    \tikzschnuller{(0,3)}
    \draw[->] (0.25,2.25) -- (-0.5,2.25); \draw[->] (-1.5,2.25) -- (-2,1.75); \draw[->] (-2.5,1) -- (-1.75,1.5); \draw[->] (-1.5,0.75) -- (-0.25,1.5);
    \draw (-2,0.75) ..controls(-2,0.75) and ($(-1,0.75) - 0.5*(0.75,0.25)$).. (-1,0.75) ..controls($(-1,0.75) + (0.75,0.25)$) and (0,1.75).. (0,1.75);
    \draw[->] (-1,0.75) -- ($(-1,0.75) + (0.75,0.25)$); \draw[->] (-2,0.75) -- (-1.75,1);
  \end{tikzpicture}}
Was haben Vektorfelder mit Differentialgleichungen zu tun? Jedes glatte Vektorfeld $X$ definiert ein Anfangswertproblem
\begin{align*}
  \begin{cases}
    \dot \gamma(t) = X_{\gamma(t)}\\
    \gamma(0) = p
  \end{cases},
\end{align*}
oder in lokalen Koordinaten:
\begin{align*}
  \begin{cases}
    \dot \gamma(t) = \xi(\tilde \gamma(t))\\
    \tilde \gamma(0) = 0, \text{ falls } \varphi(p) = 0,
  \end{cases}
\end{align*}
mit $\tilde \gamma = \varphi \circ \gamma$ für eine Karte $(\varphi,U)$ um $p$ und $X|_{\textcolor{red}{U}} = \sum \xi^i \pdifffrac{}{x^i}$.

% Definition 4.5
\begin{Dfn}
  Es sei $X \in \mathcal V(M)$ ein glattes Vektorfeld und $p \in M$, sowie $\calI \subset \R$ ein offenes, zusammenhängendes Intervall um $0$. Eine glatte Kurve $\gamma \colon \calI \to M$ mit
  \begin{align*}
    \dot \gamma(t) = X_{\gamma(t)}, \ \gamma(0) = p
  \end{align*}
  heißt \CmMark{Integralkurve} oder \CmMark{Trajektorie} von $X$ durch $p$.
\end{Dfn}

\begin{bem}
  Eine Kurve $\gamma$ ist genau dann Integralkurve von $X$ durch $p$, wenn für jede Karte $(\varphi,U)$ die Kurve $\tilde \gamma = \varphi \circ \gamma$ eine Lösung des (autonomen) Anfangswertproblems
  \begin{align*}
    \dot{\tilde \gamma} = \xi(\gamma(t)),  \ \tilde \gamma(0) = \varphi(p)
  \end{align*}
  ist, wobei $X_U = \sum \xi^i \pdifffrac{}{x^i}$ gelte.

  Für jedes $p \in M$ ist somit (lokal) ein Anfangswertproblem gestellt. Gesucht ist eine \quot{simultane} Lösung all dieser Anfangswertprobleme, also eine Abbildung $(t,p) \mapsto \gamma(t,p) = \gamma^t(p)$ mit 
  \begin{align*}
    \begin{cases}
      \dot \gamma^t(p) = X_{\gamma^t(p)}\\
      \gamma^0(p) = p
    \end{cases}.
  \end{align*}
\end{bem}

% Satz 4.6
\begin{Satz}[Lokale Existenz und Eindeutigkeit]\label{satz-4-6}
  Es sei $U \subseteq \R^n$ offen und $\calI_{\varepsilon}=(-\varepsilon,\varepsilon)$ und $F \colon \calI_{\varepsilon} \times \R^n \to \R^n$ $C^k$-differenzierbar.
  Dann existiert für alle $x \in U$ eine Umgebung $V$ von $x$ in $U$ und ein $\delta > 0$, so dass
  \begin{enumerate}[label=(\roman*),leftmargin=*,widest=iii]
  \item Für alle $x \in V$ existiert eine $C^{k+1}$-Lösung $y_x\colon \calI_{\delta} \to V$, von $y_x'(t)=F(t,y(t))$ und $y_x(0) = x$.
  \item\label{satz-4-6-ii} Diese Lösung ist lokal eindeutig, das hei\ss t falls $\tilde y_x$ eine weitere Lösung auf $\calI_{\tilde\delta}$ ist, so gilt
    \begin{align*}
      y_x(t) = \tilde y_x(t) \text{ für alle } |t| \leq \min\{\delta, \tilde\delta\}.
    \end{align*}
  \item Die Abbildung 
    \begin{align*}
      y\colon \calI_{\delta} \times V \to U, \ (t,x) \mapsto y_x(t)
    \end{align*}
    ist $C^k$-differenzierbar.
  \end{enumerate}
\end{Satz}

Zum Beweis siehe Lang: "`Differential and Riemannian Manifolds"', 3. Auflage, 1995, Chapter IV.1, P.65\cite{lang1995differential}.

% Korollar 4.7
\begin{Kor}\label{korollar-4-7}
  Es sei $X \in \mathcal V(M)$ und $p \in M$. Dann existiert eine offene Umgebung $U$ von $p$, ein $\varepsilon > 0$ und eine glatte Abbildung:
  \begin{align*}
    \gamma\colon(-\varepsilon,\varepsilon) \times U \to M
  \end{align*}
  so dass $t \mapsto \gamma^t(p)$ eine Integralkurve von $X$ durch $p$ ist. (Setze dann $F(t,x) = \xi(\varphi^{-1}(x))$.)
\end{Kor}

% Korollar 4.8
\begin{Kor}\label{korollar-4-8}
  Sind $\gamma_1 \colon \calJ_1 \to M, \ \gamma_2 \colon \calJ_2 \to M$ Integralkurven eines Vektorfeldes $X \in \mathcal V(M)$ durch $p$, dann gilt $0 \in \calJ_1 \cap \calJ_2$ und $\gamma_1(0)= p = \gamma_2(0)$.
  Nach Satz \ref{satz-4-6} \ref{satz-4-6-ii} gilt dann $\gamma_1(t) = \gamma_2(t)$ für alle $t \in \calJ_1 \cap \calJ_2$. Damit ist
  \begin{align*}
    \gamma \colon \calJ_1 \cap \calJ_2, \ t \mapsto 
    \begin{cases}
      \gamma_1(t) & t \in \calJ_1\\
      \gamma_2(t) & t \in \calJ_2
    \end{cases}
  \end{align*}
  eine Integralkurve von $X$ durch $p$.
  Damit existiert für jedes $p \in M$ ein maximaler Definitionsbereich $\calI_p$ für Integralkurven von $X$ durch $p$; dieser ist offen.
\end{Kor}

\begin{dfn}
  Für $X \in \mathcal V(M)$ heißt die, wie im vorigen Korollar definierte, Familie maximaler Integralkurven
  \begin{align*}
    \gamma(t,p) = \gamma^t(p),\ t \in \calI_p
  \end{align*}
  der \CmMark{Fluss} des Vektorfeldes $X$.
  Seinen Definitionsbereich notiert man mit:
  \begin{align*}
    \mathcal D_X = \{(t,p) \in \R \times M \mid t \in \calI_p\}.
  \end{align*}
\end{dfn}

\begin{bem}
  Es gilt: $\gamma^0(\cdot) = \Id_M$. Ist $(s,p) \in \mathcal D_{X}$, so gilt
  \[ \difffrac[t=0]{}{t}(t \mapsto \gamma^{t+s}(p)) = X_{\gamma^s(p)}, \]
  also ist $t \mapsto \gamma^{t+s}(p)$ eine Integralkurve von $X$ durch $q = \gamma^s(p)$. Aus der Eindeutigkeit folgt damit:
  \begin{align*}
    \gamma^{t+s}(p) = \gamma^t(\gamma^s(p))
  \end{align*}
  f"ur alle $s, t$, $s+t \in \calI_p$, ferner gilt $\calI_q = \calI_p - s$. Kurz geschrieben: $\gamma^{t+s} = \gamma^t \circ \gamma^s$. Also definiert $\gamma$ einen \quot{lokalen Gruppenhomomorphismus} von $\R$ in $M^M$.
\end{bem}

% Satz 4.9
\begin{Satz}\label{satz-4-9}
  Ist $X \in \mathcal V(M)$ ein glattes Vektorfeld mit Fluss $\gamma$, so ist $\mathcal D_X$ eine offene Menge und sein Fluss $\gamma \colon \mathcal D_X \to M$ glatt.
\end{Satz}


%%% 
%%% 7. Vorlesung <2012-11-6 Tue>
%%% 

\begin{bem}[Erinnerung]
\textcolor{red}{TODO: Diesen Absatz aufräumen. (ich würde das komplett weglassen weil es bloß eine Wiederholung von der Woche davor ist -Aleks)}
Ist $X \in \mathcal V(M)$, $c$ Integralkurve von $X$ durch $p$, wenn $c(t) = X_{c(t)}$ maximales Existenzintervall $I_p$\\

%%% 
%%% TODO Abbildung 7-1
\textcolor{red}{Abbildung 7-1}
%%% 

\begin{align*}
	\gamma \colon \mathcal D_X = \{(t,p) \mid p \in M, t \in I_p\} \to M
\end{align*}
Fluß von $X$, wobei
\begin{align*}
	\dot \gamma^t = X_{\gamma^t(p)}, \gamma^0 = \Id_M.
\end{align*}
Für $f \in C^{\infty}$ gilt
\begin{align*}
	X_p(t) = \dot \gamma^t(pXf) = \difffrac[t=0]{}{t}(f \circ \gamma^t)(p).
\end{align*}
Aus der lokalen Eindeutigkeit folgt (für kleine Zeiten)
\begin{align*}
	\gamma^{s-t} = \gamma^s \circ \gamma^t.
\end{align*}
Für alle $t$ ist $\gamma^t$ (lokalaer) Diffeomorphismus mit Inversen $\gamma^{-t}$.
\end{bem}

\begin{bew}
Es sei $p \in M$ und $\calJ_p^{+}$ die Menge aller $t \geq 0$, für welche eine offene Umgebung $U$ von $[0,t] \times \{p\}$ in $\R \times M$ existiert, so dass $\gamma$ auf $U$ glatt ist.
\begin{itemize}
\item $\calJ_p^{+}$ ist ein Intervall,
\item $\calJ_p^{+} \subseteq \calJ_p \cap \R_{\geq 0}$
\item $0 \in \calJ_p^{+}$
\item $\calJ_p^{+}$ ist offen in $\R_{\geq 0}$.
\end{itemize}
Es bleibt zu zeigen, dass $\calJ_p^{+}$ abgeschlossen ist.

Es sei $s \in \overline \calJ_p^{+} \cap (\calJ_p \cap \R_{\geq 0}$ und $q = \gamma^s(p)$.\marginnote{\begin{tikzpicture}[scale=0.75,font=\tiny]
	%\draw[step=0.25,gray!15] (-3,-3) grid (3,3); \draw[step=0.5,gray!30] (-3,-3) grid (3,3); \fill (0,0) circle(0.1); %Hilfsgitter
	\draw[name path=kurve] (0,0) node[anchor=north west]{$\gamma^0(p) = p$} to[out=65,in=185] (2.75,2) node[anchor=north west]{$\gamma^s(p) = q$};
	\path[name path=senkrecht] (2.5,0) -- (2.5,3);
	\path[name intersections={of=kurve and senkrecht}];
	\fill (intersection-1) circle (0.05) node[anchor=south east] {$\gamma^t(p)$} (0,0) circle(0.05) (2.75,2) circle(0.05);
\end{tikzpicture}}
Dann existieren $\varepsilon > 0$ und eine Umgebung $U$ von $q$, so dass $\gamma$ auf $(-\varepsilon,\varepsilon) \times U$ glatt ist.
Es sei $t \in \calJ_p^{+}$ mit $s-t < \varepsilon$ und $\gamma^t(p) \in U$.
Nach Definition von $\calJ_p^{+}$ existiert eine offene Umgebung $V$ von $p$ in $M$ und $\delta > 0$, so dass $(-\delta,t+\delta)\times V \subseteq \mathcal D_X$ gilt und darauf $\gamma$ glatt ist.
Dann ist $V' = \left(\left.\gamma^t\right|_V\right)^{-1}(U) = \{p' \in V \mid \gamma^t(p') \in U\}$ eine offene Umgebung von $p$.
Für alle $p' \in V'$ gilt:
\begin{align*}
	r \mapsto
	\begin{cases}
		\gamma^r(p') & \text{ falls } r < t+\delta\\
		\gamma^{r-t}(\gamma^t(p')) & \text{ falls } r \in \textcolor{red}{(t-s,t+\varepsilon)}
	\end{cases}.
\end{align*}
ist eine Integralkurve von $X$ durch $p'$.

Es gilt also $(-\delta,t+\varepsilon) \subseteq \calJ_{p'}$ für alle $p' \in V'$ und somit $(-\delta,t+\varepsilon) \times V' \subset \mathcal D_X$. An der obigen Darstellung sieht man, dass $\gamma$ darauf glatt ist.
Also gilt $s \in \calJ_p^{+}$. Analog argumentiert man für $\calJ_p^{-}$.
\end{bew}

% Definition 4.10
\begin{Dfn}[vollständiges Vektorfeld]
  Ein Vektorfeld auf $M$ heißt \CmMark{vollständig}, wenn der Definitionsbereich seines Flusses gleich $\R \times M$ ist.
\end{Dfn}

\begin{bem}
  Dies ist genau dann der Fall, wenn alle Integralkurven für alle Zeiten existieren. Ist $X \in \mathcal V(M)$ vollständig und bezeichnet $\gamma$ seinen Fluss, so ist jede $\gamma^t$ ein Diffeomorphismus von $M$ mit Inversen $\gamma^{-t}$.
\end{bem}

% Lemma 4.11
\begin{Lemma}
  Es sei $c \colon \calI \to M$ eine Integralkurve von $X \in \mathcal V(M)$ durch $p$ und $t_n \in \calI$ eine Folge, so dass die Grenzwerte $t_n \xrightarrow{n\to\infty} t_{\infty} \in \R$ und $c(t_n) \to q \in M$ existieren.
  
  Dann gilt $t_{\infty} \in \calI_{p}$ und $\gamma^{t_{\infty}}(p) = q$.
\end{Lemma}

\begin{bew}
  Nach Korollar \ref{korollar-4-7} existiert eine Umgebung $U$ von $q$ und $\varepsilon > 0$, so dass auf $(-\varepsilon,\varepsilon) \times U$ ein lokaler Fluss von $X$ definiert ist.
  
  Wählt man $k$ so groß, dass $t_{\infty} - t_k < \varepsilon$ gilt, so ist für $\overline q = c(t_k)$ und $t$ mit $|t-t_k| < \varepsilon$ der Fluss $t \mapsto \gamma^{t-t_k}(\overline q)$ erklärt. Aus Korollar \ref{korollar-4-8} folgt, dass $\gamma^{t_k}(p) = c(t_k) = \overline q$ gilt.
  Damit ist die Kurve
  \begin{align*}
    \overline c(t) =
    \begin{cases}
      c(t) & \text{ für } t < t_{\infty}\\
      \gamma^{t-t_k}(\overline q) & \text{ falls } |t-t_k| < \varepsilon
    \end{cases}
  \end{align*}
  eine glatte Fortsetzung von $c$ und eine Integralkurve von $X$ durch $p$.
  Insbesondere gilt $t_{\infty} < t_k + \varepsilon$, also $t_{\infty} \in \calI_p$ und
  \begin{align*}
    \gamma^{t_{\infty}}(p)= \gamma^{t_{\infty}-t_k}(\gamma^{t_k}(p)) = \overline c(t_{\infty}) = \lim_{t\to t_{\infty}}c(t) = q.
  \end{align*}
\end{bew}

\begin{Bem}[Moral des obigen Lemmas]
  Integralkurven existieren für alle Zeiten oder aber sie verlassen jedes Kompaktum.
\end{Bem}

\begin{Kor}
Hat $X \in \mathcal V(M)$ kompakten Träger, so ist $X$ vollständig. Ist $M$ kompakt, os ist jedes glatte Vektorfeld vollständig.
\end{Kor}

\begin{bsp}
  Das Vektorfeld $X \colon \R \to \T\R, t \mapsto t^2\pdifffrac{}{t}$ ist nicht vollständig, denn $c(t) = (1-t)^{-1}$ ist die Integralkurve von $X$ durch $1$.
\end{bsp}


%\begin{bem}
Ist $\Phi \colon M \to N$ glatt, so ist $\Phi_{\ast} \colon \T M \to \T N$ glatt.\marginnote{\begin{tikzpicture}[scale=1]
	%\draw[step=0.25,gray!15] (-3,-1) grid (3,1); \draw[step=0.5,gray!30] (-3,-1) grid (3,1); \fill (0,0) circle(0.1); %Hilfsgitter
	\def\hor{1}
	\def\vert{0.5}
	\node (0) at (-\hor,\vert) {$\T M$}; \node (1) at (\hor,\vert) {$\T N$}; \node (2) at (\hor,-\vert) {$N$}; \node (3) at (-\hor,-\vert) {$M$};
	\draw[->] (0) --node[above,font=\scriptsize]{$\Phi_*$} (1); \draw[->] (2) --node[right,font=\scriptsize]{$\Phi_*X$} (1); \draw[->] (2) --node[above,font=\scriptsize]{$\Phi^{-1}$} (3); \draw[->] (3) --node[left,font=\scriptsize]{$X$} (0);
\end{tikzpicture}}
Ist $\Phi$ ein Diffeomorphismus, so ist $\Phi^{-1}$ glatt und $q \mapsto (\Phi_{\ast}X)_q = \Phi_{*\Phi^{-1}(q)}X_{\Phi^{-1}(q)}$ ist ein glattes Vektorfeld auf $N$. Es gilt für $f \in C^{\infty}(N)$:
\begin{align*}
  (\Phi_{\ast}X)(f)(q) = X_{\Phi^{-1}(q)}(f \circ \Phi).
\end{align*}
%\end{bem}


% Lemma 4.13
\begin{Lemma}
  Es sei $X \in \mathcal V(M)$, $\Phi \colon M \to N$ ein Diffeomorphismus und es bezeichne $\gamma$ den Fluss von $X$ und $\mathcal D_X$ seinen (maximalen) Definitionsbereich.
Dann ist
\begin{align*}
  \{(t,q) \mid (t,\Phi^{-1}(q)) \in \mathcal D_X\} \to N \colon (t,q) \mapsto \Phi \circ \gamma^t \circ \Phi^{-1}(q)
\end{align*} 
der Fluss von $\Phi_{*}X$.
\end{Lemma}

\begin{bew}
Für $f \in \C^{\infty}(N)$ und $q \in N$ gilt
\begin{align*}
	(\Phi_{\ast}X)(f)(q) & = X_{\Phi^{-1}(q)}(f \circ \Phi)\\
	& = \difffrac[t=0]{}{t} (f \circ \Phi)(\gamma^t(\Phi^{-1}(q)))\\
	& = \difffrac[t=0]{}{t}f(\underbrace{\Phi \circ \gamma^t \circ \Phi^{-1}}_{\text{Fluss von }\Phi_{*}X}(q)).
\end{align*}
\end{bew}

%\begin{Bem}
\emph{Erinnerung:} $\frac{1}t(F(x+tv)-F(x))$ oder für $c$ mit $c(0) = x, \dot c(0) = v$: $\frac{1}t(F(c(t))-F(x))$.

Nun seien $X,Y \in \mathcal V(M)$.
\begin{center}\begin{tikzpicture}[font=\scriptsize]
	%\draw[step=0.25,gray!15] (-6,-1) grid (6,5); \draw[step=0.5,gray!30] (-6,-1) grid (6,5); \fill (0,0) circle(0.1); %Hilfsgitter
	\draw[name path=kurve] (-3,0) ..controls(-3,0) and (-2.75,1).. (0,1.5) node[below] {$\gamma^{t}(p)$} ..controls(1.5,1.75) and (2.5,2.5).. (2.5,2.5) node[anchor=north west] {$\gamma^{(\cdot)}(p)$};
	
	\fill (0,1.5) circle(0.05); \draw[->] (0,1.5) --node[left]{$y_{\gamma^t(p)}$} (1,2.5);
	
	\path[name path=senkrecht] (-2,0) -- (-2,3);
	\path[name intersections={of=kurve and senkrecht}];
	\fill (intersection-1) circle(0.05); \draw[->] (intersection-1) node[below] {$p$} --node[left]{$y_p$} (-2.5,2.5);
	
	\node[font=\normalsize] at (0,0) {$\gamma^t$};
\end{tikzpicture}\end{center}
Im Allgemeinen liegen $y_{\gamma^t(p)}$ und $y_p$ in unterschiedlichen Tangentialräumen.
Da aber $\gamma^t$ (für kleine Zeiten) ein (lokaler) Diffeomorphismus ist, gilt
\begin{align*}
  \gamma_{*}^{-t}(y_{\gamma^t(p)})-(\gamma_{*}^{-t}y)_p \in \T_pM.
\end{align*}
Die Differenz $(\gamma_{*}^{-t}y)_p - y_p$ ist also wohldefiniert. \textcolor{red}{(Ich bin mir nicht sicher ob die $y$ gro\ss  oder klein geschrieben sein sollen -Aleks)}
%\end{Bem}

\begin{Dfn}
  Es seien $X,Y \in \mathcal V(M)$ und $\gamma$ der Fluss von $X$.
Das durch 
\begin{align*}
  p \mapsto \lim_{t\to 0}\frac{1}t\left(\left(\gamma_{*}^{-t}y\right)_p-y_p\right)
\end{align*}
definierte glatte Vektorfeld heißt \CmMark{Lieableitung} von $Y$ längs $X$. Man schreibt $\mathcal L_XY$.
\end{Dfn}

%\begin{Bem}
Die Kurve $(\gamma_{*}^{-t}y)_p$ in $\T_pM$ ist glatt und $(\mathcal L_XY)_p = \difffrac[t=0]{}{t}(\gamma_{*}^{-t}Y)_p$. Dass $\mathcal L_XY$ glatt ist, rechnet man entweder in lokalen Koordinaten nach oder benutzt den folgenden Satz.
%\end{Bem}

% Satz 4.15
\begin{Satz}
  Für $X,Y \in \mathcal V(M)$ gilt:
  \begin{align*}
    \mathcal L_XY = [X,Y].
  \end{align*}
\end{Satz}

\begin{bew}
Es seinen $X,Y \in \mathcal V(M)$, $f \in \C^{\infty}(M)$ und $\gamma$ der Fluss von $X$.

Es ist zu zeigen: $[X,Y]_p(f) = \lim_{t \to \infty} \frac{1}{t}\left((\gamma_{*}^{-t}y)_p(f)-y_p(f)\right)$.
Dazu sei (um $(0,p)$)
\begin{align*}
  h(t,q) = f(\gamma^{-t}(q))-f(q) \text{ und } g_t(q) = \int_0^1 h'(ts,q)ds.
\end{align*}
Dann gilt:
\begin{align*}
  tg_t(q) = \int_0^1h'(ts,q)(t)ds = \int_0^th' = h(t,q) - h(0,q) = f(\gamma^{-t}(q)) - f(q)
\end{align*}
also $f \circ \gamma^t = f + tg_t$ und 
\begin{align*}
  g_0(q) = \lim_{t \to 0} \frac{tg_t(q)}{t} = \lim_{t \to 0}\frac{1}t\left(f(\gamma^{-t}(q))-f(q)\right) = \difffrac[t=0]{}{t}\left(f \circ \gamma^{-t}\right)(q) = -X_q(f)
\end{align*}
Betrachte:
\begin{align*}
  \left(\gamma_{*}^{-t}y\right)_p(f) = y\left(f \circ \gamma^{-t}\right)\left(\gamma^t(p)\right) = y(f)\left(\gamma^t(p)\right) + ty_{g_t}\left(\gamma^t(p)\right).
\end{align*}
Es folgt:
\begin{align*}
  \left(L_XY\right)_q(f) & = \lim \frac{1}t \left((\gamma_{*}^{-t}y)_p(f) - y_p(f)\right)\\
& = \lim_{t\to 0} \frac{1}t \left(yf(\gamma^t(p))-y_p(f)\right) + \lim_{t\to 0} y_{g_t}\left(\gamma^t(p)\right)\\
& = \difffrac[t=0]{}{t}\left(yf \circ \gamma^t\right)(p) + yg_0(p)\\
& = X_p(yf) - y_p(Xf) = [X,Y]_p(f).
\end{align*} 
\end{bew}


%%%
%%% 8. Vorlesung <2012-11-9 Fri>
%%% 

% Satz 4.16
\begin{Satz}
  Die Lieklammer $[X,Y]$ zweier Vektorfelder $X,Y \in \mathcal V(M)$ verschwindet genau dann, wenn ihre Flüsse (lokal) kommutieren, i.e.
  \begin{align*}
    \gamma_X^s \circ \gamma_Y^t = \gamma_Y^t \circ \gamma_X^s.
  \end{align*}
\end{Satz}

Der Beweis sei als Übungsaufgabe überlassen.

Es seien $X,Y \in \mathcal V(M)$ und $\Phi \colon M \to N$.
Bezeichnet $\gamma$ den Fluss von $X$, so gilt
\begin{align*}
  [\Phi_{\ast}X,\Phi_{\ast}Y] & = \mathcal L_{\Phi_{\ast}X} \Phi_{\ast}Y\\
  & = \difffrac[t=0]{}{t}\left(\Phi_{\ast} \circ \gamma_{\ast}^{-t} \circ \Phi_{\ast}^{-1}(\Phi_{\ast}Y)\right)\\
  & = \difffrac[t=0]{}{t} \left(\Phi_{\ast} \circ \gamma_{\ast}^{-t}Y\right) 
  = \Phi_{\ast}\left(\difffrac[t=0]{}{t}\gamma_{\ast}^{-t}Y\right)\\
  & = \Phi_{\ast}(\mathcal L_XY) = \Phi_{\ast}[X,Y].
\end{align*}
Man erhält einen alternativen Beweis der Jacobiidentität.


\begin{bew}
  Es seien $X,Y, Z \in \mathcal V(M)$ und $\gamma$ der Fluss von $X$.
  Dann gilt:
  \begin{align*}
    [X,[Y,Z]] & = \mathcal L_X[Y,Z] 
    = \difffrac[t=0]{}{t}\left(\gamma_{\ast}^{-t}[Y,Z]\right)\\
    & = \difffrac[t=0]{}{t}\left[\gamma_{\ast}^{-t}Y,\gamma_{\ast}^{-t}Z\right]\\
    & = \left[\difffrac[t=0]{}{t} \gamma_{\ast}^{-t}Y,Z\right] + \left[Y, \difffrac[t=0]{}{t}\gamma_{\ast}^{-t}Z\right]\\
    & = [\mathcal L_XY,Z] + [Y,\mathcal L_XZ]\\
    & = [[X,Y],Z] + [Y,[X,Z]]\\
    & = -[Z,[X,Y]] - [Y,[Z,X]].
  \end{align*}
\end{bew}

%%% Local Variables: 
%%% mode: latex
%%% TeX-master: "../skript-diffgeom"
%%% End: 

\chapter{Vektorb"undel}

Betrachte $\T M$ als glatte Mannigfaltigkeit durch
	\[ \begin{array}{ccc} \T M|_U = \pi^{-1}(U) &\to& \varphi(U) \times \R^m \cong U \times \R^m \\
		X_p = \sum \xi^i\pdifffrac[p]{}{x^i} &\mapsto& \left(\varphi(p),\xi\right). \end{array} \]
Fasern : $\T_pM = \pi^{-1}(p)$ $m$-dimensionaler Vektorraum und $X_p = \sum \xi^i \pdifffrac[p]{}{x^i} \mapsto \xi$ ist ein linearer Isomorphismus.

% Definition 5.1
\begin{Dfn}
  Ein \CmMark[Vektorb{\"u}ndel!glattes reelles]{glattes reelles Vektorb"undel} vom Rang $k$ "uber einer glatten Mannigfaltigkeit $M$ ist eine glatte Mannigfaltigkeit $E$, der sogenannte \CmMark{Totalraum} des B"undels, zusammen mit einer glatten Abbildung $\pi \colon E \to M$, der Projektion, sodass f"ur jedes $p \in M$ gilt:
  \begin{enumerate}[label=(\roman*)]
  \item Die Faser $E_p = \pi^{-1}(p)$ trägt die Struktur eines $k$-dimensionalen reellen Vektorraumes.
  \item Es existiert eine Umgebung $U$ von $p$ in $M$ und ein Diffeomorphismus
    \begin{align*}
      \tau \colon E|_U = \pi^{-1}(U) \to U \times \R^k,
    \end{align*}
    so dass die Einschränkung
    \begin{align*}
      \tau_p \colon E_p \to \R^k \quad ( \cong \{p\} \times \R^k)
    \end{align*}
    ein linearer Isomorphismus ist.
    Ein solches $\tau$ heißt \CmMark{B"undelkarte}.
  \end{enumerate}
\end{Dfn}

\begin{bsp}
  \begin{enumerate}[label=(\arabic*)]
  \item $E = M \times \R^k$ mit $\pi \colon E \to M, (p,x) \mapsto p$.
  \item Das Tangentialb"undel $\T M$ auf $M$.
  \item Ist $E \xrightarrow{\pi} M$ ein Vektorb"undel "uber $M$ und $U \subseteq M$ offen (oder eine Untermannigfaltigkeit), so ist $E|_U = \pi^{-1}(U)$ ein Vektorb"undel "uber $U$.
  \end{enumerate}
\end{bsp}

Ein \CmMark{Vektorb"undelmorphismus} zwischen zwei Vektorb"undeln $E \xrightarrow{\pi} M$ und $E' \xrightarrow{\pi'} N$ ist eine glatte Abbildung $F \colon E \to E'$, so dass eine glatte Abbildung $f$ existiert f"ur die das folgende Diagram kommutiert

\begin{center}\begin{tikzpicture}
	%\draw[step=0.25,gray!15] (-6,-1) grid (6,5); \draw[step=0.5,gray!30] (-6,-1) grid (6,5); \fill (0,0) circle(0.1); %Hilfsgitter
	\node (E) at (-1.5,0.75) {$E$}; \node (E') at (1.5,0.75) {$E'$}; \node (M) at (-1.5,-0.75) {$M$}; \node (N) at (1.5,-0.75) {$N$};
	
	\draw[->] (E) --node[above,font=\scriptsize]{$F$} (E'); \draw[->] (M) --node[above,font=\scriptsize]{$f$} (N);
	\draw[->] (E) --node[left,font=\scriptsize]{$\pi$} (M); \draw[->] (E') --node[right,font=\scriptsize]{$\pi'$} (N);
	
	\node at (0,0) {$\#$};
\end{tikzpicture}\end{center}

und ferner f"ur $p \in M$ die Abbildung $E_p \xrightarrow{F} E'_{f(p)}$ linear ist.

Gilt $M = N$ so ist ein $M$-Vektorb"undelmorphismus $F$ von $E$ nach $E'$ eine glatte Abbbildung $F \colon E \to E'$, so dass das folgende Diagram kommutiert und $F$ faserweise linear ist.

\begin{center}\begin{tikzpicture}
	%\draw[step=0.25,gray!15] (-6,-1) grid (6,5); \draw[step=0.5,gray!30] (-6,-1) grid (6,5); \fill (0,0) circle(0.1); %Hilfsgitter
	\node (E) at (-1.5,0.75) {$E$}; \node (E') at (1.5,0.75) {$E'$}; \node (M) at (0,-0.75) {$M$};
	
	\draw[->] (E) --node[above,font=\scriptsize]{$F$} (E');
	\draw[->] (E) --node[left,font=\scriptsize]{$\pi$} (M); \draw[->] (E') --node[right,font=\scriptsize]{$\pi'$} (M);
	
	\node at (0,0) {$\#$};
\end{tikzpicture}\end{center}

Die Vektorb"undel $E,E'$ "uber $M$ heißen \CmMark[Vektorb{\"u}ndel!isomorphes]{isomorph}, wenn ein $M$-Vektor"-b"undel"-morphismus
$G$ existiert mit $G F = \Id_E$ und $FG = \Id_{E'}$.
Dies ist genau dann der Fall, wenn $F$ faserweise ein Inverses besitzt.
(Der Beweis dieser Aussage sei als "Ubungsaufgabe "uberlassen.)

Ein Vektorb"undel $E \xrightarrow{\pi} M$ heißt \CmMark[Vektorb{\"u}ndel!triviales]{trivial}, wenn es einen Vektorb"undelisomorphismus von $E$ auf $M \times \R^k$ gibt. Jedes
\begin{align*}
  \tau \colon E|_U \to U \times \R^k
\end{align*}
ist ein Vektorb"undelisomorphismus.
Die B"undelkarten werden daher auch \CmMark[Trivialisierung!lokale]{lokale Trivialisierungen} genannt.

Es sei $(\tau_\alpha,U_{\alpha})_{\alpha \in \calI}$ eine Familie lokaler Trivialisierungen von $E$ mit $M = \bigcup_{\alpha \in J}U_{\alpha}$.
Der Diffeomorphismus
\begin{align*}
  \tau_{\alpha} \circ \tau_{\beta}^{-1}\colon (U_{\alpha} \cap U_{\beta}) \times \R^k \to (U_{\alpha} \cap U_{\beta}) \times \R^k
\end{align*}
definiert die sogenannten \CmMark{{\"U}bergangsfunktionen}
\begin{align*}
  g_{\alpha\beta} \colon U_{\alpha} \cap U_{\beta} \to \gls{GL}_k(\R)
\end{align*}
durch 
\begin{align*}
  \tau_{\alpha} \circ \tau_{\beta}^{-1}(p,x) = (p,g_{\alpha\beta}(p) x).
\end{align*}
Die "ubergangsfunktionen sind glatt und f"ur alle $p \in U_{\alpha} \cap U_{\beta} \cap U_{\gamma}$ gilt:
\begin{align*}
  g_{\alpha\gamma}(p) = g_{\alpha\beta}(p) \cdot g_{\beta\gamma}(p),
\end{align*}
denn
\begin{align*}
	(p,g_{\alpha\gamma}(p)x) & = \tau_{\alpha} \circ \tau_{\gamma}^{-1}(p,x)\\
	& = \tau_{\alpha} \circ \tau_{\beta}^{-1} \circ \tau_{\beta} \circ \tau_{\gamma}^{-1}(p,x)\\
	& = (\tau_{\alpha} \circ \tau_{\beta}^{-1})(p,g_{\beta\gamma}(p)x)\\
	& = (p,g_{\alpha\beta}(p)\cdot g_{\beta\gamma}(p)x).
\end{align*}

\begin{bsp}
  Die "ubergangsfunktionen von $\T M$ sind gegeben durch
  \begin{align*}
    \D(\psi \circ \varphi^{-1}) = \left(\partial_i(\psi^j \circ \varphi^{-1})\right)_{i,j \leq m}.
  \end{align*}
\end{bsp}

% Satz 5.2
\begin{Satz}\label{satz-5-2}
  Es sei $M$ eine glatte Mannigfaltigkeit mit einer offenen "Uberdeckung $\{U_{\alpha}\}_{\alpha \in \calI}$ und einer glatten Abbildung
  \begin{align*}
    g_{\alpha\beta} \colon U_{\alpha} \cap U_{\beta} \to \Gl_k(\R)
  \end{align*}
so dass f"ur alle $\alpha,\beta,\gamma \in \calI$ und $p \in U_{\alpha} \cap U_{\beta} \cap U_{\gamma}$ gilt:
\begin{align*}
  g_{\alpha\gamma} (p) = g_{\alpha\beta}(p)g_{\beta\gamma}(p).
\end{align*}
Dann ist
	\[ E = \bigcup_{\alpha \in \calI}^{\cdot} \FakRaum{U_{\alpha} \times \R^k}{\sim}, \]
wobei $(p,x)_{\alpha} \sim (q,y)_{\beta}$ genau dann gilt, wenn $p = q$ und $x = g_{\alpha\beta}(p)y$, ein glattes Vektorb"undel.
\end{Satz}

Der Beweis sei erneut als Aufgabe "uberlassen.

\begin{kor}
  Ist $E$ ein glattes Vektorb"undel "uber $M$ mit "Ubergangsfunktionen $\{g_{\alpha\beta}\}$, so ist das oben konstruierte Vektorb"undel isomorph zu $E$.
\end{kor}

Es sei $E \xrightarrow{\pi} N$ ein Vektorb"undel und $\Phi \colon M \to N$ glatt.
Das längs $\Phi$ zur"uckgezogene B"undel ("`\CmMark{pullback}"') ist definiert durch den Totalraum
\begin{align*}
  E' = \Phi^{\ast}E = \{(p,x) \mid x \in E_{\Phi(p)}\} \subseteq M \times E,
\end{align*}
die Projektion $\pi' \colon \Phi^{\ast}E \to M, (p,x) \mapsto p$ und die folgenden B"undelkarten:
Es sei $p \in M$ und $(\tau, U)$ eine B"undelkarte von $E$ um $\Phi(p)$, sowie $(\varphi,V)$ eine Karte von $M$ um $p$ mit $\Phi(V) \subseteq U$.
Dann definiert 
\begin{align*}
  \Phi^{\ast}E|_V \to V \times \R^k, (p,x) \mapsto \left(p, \tau_{\Phi(p)}(x)\right)
\end{align*}
eine B"undelkarte.

Sind $E \xrightarrow{\pi} M, E' \xrightarrow{\pi'} N$ Vektorb"undel, dann ist $E \times E' \xrightarrow{\pi \times \pi'} M \times N$ mit lokalen Trivialisierungen $\tau \times \tau'$ ebenfalls ein Vektorb"undel.
Insbesondere ist im Falle $M = N$ $E \times E'$ ein B"undel "uber $M \X M$.
Es sei $\Delta \colon M \to M \times M$, $p \mapsto (p,p)$.

\begin{center}\begin{tikzpicture}
	%\draw[step=0.25,gray!15] (-6,-1) grid (6,5); \draw[step=0.5,gray!30] (-6,-1) grid (6,5); \fill (0,0) circle(0.1); %Hilfsgitter
	\node (1) at (-2,0.75) {$E \oplus E' = \Delta^*(E \X E')$}; \node (2) at (2,0.75) {$E \X E'$}; \node (3) at (-2,-0.75) {$M$}; \node (4) at (2,-0.75) {$M \X M$};
	
	\draw[->] (1) -- (2); \draw[->] (3) --node[above,font=\scriptsize]{$\Delta$} (4);
	\draw[->] (1) -- (3); \draw[->] (2) -- (4);
\end{tikzpicture}\end{center}

Das längs $\Delta$ zur"uckgezogene B"undel $E \oplus E' = \Delta^{\ast}(E \times E')$ heißt die \CmMark{Whitneysumme} von $E$ und $E'$.
Faserweise gilt
\begin{align*}
  (E \oplus E')_p = E_p \oplus E'_p.
\end{align*}
 
\emph{"Uberlege:} $\Hom(E,E')$, sowie $E \otimes E', \bigotimes E$ und $\Lambda^pE, \Lambda E$ sind "`vern"unftige"' B"undel.

\section{Intermezzo: Multilineare Algebra}
Es seien $V$ und $W$ $k$-Vektorr"aume. Das \CmMark{Tensorprodukt} $V \otimes W$ ist der von den Elementen $v \otimes w$ mit $v \in V$, $w \in W$ und den Relationen \begin{enumerate}[label=(\roman*),widest=iii]
\item
	$(v + v') \otimes w = v \otimes w + v' \otimes w$
\item
	$v \otimes (w + w') = v \otimes w + v' \times w'$
\item
	$(\lambda v) \otimes w = \lambda (v \otimes w) = v \otimes (\lambda w)$
\end{enumerate} erzeugte Vektorraum.

\begin{emptythm}[Eigenschaften:]\begin{enumerate}[label=\arabic*),leftmargin=*]
\item
	Die Abbildung $b: V \X W \to V \otimes W$, $(v, w) \mapsto v \otimes w$ ist bilinear.
\item
	$v \otimes w = 0 \Leftrightarrow v = 0$ oder $w = 0$
\item
	$V \otimes k \cong V$
\item
	$V \otimes W \cong W \otimes V$
\item
	$V^* \otimes W \cong \Hom(V, W)$ verm"oge $(\varphi \otimes w)(v) = \varphi(v) \cdot w$ wobei $V^*$ der \gls{Dualraum} zu $V$ ist
\item
	Sind $\{v_i\}_{i \in \calI}$ und $\{w_j\}_{j \in \calJ}$ Basen von $V$ und $W$, so ist $\{v_i \otimes w_j\}_{(i,j) \in \calI \X \calJ}$ eine Basis von $V \otimes W$. Insbesondere gilt f"ur Vektorr"aume endlicher Dimension dass $\ddim (V \otimes W) = \ddim V \cdot \ddim W$
\end{enumerate}\end{emptythm}

\begin{emptythm}[Universelle Eigenschaft:]
Ist $U$ ein Vektorraum und $\beta$ eine bilineare Abbildung von $V \X W$ in $U$. Dann existiert genau eine lineare Abbildung $\varphi: V \otimes W \to U$ mit $\beta = \varphi \circ b$.
\begin{center}\begin{tikzpicture}
	\node (1) at (-2,0) {$V \X W$}; \node (2) at (1,0) {$U$}; \node (3) at (-2,-1.5) {$V \otimes W$};
	\draw[->] (1) --node[above,font=\scriptsize]{$\beta$} (2);
	\draw[->] (1) --node[left,font=\scriptsize] {$b$} (3);
	\draw[->,dashed] (3) --node[below,font=\scriptsize] {$\varphi$} (2);
\end{tikzpicture}\end{center}
Diese Eigenschaft bestimmt $(V \otimes W, b)$ eindeutig bis auf Isomorphie.
\end{emptythm}

Das Bilden von Tensorprodukten ist bis auf Isomorphie assoziativ. Man schreibt daher
	\[ \bigotimes^p V = \underbrace{V \otimes \ldots \otimes V}_{p\text{-mal}} \]
Setzt man $\bigotimes^0 V = k$, so wird $\bigotimes V = \bigoplus_{p=0}^{\infty} \bigotimes^p V$ mit dem durch die Zuordnungen
	\[ (v_1 \otimes \ldots \otimes v_{p+1} \otimes \ldots \otimes v_{p+q}) \mapsto v_1 \otimes \ldots \otimes v_{p+q} \in \bigotimes^{p+q} V \]
induzierten Produkt zu einer graduierten Algebra.

Das $p$-fach \CmMark[Produkt!{\"a}u{\ss}eres]{"au"sere Produkt} $\Lambda^pV$ ist der von den Elementen $v_1 \wedge \ldots \wedge v_p$, $v_i \in V$ und den Relationen \begin{enumerate}[label=(\roman*),widest=iii]
\item
	$v_1 \wedge \ldots \wedge v_i \wedge v_{i+1} \wedge \ldots \wedge v_p = - v_1 \wedge \ldots \wedge v_{i+1} \wedge v_i \wedge \ldots \wedge v_p$ (Schiefsymmetrie)
\item
	$(v_1 + w_1) \wedge \ldots \wedge v_p = v_1 \wedge \ldots \wedge v_p + w_1 \wedge v_2 \wedge \ldots \wedge v_p$
\item
	$(\lambda v_1) \wedge \ldots \wedge v_p = \lambda (v_1 \wedge \ldots \wedge v_p)$
\end{enumerate}
erzeugte Vektorraum.

\begin{emptythm}[Eigenschaften:]\begin{enumerate}[label=\arabic*),leftmargin=*]
\item
	Die Abbildung $s: V \X \ldots \X V \to \Lambda^p V$, $(v_1,\ldots ,v_p) \mapsto v_1 \wedge \ldots \wedge v_p$ ist multilinear und schief
\item
	Es gilt $v_1 \wedge \ldots \wedge v_p = 0$ genau dann wenn $v_1, \ldots , v_p$ linear abh"angig sind
\item
	Ist $\{e_i\}_{i \le n}$ eine Basis von $V$, so ist $\{e_{i_1} \wedge \ldots \wedge e_{i_p} | i_1 < i_2 < \ldots < i_p\}$ eine Basis von $\Lambda^pV$, es gilt also $\ddim \Lambda^pV = \left( \begin{smallmatrix} n \\ p \end{smallmatrix} \right)$. Insbesondere ist $\Lambda^pV = 0$ falls $p > n$ und $\ddim \Lambda^nV = 1$ und f"ur $v_i = \sum \alpha_i^j e_j$ gilt:
		\[ v_1 \wedge \ldots \wedge v_p = \ddet (\alpha_i^j) \cdot e_1 \wedge \ldots \wedge e_n \]
\end{enumerate}\end{emptythm}

\begin{emptythm}[Universelle Eigenschaft:]
Ist $U$ ein Vektorraum und $\sigma$ eine schiefsymmetrische multilineare Abbildung $\underbrace{V \X \ldots  \X V}_{p\text{-mal}} \to U$, so existiert genau eine lineare Abbildung $\varphi: \Lambda^pV \to U$ mit $\sigma = \varphi \circ s$.
\begin{center}\begin{tikzpicture}
	\node (1) at (-2,0) {$\bigotimes^pV$}; \node (2) at (1,0) {$U$}; \node (3) at (-2,-1.5) {$\Lambda^pV$};
	\draw[->] (1) --node[above,font=\scriptsize]{$\sigma$} (2);
	\draw[->] (1) --node[left,font=\scriptsize] {$s$} (3);
	\draw[->,dashed] (3) --node[below,font=\scriptsize] {$\varphi$} (2);
\end{tikzpicture}\end{center}
Die Isomorphieklasse von $(\Lambda^pV, s)$ ist durch diese Eigenschaft eindeutig bestimmt.
\end{emptythm}

Setzt man $\Lambda^0V = k$, so wird $\Lambda V = \bigoplus_{p=0}^{\infty}\Lambda^pV$ mit dem durch
	\[ (v_1 \wedge \ldots \wedge v_p, v_{p+1} \wedge \ldots \wedge v_{p+q}) \mapsto v_1 \wedge \ldots \wedge v_{p+q} \]
induzierten Produkt zu einer assoziativen, graduiert kommutativen Algebra: $v \in \Lambda^pV$, $w \in \Lambda^qV$, $v \wedge w = (-1)^{p \cdot q} w \wedge v$. Sind $V_1$, $V_2$, $W_1$, $W_2$ Vektorr"aume und $\varphi \in \Hom(V_1, W_1)$, $\psi \in \Hom(V_2, W_2)$, so definiert die Fortsetzung von 
	\[ \varphi \otimes \psi (v_1 \otimes v_2) = (\varphi(v_1)) \otimes (\psi(v_2)) \]
ein Element $\varphi \otimes \psi \in \Hom(V_1 \otimes V_2, W_1 \otimes W_2)$. Sind $V$, $W$ Vektorr"aume und $\varphi_1,\ldots ,\varphi_p \in \Hom(V, W)$, so induzieren diese eine lineare Abbildung
	\[ \varphi_1 \otimes \ldots \otimes \varphi_p : \bigoplus^p V \to \bigoplus^p W \]
und
	\[ \varphi_1 \wedge \ldots  \wedge \varphi_p : \Lambda^pV \to \Lambda^pW \]

\section{B"undelkonstruktion}
Es seien $E$ und $E'$ Vektorb"undel vom Rang $k$ und $l$. Es bezeichnen stets $\tau$ und $\tau'$ lokale Trivialisierungen mit dem gleichen Trivialisierungsgebiet $U$ sowie $g_{\alpha\beta}$ und $g'_{\alpha\beta}$ die "Ubergangsfunktionen von $E$ beziehungsweise $E'$. Das \CmMark{Tensorprodukt} $E \otimes E'$ von $E$ und $E'$ ist das Vektorb"undel mit den Fasern $(E \otimes E')_p = E_p \otimes E'_p$, also $E \otimes E' = \bigcup_{p \in M} E_p \otimes E'_p \to M$ mit lokalen Trivialisierungen
	\[ \sigma: \left\{\begin{array}{ccc} (E \otimes E')|_U = \bigcup\limits_{p \in U} E_p \otimes E'_p &\to& U \X (\R^k \otimes \R^l) \cong U \X \R^{kl} \\
		E_p \otimes E'_p \ni w = \sum v_i \otimes u_i &\mapsto& (p, \sum \tau_p(v_i) \otimes \tau'_p(u_i)) \end{array}\right. \]
das hei"s dass $\sigma = (\pi, \tau_p \otimes \tau'_p)$ ist. Wie in Kapitel 4 zeigt man dass $E \otimes E'$ ein B"undel ist. Alternativ l"asst sich $E \otimes E'$ mit Satz \ref{satz-5-2} durch die "Ubergangsfunktion definieren:
	\[\begin{array}{ccc} E \otimes E' &=& \FakRaum{\dot \bigcup U_{\alpha} \X (\R^k \otimes \R^l)}{\sim} \\
		h_{\alpha\beta} &=& g_{\alpha\beta} \otimes g'_{\alpha\beta} \end{array}\]
Die Relation $\sim$ ist durch $h_{\alpha\beta}$ wie in Satz \ref{satz-5-2} definiert. Analog definiert man h"ohere Tensorprodukte $\bigotimes^pE$, die Tensoralgebra $\bigotimes E$ und "au"sere Produkte $\Lambda^pE$ und $\Lambda E$.

Das duale B"undel $E^*$ hat die Fasern $E_p^* = \Hom(E_p, \R)$ und lokale Trivialisierungen:
	\[ \begin{array}{rcl} \sigma: E^*|_U &\to& U \X \R^k\\
	v^* \in E^*_p, \sigma(v^*) &=& (p, v^* \circ \tau_p^{-1}) \\
	&=& (p, (\tau_p^{-1})^*(v^*)) \end{array}\]
Die "Ubergangsfunktionen von $E^*$ sind die transponierten Inversen der "Ubergangsfunktionen von $E$:
	\[ h_{\alpha\beta} = (g_{\alpha\beta}^{-1})^* \]
Das B"undel
	\[ \T_s^r E = \underbrace{E \otimes \ldots \otimes E}_{r} \otimes \underbrace{E^* \otimes \ldots \otimes E^*}_{s} \]
hei"st das $(r,s)$-\CmMark{Tensorb"undel} von $E$. Das Homomorphismenb"undel $\Hom(E,E^*)$ ist definiert durch
	\[ \sigma: \left\{ \begin{array}{ccc} \Hom(E,E')|_U = \bigcup\limits_{p \in U}^{\cdot} \Hom(E_p, E_p') &\to& U \X \Hom(\R^k, \R^l)\\
		\varphi \in \Hom(E_p, E'_p) &\mapsto& (p, \tau'_p \circ \varphi \circ \tau_p^{-1}) \end{array} \right. \]
Zur Definition der "Ubergangsfunktionen schreibt man $\Hom(E, E') \cong E^* \otimes E'$ und definiert $h_{\alpha\beta} = (g_{\alpha\beta}^{-1})^* \otimes g'_{\alpha\beta}$.
\begin{center}\begin{tikzpicture}
	\node (1) at (-1.5,1) {$V^*$}; \node (2) at (1.5,1) {$V$}; \node (3) at (-1.5,-1) {$W^*$}; \node (4) at (1.5,-1) {$W$};
	\draw[->] (2) --node[right,font=\scriptsize] {$g_{\alpha\beta}$} (4);
	\draw[->,transform canvas={xshift=1mm}] (1) --node[anchor=north west,font=\scriptsize] {$(g^*_{\alpha\beta})^{-1}$} (3);
	\draw[->,transform canvas={xshift=-1mm}] (3) --node[anchor=south east,font=\scriptsize] {$g_{\alpha\beta}^*$} (1);
	\draw[->,transform canvas={yshift=1mm},decorate,decoration={snake,segment length=6mm}] (1,0) -- (-1,0);
\end{tikzpicture}\end{center}

\begin{Dfn}
Es sei $E \xrightarrow{\pi} M$ ein Vektorb"undel "uber M. Ein \CmMark{Schnitt} in $E$ ist eine glatte Abbildung $S: M \to E$ mit $\pi \circ S = \Id_M$, also $S(p) \in E_p = \pi^{-1}(p)$. Der Raum der Schnitte wird mit $\Gamma(E)$ bezeichnet. $\Gamma(E)$ ist ein $C^{\infty}$-Modul.
\begin{center}\begin{tikzpicture}[scale=0.9]
%	\draw[step=0.25,gray!15] (-6,-1) grid (6,5); \draw[step=0.5,gray!30] (-6,-1) grid (6,5); \fill (0,0) circle(0.1); %Hilfsgitter
	
	\draw[name path=kurve] (-3,2) to[out=30,in=150] (0,2) to[out=330,in=210] (3,2);
	\draw (-3,0) to[out=30,in=150] (0,0) to[out=330,in=210] (3,0);
	
	\foreach \x in {-2.5, -2, ..., 2.5}{
		\draw[name path=strich] (\x,1) -- (\x,3);
		\path[name intersections={of=kurve and strich}];
		\fill (intersection-1) circle(0.05);
	}
	
	\node at (3.5,0) {$M$};
	\node at (3.5,2.5) {$E = \bigcup E_p$};
\end{tikzpicture}\end{center}
\end{Dfn}

%%%
%%% 10. Vorlesung <2012-11-16 Fri>
%%% 

\begin{center}\begin{tikzpicture}
%	\draw[step=0.25,gray!15] (-6,-1) grid (6,5); \draw[step=0.5,gray!30] (-6,-1) grid (6,5); \fill (0,0) circle(0.1); %Hilfsgitter
	
	\node (1) at (-2.5,0) {$S|_U: U$};
	\node (2) at (0,0) {$E|_U$};
	\node (3) at (2.5,0) {$U \X \R^k$};
	
	\draw[->] (1)node[above,font=\tiny,gray,yshift=1mm]{$S_i^{-1}(p) = \tau^{-1}(p,e_i)$} -- (2); \draw[->] (2) --node[above,font=\scriptsize]{$\cong$}node[below,font=\scriptsize]{$\tau$} (3);
	\draw[->] (1) to[out=330,in=210]node[below,font=\scriptsize]{$\pi \circ S|_U$} (3);
	
	\draw[->,gray] (3.5,-0.5)node[anchor=north,font=\tiny,xshift=7mm]{Basis $\{e_i\}$ von $\R^k$} to[out=90,in=0] (3);
	
	\node (4) at (-2.5,-1.5) {$p$}; \node (5) at (2.5,-1.5) {$(p,x)$};
	\draw[|->] (4) -- (5);
	
	\node (6) at (-2.5,-2) {$p$}; \node (7) at (2.5,-2) {$(p,e_i)$};
	\draw[|->] (7) -- (6);
\end{tikzpicture}\end{center}
Die $S_i$ bilden punktweise eine Basis der Fasern.

Die Schnitte des $(r,s)$-Tensorbündels $\Gamma(\T_s^r(\T M)) = \mathcal T_s^r(M)$ bezeichnet man als \CmMark{$(r,s)$-Tensorfelder} auf $M$. Die $(1,0)$-Tensorfelder sind genau die Vektorfelder auf $M$; $\mathcal T_0^1(M) = \mathcal V(M)$. Es bezeichne $\mathcal V^{*}(M) = \mathcal T_1^0(M)$ den Raum der $(0,1)$-Tensorfelder.

\begin{Prop}
Die $(r,s)$-Tensorfelder auf $M$ entsprechen genau den $C^{\infty}(M)$-multilinearen Abbildungen
	\[ \underbrace{\mathcal V^{*}(M) \X \cdots \X \mathcal V^{*}(M)}_{r-\text{mal}} \X \underbrace{\mathcal V(M) \X \cdots \X \mathcal V(M)}_{s-\text{mal}} \to C^{\infty}(M) \]
verm"oge der linearen Forsetzung
\begin{align*}
	p \mapsto & X_1 \otimes \cdots \otimes X_r \otimes \omega_1 \otimes \cdots \otimes \omega_s (\eta_1, \ldots, \eta_r, Y_1, \ldots, Y_s)\\
	& = \eta_1(X_1)\eta_2(X_2)\cdots\eta_r(X_r) \cdot \omega(Y_1)\cdots \omega_s(Y_s) (p)
\end{align*}
\end{Prop}

Der Beweis sei als Übung überlassen.

\begin{bsp}\begin{enumerate}[label=\arabic*)]
\item
	Ist $f \in C^{\infty}(M)$, so ist durch sein Differential
	\begin{align*}
		\dop f|_p(X_p) \textcolor{red}{=} X_p(f)
	\end{align*}
	ein $(0,1)$-Tensorfeld gegeben.
\item
	Die Lieklammer $[\cdot,\cdot]$ ist \emph{nicht} $C^{\infty}(M)$-linear, also kein Tensorfeld.
\end{enumerate}\end{bsp}

Ein Element von $\T_p^*M$ bezeichnet man als \CmMark[Kotangentialvektor]{Kotangentialvektoren}, $\T_p^*M$ als \CmMark{Kotangentialvektorraum} und das Bündel $\T^{*}M$ als \CmMark{Kotangentialbündel}. \textcolor{red}{(in der Vorlesung wurde das Element von $\T_p^*M$ selbst das Kotangententialvektorraum bezeichnet und der kotangentialvektor wurde "uberhaupt nicht er"ahnt, das erscheint falsch, also haben wir es ge"andert)}

Ist $(\varphi, U)$ eine Karte von $M$, dann bilden die Differentiale $\dop \varphi^i = \dop x^i$ der Koordinatenfunktionen (punktweise) eine Basis von $\T_p^{*}M$, denn
\begin{align*}
  \dop x^i|_p\left(\pdifffrac[p]{}{x^i}\right) = \pdifffrac{}{x^j}(\varphi^i) = \delta_j^i.
\end{align*}
Die $\dop x^i$ sind also \textcolor{red}{paarweise} linear unabhängige $(0,1)$-Tensorfelder über dem Kartengebiet $U$.

Ist $\psi$ eine weitere Karte und bezeichnen $\pdifffrac{}{y^i}$ beziehungsweise $\dop y^i$ die entsprechenden Koordinaten(ko)tangentialvektoren, so gilt:
	\[ \dop x^i = \sum \alpha_j^i\dop y^j \]
mit
	\[ \alpha_j^i \overset{!}{=} \dop x^i\left(\pdifffrac{}{y^j}\right) = \sum \alpha_k^i \underbrace{\dop y^k\left(\pdifffrac{}{y^j}\right)}_{\delta_j^k} \]
denn
	\[ \alpha_j^i = \textcolor{red}{\pdifffrac{}{y^j}(\varphi^i)} = \pdifffrac{p^i}{y^i} = \partial_j(\varphi^i\circ \psi^{-1}) \]
Es gilt also $\alpha = \D(\psi \circ \varphi^{-1})^{x^{-1}}$, vergleiche Satz 2.10 \textcolor{red}{(richtige Nummer hier, Nummerierung in Kapitel 2 korrigieren)}
\begin{align*}
  \pdifffrac{}{x^i} = \sum \partial_i (\psi^j \circ \varphi^{-1})\pdifffrac{}{y^j}.
\end{align*}
Diese transponierten Inversen der Differentiale der Kartenwechsel sind geanu die Übergangsfunktionen $h_{\alpha\beta} = (g_{\alpha\beta}^{*})^{-1}$ in der Definition von $(\T M)^{*} = \T^{*}M$.

Ist $S$ ein Tensorfeld vom Typ $(0,s)$ auf einer Mannigfaltigkeit $N$ und $\Phi \colon M \to N$ eine glatte Abbildung.
\begin{center}\begin{tikzpicture}
	\node (1) at (-2,0.75) {$\calT_s^0(M)$};
	\node (2) at (2,0.75) {$\calT_s^0(N)$};
	\node (3) at (-2,-0.75) {$M$};
	\node (4) at (2,-0.75) {$N$};
	
	\draw[->] (2) --node[above,font=\scriptsize]{$\Phi^*$} (1);
	\draw[->] (1) -- (3);
	\draw[->] (2) -- (4);
	\draw[->] (3) --node[above,font=\scriptsize]{$\Phi$} (4);
\end{tikzpicture}\end{center}
	\[ \begin{array}{rccc} S \colon & \mathcal V(N) \X \cdots \X \mathcal V(N) &\to& C^{\infty}(N)\\
		\Phi^{*}S \colon & \mathcal V(M) \X \cdots \X \mathcal V(M) &\to& C^{\infty}(M) \end{array} \]
Dann ergibt
	\[ \Phi^{*}S(X_{1},\ldots,X_{s}) = S(\Phi_{*}X_{1},\ldots,\Phi_{*}X_{s}) \]
ein $(0,s)$-Tensorfeld auf $M$, das entlang $\Phi$ zurückgezogene Tensorfeld (\CmMark{pullback}).

Es gilt $\Phi^{*}(S \otimes T) = \Phi^{*}S \otimes \Phi^{*}T$ für $S \in \mathcal T_s^0(N), T \in \mathcal T_t^0(N)$.
Ist $\Psi \colon P \to M$ glatt, so gilt
	\[ (\Phi \circ \Psi)^{*} = \Psi^{*} \circ \Phi^{*}.\]

Ist $\Phi$ ein Diffeomorphismus, so lässt sich $\Phi^{*}$ für beliebige Tensorfelder $S \in \mathcal T_s^r(M)$ definieren:
	\[ \begin{array}{rl} p \mapsto& \Phi^{*}S (\omega_1,\ldots,\omega_r,X_1,\ldots,X_s)(p)\\
		& S((\Phi^{*})^{-1}\omega_1, \ldots, (\Phi^{*})^{-1}\omega_{r}, \Phi_{*}X_1,\ldots,\Phi_{*}X_s)(p) \end{array} \]
mit
	\[ \begin{array}{rccc} S \colon& \underbrace{\mathcal V^{*}(N) \cdots \mathcal V^{*}(N)}_{r\text{-mal}} \X \underbrace{ \mathcal V(N) \cdots\mathcal V(N)}_{s\text{-mal}} &\to& C^{\infty}(N)\\
		\Phi^*S \colon& \mathcal V^{*}(M) \cdots \mathcal V^{*}(M) \X \mathcal V(N) \cdots\mathcal V(N) &\to& C^{\infty}(M) \end{array} \]
	\[ \begin{array}{rccc} \textcolor{red}{\omega \in}& \mathcal V^{*}(M) = \mathcal T_1^0(M) &\to& C^{\infty}(M)\\
		\textcolor{red}{X \in}& \mathcal V(M) = \calT_0^1(M) &\to& C^{\infty}(M) \end{array} \]
Insbesondere: $X \in \mathcal T_0^1(M) = \mathcal V(M) = \{\calV^*(M) \to C^{\infty}(M)\}$
\begin{align*}
  \Phi^{*}X(\omega) = X((\Phi^{*})^{-1}\omega) = ((\Phi^{*})^{-1}\omega)(X) = \omega(\Phi_{*}^{-1}X)
\end{align*}
also $\Phi^{*}X = \Phi_{*}^{-1}X$.

\begin{bsp}[Anwendung]
  Es sei $(\varphi_{\alpha},U_{\alpha})$ ein Atlas von $M$. Dann ist jedes $\varphi_{\alpha}$ ein Diffeomorphismus von $U_{\alpha}$ auf $\varphi_{\alpha}(U_{\alpha}) = V_{\alpha} \subset \R^m$. Ist $S \in \mathcal T_s^r(M)$, so ist
  \begin{align*}
    S_{\alpha} = (\varphi_{\alpha}^{-1})^{*}S|_{U_\alpha}
  \end{align*}
  ein $(r,s)$-Tensorfeld auf $V_{\alpha}$.

  \textcolor{red}{Abbildung 10.4 Diagram Beispiel/Anwendung}

  Für alle $\alpha, \beta$ mit $U_{\alpha} \cap U_{\beta}$ gilt
  \begin{align*}
    (\varphi_{\alpha} \circ \varphi_{\beta}^{-1})^{*}S_{\alpha}|_{V_{\alpha} \cap V_{\beta}} & = (\varphi_{\alpha} \circ \varphi_{\beta}^{-1})^{*}\left((\varphi_{\alpha}^{-1})^{*}S|_{U_{\alpha}}\right)|_{V_{\alpha}\cap V_{\beta}}\\
    & = \left.(\varphi_{\beta}^{-1})^{*} \circ \varphi_{\alpha}^{*} \circ (\varphi_{\alpha}^{-1})^{*} S|_{U_{\alpha}}\right|_{V_{\alpha} \cap V_{\beta}}\\
    & = \left.(\varphi_{\beta}^{-1})^{*}S|_{U_{\beta}}\right|_{V_{\alpha} \cap V_{\beta}}\\
    & = S_{\beta}|_{V_{\alpha} \cap V_{\beta}}
  \end{align*}
  Ist umgekehrt $S_\alpha$ eine Familie von $(r,s)$-Tensorfeldern auf $V_{\alpha}$ mit obigem Transformationsverhalten, so definiert dies ein $(r,s)$-Tensorfeld auf $M$.
\end{bsp}

\begin{Dfn}
  Es sei $X \in \mathcal V(M)$ mit dem Fluss $\gamma$ und $S \in \mathcal T_s^r(M)$ ein glattes Vektorfeld auf $M$. Dann heißt
  \begin{align*}
    \mathcal L_X S = \difffrac[t=0]{}{t}(\gamma^{t*}S)
  \end{align*}
  die \CmMark{Lieableitung} von $S$ in Richtung $X$.
\end{Dfn}

\begin{emptythm}[Eigenschaften]\begin{enumerate}[label=\arabic*)]
\item
	$f \in \mathcal T_0^0(M) = C^{\infty}(M)$. Dann ist $\mathcal L_Xf = \dop f (X) = X(f)$.
\item
	$X,Y \in \mathcal T_0^1(M) = \mathcal V(M)$, so gilt
	\begin{align*}
		\mathcal L_XY = [X,Y].
	\end{align*}
\item
	Für $S \in \mathcal T_s^r(M), T \in \mathcal T_{s'}^{r'}(M)$ gilt:
	\begin{align*}
		\mathcal L_X(S \otimes T) = (\mathcal L_XS) \otimes T + S \otimes (\mathcal L_X T).
	\end{align*}
\end{enumerate}\end{emptythm}


\section{Differentialformen und die äußere Ableitung}

\begin{Dfn}
  Das Vektorbündel $\Lambda^k(\T^{*}M)$ wird mit $\Lambda^k(M)$ bezeichnet und der Raum seiner Schnitte $\Gamma(\Lambda^k(\T^{*}M))$ mit $\Omega^k(M)$. Die Elemente von $\Omega^k(M)$ heißen \CmMark{Differentialformen} vom Grad $k$ oder kurz \CmMark{$k$-Formen} auf $M$.
\end{Dfn}

Ist $(\varphi, U)$ eine Karte von $M$, so bilden die Differentiale der Koordinatenfunktionen $\dop x^i = \dop \varphi^i$ eine Basis von $\T^{*}M$.
Diese sind (lokale) Schnitte in $\Lambda^1(\T^{*}M)$, also lokal $1$-Formen. Das Differential von $f \in C^{\infty}(M)$ ist eine $1$-Form $\dop f(X) = X(f)$.
Lokal gilt $\dop f = \sum f_i \dop x^i$, wobei $f_i = \dop f \left(\pdifffrac{}{x^i}\right) = \pdifffrac{}{x^i}(f) = \pdifffrac{f}{x^i}$.

Desweiteren sind (lokal) die $\left( \begin{smallmatrix} m \\ k\end{smallmatrix} \right)$ $k$-Formen $\dop x^{i_1} \wedge \ldots \wedge \dop x^{i_k}$ mit $i_1 < \cdots < i_k$ eine Basis von $\Omega^k$. Jede $k$-Form $\omega$ ist lokal von der Gestalt
\begin{align*}
  \omega = \sum_{\mathclap{i_1 < \cdots < i_k}}f_{i_1,\ldots,i_k}\dop x^{i_1} \wedge \cdots \wedge \dop x^{i_k}.
\end{align*}


%%% Local Variables: 
%%% mode: latex
%%% TeX-master: "../skript-diffgeom"
%%% End: 

%% 
%% Skript Differentialgeometrie im Wintersemester 12/13
%% Zur Vorlesung von Dr. Grensing am KIT Karlsruhe
%% 
%% Kapitel 6
%% 

\chapter{Riemannsche Metriken}

\begin{emptythm}[Was ist Geometrie?]
Vereinfacht ausgedr"uckt suchen wir eine M"oglichkeit um Distanzen und Winkel auszudr"ucken. Betrachte im Folgenden die Einheitssph"are, auf der wir den eine Reise von $x$ nach $y$ unternehmen m"ochten.
\begin{center}\begin{tikzpicture}
	%\draw[step=0.25,gray!15] (-4,-4) grid (4,4); \draw[step=0.5,gray!30] (-4,-4) grid (4,4); \fill (0,0) circle(0.1); %Hilfsgitter
	
	\draw (0,0) circle (3);
	\begin{scope}
		\clip (-3,0) rectangle (3,1.6);
		\draw[dashed] (0,0) ellipse (3 and 1);
	\end{scope}
	\begin{scope}
		\clip (-3,0) rectangle (3,-1.6);
		\draw (0,0) ellipse (3 and 1);
	\end{scope}
	
	\fill (0,3) circle(0.05) node[anchor=south west]{$N$};
	\fill (0,-3) circle(0.05) node[anchor=north west]{$S$};
	
	\coordinate (x) at (-0.5,2); \coordinate (y) at (0.75,-2);
	\fill (x) circle(0.05)node[above]{$x$};
	\fill (y) circle(0.05)node[below]{$y$};
	\draw (x) -- (y);
	\coordinate (a) at (1.25,0.75); \coordinate (b) at (1,0.25); \coordinate (c) at (1.5,-0.25);
	\draw (x) ..controls(x) and ($(a) + (-1.25,1.25)$).. (a) ..controls($(a) + (0.25,-0.25)$) and ($(b) + (0,0.25)$).. (b) ..controls($(b) + (0,-0.25)$) and ($(c) + (0,0.25)$).. (c) ..controls($(c) + (0,-0.75)$) and (y).. (y);
	
	\node at (2.75,2.75) {$S^2 \subset \R^3$};
	\draw[->] (1.5,-2)node[anchor=north]{$c$} to[out=90,in=310] (1.25,-1.25);
	
	\node at (-3.25,-2.5) {$c: [0,1] \to S^2$};
	\node at (-3.25,-3.25) {$c(0) = x,\quad c(1) = y$};
\end{tikzpicture}\end{center}
Wir definieren mit $\calL(c) = \int_0^1 \|\dot c\| \dop t$ die \CmMark[Metrik!Riemann-]{Riemann-Metrik}, also das Skalarprodukt mit allen $\T_pM$. Damit folgt dass wenn $c: [0,1] \to M$ glatt ist, dass $\calL(c) = \int_0^1 \sqrt{\langle \dot c, \dot c \rangle}$ und der Abstand auf $M$ kann ausgedr"uckt werden durch $d_M(x,y) = \inf \{\calL(c) | c$ von $x$ nach $y\}$.

Das wirft Fragen auf nach der Existenz k"urzester Abst"ande, Unterschieden zwischen lokal K"urzestem und global K"urzesten und der Eindeutigkeit.
\end{emptythm}

\begin{Dfn}
Es sei $M$ eine glatte Mannigfaltigkeit. Eine \CmMark[Metrik!Riemann-]{Riemannsche Metrik} $g$ auf $M$ ist gegeben durch ein Skalarprodukt auf jedem $\T_pM$, welches glatt von $p$ abh"angt, das hei"st $g \in \calT_2^0(M)$, so dass $g_p = \langle \cdot, \cdot \rangle_p : \T_pM \X \T_pM \ \to \R$ symmetrisch und positiv definit ist. Ist $g$ eine Riemann-Metrik auf M, so hei"st $(M,g)$ eine \CmMark[Mannigfaltigkeit!Riemannsche]{Riemannsche Mannigfaltigkeit}.
\end{Dfn}

Ist $(M,g)$ eine Riemannsche Mannigfaltigkeit, $X, Y \in \calV(M)$, $X = \sum \xi^{i} \pdifffrac{}{x^{i}}$, $Y = \sum \eta^j \pdifffrac{}{y^j}$, dann ist
\begin{align*}
	g(X, Y) &= g\left(\sum \xi^{i} \pdifffrac{}{x^{i}}, \sum \eta^j \pdifffrac{}{y^j}\right)\\
	&= \sum_{i,j}\xi^{i} \eta^j g\left(\pdifffrac{}{x^{i}}, \pdifffrac{}{y^j}\right)\\
	&= \sum_{i,j} \xi^{i} \eta^j g_{ij} \qquad (g_{ij} \text{ glatt, } g_{ij} = g_{ji})
\end{align*}

\begin{bsp}\begin{enumerate}[label=\arabic*),leftmargin=*]
\item
	$\R^m$ tr"agt eine nat"urliche Riemannsche Metrik: F"ur $x \in \R^m$ ist $\calI_x: \T_x\R^m \to \R^n$ ein (nat"urlicher) Isomorphismus. Damit definiert
		\[ g_x(\cdot,\cdot) = \langle \calI_x(\cdot,\cdot), \calI_x(\cdot,\cdot) \rangle \]
	eine Riemannsche Metrik auf $\R^m$. Bez"uglich der Karte $(\Id, \R^m)$ gilt
		\[ g_{ij} = \sum_{ij} \delta_{ij} \dop x^{i} \otimes \dop x^j = \sum_i \dop x^{i} \otimes \dop x^j \]
\item
	Betrachtet man Polarkoordinaten auf $\R^2(r, \theta)$:
	\begin{align*}
		\pdifffrac[(r,\theta)]{}{r} &= (\cos \theta, \sin \theta)\\
		\pdifffrac[(r,\theta)]{}{\theta} &= r(-\sin \theta, \cos \theta)
	\end{align*}
	\begin{align*}
		g_{rr} &= g\left( \pdifffrac{}{r}, \pdifffrac{}{r} \right) = 1\\
		g_{\theta\theta} &= g\left( \pdifffrac{}{\theta}, \pdifffrac{}{\theta} \right) = r^2\\
		g_{r\theta} &= g_{\theta r} = 0
	\end{align*}
\item
	Sei $M \subseteq \R^n$ $m$-dimensionale glatte Untermannigfaltigkeit. $M$ tr"agt eine nat"urliche Riemann-Metrik:
	\begin{center}\begin{tikzpicture}
		%\draw[step=0.25,gray!15] (-4,-4) grid (4,4); \draw[step=0.5,gray!30] (-4,-4) grid (4,4); \fill (0,0) circle(0.1); %Hilfsgitter
		
		\draw (0,0) circle(2);
		\begin{scope}
			\clip (-2,0) rectangle (2,2);
			\draw[dashed] (0,0) ellipse(2 and 0.75);
		\end{scope}
		\begin{scope}
			\clip (-2,0) rectangle (2,-2);
			\draw (0,0) ellipse(2 and 0.75);
		\end{scope}
		
		\path[name path=laenge] (0,2) to[out=335,in=33] (0,-2);
		\path[name path=breite] (0,2) ellipse(2 and 0.75);
		\path[name path=aequator] (0,0) ellipse(2 and 0.75);
		\path[name intersections={of=laenge and breite, by=p}];
		\path[name intersections={of=laenge and aequator, by={grenze1, grenze2}}];
		
		\coordinate (a) at (1,2.25); \coordinate (b) at (-1,0.5); \coordinate (c) at (1.75,-1.5); \coordinate (d) at ($(a) + (c) - (b)$);
		\path[draw,name path=raute] (a) -- (b) -- (c) -- (d) -- cycle;
		
		\begin{scope}
			\clip (0,2) rectangle ($(grenze2) + (0.5,0)$);
			\draw (0,2) to[out=335,in=33] (0,-2);
			\clip (0,0) circle(2);
			\draw (0,2) ellipse(2 and 0.75);
		\end{scope}
		\begin{scope}
			\clip (0,0) circle(2);
			\draw (0,2) ellipse(2 and 0.75);
		\end{scope}
		
		\node at (-2,2) {$S^2 \subset \R^3$};
		\fill (p) circle(0.05)node[anchor=south west,font=\scriptsize] at (p) {$p$};
		\draw[->] (p) -- ($(p) + 0.6*(1,-1.7)$);
		\draw[->] (p) -- ($(p) + 1.1*(1,0.15)$);
		
		\draw[->] (2.25,2.25)node[right,font=\scriptsize]{$\pdifffrac{}{\phi}$} to[out=210,in=80] (1.65,1.5);
		\draw[->] (3.25,1.25)node[right,font=\scriptsize]{$\pdifffrac{}{\theta}$} to[out=180,in=30] (1.2, 0.75);
	\end{tikzpicture}\end{center}
	F"ur jedes $p \in M$ ist $\T_pM$ kanonisch isomorph zum von partiellen Ableitungen $\partial_1F|_p, \ldots ,\partial_mF|_p$ einer lokalen Parametrisierung $F$ aufgespannten Untervektorraum $\R^m$. Mit diesem (lokalen) isomorphismus definiert
		\[ g_{ij} = \langle \partial_i F, \partial_j F \rangle \]
	eine Riemann-Metrik auf $M$.
\end{enumerate}\end{bsp}

\begin{bem}
Sind $\phi$ und $\psi$ Karten einer Riemannschen Mannigfaltigkeit $(M,g)$ um $p$ und sind $g = \sum g_{ij} \dop x^{i} \otimes \dop x^j$ und $h = \sum h_{ij} \dop y^{i} \otimes \dop y^j$ die lokalen Darstellungen bez"uglich $\phi$ beziehungsweise $\psi$, so gilt
	\[ h_{kl} = g\left( \pdifffrac{}{y^k}, \pdifffrac{}{y^l} \right) = \sum_{i,j} \pdifffrac{x^{i}}{y^k} \underbrace{\pdifffrac{x^j}{y^l}}_{\mathclap{\qquad \partial_l(\phi^{i}\circ \psi^{-1})}} g_{ij} \]
\end{bem}

Eine Riemannsche Metrik induziert eine Metrik auf dem Kotangentialb"undel: Die Isomorphismen $\T_pM \to \T_p^*M$, $X \mapsto \langle X, \cdot \rangle_p$ einen Isomorphismus von $\T M$ nach $\T^*M$. F"ur $\omega \in \T_p^*M$ sei $X(\omega) \in \T_pM$ mit $\omega = \langle X(\omega), \cdot \rangle_p$. Man definiert nun durch
	\[ \langle \omega, \tilde \omega \rangle = \langle X(\omega), X(\tilde \omega) \rangle \]
ein Skalarprodukt auf $\T_p^*M$. F"ur $\omega = \sum \omega_i \dop x^{i}$, $X(\omega) = \xi^{i} \pdifffrac{}{x^{i}}$ gilt
	\[ \omega_i = \omega \left( \pdifffrac{}{x^{i}} \right) = \left\langle X(\omega), \pdifffrac{}{x^{i}} \right\rangle = \sum_j \xi^{i} g_{ij} \]
Also $\xi^{i} = \sum g^{ij} \omega_i$, wobei $(g^{ij})$ die zu $(g_{ij})$ inverse Matrix ist. Damit gilt:
\begin{align*}
	\langle \omega, \tilde \omega \rangle &= \langle X(\omega), X(\tilde \omega) \rangle \\
	&= \sum g_{kl} \xi^k \xi^l\\
	&= \sum g_{kl} g^{ki} \omega_i g^{lj} \tilde \omega_j\\
	&= \sum \delta_l^i g^{lj} \omega_i \tilde \omega_j\\
	&= \sum g^{ij} \omega_i \tilde \omega_j\\
\end{align*}


%%%
%%% 13. Vorlesung <2012-11-27 Tue>
%%%

F"ur beliebige Tensoren $S, S' \in T_q^p(\T M)$ und $T, T' \in T_l^k(\T M)$ definiert man induktiv durch lineare Fortsetzung Skalarprodukte wie folgt:
\begin{align*}
  \left<S \otimes T, S' \otimes T'\right> = \left<S,S'\right> \otimes \left<T,T'\right>.
\end{align*}
Auf $\T M \otimes \T M$ hat die Metrik die folgende Gestalt:
\begin{align*}
  \left<X \otimes Y,\tilde X \otimes \tilde Y\right> = \sum g_{ij}g_{kl}\xi^i\tilde\xi^j\eta^k\tilde\eta^l.
\end{align*}

% Definition 6.2
\begin{Dfn}
  Es seien $(M, g)$ und $(N,h)$ Riemannsche Mannigfaltigkeiten.
Ein Diffeomorphismus $\Phi \colon M \to N$ hei"st \CmMark{Isometrie}, falls $\Phi^{*}h = g$, das hei"st f"ur alle $p \in M$ und $X,Y \in \T_pM$ gilt:
\begin{align*}
  g_p(X,Y) = \underbrace{h_{\Phi(p)}(\Phi_{*p}X,\Phi_{*p}Y)}_{\mathclap{= \Phi^{*}h(X,Y) \; \rightsquigarrow \text{Pullback Metrik}}}
\end{align*}
Ist umgekehrt $\Phi \colon M \to N$ ein Diffeomorphismus und $h$ eine Riemannsche Metrik auf $N$, so ist $\Phi^{*}h$ eine Riemannsche Metrik auf $M$.
\end{Dfn}

% Satz 6.3
\begin{Satz}\label{Satz-6-3}
  Jede glatte Mannigfaltigkeit tr"agt eine Riemannsche Metrik.
\end{Satz}

Um Metriken in den "Uberlappungsgebieten von Karten "`verkleben"' zu k"onnen, ben"otigt man das folgende Hilfsmittel.

% Hilfssatz
\begin{satz}[Zerlegung der Eins]
  Es sei $M$ eine glatte Mannigfaltigkeit und $\{U_i\}_{i \in \calI}$ eine offene "Uberdeckung von $M$.
  Dann existiert eine \CmMark{Zerlegung der Eins} auf einer abz"ahlbaren, lokal endlichen Verfeinerung von $\{U_i\}_{i \in \calI}$, das hei"st es existiert eine abz"ahlbare offene "Uberdeckung $\{V_k\}_{k\in\N}$ von $M$ und glatte Funktionen mit kompaktem Tr"ager $\alpha_k \colon M \to \R$, so dass gilt:

  \begin{enumerate}[label=(\roman*)]
  \item $\forall k \in \N \ \exists i(k) \in \calI: V_k \subseteq U_{i(k)}$ (Verfeinerung),
  \item $\forall p \in M \ \exists U \ni p: \# \{k \mid V_k \cap U \neq \emptyset \} < \infty$ (lokal endlich),
  \item $\forall k \in \N: \supp (\alpha_k) \subseteq V_k$,
  \item $\forall k \in \N \ \forall p \in M: 0 \leq \alpha(p) \leq 1$,
  \item $\forall p \in M: \sum_{k\in\N}\alpha_k(p) = 1$.
  \end{enumerate}
  (Wegen (ii) und (iii) ist die Summe in (v) endlich).
\end{satz}
An dieser Stelle geht ma"sgeblich ein, dass die Topologie von $M$ eine abz"ahlbare Basis besitzt. Beweis siehe Boothby, Kapitel V.4 \cite{boothby1986introduction}.

\begin{bew}(von Satz \ref{Satz-6-3})
Es sei $M$ eine glatte, $m$-dimensionale Mannigfaltigkeit. $\{(\phi_i,U_i)\}_{i \in \calI}$ ein Atlas von $M$ und $\{(V_k,\alpha_k)\}_{k \in \N}$ eine Zerlegung der Eins auf einer abz"ahlbaren, lokal endlichen Verfeinerung von $\{U_i\}_{i \in \calI}$. Es sei $\beta$ ein Skalarprodukt auf $\R^m$. F"ur jedes $k \in \N$ ist dann
\begin{align*}
	g_k = \left.\phi_{i(k)}\right|_{V_k}^{*}\beta
\end{align*}
eine Riemannsche Metrik auf $V_k$. Damit ist $g = \sum g_k\alpha_k$ eine Riemannsche Metrik auf $M$.
Die Summe ist punktweise endlich und $g$ ist als Komposition glatter Abbildungen selbst glatt.
Symmetrie und Bilinearit"at folgen sofort.
F"ur jedes $p \in M$ gilt $\sum_{k \in \N}\alpha_k(p) = 1$, das hei"st es existiert ein $l \in \N$ mit $\alpha_l(p) > 0$ und f"ur $X \in \T_pM$ mit $X \neq 0$ folgt:
\begin{align*}
	g_p(X,X) & = \sum \underbrace{g_k(p)(X,X)}_{> 0}\alpha_k(p)\\
	& \geq g_l(p)(X,X)\alpha_l(p) > 0.
\end{align*}
Damit ist $g$ positiv definit.
\end{bew}

F"ur eine glatte Kurve $\gamma \colon [a,b] \to M$ hei"st
\begin{align*}
  \mathcal L(\gamma) = \int_{a}^b\|\dot \gamma\| = \int_a^b \sqrt{g_{\gamma(t)}(\dot\gamma(t),\dot\gamma(t))}\dop t
\end{align*}
die \CmMark[L\"ange!Kurven-]{(Kurven-)L"ange} von $\gamma$. Ist $\tau \colon [\alpha,\beta] \to [a,b]$ glatt und monoton, so gilt\marginnote{\textcolor{gray}{\scriptsize{$\tau'$ ist die Ableitung von $\tau$, der Strich wurde aus "asthetischen Gr"unden statt dem Punkt gew"ahlt}}}
\begin{align*}
  \mathcal L(\gamma \circ \tau) & = \int_{\alpha}^{\beta}\|\dot\gamma(\tau(s))\||\tau'(s)|\dop s\\
  & = \int_a^b\|\dot\gamma\| = \mathcal L(\gamma).
\end{align*}
Damit ist die Kurvenl"ange invariant unter Reparametrisierungen. Ist $\gamma$ \CmMark[regul\"ar!Kurve]{regul\"ar}, das hei"st $\dot\gamma(t) \neq 0$ f"ur alle $t \in [a,b]$, so ist ihre sogenannte \CmMark{Bogenl"ange}
\begin{align*}
  \sigma \colon [a,b] \to [0,\mathcal L(\gamma)], t \mapsto \mathcal L(\gamma|_{[a,t]}) = \int_a^t\|\dot\gamma\|.
\end{align*}
streng monoton steigend, also $\sigma'(s) = \|\dot\gamma(s)\| > 0$.
F"ur $\tilde\gamma = \gamma \circ \sigma^{-1}\colon [0,\mathcal L(\gamma)] \to M$ gilt $\|\dot{\tilde\gamma}\| \equiv 1$.
Die Kurve $\tilde \gamma$ hei"st \CmMark{Bogenl"angenparametrisierung} von $\gamma$.
Gilt f"ur $\gamma \colon [a,b] \to M$ dass $\|\dot\gamma\| \equiv \lambda$, so hei"st $\gamma$ \CmMark[Parametrisierung!proportional zur Bogenl"ange]{proportianal zur Bogenl"ange} parametrisiert.

\marginnote{\begin{tikzpicture}[font=\scriptsize,scale=0.6]
	%\draw[step=0.25,gray!15] (-4,-4) grid (4,4); \draw[step=0.5,gray!30] (-4,-4) grid (4,4); \fill (0,0) circle(0.1); %Hilfsgitter
	\fill (0,0) circle(0.05)node[left]{$b$} (-3,-2) circle(0.05)node[left]{$a$} (1,-2) circle(0.05)node[right]{$c$};	
	\draw (-3,-2) ..controls(-3,-2) and (-2.5,-1).. (-1.5,-1)node[above]{$\gamma$} ..controls(-0.5,-1) and (0,0).. (0,0);
	\draw (0,0) ..controls(0,0) and (1,-0.5).. (0.75,-1)node[left]{$\tilde\gamma$} ..controls(0.5,-1.5) and (1,-2).. (1,-2);
	\draw[->] (0,0) -- (0.4,0.75)node[left]{$\dot\gamma(b)$};
	\draw[->] (0,0) -- (1,-0.5)node[above]{$\dot{\tilde\gamma}(b)$};
\end{tikzpicture}}
Sind $\gamma \colon [a,b] \to M, \tilde \gamma \colon [b,c] \to M$ glatte Kurven mit $\gamma(b) = \tilde \gamma(b)$, so sei
\begin{align*}
  \mathcal L(\gamma \cup \tilde\gamma) = \mathcal L(\gamma) + \mathcal L(\tilde \gamma).
\end{align*}
Eine Kurve $\gamma \colon [a,b] \to M$ hei"st \CmMark{st"uckweise glatt}, wenn $t_0, \ldots, t_k$ mit $a = t_0 < t_1 < \cdots < t_k = b$ existieren, so dass $\gamma|_{[t_{i-1},t_i]}$ f"ur alle $i \leq k$ glatt ist.

% Definition 6.4
\begin{Dfn}
  F"ur Punkte $p, q \in M$ ist der \CmMark{Abstand} definiert durch:
  \begin{align*}
    \dop(p,q) = \inf\{ \mathcal L(\gamma) \mid \gamma \colon [0,1] \to M \text{ st"uckweise glatt mit } \gamma(0) = p, \gamma(1) = q\}.
  \end{align*}
\end{Dfn}

% Satz 6.5
\begin{Satz}
  Es sei $(M,g)$ eine zusammenh"angende Riemannsche Mannigfaltigkeit.
  Die Abstandsfunktionen bilden eine Metrik auf $M$, welche die urspr"ungliche Topologie induziert.
\end{Satz}

Der Beweis sei zur "Ubung "uberlassen.

% Satz 6.6
\begin{Satz}
  Es seien $(M,g)$ und $(N,h)$ zusammenh"angende Riemannsche Mannigfaltigkeiten und $\Phi \colon M \to N$ ein Diffeomorphismus.
  Dann ist $\Phi$ genau dann eine Isometrie, wenn $\mathcal L(\Phi \circ \gamma) = \mathcal L(\gamma)$ f"ur alle glatten $\gamma \colon [0,1] \to M$ gilt. 
\end{Satz}

\begin{bew}
  Dass eine Isometrie die Kurvenl"angen erh"alt gilt offensichtlich. Erh"alt $\Phi$ die Kurvenl"angen, so erh"alt $\Phi$ auch die Norm von Tangentialvektoren, den andernfalls g"abe es $X_p \in \T_pM$ mit (ohne Einschr"ankung)
  \begin{align*}
    h_{\Phi(p)}(\Phi_{*p}X,\Phi_{*p}X) > g_p(X,X)
  \end{align*}
  und eine Kurve $\gamma\colon [0,1] \to M$ mit $\gamma(0) = X$ und es g"alte (f"ur hinreichend kleines $\epsilon$):
  \begin{align*}
    \mathcal L(\gamma|_{[0,\epsilon]}) & = \int_0^{\epsilon}\sqrt{g_{\gamma(t)}\left(\dot\gamma(t),\dot\gamma(t)\right)}\dop t\\
    & < \int_0^{\epsilon}\sqrt{h_{\Phi(\gamma(t))} \left(\Phi_{*\gamma(t)}\dot\gamma(t), \Phi_{*\gamma(t)}\dot\gamma(t)\right)}\dop t\\
    & = \int_0^{\epsilon}\sqrt{h_{\Phi(\gamma(t))} \left(\dot{(\Phi \circ \gamma)}(t), \dot{(\Phi \circ \gamma)}(t) \right)}\dop t\\
    & = \mathcal L \left((\Phi \circ \gamma)|_{[0,\epsilon]}\right).
  \end{align*}
  Mit der Polarisationsformel $\left<x,y\right> = - \frac{1}2 (\|x-y\|^{2} - \|x\|^2-\|y\|^2)$ folgt dann, dass $\Phi$ auch die Skalarprodukte erh"alt.
\end{bew}

% Definition 6.7
\begin{Dfn}
  Eine Kurve $\gamma \colon [a,b] \to M$ hei"st \CmMark[Geod\"atische!minimale]{minimale Geod"atische} von $\gamma(\alpha)$ nach $\gamma(\beta)$, falls ein $\lambda \geq 0$ existiert, so dass f"ur alle $a \leq s < t \leq b$ gilt:
  \begin{align*}
    \mathcal L(\gamma|_{[s,t]}) = \lambda(t-s) = \dop(\gamma(s),\gamma(t)).
  \end{align*}

  Eine Kurve $\gamma$ hei"st \CmMark{Geod\"atische}, falls sie lokal minimierende Geod"atische ist, das hei"st f"ur alle $t \in [a,b]$ existiert ein $\epsilon > 0$, so dass $\gamma|_{[t-\epsilon,t+\epsilon]}$ minimierende Geod"atische ist.
\end{Dfn}

Eine bessere Vorstellung erh"alt man durch Betrachtung von Geod"atischen als Isometrien von Intervallen in den euklidischen Raum, denn $\dop(\gamma(s),\gamma(t)) = t-s = \dop_{\R}(t,s)$.

\begin{center}\begin{tikzpicture}
    % \draw[step=0.25,gray!15] (-4,-4) grid (4,4); \draw[step=0.5,gray!30] (-4,-4) grid (4,4); \fill (0,0) circle(0.1); %Hilfsgitter
    \draw (0,0) circle (2);
    \fill (0,2) circle(0.05)node[above]{$N$} (0,-2) circle(0.05)node[below]{$S$};
    \def\vert{0.5}
    \begin{scope}
      \clip (-2,0) rectangle (2,-2);
      \draw (0,0) ellipse(2 and \vert);
    \end{scope}\begin{scope}
      \clip (-2,0) rectangle (2,2);
      \draw[dashed] (0,0) ellipse(2 and \vert);
    \end{scope}\begin{scope}
      \clip (0,-2) rectangle (2,2);
      \draw (0,0) ellipse(1 and 2);
    \end{scope}\begin{scope}
      \clip (0,-2) rectangle (-2,2);
      \draw[dashed] (0,0) ellipse(1 and 2);
    \end{scope}
    
    \fill (-1.25,-0.75) circle (0.05) (-0.25,1) circle (0.05);
    \draw (-1.25,-0.75) ..controls(-1.25,-0.25) and (-1,0.5).. (-0.25,1);
    
    \node at (2,2) {$S^2 \subset \R^3$};
  \end{tikzpicture}\\
  Geod"atische = Gro"skreissegmente\end{center}


%%% 
%%% 14. Vorlesung <2012-11-30 Fri>
%%% 

\begin{bem}\begin{enumerate}[label=\arabic*),leftmargin=*]
\item
	Die Geod"atischen von $\R^n$ mit Standardmetrik sind genau die Geradensegmente.
\item
	Ist $\gamma$ eine minimale Geod"atische, so gilt $\|\dot\gamma\| = \lambda$, falls $\lambda > 0$, existiert eine Bogenl"angenparametrisierung $\tilde\gamma$ von $\gamma$ auf $[0,l]$ mit $l = \mathcal L(\gamma) = \mathcal L(\tilde\gamma)$ und $\dop(\tilde\gamma(0),\tilde\gamma(t)) = t$.
	Damit ist $\tilde\gamma$ eine isometrische Einbettung von $[0,l]$ in $M$.
\end{enumerate}\end{bem}

% Definition 6.8
\begin{Dfn}
  Es sei $\gamma \colon [a,b] \to M$ eine (st"uckweise) glatte Kurve auf $M$.
  Das Integral
  \begin{align*}
    E(\gamma) = \frac{1}2 \int_a^b\|\dot\gamma\|^2
  \end{align*}
  hei"st \CmMark{Energie} von $\gamma$.
\end{Dfn}

% Lemma 6.9
\begin{Lemma}\label{lemma-6-9}
  F"ur eine (st"uckweise) glatte Kurve $\gamma \colon [0,1] \to M$.
  Dann gilt:
  \begin{align*}
    \frac{1}2 \mathcal L(\gamma)^2 \leq E(\gamma),
  \end{align*}
  wobei Gleichheit genau dann gilt, wenn $\gamma$ proportional zur Bogenl"ange parametrisiert ist.
\end{Lemma}

\begin{bew}
  Es gilt die Cauchy-Schwarzsche Ungleichung f"ur das Skalarprodukt $(f,g) \mapsto \int_0^1 fg$ mit $f,g \in \C^{\infty}([0,1], \R)$.

  Nun sei $f \equiv 1$ und $g = \|\dot\gamma\|$, so folgt:
  \begin{align*}
    \mathcal L(\gamma) = \int_0^1 \|\dot\gamma\| \leq \left|\left(\int_0^1 f^2\right)^{\frac{1}2} \left(\int_0^1g^2\right)^{\frac{1}2}\right| = \sqrt{2 E(\gamma)}
  \end{align*}
  Gleichheit gilt genau dann, wenn $1$ und $\|\dot\gamma\|$ $\R$-linear abh"angig sind, das hei"st wenn $\|\dot\gamma\| \equiv \lambda$ gilt.
\end{bew}

% Satz 6.10
\begin{Satz}
  Eine (st"uckweise) glatte Kurve ist genau dann minimale Geod"atische, wenn ihre Energie minimal ist.
\end{Satz}

\begin{bew}\begin{description}[font=\normalfont]
\item[\quot{$\Rightarrow$}:]
	Es sei $\gamma$ minimale Geod"atische, das hei"st $\mathcal L(\gamma|_{[0,t]}) = \lambda t = \dop (\gamma(0),\gamma(t))$.
	Also gilt $E(\gamma) = \frac{1}2 \mathcal L(\gamma)^2 \leq \frac{1}2 \mathcal L(c)^2 \leq E(c)$, wobei $c$ eine Kurve zwischen den Endpunkten von $\gamma$ ist und die letzte Ungleichung aus Lemma \ref{lemma-6-9} folgt.
\item[\quot{$\Leftarrow$}:]
	Sei $\gamma$ energieminimierend.
	\begin{align*}
		\frac{1}2 \dop (\gamma(0), \gamma(1))^2 \leq \frac{1}2 \calL(\gamma)^2 \leq E(\gamma) \leq E(\underbrace{c_n}_{\mathclap{\text{regul"are Kurven}}}) = \frac{1}2 \mathcal L(c_n)^2 \xrightarrow{n\to\infty} \frac{1}2 \dop(\gamma(0),\gamma(1))^2
	\end{align*}
	\begin{center}\begin{tikzpicture}
		%\draw[step=0.25,gray!15] (-6,-4) grid (6,4); \draw[step=0.5,gray!30] (-6,-4) grid (6,4); \fill (0,0) circle(0.1); %Hilfsgitter,
		
		\coordinate (1) at (-0.5,0.5); \coordinate (2) at (0,1.55); \coordinate (3) at (0.5,0.5);
		\coordinate (ctrl1) at (-1,-1.5); \coordinate (ctrl3) at (1,-1.5);
		\draw (-6,-1.5) ..controls(-6,-1.5) and ($(1) + (ctrl1)$).. (1) ..controls($(1) - 1/3*(ctrl1)$) and (2).. (2) ..controls(2) and ($(3) - 1/3*(ctrl3)$).. (3) ..controls($(3) + (ctrl3)$) and (6,-1.5).. (6,-1.5);
		
		\coordinate (4) at (0,0.75);
		\draw[very thick] (1) ..controls($(1) - 1/3*(ctrl1)$) and (2).. (2) ..controls(2) and ($(3) - 1/3*(ctrl3)$).. (3) ..controls($(3) - 1/15*(ctrl3)$) and ($(4) + (0.25,0)$).. (4) ..controls($(4)-(0.25,0)$) and ($(1) - 1/15*(ctrl1)$).. (1) -- cycle;
		
		\draw[dashed] (0,1.25) circle(1.25); \node[font=\scriptsize] at (0,2) {$c(t_i)$}; \node at (3,2) {$U$ Kartenumgebung};
		\node[font=\scriptsize] at (0,-1.25) {Die L"ange des fetten Dreiecks wird beliebig klein};
	\end{tikzpicture}\end{center}

	Damit gilt: $\mathcal L(\gamma) = \dop(\gamma(0),\gamma(1))$ und wegen Cauchy-Schwarz ist $\gamma$ proportional zur Bogenl"ange parametrisiert.
	Wendet man dieses Argument auf beliebige Teilst"ucke an, erh"alt man:
	\begin{align*}
		\mathcal L(\gamma|_{[s,t]}) = \dop(\gamma(s),\gamma(t)) = \lambda(s-t).
	\end{align*}
\end{description}\end{bew}

%%% Local Variables: 
%%% mode: latex
%%% TeX-master: "../skript-diffgeom"
%%% End: 

%% 
%% 14. Vorlesung <2012-11-30 Fri>, Fortsetzung
%% 

\chapter{Kovariante Ableitungen}

\paragraph{Frage:} Was ist eine \quot{gute} Differentialrechnung für Vektorfelder?

Das gewöhnliche Differential im $\R^n$ für $Y \colon \R^n \to \R^n$ ist gerade die lineare Abbildung $\D Y|_p \cdot v = \lim \frac{1}t \left(Y(p+tv) -Y(p)\right) = \difffrac[t=0]{}{t} Y(p+tr)$.
Betrachte im euklidischen Fall einen Punkt $p$, sowie einen Tangentialvektor $Y_p$.
\begin{center}\begin{tikzpicture}[font=\scriptsize]
	\coordinate (end) at (1.5,0.75); %Endrichtung
	\draw (0,0) -- ($(0,0) + 4*(end)$);
	% die Punkte
	\coordinate (p) at ($(0,0) + 0.5*(end)$); \coordinate (q) at ($(0,0) + 2.75*(end)$);
	\fill (p) circle(0.1)node[anchor=north west]{$p$}; \fill (q) circle(0.1)node[anchor=north west]{$p + tv$};
	% die Pfeile
	\draw[->] (p) --node[left]{$Y_p$} ($(p) + 1.25*(-0.75,2)$);
	\coordinate (dir) at (0.15,1);
	\def\scl{1.5}
	\draw[->] (q) --node[right]{$Y_{p+tv}$} ($(q) + \scl*(dir)$); \draw[->] (p) -- ($(p) + \scl*(dir)$);
	% Parallelverschiebung
	\draw[->,dashed] ($(q) + 0.5*\scl*(dir)$) --node[above,sloped]{Parallelverschiebung} ($(p) + 0.5*\scl*(dir)$);
\end{tikzpicture}\\
\textcolor{red}{evtl. auch noch die Idee der Parallelverschiebung erklären.}\end{center}

Nun gehe zur Betrachtung von Vektorfeldern $X \colon \R^n \to \R^n$ über und setze $\D_XY|_p = \D Y|_p\cdot X_p$. Hierfür gilt:
\begin{itemize}
\item $\D$ ist $\R$-linear in $Y$: $\D(Y + \tilde Y) = \D Y + \D \tilde Y$.
\item Es gilt die Leibnizregel: $\D(fY) = \D f \cdot Y + f\D Y$.
\item $\D$ ist $C^{\infty}(\R^n)$-linear in $X$:
  \begin{align*}
    \D_{fX}Y|_p = \D Y|_p\cdot(fX)_p = \D Y|_p \cdot f(p)X_p = f(p) \D Y|_p \cdot X_p = (f \D_XY)(p).
  \end{align*}
\end{itemize}

\emph{Erinnerung:} Die Lieableitung $\mathcal L_{(\cdot)}Y$ ist \emph{nicht} $C^{\infty}$-linear.

% Definition 7.1
\begin{Dfn}
  Es seien $M$ eine glatte Mannigfaltigkeit und $E$ ein Vektorbündel über $M$.
  Eine \CmMark{kovariante Ableitung} (oder \CmMark{Zusammenhang} ([engl.] \quot{connection}) auf $E$ ist eine Abbildung
  \begin{align*}
    \nabla \colon \mathcal V(M) \times \Gamma(E) \to \Gamma(E), \quad \nabla(X,S) = \nabla_XS
  \end{align*}
  mit den folgenden Eigenschaften:
  \begin{enumerate}[label=(\roman*),widest=iii]
  \item $\nabla S$ ist $C^{\infty}(M)$-linear, das hei"st
    \begin{align*}
      \nabla_{X+Y}S = \nabla_XS+\nabla_YS \text{ und } \nabla_{fX}S = f\nabla_XS
    \end{align*}
    f"ur alle $X, Y \in \calV(M)$ und $f \in C^{\infty}(M)$.
  \item $\nabla_X$ ist $\R$-linear, das hei"st
    \begin{align*}
      \nabla_X(\mu S + \nu T) = \mu\nabla_XS + \nu\nabla_XT.
    \end{align*}
  \item $\nabla_X$ erfüllt die folgende Leibnizregel:
    \begin{align*}
      \nabla_X(fS) = \dop f(X) \cdot S + f\cdot \nabla_XS = X(f)\cdot S + f \cdot \nabla_XS.
    \end{align*}
  \end{enumerate}
  Kurzform: $\nabla \colon \Gamma(E) \to \Gamma(\T^{*}M \otimes E), S \mapsto \nabla_{(.)}S$ ist eine $C^{\infty}(M)$-Modulderivation.
\end{Dfn}

\begin{bsp}\begin{enumerate}[label=\arabic*),leftmargin=*]
\item
	Das gewöhnliche Differential $\D$ definiert in kanonischer Weise eine kovariante Ableitung auf $\T\R^n$.
	\begin{align*}
		X \in \mathcal V(\R^n), X \colon \R^n \to \T\R^n \cong \R^n \times \R^n \text{ via } \calI\colon X_p \mapsto (p,\underbrace{\calI_p(X_p)}_{=:\overline X_p}).
	\end{align*}
	Nun ist wie folgt eine kovariante Ableitung gegeben: $(\nabla_XY)_p = \calI^{-1}(p,\D_{\overline X_p}\overline Y)$.
\item
	$E = M \times \R^n$, ein Schnitt $S$ von $E$ ist von der Form $S_p = (p,s(p))$, $s \colon M \to \R^n$ glatt.

	Hier definiert man die kovariante Ableitung:
	\begin{align*}
		& \nabla_XS = (p,\calI_{s(p)}^{-1}(s_{*p},X_p))\\
		&  s_{*p}\colon \T_{*p}M \to \T_{*p}\R^n, s_{*p}\colon X_p \in \T_{*p}\R^n \xrightarrow{\calI_{s(p)}} \R^n.
	\end{align*}
\item
	Sei $E = M \times \R^n$, ein Schnitt $S = (\Id, \sigma)$, $\sigma: M \to \R^n$. Dann ist $(\nabla_X S)_p = (p, \calI_p(\sigma_{*p}(X_p))$, $\sigma_{*p}: \T_pM \to \T_{\sigma(p)}\R^n$. Sei $ \omega = (\omega_j^k)_{j,k \le n}$ eine $(n\times n)$-Matrix von 1-Formen auf $M$, das hei"st $\omega(X)|_p \in \mathfrak M^{n\times n}(\R)$.
	Für einen Schnitt $S = (\Id, \sigma)$ und sei dann
		\[ (\nabla_XS)_p = (\Id, \calI_p(\sigma_{*p}(X_p)) + \omega(X)|_p \cdot \sigma(p). \]
	Dies definiert eine kovariante Ableitung auf $E = M \X \R^n$.
\item
	$\dop \colon \Omega^0(M) = C^{\infty}(M) = \Gamma(M\times \R) \to \Omega^1(M) = \Gamma(\T^{*}M) = \Gamma(\underbrace{\T^{*}M \otimes (M \times \R)}_{\mathclap{\text{Fasern: } \T_p^{*}M\otimes\R \cong \T_p^{*}M}})$ mit $f \mapsto [\dop f \colon X \mapsto \dop f(X) = X(f)]$.

	Dann ist
	\begin{align*}
		& \dop \colon \mathcal V(M) \times C^{\infty}(M) \to C^{\infty}(M),\\
		& \nabla_Xf = \dop (X,f) \mapsto X(f)
	\end{align*}
	eine kovariante Ableitung auf $C^{\infty}(M)$.
\item
	Es sei $M \subseteq \R^k$ eine glatte Untermannigfaltigkeit und $\nabla$ die kanonische kovariante Ableitung auf $\T \R^k$.

	Erster Ansatz für eine kovariante Ableitung:
	\begin{align*}
		\tilde \nabla_XY = \nabla_{\tilde X}\tilde Y|_M \text{ das funktioniert noch nicht.}
	\end{align*}
	Für $X,Y \in \mathcal V(M)$ seien $\tilde X, \tilde Y$ Fortsetzungen, das hei"st $\tilde X|_M = X$ und $\tilde Y|_M = Y$.
	\begin{align*}
		(\nabla_{\tilde X}\tilde Y)_p \in \T_p\R^k \supseteq \T_pM.
	\end{align*}

	Nächster Ansatz, der tasächlich eine kovariante Ableitung definiert.
	\begin{align*}
		\tilde \nabla_XY = (\nabla_{\tilde X}\tilde Y|_M)^{\text{proj}\T_pM},
	\end{align*}
	wobei $X^{\text{proj}\T_pM}$ die orthogonale Projektion von X auf den Tangentialraum $\T_pM$ bzgl. des Standardskalarproduktes ist.
\end{enumerate}\end{bsp}

Schreibt man in Beispiel 3) $\sigma = ( \sigma^1, \ldots ,\sigma^n)$, so kann man $\dop \sigma = (\dop \sigma^1,\ldots ,\dop\sigma^n)$ als 1-Form auf $M$ mit Werten in $\R^n$ auffassen:
	\begin{align*}
		\dop \sigma(X)_p &= (\dop \sigma^1(X)_p,\ldots , \dop \sigma^n(X)_p)\\
		&= (X(\sigma^1)_p,\ldots ,X(\sigma^n)_p)\\
		&= \calI_p(\sum X(\sigma^{i}) \partial_i),
	\end{align*}
	wobei $\partial_i$ das $i$-te Koordinatenfeld in der Karte $(\Id, \R^n)$ ist. Identifiziert man $E = M \X \R^n$ mit $C^{\infty}(M, \R^n)$, so gilt $\nabla_X S = \dop \sigma(X) \omega(X) \sigma$ (Kurzschreibweise f"ur die zweite Komponente von $S$). Lokal ist \emph{jede} kovariante Ableitung von dieser Form.

\begin{Lemma}
Die kovariante Ableitung $(\nabla_XS)_p$ h"angt nur von den Werten von $X$ und $S$ in einer Umgebung von $p$ ab.
\end{Lemma}

\begin{bew}
Es seien $p \in M$ und $X_1, X_2 \in \calV(M)$ sowie $S_1, S_2 \in \Gamma(E)$ und $U$ eine Umgebung von $p$ mit $X_1|_U = X_2|_U$ und $S_1|_U = S_2|_U$. W"ahle nun ein $\sigma \in C^{\infty}(M)$ mit dem Tr"ager $\supp \sigma \subseteq U$ und $\sigma|_V \equiv 1$ auf einer Umgebung $V$ von $p$. Dann gilt: $\sigma X_1 = \sigma X_2$ und $\sigma S_1 = \sigma S_2$. F"ur $q \in V$ folgt dann:
\begin{align*}
	(\nabla_{\sigma X_i} \sigma S_i)_q &= \sigma(q)(\nabla_{X_i} \sigma S_i)|_q\\
	&= \sigma(q)(\underbrace{X_i(\sigma)|_q}_{=0} S_i + \underbrace{\sigma(q)}_{=1}\nabla_{X_i} S_i|_q)\\
	&= \nabla_{X_i} S_i|_q
\end{align*}
Damit folgt $\nabla_{X_1} S_1 = \nabla_{X_2} S_2$
\end{bew}

\section{Lokale Koordinaten}

Es sei $(\varphi, U)$ eine Karte von $M$ um $p \in M$ und $E|_U \overset{\tau}{\to} U \X \R^n$. Dann ist $s_i(p) = \tau^{-1}(p, e_i)$ eine lokale Basis. Jeder Schnitt $S$ ist also lokal von der Form $S|_U = \sum_i \sigma^{i} s_i$. Somit existieren glatte Funktionen $\Gamma_{ij}^k$, die sogenannten \CmMark{Christoffelsymbole} mit 
	\[ \nabla_{\pdifffrac{}{x^{i}}} s^j = \sum_k \Gamma_{ij}^k s^k. \]
F"ur $S = \sum \sigma^j s_j$ und $X = \sum \xi^{i} \pdifffrac{}{x^{i}}$ folgt dann:
\begin{align*}
	(\nabla_XS)_p &= \sum_{i,j} \xi_p^{i} \nabla_{\pdifffrac{}{x^{i}}} \left(\sigma^j s_j\right)\\
	&= \sum_{i,j} \xi_p^{i} \left(\pdifffrac{\sigma^j}{x^{i}} \cdot s_j(p) \nabla_{\pdifffrac{}{x^{i}}} s_j|_p\right)\\
	&= \sum_{i,j} \xi_p^{i} \left(\pdifffrac[p]{\sigma^j}{x^{i}} s_j(p) + \sigma^j(p) \sum_k \Gamma_{ij}^k(p) s_k(p)\right)\\
	&= \sum_k \Bigg(\underbrace{\sum_i \xi_p^{i} \pdifffrac[p]{\sigma^k}{x^{i}}}_{\mathclap{= X(\sigma^k)|_p = \difffrac[t=0]{}{t} (\sigma^k \circ \gamma) \text{ mit } \dot\gamma(0) = X_p}} + \sum_{i,j} \xi_p^{i} \sigma^j(p) \Gamma_{ij}^k(p)\Bigg) s_k(p)
\end{align*}

\begin{bem}\begin{enumerate}[label=\arabic*),leftmargin=*]
\item
	$X \mapsto (\nabla_XS)_p$ h"angt nur von dem Wert $X_p$ von  $X$ in $p$ ab, Schreibweise $(\nabla_XS)_p = \nabla_{X_p}S$.
\item
	$S \mapsto \nabla_{X_p}S$ h"angt nur von den Werten von $S$ entlang einer Kurve $\gamma$ mit $\dot\gamma(0) = X_p$ ab. Es gilt
		\[ \nabla_XS = \sum_k X(\sigma^k)S_k + \sum_k \sum_j\left(\left(\sum_i \Gamma_{ij}^k \xi^{i}\right) \sigma^j\right) s_k. \]
	Schreibt man $\sigma = (\sigma^1,\ldots ,\sigma^n)$ und fasst $\dop \sigma = (\dop \sigma^1,\ldots ,\dop \sigma^n)$ also lokale 1-Form mit Werten in $\R^n$ auf, so ist f"ur $s=(s_1,\ldots ,s_n)$ $\dop\sigma \cdot s = \sum \dop \sigma^j s^j$ eine lokale 1-Form mit Werten in $E$. Es gilt: $\dop \sigma \cdot s(X) = \D_X \sigma \cdot s$. Analog definiert $\omega(X) = (\omega_j^k(X))_{k,j}$ eine lokale 1-Form mit Werten in den reellen $(n\X n)$-Matrizen. Dann ist 
	\[ \omega \sigma : X \mapsto \omega(X) \sigma = \left( \sum_{i,j} \Gamma_{ij}^k \xi^{i} \sigma^j \right)^k \]
eine lokale 1-Form mit Werten in $\R^n$ und $\omega\sigma \cdot s$ eine lokale 1-Form mit Werten in $E$. Damit gilt
	\[ \nabla_XS = (\dop \sigma(X) + \omega(X) \sigma) \cdot s \]
oder kurz
	\[ \nabla = \dop + \omega. \]
\end{enumerate}\end{bem}

\section{Transformationsverhalten}

Es seien $E|_{U_\alpha} \overset{\tau_\alpha}{\to} U_\alpha \X \R^n$ und $E|_{U_\beta} \overset{\tau_\beta}{\to} U_{\beta} \X \R^n$ lokale Trivialisierungen. Die "Ubergangsfunktion
	\[ \psi = \psi_{\beta\alpha}: U_\alpha \cap U_\beta \to \GL_n(\R) \]
war durch
	\[ \tau_\beta \circ \tau_\alpha^{-1} (p,x) = (p, \psi x) \]
definiert. F"ur die lokalen Darstellungen $S = \sum \sigma^j s_j = \sum \tilde\sigma^j s_j$ in $\tau_\alpha$ beziehungsweise $\tau_\beta$ gilt damit $\tilde\sigma^{i} = \sum_k \psi_k^{i} \sigma^k$, kurz $\tilde\sigma = \psi \sigma$. Es folgt daraus:
	\[ (\dop \sigma(X) + \omega(X) \sigma) S = \nabla_X S = (\dop \tilde\sigma(X) + \tilde\omega(X)\tilde\sigma) \tilde S \]
also
\begin{align*}
	\dop \sigma(X) + \omega(X) \sigma &= \psi^{-1}(\dop \tilde\sigma(X) + \tilde\omega(X) \tilde\sigma)\\
	&= \psi^{-1} (\dop(\psi\sigma)(X) + \tilde\omega(X) \psi \sigma)\\
	&= \psi^{-1} ((D_X f) \sigma + \psi \dop \sigma(X) + \tilde\omega(X) \psi \sigma)\\
	&= \dop \sigma(X) + (\underbrace{\psi^{-1}(D_X \psi) + \psi^{-1} \tilde\omega(X) \psi}_{=\omega(X)}) \sigma.
\end{align*}
Damit gilt
	\[ \tilde\omega(X) = \psi \omega(X) \psi^{-1} - D_X \psi \cdot \psi^{-1}. \qquad (*) \]
Daher definiert $\omega(X)$ \emph{keinen} Schnitt in $\Hom(E, E)$, denn in Kapitel \ref{kapitel-5} wurde gezeigt, dass die "Ubergangsfunktion von $\Hom(E, E)$ gegeben ist durch
	\[ (p, \eta) \to (p, \psi \circ \eta \circ \psi^{-1}). \]

\begin{bem}
Der zweite Summand in (*) h"angt \emph{nur} von der "Ubergangsfunktion $\psi$ und $X$ ab, und somit \emph{nicht} von $\nabla$. Das hei"st sind $\nabla$ und $\tilde\nabla$ kovariante Ableitungen auf $E$, so ist ihre Differenz $\nabla - \tilde\nabla$ eine globale 1-Form mit Werten in $\Hom(E,E)$.
\end{bem}

Durch eine kovariante Ableitung auf einem Vektorb"undel $E$ erhalten wir kovariante Ableitungen auf dem dualen Vektorb"undel $E^*$ und Tensorprodukten von Vektorb"undeln wie folgt:

\begin{Prop}
Die f"ur $X \in \calV(M)$, $S^* \in \Gamma(E^*)$ und $v \in E_p$ sowie eine Fortsetzung $\tilde v \in \Gamma(E)$ von $v_p$ durch
	\[ (\nabla_X^* S^*)_p(v) = X_p(S^*(\tilde v)) - S^*|_p (\nabla_X \tilde v) \]
definierte Abbildung ist eine kovariante Ableitung auf $E^*$. Dass $S^*(\tilde v) = (S^*, \tilde v)$ ist führt zu $X_p(S^*, \tilde v) = (\nabla_X^* S^*, \tilde v) + (S^*, \nabla_X \tilde v)$.
\end{Prop}
Der Beweis sei zur "Ubung "uberlassen.

\begin{Prop}
Es seien $E_1$ und $E_2$ Vektorb"undel mit kovarianten Ableitungen  $\nabla^1$ und $\nabla^2$ "uber $M$. Dann definiert f"ur $X \in \calV(M)$, $S_i \in \Gamma(E_i)$
	\[ \nabla_X (S_1 \otimes S_2) = \nabla_X^1 S_1 \otimes S_2 + \nabla_X^2 S_1 \otimes S_2 \]
durch lineare Fortsetzungen eine kovarainte Abbildung auf $E_1 \otimes E_2$.
\end{Prop}


%%% Local Variables: 
%%% mode: latex
%%% TeX-master: "../skript-diffgeom"
%%% End: 

%% 
%% Vorlesung <2012-12-14 Fri>, Fortsetzung
%%

\chapter{Geod\"atische und die Exponentialabbildung}

\begin{emptythm}[Heuristik:] Geodätische sind Minimalstellen des Energiefunktionals $\gamma \mapsto E(\gamma) = \int \|\dot\gamma\|^2$. 
Was sind kritische Punkte dieser Abbildung? Für $f \in C^{\infty}(M)$ ist $p$ kritischer Punkt, wenn alle Richtungsableitungen verschwinden, das hei"st $0 = X(f) = \difffrac[t=0]{}{t}(f(c(t)))$.
\end{emptythm}

\begin{center}\begin{tikzpicture}[font=\scriptsize]
%	\draw[step=0.25,gray!15] (-6,-4) grid (6,4); \draw[step=0.5,gray!30] (-6,-4) grid (6,4); \fill (0,0) circle(0.1); %Hilfsgitter
	
	\coordinate (p) at (-2,-1); \coordinate (q) at (2,1); \coordinate (wirbel) at (-0.5,-0.5);
	\coordinate (ctrl1) at (1,0);
	\def\left{0.75}
	\def\right{1.75}
	\fill (p) circle(0.05)node[below]{$p$}; \fill (q) circle(0.05)node[right]{$q$};
	
	\draw[name path=kurve] (p) ..controls(p) and ($(wirbel) - \left*(ctrl1)$).. (wirbel)node[below]{$\gamma(t)$} ..controls($(wirbel) + \right*(ctrl1)$) and (q)..node[below]{$\gamma$} (q);
	\fill (wirbel) circle(0.05);
	
	\coordinate (vec) at (-0.5,1);
	\foreach \shift in {0.2,0.4,...,1}{
		\coordinate (neuwirbel) at ($(wirbel) + \shift*(vec)$);
		\draw[name path=obere kurve] (p) ..controls(p) and ($(neuwirbel) - \left*(ctrl1)$).. (neuwirbel) ..controls($(neuwirbel) + \right*(ctrl1)$) and (q).. (q);
	}
	\draw[->] (wirbel) -- ($(wirbel) + 1.3*(vec)$);
	
	\path[name path=vert] (0,-1) -- (0,1);
	\path[name intersections={of={kurve and vert}}];
	\fill (intersection-1) circle(0.05);
	\draw[->] (intersection-1) -- ($(intersection-1) + (0.5,1)$);
	\path[name intersections={of={obere kurve and vert}}];
	\node[above] at (intersection-1) {$h_s$};
\end{tikzpicture}\end{center}

Eine \quot{Kurve} durch $\gamma$ ist eine sogenannte \CmMark[Variation!glatte]{glatte Variation} $h\colon[0,1]\times[0,1] \to M$, $h(s,t) = h_s(t)$ mit $h_0 = \gamma$ und $h_s(0) = p$, sowie $h_s(1) = q$ f"ur alle $s \in [0,1]$. Dann ist
\begin{align*}
  X(t) = \difffrac[s=0]{}{s}h_s(t)
\end{align*}
ein glattes Vektorfeld entlang $\gamma$.
Ferner gilt $X(0) = 0$ und $X(1) = 0$.
Nun betrachte
\begin{align*}
	0  = \difffrac[s=0]{}{s}E(h_s) &= \int_{0}^{1} \difffrac[s=0]{}{s} \left<\difffrac{}{t} h_s(t),\difffrac{}{t}h_s(t)\right>\\
	& = \int_0^1 2 \left<\nabla_s \difffrac{}{t}h_s(t), \difffrac{}{t}h_s(t)\right>\\
	& = \int_0^1 2 \left<\nabla_t\smash{\underbrace{\difffrac{}{s}h_s(t)}_{=X(t)}}, \difffrac{}{t}h_s(t)\right> \vphantom{\underbrace{\difffrac{}{s}h_s(t)}_{=X(t)}}\\
	& = \int_0^1 2 \left<\nabla_tX,\difffrac{}{t}h_s(t)\right>\\
	& = 2 \int_0^1 \difffrac{}{t}\left<X,\difffrac{}{t}h_s(t)\right> - \left<X,\nabla_t\difffrac{}{t}h_s(t)\right>\\
	& = \underbrace{2 \int_0^1 \difffrac{}{t}\left<X,\difffrac{}{t}h_s(t)\right>}_{=0} - 2 \int_0^1 \left<X,\nabla_t\difffrac{}{t}h_s(t)\right>\\
	& = -2 \int_0^1 \left<X(t),\nabla_t\dot\gamma(t)\right>\dop t
\end{align*}

% Definition 8.1
\begin{Dfn}\label{dfn-8-1}
  Eine glatte Kurve $c$ in $M$ heißt \CmMark{Geod\"atische}\footnote{Die Äquivalenz zur bereits bekannten Definition wird in Kürze gezeigt.}, wenn $\nabla_t\dot c \equiv 0$ gilt.
\end{Dfn}

Ist $c$ Geodätische, so ist $c$ proportional zur Bogenlängenparametrisierung, das hei"st $\|\dot c\| = $const, denn $\difffrac{}{t}\|\dot c(t)\|^{2} = \difffrac{}{t}\left<\dot c(t),\dot c(t) \right> = 2\left<\nabla_t\dot c(t), \dot c(t)\right> = 0$.
Mit $c$ ist auch jede affine Umparametrisierung $t \mapsto c(at + b)$ eine Geodätische.

% Proposition 8.2
\begin{Prop}
Für jedes $p \in M$ und $v \in \T_pM$ existiert genau eine Geodätische $\gamma_{p,v}\colon[0,\epsilon] \to M$ mit $\gamma_{p,v}(0) = p$ und $\dot \gamma_{p,v}(0) = v$.
Zudem hängt $\gamma_{p,v}$ glatt von $p$ und $v$ ab.
\end{Prop}

\begin{bew}\begin{enumerate}[label=(\Alph*),leftmargin=*,widest=B]
\item
	Es sei $(\varphi, U)$ eine Karte um $p$, $\gamma^i(t) = \varphi^i(\gamma(t))$. Dann besitzt das folgende Anfangswertproblem
	\begin{align*}
		\begin{cases}
			0 = \nabla_t\dot \gamma|_t = \sum_k\left(\ddot \gamma^k(t) + \sum_{ij}\Gamma_{ij}^k\big(\gamma(t)\big)\dot \gamma^i(t)\gamma^j(t)\right) \pdifffrac[\gamma(t)]{}{x^k}\\
			\gamma^i(0) = \varphi^i(p)\\
			\dot\gamma^i(0) = \xi_p^i, \quad v = \sum \xi^i_p\pdifffrac[p]{}{x^i}
		\end{cases}
	\end{align*}
	eine eindeutige Lösung (lokal), wleche glatt von den Startwerten $p$ und $v$ abhängt.
\item
	(Alternativ) Ist $(\varphi, U)$ eine Karte von $M$ um $p$, dann ist
	\[ \overline \varphi \colon \left\{\begin{array}{cccl}
		\T M|_U &\to& \R^{2m}&\\
		X_p=\sum \xi_p^i\pdifffrac[p]{}{x^i} &\mapsto& \overline\varphi(X_p) &= (\varphi^1(p), \ldots, \varphi^m(p), \xi_p^1, \ldots, \xi_p^m)\\
		&&& =: (y^1, \ldots, y^{2m})
	\end{array}\right.\]
	eine Karte von $\T M$.	
	Es sei $S$ das durch
	\[ S \colon \left\{ \begin{array}{ccc}
		\T M &\to& \T\T M\\
		X = \sum \xi^i \pdifffrac{}{x^i} &\mapsto& \sum_i^m \xi^i \pdifffrac{}{y^i} - \sum_{i,j,k=1}^{m} \Gamma_{ij}^k \xi^i\xi^j\pdifffrac{}{y^{m+k}}
	\end{array}\right.\]
	defninierte glatte Vektorfeld auf $\T M$.	
	$g^t$ ist genau dann Integralkurve von $S$ durch $X_p = \sum \xi_p^i\pdifffrac[p]{}{x^i}$, wenn
	\begin{align*}
		\difffrac{}{t}g^t = \dot g^t = S(g^t) \text{ und } g^0 = X_p.
	\end{align*}
	Setzt man $\overline \varphi(g^t) = (\gamma^1(t), \ldots, \gamma^m(t),\eta^1(t), \ldots, \eta^m(t))$, so ist dies genau dann der Fall, wenn gilt:
	\begin{align*}
		& (\dot\gamma^1,\ldots, \dot\gamma^m,\dot\eta^1,\ldots, \dot\eta^m) = \left(\eta^1, \ldots, \eta^m, -\sum_{i,j}\Gamma_{ij}^1\eta^i\eta^j, \ldots, -\sum_{i,j}\Gamma_{ij}^m\eta^i\eta^j\right)\\
		& \rightsquigarrow \eta^i = \dot\gamma^i \text{ und } \ddot\gamma = -\sum_{i,j}\Gamma_{ij}^k\dot\gamma^i \dot\gamma^j
	\end{align*}
	und 
	\begin{align*}
		(\gamma^1(0), \ldots, \gamma^m(0), \eta^1(0), \ldots, \eta^m(0) = \overline\varphi(X_p) = (\varphi^1(p), \ldots, \varphi^m(p), \xi_p^1, \ldots, \xi_p^m)
	\end{align*}
	also genau dann, wenn
	\begin{align*}
		\gamma(t) = \overline\varphi^{-1}(\gamma^1(t), \ldots, \gamma^m(t))
	\end{align*}
	eine Geodätische durch $p$ mit $\dot \gamma(0) = X_p$ ist.	
	Der maximale Fluss $g^t$ von $S$ heißt \CmMark[Fluss!geod\"atischer]{geod"atischer Fluss}.
	Mit Satz \ref{satz-4-9} folgt die Aussage der Proposition.
\end{enumerate}\end{bew}

\begin{center}\begin{tikzpicture}[font=\scriptsize]
%	\draw[step=0.25,gray!15] (-6,-1) grid (6,5); \draw[step=0.5,gray!30] (-6,-1) grid (6,5); \fill (0,0) circle(0.1); %Hilfsgitter
	
	\def\breite{2.5}
	\def\hoehe{2}
	\def\shift{1}
	\def\vert{2.5}
	\draw (-\breite, \vert) -- (-\breite+\shift, \vert+\hoehe) -- (\breite+\shift, \vert+\hoehe) -- (\breite, \vert) -- cycle;
	\fill (-0.5,3.25) circle(0.05) node[left]{$0_p$};
	\draw[->] (-0.5,3.25) --node[above]{$v$} (0.75,3.75);
	\node at (3.5,3.5) {$\T_pM$};
	
	\coordinate (segel) at (-2.25,-0.5); \node at ($(segel) + (4.5,1.25)$) {$M$};
	\tikzsegel[1.5]{(segel)}
	\coordinate (pkt) at ($0.75*(-0.5,0.5)+(segel3)$);
	\coordinate (ctrl1) at (1,1); \coordinate (ctrl2) at (-1,1); \coordinate (ctrl3) at (0,1); \coordinate (ctrl4) at (-1,0);
	\draw[dashed] (segel1) ..controls($(segel1) + 0.5*(ctrl1)$) and ($(pkt) + 0.25*(ctrl2)$).. (pkt) ..controls($(pkt) + 0.5*(ctrl3)$) and ($(segel2) + 0.75*(ctrl4)$).. (segel2);
	
	\coordinate(pkt1) at ($(segel) + (1.25,0.75)$); \coordinate(pkt2) at ($(segel) + (2.5,1)$); \coordinate(pkt3) at ($(segel) + (3.75,1.75)$);
	\fill(pkt1) circle(0.05)node[anchor=north east,font=\tiny]{$p$}; \fill(pkt2) circle(0.05) node[below,font=\tiny]{$\gamma_v(1)= \exp_p(v)$};
	\coordinate (ctrl1) at (1.5,1); \coordinate (ctrl2) at (-1,-0.25);
	\draw[->](pkt1) --node[above,sloped,font=\tiny]{$\dot\gamma_v(0)=v$} ($(pkt1) + 0.75*(ctrl1)$);
	\draw (pkt1) ..controls($(pkt1) + 0.25*(ctrl1)$) and ($(pkt2) + 0.5*(ctrl2)$).. (pkt2) ..controls($(pkt2) - 0.5*(ctrl2)$) and (pkt3).. (pkt3);
\end{tikzpicture}\end{center}

Für $v \in \T_pM$ sei $\gamma_v(t) = \pi(g^t(v))$ die eindeutige Geodätische mit $\gamma_v(0) = p$ und $\dot \gamma_v(0) = v$.
Ist $\delta \in \R$ und $c(t) = \gamma_v(\delta t)$, so ist $c$ eine Geodätische durch $p$ mit $\dot c(0) = \delta v$, das hei"st $c = \gamma_{\delta v}$, beziehungsweise $\gamma_{\delta v}(t) = \gamma_v(\delta t)$.

Der Definitionsbereich $\mathcal D_S$ des geodätischen Flusses ist eine offene Menge in $\R \X \T_pM$ und somit sind sowohl $\mathcal D = \{v \in \T M \mid (1,v) \in \mathcal D_S\}$, als auch $\mathcal D_p = \mathcal D \cap T_pM$ offen für alle $p \in M$ (in $\T M$, beziehungsweise $\T_pM$). Weiterhin gilt $0_p \in \mathcal D_p$.

% Definition 8.3
\begin{Dfn}
  Die Abbildung $\exp_p\colon\mathcal D_p \to M$, $v \mapsto \gamma_v(1)$ heißt \CmMark{Exponentialabbildung}.
\end{Dfn}


%% 
%% Vorlesung <2012-12-18 Tue>
%%

Es wurde bereits gezeigt, dass $\nabla_t \dot \gamma_v \equiv 0$ ist (Geodätische Differentialgelichung).
Die Exponentialabbildung is nach Satz \ref{satz-4-6} glatt.
Es gilt $\exp_p(0_p) = p$.
Zur Berechnung des Differentiales von $\exp_p$ in $0_p$
\begin{align*}
  \exp_{p*0_p} \colon \T_{0_p}\T_pM \to \T_pM
\end{align*}
identifiziert man $\T_{0_p}\T_pM$ mit $\T_pM$.
Es gilt
\begin{align*}
  \exp_{p*0_p}(v) = \difffrac[t=0]{}{t}\exp_p(tv) = \difffrac[t=0]{}{t}\gamma_{tv}(1) = \difffrac[t=0]{}{t}\gamma_v(t) = \dot \gamma_v(0) = v,
\end{align*}
also $\exp_{p*0_p} = 1 \dop_{\T_pM}$.
Es existiert für alle $p \in M$ eine Umgebung $V$ von $0_p \in \T_pM$ und $U$ von $p$, so dass $\exp_p \colon V \to U$ ein Diffeomorphismus ist.
Wählt man eine Orthonormalbasis $e_1, \ldots, e_m$ von $\T_pM$ und setzt
\begin{align*}
  \psi \colon \T_pM \to \R^m, v = \sum_i b^ie_i \mapsto (b^1, \ldots, b^m),
\end{align*}
so ist $(\psi \circ \exp_p|_U^{-1}, U)$ eine Karte von $M$ um $p$.
Im Allgemeinen ist dies keine Isometrie!

% Definition 8.4
\begin{Dfn}
Diese Karte bezeichnet man als \CmMark[Normalkoordinaten!Riemannsche]{Riemannsche Normalkoordinaten}.
\end{Dfn}

% Proposition 8.5
\begin{Prop}
In Riemannschen Normalkoordinaten gilt für alle $i,j,k \leq m$:
\begin{enumerate}[label=(\roman*)]
\item
	$g_{ij}(0) = \delta_{ij}$
\item
	$\Gamma^k_{ij}(0) = 0$
\item
	$\partial_k g_{ij}(0) = \pdifffrac[0]{g_{ij}}{x^k} = 0$
\end{enumerate}\end{Prop}

Der Beweis sei zur "Ubung "uberlassen.

\section{Polarkoordinaten}

Es ist $\varphi = (r, \vartheta^1, \ldots, \vartheta^{m-1})$ die Hintereinanderausführung von Riemannschen Normalkoordinaten des $\R^m$.

\begin{center}
  \textcolor{red}{Abbildung: Zurückziehen von Polarkoordinaten via Riemannscher Normalkoordinaten.}
\end{center}

Die Umkehrabbildung ist ein Diffeomorphismus
\begin{align*}
  f \colon (0, \varepsilon) \times S^{m-1} \to U \subseteq M, \ 
  (t,v) \mapsto \exp_p(tv) = \gamma_v(t).
\end{align*}
Für jedes $v \in S^{m-1}$ ist $t \mapsto f(t,v) = \gamma_v(t)$ eine Geodätische in $M$. Wir bezeichnen solche Geod"atischen im Folgenden als \CmMark[Geod\"atische!radiale]{radiale Geod\"atische}.

% Lemma 8.6
\begin{Lemma}[Gauß-Lemma]
  Jede radiale Geodätische $\gamma_v$ ist orthogonal zu der geodätischen Sphäre
  \begin{align*}
    S_r = \{q \in M \mid \ \exists v \in \T_pM: \|v\| = r \text{ und } q = \exp_p(v) \}.
  \end{align*}
\end{Lemma}

\begin{bew}
Man zeigt das Folgende:
Ist $X$ ein Vektorfeld auf $S^{m-1}$ und bezeichnet man seine Fortsetzung auf $(0,\varepsilon) \times S^{m-1}$ \quot{$\subseteq$} $\B_{\varepsilon}(0)\setminus\{0\}$ bzw. $\B_{\varepsilon}(0_p)\setminus\{0_p\} \subseteq \T_pM$ mit $X_{rv} = X_v$, so ist
\begin{align*}
	Y_q = Y_{f(r,v)} = f_{*(r,v)}(0,X_v) = \exp_{p*}(r X_v)
\end{align*}
orthogonal zu 
\begin{align*}
	\pdifffrac[q]{}{r} = \difffrac[t=r]{}{t}\exp_p(tv) = \dot \gamma_v(r)
\end{align*}

\begin{center}
	\textcolor{red}{Abbildung: Erhalten der Orthogonalität durch $\exp_p$.}
\end{center}

$Y(t) = Y_{\gamma_v(t)}$ als Vektorfeld entlang $\gamma_v$.
Dann gilt:
\begin{align*}
	\difffrac[t=r]{}{t} \left<Y,\pdifffrac{}{r}\right>_{\gamma_v(t)} & = \left<\nabla_tY|_r, \dot \gamma_v(r)\right> + \left<Y(r), \smash{\underbrace{\nabla_t\dot\gamma_v|_r}_{=0}}\right> \vphantom{\underbrace{\gamma_v}_{0}}\\
	& = \left<\nabla_{Y(r)}\dot\gamma_v(r), \dot \gamma_v(r)\right> + \left< \smash{\underbrace{[\dot\gamma_v(r),Y(r)]}_{\mathclap{\begin{subarray}{l}= [f_{*}(\pdifffrac{}{r}),f_{*}(0,X_v)]\\ = f_{*}[\pdifffrac{}{r},X] = 0\end{subarray}}}}, \dot \gamma_v(r) \vphantom{\nabla_{Y(r)}} \right>  \vphantom{\underbrace{\gamma_v(r)}_{\pdifffrac{}{r}} }\\
	&= \frac{1}{2}Y(t) \|\dot\gamma_v\|^2 = 0.
\end{align*}
Ferner gilt
\begin{align*}
	\left<Y(r),\pdifffrac{}{r}\right>_{\gamma_v(r)} = \left<\exp_{p*}(rX_v),\dot\gamma_v(r)\right> \xrightarrow{r \to 0}\left<\exp_{p*}(0_p),v\right> = 0,
\end{align*}
also $\left<Y,\pdifffrac{}{r}\right> \equiv 0$.
\end{bew}

\begin{bem}
  Insbesondere gilt für alle $i \leq m-1$:
  \begin{align*}
    \left<\pdifffrac{}{r}, \pdifffrac{}{\vartheta^i}\right> = 0.
  \end{align*}
\end{bem}

% Satz 8.7
\begin{Satz}
Für jedes $p \in M$ existiert ein $\varepsilon > 0$, so dass für alle $q \in \B_{\varepsilon}(p)$ genau eine minimierende Geodätische von $p$ nach $q$ existiert, das hei"st eine Geodätische $\gamma$ im Sinne der Definition \ref{dfn-8-1} mit $\mathcal L(\gamma) = \dop(p,q)$.
Ist $q \notin \exp_p(\B_{\varepsilon}(0_p)) = \B_{\varepsilon}(p)$, so existiert ein $q' \in \partial \B_{\varepsilon}(p)$ mit
\begin{align*}
	\dop(p,q) = \varepsilon + \dop(q',q).
\end{align*}
\end{Satz}

\begin{bew}
Es sei $\varepsilon > 0$ so, dass auf $\B_{\varepsilon}(p)$ Polorkoordinaten $\varphi = (r,\vartheta^1, \ldots, \vartheta^{m-1})$ existieren. Sei weiter $c \colon [0,1]$ eine beliebige glatte Kurve von $p$ nach $q$ mit Koordinaten $\varphi(c(t)) = (r(t), \vartheta^1(t), \ldots, \vartheta^{m-1}(t))$.
\begin{center}
	\textcolor{red}{Abbildung: Bild von $c$.}
\end{center}
Für $t_0 = \inf\{t \in [0,1] \mid c(t) \notin \B_{\varepsilon}(p)\}$ ist $c|_{[0,t_0]}$ eine Kurve zu $\B_{\varepsilon}(p)$.
Es gilt
\begin{align*}
	\left\|\pdifffrac[t]{}{r}\right\| = \|\dot\gamma_w(t)\| = \|w\| = 1.
\end{align*}
Aus der Cauchy-Schwarz-Ungleichung folgt
\begin{align*}
	\|\dot c(t)\| & = \|\dot c(t)\|\left\|\pdifffrac[c(t)]{}{r}\right\|\\
	& \geq \left|\left< \dot c(t), \pdifffrac[c(t)]{}{r}\right>\right|\\
	& = \left|\left<\dot r(t) \pdifffrac{}{r} + \sum_{i=1}^{m-1}\dot\vartheta^i(t)\pdifffrac{}{\vartheta^i},\pdifffrac[c(t)]{}{r}\right>\right|\\
	& = \left|\left<\dot r(t)\pdifffrac[c(t)]{}{r}, \pdifffrac[c(t)]{}{r}\right>\right|\\
	& = \left|\dot r(t)\right|,
\end{align*}
wobei die Gleichheit genau dann gilt, wenn $\dot c(t)$ und $\pdifffrac[c(t)]{}{r}$ linear abhängig sind.
\begin{align*}
	\mathcal L(c) = \int_0^{t_0}\|\dot c\| + \int_{t_0}^T\|\dot c\| \geq \int_0^{t_0} \left|\left<\dot c,\pdifffrac{}{r}\right>\right| = \int_0^{t_0}|\dot r| = r(t_0)
\end{align*}
Gleichheit gilt genau dann, wenn $\vartheta^1(t), \ldots, \vartheta^{m-1}(t)$ konstant sind und $\dot r(t) \geq 0$ gilt, also genau dann, wenn $c$ eine monotone Umparametrisierung von $t \mapsto \exp_p(tv)$ f"ur $v \in S^{m-1}$ ist.
\end{bew}


%%% Local Variables: 
%%% mode: latex
%%% TeX-master: "../skript-diffgeom"
%%% End: 



%-_-_-_-_-_-_-_-_-_-_-_-_-_-_ Anhang -_-_-_-_-_-_-_-_-_-_-_-_-_-_-_-_

\appendix

%-_-_-_-_-_-_-_-_-_-_-_-_-_-_ Uebungen -_-_-_-_-_-_-_-_-_-_-_-_-_-_-_-_

% Die Benennung der "section" so aendern, dass "\"Ubung 123 vom " am Anfang steht
% Der Code ist fast genau der vom Anfang der Praeambel, dort steht die Erklaerung
\renewcommand*{\othersectionlevelsformat}[3]{\ifstr{#1}{section}{\"Ubung\ #3\ vom\ }{#3\autodot\enskip}}

% Das Format der "section" in Kopfzeile der rechten Seiten
\renewcommand*{\sectionmarkformat}{\"Ubung \thesection\autodot\ vom\enskip}


\section{22. Oktober 2012}
\setcounter{Aufg}{0} %Damit die Aufgaben jedes Mal bei Aufgabe 1 anfangen
\setcounter{Loes}{0}

\begin{dfn}[Gra\ss mann-Manningfaltigkeiten]
Sei $k \le n$ und $\Gr_k(\R^n) = \{ V \subseteq \R^n | \ddim V = k\}$
\end{dfn}

\emph{Behauptung:} $\Gr_k(\R^n)$ ist eine glatte Manningfaltigkeit.

\begin{bem}
F"ur $k = 1$ ist $\Gr_1(\R^n) = \R \P^n$
\end{bem}

$X_0 \in \Gr_k(\R^n) \Rightarrow \R^n = X_0 \oplus X_0^\perp$, $X_0 = \mspan\{e_1,\ldots ,e_k\}$ \marginnote{\begin{tikzpicture}
\draw[->] (-1.5, 0) -- (1.5,0) node[below]{$X_0$};
\draw[->] (0,-1.5) -- (0,1.5) node[left]{$X_0^\perp$};
\draw (-1,-1) --(1,1) node[below]{$Y$};
\end{tikzpicture}}

Definiere $U_{X_0} := \{Y \in \Gr_k(\R^n) | Y \cap X_0^\perp = \{0\}\}$. F"ur $Y \in U_{X_0}$ gilt dann: $\pr_{X_0}(Y) = X_0 \Rightarrow \pr_{X_0}$ ist ein Isomorphismus
	\[X_0 \xrightarrow{(\pr_{X_0}|_Y)^{-1}} Y \xrightarrow{\pr_{X_0^\perp}} X_0^\perp \]
Definiere
	\[ \varphi_{X_0}: \left\{\begin{array}{ccl} U_{X_0} &\to& \Hom(\underbrace{X_0, X_0^\perp}_{\cong \R^{k \cdot (n-k)}}) \\
		Y &\mapsto& \pr_{X_0^\perp} \circ (\pr_{X_0}|_Y)^{-1} \end{array}\right.\]
	\[ \varphi_{X_0}^{-1}: \left\{\begin{array}{ccl} \Hom(X_0, X_0^\perp) &\to& U_{X_0} \\
		f &\mapsto& \Graph(f) = \{x + xf | x \in X_0\} \end{array}\right.\]
\emph{Zu zeigen:}\begin{enumerate}
\item
	$U_{X_0}$ ist offen
\item
	$\varphi_{X-0}, \varphi_{X_0}^{-1}$ sind beide stetig
\item
	$\varphi_{X-0} \circ \varphi_{X_0}^{-1}$ ist glatt
\item
	$\Gr_k(\R^n)$ ist Hausdorffsch und hat eine abz"ahlbare Basis der Topologie
\end{enumerate}

\textbf{Welche Topologie eigentlich?} Sei $V = \{ (v_1,\ldots, v_k) \in (\R^n)^k | v_1,\ldots, v_k$ linear unabh"angig$\}$ und $\pi: V \to \Gr_k(\R^n), (v_1,\ldots ,v_k) \mapsto \mspan\{v_1,\ldots ,v_k\}$. Topologie auf $\Gr_k(\R^n)$: induziert von der Quotientopologie auf $V\modulo{\sim\pi}$, also
	\[U \subset \Gr_k(\R^n) \text{ offen } \Leftrightarrow \pi^{-1}(U) \text{ offen} \]
$V$ ist offen in $(\R^n)^k$: $V = \widetilde{\ddet}^{-1}(\R^{\left(\begin{smallmatrix}n \\ k\end{smallmatrix}\right)} \setminus \{0\})$ mit $\widetilde{\ddet}(v_1,\ldots,v_k) = (\ddet(k \times k\text{-Untermatrizen}))$

\emph{Zu zeigen:} $\pi^{-1}(U_{X_0})$ offen

$\pi^{-1}(U_{X_0}) = \{(v_1,\ldots ,v_k) \in V |\ \pr_{X_0}|_{\mspan\{v_i\}} \text{ hat vollen Rang}\} = \{(v_1,\ldots ,v_k) \in V|\ \pr_{X_0}(V-i) \text{ sind linear unabh"angig} \} = (\widetilde{\ddet} \circ (\pr_{X_0},\ldots ,\pr_{X_0}))^{-1}(\R^{\left(\begin{smallmatrix}n \\ k\end{smallmatrix}\right)} \textcolor{red}{\setminus \{0\}})$

$\Rightarrow U_{Y_0}$ ist offen.
\begin{description}[font=\normalfont\bfseries]
\item[zu 2)]\begin{description}[font=\normalfont\itshape]
	\item[Behauptung:] f"ur alle $Y \in U_{X_0}$ gibt es genau eine Basis $(y_1,\ldots ,y_k)$ von $Y$ sodass $\pr_{X_0}(y_i) = x_i$ f"ur eine feste Orthonormalbasis $(x_1,\ldots ,x_k)$ von $X_0$. Bezeichnet $B(Y)$ diese Basis, so ist $B: U_{X_0} \to V$ stetig
	\item[Beweis:] Existenz und Eindeutigkeit $\checkmark$ ($\pr_{X_0}$ ist Isomorphismus)
		
		F"ur $(v_1,\ldots ,v_k) \in \pi^{-1}(U_{X_0})$ ist $B \circ \pi(v_1,\ldots ,v_k) = ((\pr_{X_0}|_{\mspan\{v_1,\ldots ,v_k\}})^{-1} X_i)_{i \le k}$. Die Darstellungsmatrix von $(\pr_{X_0}|_{\mspan\{v_1,\ldots ,v_k\}})^{-1}$ bez"uglich $\{x_i\}, \{y_i\}$ h"angt stetig von den $v_i$ ab. Daraus folgt dass $B \circ \pi|_{\pi^{-1}(U_{X_0}}$ stetig ist, womit auch $B$ stetig ist. Es gilt:
			\[ B(Y)_i = \underbrace{x_i}_{\in X_0} + \underbrace{\varphi_{X_0}(Y)_{X_i}}_{\in X_0^\perp} \qquad \text{(*)} \]
		$\Rightarrow \varphi_{X_0}(Y)_{x_i}$ h"angt stetig von $Y$ ab.
		
		$\Rightarrow$ Darstellende Matrix von $\varphi_{X_0}(Y)$ h"angt stetig von $Y$ ab $\Rightarrow \varphi_{X_0}$ ist stetig
		
		(*) $\Rightarrow B(\varphi_{X_0}^{-1}(A))_i = x_i + Ax_i \Rightarrow B \circ \varphi_{X_0}^{-1}$ ist stetig (sogar glatt)
			\[ \varphi_{X_0}^{-1} = (\pi \circ B) \circ \varphi_{X_0}^{-1} \text{ ist stetig} \]
	\end{description}
\item[zu 3)]
	$\varphi_{X_0} \circ \varphi_{\tilde X_0}^{-1} = \varphi_{X_0} \circ \pi \circ (\underbrace{B_{\tilde X_0} \circ \varphi_{\tilde X_0}^{-1}}_{\text{ist glatt, s. o.}})$ ist glatt.
	
	$\varphi_{X_0} \circ \pi$ ist glatt, da $\varphi_{X_0} \circ \pi(v_1,\ldots ,v_k)(x_i) = (\underbrace{B_{X_0} \circ \pi}_{\substack{\text{glatt (Darst.}\\ \text{aus Beh.)}}})(v_1,\ldots ,v_k) - x_i$
\item[zu 4)]
	Abz"ahlbare Basis der Topologie wird von $V$ geerbt. \emph{Hausdorffsch}: Seien $X_0 \ne \tilde X_0 \in \Gr_k(\R^n) \xRightarrow[\text{"Ub. Aufg.}]{\text{L. A.}} \exists Z \subseteq \R^n, \ddim Z = n-k: Z \cap X_0 = \{0\} = Z \cap \tilde X_0, U_{\underbrace{Z^\perp}_{k\text{-dim}}} \ni X_0, \tilde X_0$
	
	\emph{Alternativ:} Sei $w \in X_0 \setminus \tilde X_0$ und $d_w^2: \Gr_k(\R^n) \to \R, Y \mapsto (\dist(w, Y))^2 \Rightarrow d_w^2(X_0) = 0, d_w^2(\tilde X_0) > 0$. Falls $d_w^2$ stetig ist, gilt: $(d_w^2)^{-1}((-\infty, \frac{d_w^2(\tilde X_0)}{2}))$ und $(d_w^2)^{-1}((\frac{d_w^2(\tilde X_0)}{2}, \infty))$ trennen und sind offen.
\end{description}
\section{29. Oktober 2012}
\setcounter{Aufg}{0} %Damit die Aufgaben jedes Mal bei Aufgabe 1 anfangen
\setcounter{Loes}{0}

\begin{Loes}
%asdf
\textcolor{red}{[BILD]}
\begin{enumerate}[label=\alph*),leftmargin=*,widest=a,font=\normalfont]
\item
	$S^n = (S^n \setminus \{N\}) \cup (S^n\setminus \{S\}) \checkmark$
	
	$\varphi, \psi$ Hom"oomorphismen, $\Phi: \{(x^0,\ldots ,x^n) \in \R^n | x^0 < 1\} \to \R^n, x \mapsto \frac{1}{1-x^0}(x^1, \ldots ,x^n) \Rightarrow \Phi$ ist stetig $\Rightarrow  \varphi = \Phi|_{S^n \setminus \{N\}}$ ist stetig. Es ist
		\[ \varphi^{-1}(y) = \frac{1}{1+|y|^2}(\|y\|^2 - 1, 2y)\]
	also ist $\varphi^{-1}$ stetig. Analog f"ur $\psi$:
		\[\varphi \circ \psi^{-1}(y) = \frac{y}{\|y\|^2} = \psi \circ \varphi{-1}(y) \]
	f"ur $y \in \R^n \setminus \{0\}$. Also glatter Kartenwechsel.\marginnote{\textcolor{red}{[BILD]}}
		\[ \varphi_i^\pm: U_i^\pm \to B_1(0) \subset \R^n, x \mapsto (x^0,\ldots ,x^{i-1}, x^{i+1},\ldots ,x^n) \]
		\[ (\varphi_i^\pm)^{-1}: B_1(0) \to U_i^\pm, y \mapsto (y^0,\ldots ,y^{i-1}, \pm (1 - \|y\|^2), \textcolor{red}{y^i},\ldots ,y^{n+1}) \]
	$\varphi_i^\pm \circ (\varphi_j^\pm)^{-1}$ glatt
	
	$\psi \circ (\varphi_j^\pm)^{-1}$ glatt
	
	$\varphi \circ (\varphi_j^\pm)^{-1}$ glatt
	
	$\varphi_i^\pm \circ \varphi$ glatt
	
	$\varphi_i^\pm \circ \psi$ glatt
\item
	asdf
\end{enumerate}
\end{Loes}

\begin{Loes}
$\varphi: \R \to \R, x \mapsto x^3$

\emph{Behauptung:} $\varphi$ induziert eine $C^\infty$-Struktur auf $\R$, die von der Standardstruktur abweicht.

Dazu müssen wir zeigen:\begin{enumerate}[font=\normalfont,label=(\roman*)]
\item
	$\{(\varphi, \R)\}$ ist ein $C^\infty$-Atlas
\item
	$\varphi$ ist nicht vertr"aglich mit $(\Id, \R)$
\end{enumerate}
\emph{Beweis:}\begin{enumerate}[leftmargin=*,widest=ii,font=\normalfont,label=(\roman*)]
\item
	$\varphi$ ist Hom"oomorphismus, da $\varphi$ und $\varphi^{-1}: x \mapsto \sqrt[3]{x}$ stetig sind. Offensichtlich "uberdeckt $\varphi$ ganz $\R$. Der einzige Kartenwechsel $\varphi \circ \varphi^{-1} = \Id_{\R}$ ist glatt.
\item
	Betrachte
		\[ \Id_{\R} \circ \varphi^{-1} = \varphi^{-1}: x \mapsto \sqrt[3]{x} \]
	$\Id_{\R} \circ \varphi^{-1}$ ist in $0$ nicht differenzierbar $\Rightarrow$ (ii) $\checkmark$
\end{enumerate}
\begin{description}[font=\normalfont\itshape]
\item[Behauptung:]
	Die beiden $C^\infty$ Strukturen sind diffeomorph
\item[Beweis:]
	Sei
		\[\begin{array}{cccc} f:&  \overset{\text{von } \Id \text{ induziert}}{(\R, \tau_{\text{std}})} &\to& \overset{\text{von } \varphi \text{ induziert}}{(\R, \tau)} \\
			& x &\mapsto& \sqrt[3]{x} \end{array}\]
	\marginnote{\textcolor{red}{[BILD]}}
	Dann ist $f$ bijektiv. Es gilt f"ur $x \in \R$:
		\[ \varphi \circ f \circ (\Id_{\R})^{-1} (x) = (\sqrt[3]{x})^3 = x \]
	ist glatt. Betrachte nun $f^{-1}$: $\Id_{\R} \circ f^{-1} \circ \varphi^{-1}(x) = (\sqrt[3]{x})^3 = x$ ist glatt. Damit ist $f$ ein Diffeomorphismus.
\end{description}
\end{Loes}

\begin{Loes}
$k \in \N \cup \{\infty\}$, $M_1, M_2$ $C^k$-Mannigfaltigkeiten, $N_i \subseteq M_i$ Untermannigfaltigkeit, $f \in C^j(M_1, M_2)$ wobei $j \le k$, $f(N_1) \subseteq N_2$.
\begin{description}[font=\normalfont\itshape]
\item[Behauptung:]
	$f|_{N_1} \in C^j(N_1, N_2)$
\item[Beweis:]
	Sei $p \in N_1$, sei $(\varphi_1, U_1)$ eine adoptierte Karte von $M_1$ in $p$, das hei\ss t $p \in U_1$.
		\[ \varphi_1(U_1 \cap N_1) = \varphi_1(U_1) \cap \left(\R^{\ddim N_1} \times \{0\}^{n-\ddim N}\right) \]
	Sei $(\varphi_2, U_2)$ eine adoptiere Karte von $N_2$ um $f(p)$. Dann erhaltenen wir Karten von $N_i$, indem wir die Projektion $\pi_i: \R^{\ddim M_i} \to \R^{\ddim N_i}, x \mapsto (x^1,\ldots ,x^{\ddim N_i})$ hinter die Karten $\varphi_i$ schalten (das hei\ss t betrachte $\pi_i \circ \varphi_i$).
	
	Es ist
		\[(\pi_2 \circ \varphi_2) \circ f|_{N_1} \circ (\pi_1 \circ \varphi_1)^{-1} = \underbrace{\pi_2}_{C^\infty} \circ (\underbrace{\varphi_2 \circ f \circ \varphi_1^{-1}}_{C^j}) \circ C_1\]
	mit $C_1: \R^{\ddim N_1} \ni x \mapsto (x, 0,\ldots ,0) \in \R^{\ddim M_1}$. Also $(\pi_2 \circ \varphi_2) \circ f \circ (\pi_1 \circ \varphi_1)^{-1} \in C^j$ und damit $f|_{N_1} \in C^j(N_1, N_2)$
\end{description}
\end{Loes}

\begin{Loes}\begin{enumerate}[label=\alph*),leftmargin=*,widest=a,font=\normalfont]
\item
	$M = S^n$, $N = \{(x^0, x^1, \ldots ,x^n) \in S^n | x^2 = \ldots = x^n = 0\}$; Skizze f"ur $n = 2$:
	\begin{center}\begin{tikzpicture}
		\draw (0,0) circle (0.75); \draw[dashed] (-0.75,0) to[out=20,in=160] (0.75,0); \draw (-0.75,0) to[out=340,in=200] (0.75,0);
		\node at (0.75,0.75) {$S^2$}; \node (N) at (1,-0.5) {$N$};
		\draw[->] (N) to[out=180,in=270] (0.5,-0.2);
	\end{tikzpicture}\end{center}
	\begin{description}[font=\normalfont\itshape]
	\item[Behauptung:]
		$N$ ist eine Untermannigfaltigkeit von $M$.\marginnote{\begin{tikzpicture}
			\draw[->] (-1.25,0) -- (1.25,0); \draw[->] (0,-1.25) -- (0,1.25);
			\draw (0,0) circle (0.75); \fill (-0.75,0) circle (0.05) (0.75,0) circle (0.05); \node at (0.75,-0.75) {$S^1$};
		\end{tikzpicture}}
	\item[Beweis:]
		Sei $\varphi: \overbrace{S^n \setminus \{(1, 0,\ldots ,0)\}}^{=:U} \to \R^n$, $\varphi(x) = \frac{1}{1-x^0}(x^1,\ldots ,x^n)$
		
		\emph{Zu zeigen:} $\varphi(U \cap N) = \varphi(U) \cap (\R \times \{0\}^{n-1})$
			\[ \varphi(U \cap N) = \varphi(N \setminus \{(1,\ldots ,0)\}) = \R \times \{0\}^{n-1} \]
		F"ur $p \in N \setminus \{(1,0, \ldots ,0)\}$ ist $\varphi$ also eine adoptierte Karte um $p$. F"ur $p = (1, 0, \ldots, 0)$ ist analog $\psi$ (aus 1 a)) eine adoptierte Karte.
	\end{description}
\item
	$M = \R^2$, $N = \{(x, 0) | x \ge 0\} \cup \{(0,y)|y \ge 0\}$; Skizze:
	\begin{center}\begin{tikzpicture}
		\draw (-1.5,0) -- (0,0); \draw[very thick,->] (0,0) -- (1.5,0);
		\draw (0,-0.5) -- (0,0); \draw[very thick,->] (0,0) -- (0,1.5);
		\node at (-1,1) {$M$}; \node at (0.75,0.75) {$N$};
	\end{tikzpicture}\end{center}
	\begin{description}[font=\normalfont\itshape]
	\item[Behauptung:]
		$N$ ist keine glatte Untermannigfaltigkeit von $\R$.
	\item[Beweis:]
		Angenommen $N$ w"are Untermannigfaltigkeit von $\R^2$. Da $N$ hom"oomorph zu $\R$ ist, w"are es eine eindimensionale Untermannigfaltigkeit. Damit existiert eine Karte $(\varphi, U)$ von $\R^2$ um $(0,0)$ mit $\varphi(U\cap N) = \varphi(U) \cap (\R \times \{0\})$. Betrachte $\varphi^{-1}$, beziehungsweise $f(t) = \varphi^{-1}(t,0)$. Es sei $t_0 \in \R$ mit $f(t_0) = (0,0)$. Da $f(t) = N \cap U$ ist entweder $f(t) \in \{(0, y) | y \ge 0\}$ f"ur $t > t_0$ und $f(t) \in \{(x, 0) | x \ge 0\}$ f"ur $t < t_0$ oder umgekehrt. Dann ist $f'(t) \in \R e_2$ f"ur $ t > t_0$ und $f'(t) \in \R e_1$ f"ur $t < t_0$ oder umgekehrt.
		
		$\Rightarrow f'(0) \in \R e_1 \cap \R e_2 = \{(0,0)\}$
		
		\emph{Andererseits:} $f'(0) = \underbrace{(D \varphi^{-1}|_{\varphi(0,0)})}_{\text{Isom., da } \varphi^{-1} \text{ Diffeom.}}(\left(\begin{smallmatrix}1\\0\end{smallmatrix}\right)) \ne \left(\begin{smallmatrix}1\\0\end{smallmatrix}\right)$
	\end{description}
\end{enumerate}\end{Loes}
\section{5. November 2012}
\setcounter{Aufg}{0} %Damit die Aufgaben jedes Mal bei Aufgabe 1 anfangen
\setcounter{Loes}{0}

\begin{Loes}
Es sei $\psi: (0, \infty) \times (0, 2\pi) \to \R^2 \setminus (\R_{\ge0} \times \{0\})$, $(r, \vartheta) \mapsto r(\cos \vartheta, \sin \vartheta)$, die Inverse $\varphi = \psi^{-1}$ ist eine Karte von $\R^2$.
\begin{align*}
	\pdifffrac[p]{}{r} &= \pdifffrac{\left(\Id^1 \circ \varphi^{-1}\right)}{r} \left(\varphi(p)\right) \pdifffrac[p]{}{x} + \pdifffrac{\left(\Id^2 \circ \varphi^{-1}\right)}{r} \left(\varphi(p)\right) \pdifffrac[p]{}{y}\\
	&= \pdifffrac{\left(\Id^1 \circ \psi\right)}{r} \left(\varphi(p)\right) \pdifffrac[p]{}{x} + \pdifffrac[p]{}{x} + \pdifffrac{\left(\Id^2 \circ \psi\right)}{r} \left(\varphi(r)\right)^2 \pdifffrac[p]{}{y}\\
	&= \cos\left(\vartheta(p)\right) \pdifffrac[p]{}{x} + \sin \left(\vartheta(p)\right) \pdifffrac[p]{}{y}\\
	&= \frac{1}{r(p)} \left(r(p) \cos\left(\vartheta(p)\right) \pdifffrac[p]{}{x} + r(p) \sin\left(\vartheta(p)\right) \pdifffrac[p]{}{y}\right)\\
	&= \frac{1}{r(p)} \left(\psi^1\left(\varphi(p)\right) \pdifffrac[p]{}{r} + \psi^2\left(\varphi(p)\right) \pdifffrac[p]{}{y}\right)\\
	&= \frac{1}{\|p\|} \left(p^1 \pdifffrac[p]{}{x} + p^2 \pdifffrac[p]{}{y}\right)
\end{align*}
Als Vektorfeld:
	\[ \pdifffrac{}{r} = \frac{1}{\|(x,y)\|} \left( x \pdifffrac{}{x} + y \pdifffrac{}{y} \right) \]
Desweiteren gilt:
\begin{align*}
	\pdifffrac[p]{}{\vartheta} &= \pdifffrac{\left(\Id^1 \circ \varphi^{-1}\right)}{\vartheta} \left( \varphi(p) \right) \pdifffrac[p]{}{x} + \pdifffrac{\left(\Id^2 \circ \varphi^{-1}\right)}{\vartheta} \left( \varphi(p) \right) \pdifffrac[p]{}{x}\\
	&= \ldots = -p_2\pdifffrac[p]{}{x} + p_1 \pdifffrac[p]{}{y}
\end{align*}
Also:
	\[ \pdifffrac{}{\vartheta} = -y \pdifffrac{}{x} + x \pdifffrac{}{y} \]
\begin{center}\begin{tikzpicture}
%\draw[step=0.25,gray!15] (-6,-3) grid (6,3); \draw[step=0.5,gray!30] (-6,-3) grid (6,3); \fill (0,0) circle(0.1); %Hilfsgitter
\draw[->] (-5,0) -- (-1,0); \draw[->] (1,0) -- (5,0); \draw[->] (-3,-2) -- (-3,2); \draw[->] (3,-2) -- (3,2);

\draw[->] (-1.5,0) -- (-1.5,1); \draw[->] (-3,1.5) -- (-4,1.5); \draw[->] (-4.5,0) -- (-4.5,-1); \draw[->] (-3,-1.5) -- (-2,-1.5);
\draw[->] (-1.875,0.875) -- ($(-1.875,0.875)+(-0.75,0.75)$); \draw[->] (-3.875, 1.125) --($(-3.875, 1.125)+(-0.75,-0.75)$);
\draw[->] (-4.125,-0.875) -- ($(-4.125,-0.875)+(0.75,-0.75)$); \draw[->] (-2.125, -1.125) -- ($(-2.125, -1.125)+(0.75,0.75)$);

\draw[->] (-2,0) -- (-2,0.5); \draw[->] (-3,1) -- (-3.5,1); \draw[->] (-4,0) -- (-4,-0.5); \draw[->] (-3,-1) -- (-2.5,-1);
\draw[->] (-2.25,0.5) -- ($(-2.25,0.5)+(-0.35,0.35)$); \draw[->] (-3.5,0.75) -- ($(-3.5,0.75)+(-0.35,-0.35)$);
\draw[->] (-3.75,-0.5) -- ($(-3.75,-0.5)+(0.35,-0.35)$); \draw[->] (-2.5,-0.75) -- ($(-2.5,-0.75)+(0.35,0.35)$);

\draw[->] (-2.5,0) -- (-2.5,0.25); \draw[->] (-3,0.5) -- (-3.25,0.5); \draw[->] (-3.5,0) -- (-3.5,-0.25); \draw[->] (-3,-0.5) -- (-2.75,-0.5);
\draw[->] (-2.625,0.25) -- ($(-2.625,0.25)+(-0.15,0.15)$); \draw[->] (-3.25,0.375) -- ($(-3.25,0.375)+(-0.15,-0.15)$);
\draw[->] (-3.375,-0.25) -- ($(-3.375,-0.25)+(0.15,-0.15)$); \draw[->] (-2.75,-0.375) -- ($(-2.75,-0.375)+(0.15,0.15)$);

\draw[->] (3.25,0.25) -- (3.75,0.75); \draw[->] (2.75,0.25) -- (2.25,0.75); \draw[->] (2.75,-0.25) -- (2.25,-0.75); \draw[->] (3.25,-0.25) -- (3.75,-0.75);
\draw[->] (2.75,0.75) -- (2.75,1.25); \draw[->] (2.25,0.25) -- (1.75,0.25);
\draw[->] (3.25,1) -- (3.5,1.5); \draw[->] (4.25,0.25) -- (4.75,0.5); \draw[->] (4,0.5) -- (4.5,0.875); \draw[->] (3.625,0.875) -- (4,1.375);
\draw[->] (3.625,1.625) -- (4.125,2.125); \draw[->] (4.25,1.5) -- (4.75,2); \draw[->] (4.75,1.25) -- (5.375,1.5);
\end{tikzpicture}\end{center}
\end{Loes}

\begin{Loes}\begin{enumerate}[label=\alph*),widest=b,leftmargin=*]
\item
	Zeige dass $\pi_i: M_1 \times M_2 \to M_i$, $(p_1,p_2) \mapsto p_i$ eine Submersion ist.
	
	Sei $(p_1,p_2) \in M_1 \times M_2$. Seien $\varphi_i$ Karten von $M_i$ um $p_i$ mit Kartengebieten $U_i$. Dann ist $\varphi_1 \times \varphi_2: U_1 \times U_2 \to \varphi_1(U_1) \times \varphi_2(U_2)$ eine Karte von $M_1 \times M_2$ um $(p_1,p_2)$. Es ist
		\[ \varphi_i \circ \pi_i \circ (\varphi_1 \times \varphi_2)^{-1} = \varphi_i \circ \pi_i \circ (\varphi_1^{-1} \times \varphi_2^{-1}). \]
	F"ur $(x_1,x_2) \in \varphi_1(U_1) \times \varphi_2(U_2)$ ist
		\[ \varphi_i \circ \pi_i \circ (\varphi_1 \times \varphi_2)^{-1} (x_1,x_2) = \varphi_i(\pi_i(\varphi_1^{-1}(x_1), \varphi_2^{-1}(x_2)))	= \varphi_i(\varphi_i^{-1}(x_i)) = x_i. \]
	Daraus folgt dass $\varphi_i \circ \pi_i \circ (\varphi_1 \times \varphi_2)^{-1}$ glatt ist. Der Rest des Beweises kann auf zwei Arten erfolgen.
	\begin{description}[font=\normalfont\itshape]
	\item[Variante 1:]
		Es folgt dass $\D(\varphi_i \circ \pi_i \circ (\varphi_1 \times \varphi_2)^{-1}) = \left( \begin{smallmatrix} 1 & & \\ & \ddots & \\ & & 1\\ & 0 &\end{smallmatrix} \right)$
		
		Dies ist die Darstellungsmatrix von $\pi_{1*}$ bez"uglich den Basen $\pdifffrac{}{\varphi_1^1}, \ldots, \pdifffrac{}{\varphi_1^d}$ und $\pdifffrac{}{(\varphi_1 \times \varphi_2)^1}, \ldots, \pdifffrac{}{(\varphi_1 \times \varphi_2)^{\ddim M_1 + \ddim M_2}}$. Also ist $\pi_{1*}$ surjektiv. Auch $\pi_1$ ist surjektiv, also ist $\pi_1$ eine Submersion. Der Beweis f"ur $\pi_2$ folgt analog.
	\item[Variante 2:]
		Sei $X = \sum_{j = 0}^{\ddim M_1} \xi^j \pdifffrac[p_1]{}{\varphi_1^j} \in \T_{p_1}M_1$. Setze
			\[ \tilde X = \sum_{j=1}^{\ddim M_1} \xi^j \pdifffrac[p]{}{(\varphi_1 \times \varphi_2)^j} \in \T_p(M_1 \times M_2). \]
		Dann ist
		\begin{align*}
			\pi_{1*_p} \tilde X &= \sum_{k=1}^{\ddim M_1} \left( \sum_j \underbrace{\partial_j \left( \varphi^k \circ \pi_1 \circ (\varphi_1 \times \varphi_2)^{-1} \right)}_{= \delta(k,j)} \left( \varphi_1(p_1), \varphi_2(p_2) \right) \xi^j \right) \pdifffrac[p_1]{}{\varphi_1^k}\\
			&= \sum_{k=1}^{\ddim M_1} \xi^k \pdifffrac[p_1]{}{\varphi_1^k} = X
		\end{align*}
		Daraus folgt dass $\pi_{1*}$ surjektiv ist.
	\end{description}
\item
	Zeige dass $f: (0,2\pi) \to \R^2$, $t \mapsto (\sin(t), \sin(2t))$ eine injektive Immersion aber keine Einbettung ist.
	\begin{description}[font=\normalfont\itshape]
	\item[$f$ ist injektiv:]
		Seien $t_1, t_2 \in (0,2\pi)$ mit $f(t_1) = f(t_2)$. Damit muss auch gelten dass $\sin(t_1) = \sin(t_2)$ und $\sin(2t_1) = \sin(2t_2)$. Aus diesen beiden Bedingungen folgt dass f"ur $t_1, t_2$ gelten muss:
		\begin{itemize}
		\item
			$t_1=t_2$ oder $\frac{\pi}{2}-t_1 = t_2-\frac{\pi}{2}$ oder $\frac{3\pi}{2}-t_1 = t_2 - \frac{3\pi}{2}$
		\item
			$2t_1=2t_2$ oder $\frac{\pi}{2}-2t_1=2t_2-\frac{\pi}{2}$ oder $\frac{3\pi}{2}-2t_1=2t_2-\frac{3\pi}{2}$
		\end{itemize}
		Aus den beiden Bedingungen folgt somit dass $t_1 = t_2$ gilt.
	\item[$f$ ist eine Immersion:]
		Es reicht zu zeigen, dass $f_{*t} = 0$ f"ur alle $t$ gilt. Es gilt $\D f(t) = (\cos(t), 2 \cos(2t))$, also
		\begin{align*}
			\D f(t) = 0 & \Leftrightarrow \cos(t) = 0 \text{ und } \cos(2t)=0\\
			& \Leftrightarrow \left(t = \frac{\pi}{2} \vee t = \frac{3\pi}{2} \right) \bigvee \left( 2t = \frac{\pi}{2} \vee 2t = \frac{3\pi}{2} \vee 2t = \frac{5\pi}{2} \vee 2t = \frac{7\pi}{2}\right)
		\end{align*}
		Das ist aber nicht m"oglich, also ist $\D f(t) \ne 0$. $\D f(t)$ ist die Darstellungsmatrix, also ist auch $f_{*_t} \ne 0$.
	\item[$f$ ist keine Einbettung:]
		\marginnote{\emph{Skizze:}\\
			\begin{tikzpicture}[scale=0.25,baseline=0]
				%\draw[step=0.25,gray!15] (-6,-3) grid (6,3); \draw[step=0.5,gray!30] (-6,-3) grid (6,3); \fill (0,0) circle(0.1); %Hilfsgitter
				\draw[->] (-3,0) -- (3,0); \draw[->] (0,-3) -- (0,3);
				\def\streck{4}		
				\draw (0,0) ..controls(0,0) and (2.5,\streck).. (2.5,2.5) -- (2.5,-2.5) ..controls(2.5,-\streck) and (0,0).. (0,0) ..controls(0,0) and (-2.5,\streck)..  (-2.5,2.5) -- (-2.5,-2.5) ..controls(-2.5,-\streck) and (0,0).. (0,0) -- cycle;
			\end{tikzpicture}}
		Es gilt dass $\left( \frac{1}{k} \right)_{k \in \N}$ nicht in $(0,2\pi)$ konvergiert, aber es ist $f\left( \frac{1}{k} \right) \to (0,0) = f(\pi) \in \Bild f$. Damit ist $f$ kein Hom"oomorphismus auf das Bild.
	\end{description}
\end{enumerate}\end{Loes}

\setcounter{Loes}{3}
\begin{Loes}
asdf
\end{Loes}
\section{12. November 2012}
\setcounter{Aufg}{0} %Damit die Aufgaben jedes Mal bei Aufgabe 1 anfangen
\setcounter{Loes}{0}

\begin{Loes}
Sei $M$ eine glatte Mannigfaltigkeit. Sei desweiteren f"ur alle drei Teilaufgaben $p \in M$, $\varphi$ eine Karte um $p$ mit Kartengebiet $U$ und
	\[ \overline \varphi: \left\{ \begin{array}{ccc}  \T M|_U &\to& \R^{2n} \\
		\sum_i \xi^i\pdifffrac[q]{}{x^i} &\mapsto& (\varphi(q), \xi) \end{array} \right.\]
eine Karte von $\T M$. Alle diese Karten bilden dann einen Atlas von $\T M$.
\begin{enumerate}[label=\alph*),widest=a,leftmargin=*]
\item
	\emph{Zeige:} $\pi: \T M \to M$, $\T_p M \ni x \mapsto p$ ist eine Submersion.
	
	Es ist
	\begin{align*} \varphi \circ \pi \circ \overline \varphi^{-1} \underbrace{(y, \xi)}_{\mathclap{\in \varphi(U) \X \R^n}} &= \varphi\left(\pi\left(\sum_i \xi^i \pdifffrac[\varphi^{-1}(y)]{}{x^i}\right)\right)\\
		&= \varphi(\varphi^{-1}(y)) = y
	\end{align*}
	also ist $\pi$ glatt. Desweiteren ist
		\[ \D(\varphi \circ \pi \circ \overline \varphi^{-1})|_{(y, \xi)} = \left(\begin{smallmatrix}
        1 &  & \\
        & \ddots & & \\
         & & 1
      \end{smallmatrix} 0\right) \]
     surjektiv und damit auch $\pi_{*\overline\varphi^{-1}(y, \xi)}$. Offensichtlich ist $\pi$ surjektiv und damit eine Submersion.
\item
	\emph{Zeige:} $\sigma: M \to \T M$, $p \mapsto 0_{\T_pM}$ ist eine Einbettung.
	
	Es gilt
		\[\overline \varphi \circ \sigma \circ \varphi^{-1} (y) = \overline \varphi(0_{\T_{\varphi^{-1}(y)}M}) = (\varphi(\varphi^{-1}(y)),0) = (y,0) \]
	Daraus folgt folgt dass $D(\overline \varphi \circ \sigma \circ \varphi^{-1})|_y = \left( \begin{smallmatrix} 1 & &  \\ & \ddots & \\ & & 1 \\ & 0 & \end{smallmatrix} \right)$ injektiv ist und damit auch $\sigma_{* \varphi^{-1}(y)}$. $\sigma$ ist injektiv und stetig, $\pi \circ \sigma = \Id_m$, also ist $\sigma^{-1} = \pi|_{\Bild(\sigma)}$ stetig.
\item
	\emph{Zeige:} Ist $\Phi: M \to N$ eine glatte Abbildung, so auch $\Phi_*: \T M \to \T N$.
	
	Sei $\psi$ eine Karte um $\Phi(p)$ und $\overline \psi$ die zugeh"orige Karte von $\T N$.
	\begin{align*}
		(\overline \psi \circ \Phi_* \circ \overline \varphi^{-1})\underbrace{(y, \xi)}_{\mathclap{\in \varphi(U) \X \R^n}} &= (\overline \psi \circ \Phi_*)\left(\sum \xi^{i} \pdifffrac[\varphi^{-1}(y)]{}{x^{i}}\right)\\
		&= \overline \psi\left(\sum_i\left(\sum_j \pdifffrac[\varphi^{-1}(y)]{\Phi^{i}}{x^j} \xi^j\right) \pdifffrac[\Phi(\varphi^{-1}(y))]{}{\psi^{i}}\right)\\
		&= \left( \underbrace{(\psi \circ \Phi \circ \varphi^{-1})}_{\text{glatt}}(y), \underbrace{\left( \sum_j \underbrace{\pdifffrac[_\varphi^{-1}(y)]{\Phi^{i}}{x^j}}_{\text{glatt in }y} \overbrace{\xi^j}^{\substack{\text{glatt}\\ \text{in }\xi}} \right)_{i = 1,\ldots ,n}}_{\text{glatt}} \right)
	\end{align*}
	Daraus folgt dass $\Phi_*$ glatt ist.
\end{enumerate}\end{Loes}

\begin{Loes}\begin{enumerate}[label=\alph*),leftmargin=*,widest=a]
\item
	\emph{Zu zeigen:} $[\cdot,\cdot]$ ist im Allgemeinen nicht $C^{\infty}(M)$-bilinear.
	
	$M = \R$, $\pdifffrac{}{x} = X = Y$, $f = \Id \quot{= x}$
		\[\begin{array}{rcl} [X,\underbrace{fY}_{=x \pdifffrac{}{x}}] &\overset{\text{c)}}{=}& \left( 1 \underbrace{\pdifffrac{x}{x}}_{=1} - x \underbrace{\pdifffrac{1}{x}}_{=0} \right) \pdifffrac{}{x} = \pdifffrac{}{x} \\
			f[X,Y] &\overset{\text{c)}}{=}& f \left( 1 \underbrace{\pdifffrac{1}{x}}_{=0} - 1 \underbrace{\pdifffrac{1}{x}}_{=0} \right) \pdifffrac{}{x} = 0 \end{array}\]
\item
	\emph{Zu zeigen:} f"ur $X, Y \in \calV(M)$ ist $XY$ mit $(XY)|_p(f) = X_p(Y(f))$ im Allgemeinen keine Derivation.
	\begin{align*}
		(XY)_p(fg) &= X_p(Y(fg))\\
		&= X_p(q \mapsto Y_q(fg))\\
		&= X_p(q \mapsto f(q) Y_q(g) + g(q) Y_q(f))\\
		&= X_p(fY(g) + gY(f))\\
		&= X_p(fY(g)) + X_p(gY(f))\\
		&= f(p) \cdot X_p(Y(g)) + Y_p(g) \cdot X_p(f) + g(p) \cdot X_p(Y(f)) + Y_p(f)X_p(g)\\
		&= f(p) \cdot (XY)|_p(g) + g(p)(XY)|_p(f) + \underbrace{Y_p(g)X_p(f) + Y_p(f)X_p(g)}_{\ne 0}
	\end{align*}
	$M = \R$, $X = Y = \pdifffrac{}{x}$, $f = g = \Id$ $\Rightarrow $ Leibnitz-Regel gilt nicht.
\item
	\emph{Bemerkung:} Ist $X|_U = \sum_{i=1}^{n}\xi^{i}\pdifffrac{}{x^{i}}$ lokale Darstellung bez"uglich $\varphi$ von $X \in \calV(M)$, so ist
		\[ \xi^{i} = X(\varphi^{i}) \]
	Seien $X|_U = \sum_i \xi^i \pdifffrac{}{x^{i}}$, $Y|_U = \sum_i \eta^{i} \pdifffrac{}{x^{i}}$. Damit gilt dann
	\begin{align*}
		[X,Y](x^j) &= (XY - YX)(x^j)\\
		&= X(Y(x^j)) - Y(X(x^j))\\
		&= X\left(\sum_i \eta^{i}\underbrace{\pdifffrac{}{x^{i}}(x^j)}_{\delta_{ij}}\right) - Y\left(\sum_i \xi^{i}\underbrace{\pdifffrac{}{x^{i}}(x^j)}_{\delta_{ij}}\right)\\
		&= X(\eta^j) - Y(\xi^j)\\
		&= \sum_i \left( \xi^{i} \pdifffrac{}{x^{i}}(\eta^j) - \eta^{i} \pdifffrac{}{x^{i}}(\xi^j) \right)
	\end{align*}
\end{enumerate}\end{Loes}

\begin{Loes}\begin{enumerate}[label=\alph*),widest=a,leftmargin=*]
\item
	Es seien $X = -y\pdifffrac{}{x} + x\pdifffrac{}{y}, Y = -2y\pdifffrac{}{x} + \frac{1}{2}x\pdifffrac{}{y} \in \calV(\R)$, bestimme $\gamma_x^t$ und $\gamma_y^t$.\begin{description}[font=\normalfont\itshape]
	\item[F"ur $X$:]
		$t \mapsto \gamma_*^t(p)$ ist Integralkurve von $X$ mit $\gamma_x^0(p) = p$. \emph{Gesucht:} Kurve mit $\gamma(0) = p$, $\gamma_{*t}\pdifffrac{}{t} = X(\gamma(t)) \Leftrightarrow \gamma_{*t}\pdifffrac{}{t}(x) = X(\gamma(t))(x)$ und $\gamma_{*t}\pdifffrac{}{t}(y) = X(\gamma(t))(y) \Leftrightarrow \gamma_1'(t) = - \gamma_2(t)$ und $\gamma_2'(t) = \gamma_1(t)$. Das Anfangswertproblem
			\[ \begin{pmatrix} \gamma_1 \\ \gamma_2 \end{pmatrix}' (t) = \begin{pmatrix} 0 & -1 \\ 1 & 0 \end{pmatrix} \begin{pmatrix} \gamma_1(t) \\ \gamma_2(t) \end{pmatrix} \text{ und } \gamma(0) = p \]
		hat als L"osung $t \mapsto \exp(t \left( \begin{smallmatrix} 0 & -1 \\ 1 & 0 \end{smallmatrix} \right) ) \cdot p$. Es gilt:
			\[ \begin{pmatrix} 0 & -1 \\ 1 & 0 \end{pmatrix}^{2n} = \begin{pmatrix} (-1)^n & 0 \\ 0 & (-1)^n \end{pmatrix} \text{ und } 
				\begin{pmatrix} 0 & -1 \\ 1 & 0 \end{pmatrix}^{2n+1} = (-1)^n \begin{pmatrix} 0 & -1 \\ 1 & 0 \end{pmatrix} \]
		Damit gilt:
		\begin{align*}
			\exp\left(t \begin{pmatrix} 0 & -1 \\ 1 & 0 \end{pmatrix} \right) &= \sum_{k=0}^{\infty} \frac{1}{k!} t^k (\ldots)^k\\
			&= \begin{pmatrix} \sum\limits_{k=0}^{\infty} \frac{(-1)^k}{2k!} t^{2k} & -\sum\limits_{k=0}^{\infty} \frac{t^{2k+1}}{(2k+1)!} (-1)^{k} \\
				-\sum\limits_{k=0}^{\infty} \frac{t^{2k+1}}{(2k+1)!} (-1)^{k} & \sum\limits_{k=0}^{\infty} \frac{(-1)^k}{2k!} t^{2k} \end{pmatrix} \\
			&= \begin{pmatrix} \cos(t) & -\sin(t) \\ \sin(t) & \cos(t) \end{pmatrix}
		\end{align*}
	Daraus folgt $\gamma(t) = \left( \begin{smallmatrix} \cos(t) & -\sin(t) \\ \sin(t) & \cos(t) \end{smallmatrix} \right) \cdot p = \gamma_x^t(p)$
	\item[F"ur $Y$:]
		$\gamma_{*t}\pdifffrac{}{t} = Y(\gamma(t))$, $\gamma(0) = p \xLeftrightarrow{\text{analog}} \gamma'(t) = \left( \begin{smallmatrix} 0 & -2 \\ \frac{1}{2} & 0 \end{smallmatrix} \right) \gamma(t)$
			\[ \begin{pmatrix} 0 & -2 \\ \frac{1}{2} & 0 \end{pmatrix}^{2n} = \begin{pmatrix} (-1)^n & 0 \\ 0 & (-1)^n \end{pmatrix} \]
		Daraus folgt $\underbrace{\gamma(t)}_{\mathclap{= \gamma_x^t(p)}} = \exp(t \left( \begin{smallmatrix} 0 & -2 \\ \frac{1}{2} & 0 \end{smallmatrix} \right) )\cdot p = \left( \begin{smallmatrix} \cos(t) & -2\sin(t) \\ \frac{1}{2}\sin(t) & \cos(t) \end{smallmatrix} \right) \cdot p$
	\end{description}
\item
	asdf
\end{enumerate}\end{Loes}
\section{19. November 2012}
\setcounter{Aufg}{0} %Damit die Aufgaben jedes Mal bei Aufgabe 1 anfangen
\setcounter{Loes}{0}

\begin{Loes}
asdf
\end{Loes}

\begin{Loes}
Sei $\{U_\alpha | \alpha \in I\}$ eine offene "Uberdeckung vom $M$, $g_{\alpha\beta}: U_\alpha \cap U_\beta \to \GL(k,\R)$, $g_{\alpha\gamma}(p) = g_{\alpha\beta}(p) \cdot g_{\beta\gamma}(p)$ f"ur alle $ \in U_\alpha \cap U_\beta \cap U_\gamma$. Sei $E := \dot\bigcup_{\alpha \in I} (U_\alpha \X \R^k)_{/\sim}$, wobei f"ur $(p,v)_\alpha \in U_\alpha \X \R^k$, $(q,w)_\beta \in U_\beta \X \R^k$ gilt $(p,v)_\alpha \sim (q,w)_\beta \Leftrightarrow p=q$ und $v = g_{\alpha\beta}(p) \cdot w$.

\emph{Behauptung:} $\pi: E \to M$, $[p,v] \mapsto p$ ist ein Vektorb"undel.

\begin{description}[leftmargin=*]
\item[\quot{$\bm{\sim}$} ist "Aquivalenzrelation:]\begin{itemize}[leftmargin=*]
	\item
		$(p,v)_\alpha \sim (p,v)_\alpha$ gilt: $g_{\alpha\alpha}(p) = \Id v$ ($g_{\alpha\alpha}(p) = \underbrace{g_{\alpha\alpha}(p)}_{\mathclap{\in \GL(k,\R)}} \cdot g_{\alpha\alpha}(p)$)
	\item
		$(p,v)_\alpha \sim (q,w)_\beta \Rightarrow (q,w)_\beta \sim (p,v)_\alpha$ gilt: $p=q$, $v=g_{\alpha\beta}(p)w$ $\Rightarrow w = (g_{\alpha\beta}(p))^{-1} v = g_{\beta\alpha} v$ ($g_{\alpha\alpha}(p) = g_{\alpha\beta}(p) g_{\beta\alpha}(p)$)
	\item
		Transitivit"at folgt aus $g_{\alpha\gamma} = g_{\alpha\beta} g_{\beta\gamma}$
	\end{itemize}
\item[$\bm{E_p}$ ist $\bm{k}$-dimensionaler Vektorraum:]
	\[ [(p,v)_\alpha] + \lambda[(p,w)_\alpha] := [(p, v + \lambda w)_\alpha] \]
	\begin{description}[font=\normalfont\itshape,leftmargin=*]
	\item[unabh"angig von $\alpha$:]
		\[\begin{array}{rl}
			[(p,v)_\beta] + \lambda[(p,w)_\beta] &= [(p,g_{\alpha\beta}(p)v)_\alpha] + \lambda[(p,g_{\alpha\beta}(p)w)_\alpha]\\
				&= [(p,g_{\alpha\beta}(p)v + \lambda g_{\alpha\beta}(p)w)_\alpha]\\
				&= [(p,g_{\alpha\beta}(p) \cdot(v + \lambda w))_\alpha] = [(p,v + \lambda w)_\beta]
		\end{array}\]
	\item[$k$-dimensional:]
		$q|_{\{p\} \X \R^k} : \{p\} \X \R^k \to E_p$ ist Vektorraum-Isomorphismus (wobei $q: \dot \bigcup_{\alpha \in I} (U_\alpha \X \R^k) \to E$)
	\end{description}
\item[B"undelkarten (glatt):]
	$\Phi_\alpha: U_\alpha \X \R^k \to E|_{U_\alpha}$, $(p,v) \mapsto [(p,v)_\alpha]$ ist Hom"oomorphismus, da $\sim|_{(U_\alpha \X \R^k) \X (U_\alpha \X \R^k)}$ die triviale "Aquivalenzrelation ist.
	\[\begin{array}{rl} \Phi_\alpha \circ \Phi_\beta^{-1}(p,v) &= \Phi_\alpha([(p,v)_\beta])\\
		&= \Phi_\alpha([(p,g_{\alpha\beta}(p)v)_\alpha])\\
		&= (p, g_{\alpha\beta}(p)v) \end{array}\]
	$\Rightarrow \Phi_\alpha \circ \Phi_\beta^{-1}$ ist glatt. $\Phi_\alpha|_{E_p}$ ist Vektorraum-Isomorphismus.
\item[\quot{normale} Karten:]
	Sei $\varphi$ Karte von $M$ mit Kartengebiet $U \subset U_\alpha$ $\leadsto$ $\overline\varphi_\alpha: E|_U \to \varphi(U) \X \R^k$, $e \mapsto (\varphi(\pi(e)), (\Phi_\alpha)^2(e))$.
	
	Glatte Kartenwechsel $\checkmark$
\item[$\bm{E}$ Hausdorffsch:]
	$[(p,v)_\alpha] \ne [(q,w)_\beta] \in E$
	\begin{description}[font=\normalfont,leftmargin=*]
	\item[$p\ne q$:]
		Die Urbilder in $M$ trennender Umgebungen von $p$ und $q$ unter $\pi$ trennen die Punkte in $E$.
	\item[$p=q$:]
		$v \ne g_{\alpha\beta}(p) w$ $\leadsto$ trennen im $\R^k$ und "uber $\Phi_\alpha$ zur"uckziehen.
	\end{description}
\item[abz"ahlbare basis der Topologie (f"ur $\bm{U_{\alpha}} \bm{\X} \R^{\bm{k}} \bm{\checkmark}$):]
	Es gibt ein $I ' \subseteq I$ mit $I$ abz"ahlbar und $M = \bigcup_{\alpha \in I'} U_\alpha$. Sei $\{V_j | j \in J'\}$ abz"ahlbare Basis der Topologie von $M$. Dann ist mit $J = \{j \in J' | V_j \subset U_\alpha \text{ f"ur ein } \alpha \in I\}$, $\{V_j | j \in J\}$ auch abz"ahlbare Basis der Topologie von $M$, denn $U \underset{\mathclap{\text{offen}}}{\subset} M$ $\Rightarrow$
		\[\begin{array}{rcl} U &=& \bigcup\limits_{\alpha \in I} (U_\alpha \cap U)\\
			&=& \bigcup\limits_{\alpha \in I} \bigcup\limits_{\substack{j \in J' \\ U_j \subset U_\alpha \cap U}} V_j \qquad \textcolor{gray}{(U_j \subset U_\alpha \cap U \Rightarrow j \in J)}\\
			&=& \bigcup\limits_{\alpha \in I} \bigcup\limits_{\substack{j \in J \\ V_j \subset U_\alpha \cap M}} V_j \end{array}\]
	F"ur $j \in J$ sei $\alpha(j) \in I$, sodass $V_j \subset U_{\alpha(j)}$. Setze $I' := \{ \alpha(j) | j \in J\}$.
		\[ \bigcup_{\alpha \in I'} U_\alpha = \bigcup_{j \in J} U_{\alpha(j)} \supseteq \bigcup_{j \in J} V_j = M \]
\end{description}\end{Loes}

\begin{Loes}
Es seien $E, E'$ Vektorb"undel "uber $M$ und $F: E \to E'$ sei ein B"undelmorphismus mit dem Isomorphismus $F_p: E_p \to E_p'$ f"ur alle $p \in M$.

\emph{Behauptung:} $F$ ist ein B"undelisomorphismus

$F$ ist surjektiv, denn f"ur $e \in E'$ ist $F_{\pi'(e)}: E_{\pi'(e)} \to E'_{\pi'(e)}$ bereits ein Isomorphismus, also existiert ein Urbild $\tilde e \in E_{\pi'(e)} \subset E$ mit $F(\tilde e) = F_{\pi'(e)}(\tilde e) = e$. Dass $F$ injektiv ist folgt analog, da $\pi' \circ F = \pi$.

Damit existiert ein $G: E' \to E$ mit $G \circ F = \Id$, $F \circ G = \Id$ und $\pi \circ G = \pi'$. Damit gilt auch
	\[ G(e') = (F_{\pi'(e)})^{-1}(e') \]
Es sei nun ein offenes $U  \subseteq M$ mit den Trivialisierungen $E|_U$ und $E'|_U$ gegeben.
\begin{center}\begin{tikzpicture}
	%\draw[step=0.25,gray!15] (-4,-4) grid (4,4); \draw[step=0.5,gray!30] (-4,-4) grid (4,4); \fill (0,0) circle(0.1); %Hilfsgitter
	
	\def\hor{2.5}
	\def\vert{1.25}
	\def\angle{30}
	\node (1) at (-\hor,\vert) {$E|_U$}; \node (2) at (\hor,\vert) {$U \X \R^k$}; \node (3) at (-\hor,-\vert) {$E'|_U$}; \node (4) at (\hor,-\vert) {$U \X \R^k$};
	
	\draw[->] (1) --node[font=\scriptsize,above]{$\Phi$}node[font=\scriptsize,below]{$\cong$} (2);
	\draw[->] (3) --node[font=\scriptsize,above]{$\Phi'$}node[font=\scriptsize,below]{$\cong$} (4);
	
	\draw[->] (1) to[out=270-\angle,in=90+\angle]node[left,font=\scriptsize]{$F$} (3);
	\draw[->] (3) to[out=90-\angle,in=270+\angle]node[right,font=\scriptsize]{$G$} (1);
	
	\draw[->] (2) to[out=270-\angle,in=90+\angle]node[left,font=\scriptsize]{$\Phi' \circ F \circ \Phi^{-1} = \tilde F$} (4);
	\draw[->] (4) to[out=90-\angle,in=270+\angle]node[right,font=\scriptsize]{$\tilde G = \Phi \circ G \circ \Phi'^{-1}$} (2);
\end{tikzpicture}\end{center}
Da $\pi' \circ F = F$ ist, existiert eine Abbildung $f: U \to \GL(k, \R)$ sodass $\tilde F(p,v) = (p, f(p) \cdot v)$ ist. Daraus folgt dass $\tilde G(p,w) = (p, (f(p))^{-1}w)$ glatt ist, da $\cdot^{-1}: \GL(k,\R) \to \GL(k, \R)$ glatt ist (denn $A^{-1} = \frac{1}{\ddet A}((-1)^{i+j} \ddet A[i,j])_{i,j}^T$). Damit ist $G$ glatt und somit auch ein B"undelmorphismus.
\end{Loes}

\begin{Loes}\begin{enumerate}[label=(\alph*),leftmargin=*,widest=a]
\item
	\emph{Behauptung:} Es sei $E$ ein Vektorb"undel vom Rang $k$ "uber $M$ auf dem $k$ punktweise linear unabh"angige Schnitte existieren. Dann ist $E$ trivial.
	
	Es seien $\sigma_1, \ldots, \sigma_k: M \to E$ Schnitte, die punktweise linear unabh"angig sind. Dann hat $E_p$ als Basis $\sigma_1(p),\ldots ,\sigma_k(p)$. Definiere nun $F: E \to M \X \R^k$, $\sum_{i=k}^k a_i \sigma_i(p) \mapsto (p, a_1,\ldots ,a_k)$. Es gilt $\pi_1 \circ F = \pi$ und $F_p$ ist ein Isomorphismus. $F$ ist glatt, denn f"ur eine B"undelkarte $\Phi$ ist $\tilde \sigma_i = \Phi^2 \circ \sigma_i$\marginnote{\scriptsize{$\Phi^2$ ist die zweite Komponente}}, das hei"st $\Phi(\sigma(p)) = (p, \tilde\sigma(p))$. Mit $A(p) = (\tilde\sigma_1(p),\ldots ,\tilde\sigma_k(p))^{-1}$ gilt:
	\[ F \circ \Phi^{-1}(p,v) = (p, A(p) \cdot V) \]
Daher ist $F$ glatt und mit Aufgabe 3 folgt, dass $F$ ein B"undelmorphismus ist.
\item
	\emph{Zu zeigen:} $\T S^3 \cong S^3 \X \R^3$
	
	Es gilt:
		\[ \T_pS^3 \cong p^\perp \ni \underbrace{\begin{pmatrix} -p_2 \\ p_1 \\ -p_4 \\ p_3 \end{pmatrix}}_{=:\sigma_1(p)} \cdot \underbrace{\begin{pmatrix} p_3 \\ -p_4 \\ -p_1 \\ p_2 \end{pmatrix}}_{\sigma_2(p)} \cdot \underbrace{\begin{pmatrix} -p_4 \\ -p_3 \\ p_2 \\ p_1 \end{pmatrix}}_{\sigma_3(p)} \]
	Dadurch sehen wir dass $\sigma_i(p) \perp \sigma_j(p)$ f"ur $ i \ne j$, woraus folgt dass $\sigma_1, \ldots ,\sigma_3$ punktweise linear unabh"agige Schnitte sind. Mit (a) folgt dann die Behauptung.
\end{enumerate}\end{Loes}

\begin{emptythm}[Anmerkung]
Der Raum der Schnitte $\Gamma(M,E)$ ist ein $\R$-Vektorraum, also ist lineare Unabh"angigkeit f"ur Schnitte definiert. Linear unabh"angige Schnitte sind im Allgemeinen \emph{nicht} punktweise linear unabh"angig. Betrachte beispielsweise
	\[ \pi_1: \underset{\mathclap{\text{\textcolor{gray}{Mf.}}}}{\R} \X \underset{\mathclap{\text{\textcolor{gray}{VR}}}}{\R} \to \R \]
Dann sind $\sigma_1(t) = (t,1)$ und $\sigma_2(t) = (t,t)$ lineare unabh"angig, aber in jedem Punkt linear abh"angig.
\end{emptythm}
%%
%% Skript Differentialgeometrie im Wintersemester 12/13
%% Zur Vorlesung von Dr. Grensing am KIT Karlsruhe
%%
%% Uebung 5
%%

\section{26. November 2012}
\setcounter{Aufg}{0} %Damit die Aufgaben jedes Mal bei Aufgabe 1 anfangen
\setcounter{Loes}{0}

\begin{emptythm}[Einschub]
F"ur ein Tensorprodukt $V \otimes W$ und ein Element $v_1 \otimes w_1 + v_2 \otimes w_2 \in V \otimes W$ gilt im Allgemeinen
	\[v_1 \otimes w_1 + v_2 \otimes w_2 \ne v_3 \otimes w_3 \]
\emph{Beispiel:} $\left( \begin{smallmatrix} 1 \\ 0 \end{smallmatrix} \right) \otimes \left( \begin{smallmatrix} 1 \\ 0 \end{smallmatrix} \right) + \left( \begin{smallmatrix} 0 \\ 1 \end{smallmatrix} \right) \otimes \left( \begin{smallmatrix} 0 \\ 1 \end{smallmatrix} \right)$
\end{emptythm}

\begin{Loes}
Wir zeigen dass die $(r,s)$-Tensorfelder den $C^\infty(M)$-multilinearen Abbildungen entsprechen.
	\[ \underbrace{\calV^*(M) \X \ldots \X \calV^*(M)}_{r\text{-mal}} \X \underbrace{\calV(M) \X \ldots \X \calV(M)}_{s\text{-mal}} \]
Zun"achste zeigen wir die Behauptung punktweise. Sei dazu $p \in M$ und die Abbildung
	\[ F_p: \T_pM \otimes \ldots \otimes \T_pM \otimes \T_p^*M \otimes \ldots \otimes \T_p^*M \to \text{Multilin}_{\R}(\T_p^*M \otimes \ldots \otimes \T_pM) \]
definiert durch
\begin{align*}
	F_p(\sum_i a_i \overbrace{X_1^{i}}^{\mathclap{\in \T_pM}} \otimes \ldots \otimes X_r^{i} \otimes \overbrace{\omega_1^{i}}^{\mathclap{\in \T_p^*M}} \otimes \ldots \otimes \omega_s^{i})(\eta_1,\ldots ,\eta_r, Y_1, \ldots ,Y_s)\\
	\qquad := \sum_i a_i \eta_1(X_1^{i}) \cdot \ldots \cdot \eta_n(X_r^{i}) \cdot \omega_1^{i}(Y_1) \cdot \ldots \cdot \omega_s^{i}(Y_s)
\end{align*}
\begin{description}
\item[$\bm{F_p}$ ist wohldefiniert:]
	\begin{itemize}[leftmargin=*]
		\item
			$F_p(\ldots)$ ist $\R$-multilinear $\checkmark$
		\item
			Sei $Z_1,\ldots ,Z_r$ Basis von $\T_pM$, $\mu_1,\ldots ,\mu_s$ die dazu duale Basis von $\T_p^*M$. Damit ist $\{Z_{i_1} \otimes \ldots \otimes Z_{i_r} \otimes \mu_{j_1} \otimes \ldots \otimes \mu_{j_s} | i_1, \ldots i_r, j_1, \ldots , j_s \in \{1,\ldots ,n\}\}$ eine Basis von $\T_pM \otimes \ldots \otimes \T_p^*M$. Sei $X_k^{i} = \sum_\alpha \chi_{k, \alpha}^{i} Z_\alpha$, $\omega_l^{i} = \sum_\beta w_{l, \beta}^{i} \mu_\beta$, dann folgt
			\begin{align*}
				& \sum_i a_i X_1^{i} \otimes \ldots \otimes \omega_s^{i}\\
				=& \sum_i a_i (\sum_{\alpha_1} \chi_{k, \alpha_1}^{i} Z_{\alpha_1}) \otimes \ldots  \otimes (\sum_{\beta_s} w_{s, \beta_s}^{i} \mu_{\beta_s})\\
				=& \sum_{\mathclap{\substack{\alpha_1,\ldots ,\alpha_r \\ \beta_1,\ldots ,\beta_s}}} \Big( \underbrace{\sum_i a_i \chi_{1,\alpha_1}^{i} \cdot \ldots \cdot \chi_{r,\alpha_r}^{i} \cdot w_{1, \beta_1}^{i} \cdot \ldots w_{s, \beta_s}^{i} }_{= A_{\alpha_1,\ldots , \alpha_r, \beta_1, \ldots ,\beta_s}} \Big) Z_{\alpha_1} \otimes \ldots \otimes Z_{\alpha_r} \otimes \mu_{\beta_1} \otimes \ldots \otimes \mu_{\beta_s}
			\end{align*}
			Damit folgt insgesamt
			\begin{align*}
				F_p\left(\sum a_i X_1^{i} \otimes \ldots \right) &= \sum a_i \eta_1 (X_1^{i}) \cdot \ldots \cdot \omega_s^{i} (Y_s) \\
					& = \ldots \\
					& = \sum_{\mathclap{\alpha_1,\ldots ,\beta_s}} A_{\alpha_1,\ldots ,\beta_s} \eta_1(Z_{\alpha_1}) \ldots \mu_{\beta_s}(Y_s)\\
					& = F_p \left(\sum A_{\alpha_1,\ldots ,\beta_s} Z_{\alpha_1} \otimes \ldots \otimes \mu_{\beta_s}\right)
			\end{align*}
	\end{itemize}
\item[$\bm{F_p}$ ist $\R$-linear]
\item[$\bm{F_p}$ ist surjektiv:]
	Sei $g: \T_p^*M \X \ldots \T_pM \to \R$ eine $\R$-multilineare Abbildung, dann ist $g$ eindeutig bestimmt durch
		\[ \overset{\R \ni}{A_{\alpha_1, \ldots , \alpha_r, \beta_1, \ldots , \beta_s}} = g(\mu_{\alpha_1}, \ldots , \mu_{\alpha_r}, \beta_1, \ldots , \beta_s) \text{, mit } \alpha_1, \ldots , \beta_s \in \{1,\ldots ,n\} \]
	Damit ist dann
	\begin{align*}
		F_p\left(\sum A_{\alpha_1,\ldots ,\beta_s} Z_{\alpha_1} \otimes \ldots \otimes \mu_{\beta_s}\right) (\mu_{\tilde\alpha_1},\ldots ,Z_{\tilde\beta_s}) &= \sum A_{\alpha_1,\ldots ,\beta_s} \underbrace{\mu_{\tilde \alpha_1}(Z_{\alpha_1})}_{\delta_{\tilde \alpha_1 \alpha_1}} \ldots \underbrace{\mu_{\beta_s} (Z_{\tilde \beta_s})}_{\delta_{\beta_s\tilde\beta_s}}\\
		&= A_{\tilde\alpha_1,\ldots ,\tilde\beta_s}\\
		&= g(\mu_{\tilde\alpha_1},\ldots ,Z_{\tilde\beta_s})
	\end{align*}
	Insgesamt folgt
		\[ g = F_p\left(\sum A_{\alpha_1,\ldots ,\beta_s} Z_{\alpha_1} \otimes \ldots  \otimes Z_{\beta_s}\right) \]
\item[$\bm{F_p}$ ist injektiv:]
	Ist $0 = F_p(\sum A_{\alpha_1,\ldots ,\beta_s} Z_{\alpha_1} \otimes \ldots \otimes \mu_{\beta_s})$, so folgt
	\begin{align*}
		0 &= F_p() (\mu_{\tilde\alpha_1},\ldots ,\mu_{\tilde\alpha_r}, Z_{\tilde\beta_1}, \ldots , Z_{\tilde\beta_s})\\
		&= A_{\tilde\alpha_1,\ldots ,\tilde\beta_s} \text{ f"ur alle } \tilde\alpha_1, \ldots , \tilde\beta_s \in \{1,\ldots ,n\}
	\end{align*}
	Daraus folgt $\sum A_{\alpha_1,\ldots ,\beta_s} Z_{\alpha_1} \otimes \ldots \otimes \mu_{\beta_s} = 0$
\end{description}
Insgesamt folgt damit dass $F_p$ ein Isomorphismus von $\R$-Vektorr"aumen ist. Wir definieren nun
	\[ F: \calT_s^r(M) \to \text{Multilin}_{C^\infty(M)}(\underbrace{\calV^*(M) \X \ldots \X \calV^*(M)}_{r\text{-mal}} \X \underbrace{\calV(M) \X \ldots \calV(M)}_{s\text{-mal}}, C^\infty(M)) \]
durch
	\[ F(S)(\overset{\in \calV^*(M) \ni}{\omega_1,\ldots ,\omega_r}, \overset{\in \calV(M) \ni}{X_1,\ldots ,X_s})(p) := F_p(S_p)(\omega_1|_p,\ldots ,\omega_r|_p, X_1|_p,\ldots ,X_s|_p) \]
\begin{description}
\item[$\bm{F(S)(\omega_1,\ldots ,X_s) \in C^\infty(M)}$:]
	lokale Koordinaten $\leadsto \pdifffrac{}{x^{i}} x^{i}$, Koeffizienten von $\omega_1, \ldots ,X_s$ glatt $\Rightarrow F(S)(\omega_1,\ldots ,X_s)$ glatt
\item[$\bm{F(S)}$ ist $\bm{C^\infty(M)}$-multilinear:]
	Seien $f \in C^\infty(M)$ und $\tilde\omega_i \in \calV^*(M)$, damit ist dann:
	\begin{align*}
		F(S)(\omega_1,\ldots ,\omega_i + f \tilde\omega_i, \ldots ,X_s)(p) =& F_p(S_p)(\omega_1|_p,\ldots ,\omega_i|_p + f(p)\tilde\omega_i|_p,\ldots ,X_s|_p)\\
		=& F_p(S_p)(\omega_1|_p,\ldots ,\omega_i|_p,\ldots ,X_s|_p)\\
		 & + f(p)F_p(S_p)(\omega_1|_p,\ldots ,\tilde\omega_i|_p,\ldots ,X_s|_p)\\
		=& F(S)(\omega_1,\ldots ,\omega_i,\ldots ,X_s)(p)\\
		 & + f(p)F(S)(\omega_1,\ldots ,\tilde\omega_i,\ldots ,X_s)(p)
	\end{align*}
\item[$\bm{F}$ ist $\bm{C^\infty(M)}$-linear] $\checkmark$
\item[$\bm{F}$ ist injektiv:]
	$F(S) = 0$, also ist $F_p(S_p) = 0$ f"ur alle $p \in M$. Da $F_p$ injektiv ist, ist $S_p = 0$ f"ur alle $p \in M$ und damit $S = 0$.
\item[$\bm{F}$ ist surjektiv:]
	Sei $g: \calV^*(M) \X \ldots \calV^*(M) \X \calV(M) \X \ldots \X \calV(M) \to C^\infty(M)$ eine $C^\infty(M)$-multilineare Abbildung. Seien weiter $p \in M$, $\phi$ eine Karte um $p$ und $\chi$ eine glatte cut-off Funktion mit $\supp \chi \subset$\marginnote{\scriptsize{$\supp$ \quot{Tr"ager}}} Kartengebiet von $\phi$ und $\chi \equiv 1$ auf einer Umgebung $V$ von $p$. F"ur $q \in V$ ist
	\begin{align*}
		g(\omega_1, \ldots ,X_s)(q) =&\, g(\chi\omega_1 + (1-\chi)\omega_1,\ldots ,\chi X_s + (1 - \chi) X_s)(q)\\
		=&\, g(\chi \omega_1, \chi \omega_2 + (1 - \chi) \omega_2, \ldots \chi X_s + (1-\chi) X_s)(q)\\
		 &\, + \underbrace{(1-\chi)(q)}_{=0} g(\omega_1, \chi \omega_2 + (1-\chi) \omega_2,\ldots )\\
		=&\, g(\chi\omega_1, \chi\omega_2 + (1-\chi)\omega_2,\ldots ,\chi X_s+ (1-\chi) X_s)(q)\\
		=&\, \ldots = g(\chi \omega_1, \ldots ,\chi X_s)(q) \qquad (*)
	\end{align*}
	Sei $S_p := \sum A_{\alpha_1, \ldots, \beta_s}(p) \pdifffrac[p]{}{x^{\alpha_1}} \otimes \ldots \otimes \dop x^{\beta_s}|_p$ mit
		\[ A_{\alpha_1, \ldots, \beta_s}(p) = g(\chi \dop x^{\alpha_1},\ldots ,\chi \pdifffrac{}{x^{\beta_s}}) \]
	Bachrechnen ergibt dass andere Karten das gleiche $S_p$ liefern. Daher ist $S$ auf ganz $M$ definiert. Wegen (*) und der lokalen Darstellung gilt $F(S) = g$.
\end{description}
\end{Loes}

\begin{Loes}\begin{enumerate}[label=\alph*), leftmargin=*]
\item
	$\omega_1 = yz \dop x + xz \dop y + xy \dop z$ ist geschlossen und exakt
		\[ \textcolor{gray}{\left(\dop f = \pdifffrac{f}{x} \dop x + \pdifffrac{f}{y} \dop y + \pdifffrac{f}{z} \dop z \right)} \]
	$\dop(xyz) = (yz) \dop x + (xz) \dop y + (xy) \dop z \Rightarrow 0 = \dop \circ \dop(xyz) = \dop \omega_1$
\item
	$\omega_2 = y^2 \dop x + x^3yz \dop y + x^2y \dop z$ ist weder geschlossen noch exakt.
	
	$\dop \omega_2 \ne 0$ (nachrechnen); angenommen $\exists \eta : \dop \eta = \omega_2 \Rightarrow 0 = \dop^2 \eta = \dop \omega_2 \ne 0 \lightning \Rightarrow \omega_2$ nicht exakt
\item
	$\dop \omega_3 \ne 0$
\item
	$\omega_4$ ist exakt
\end{enumerate}\end{Loes}

\begin{Loes}
Skizze:
\begin{center}\begin{tikzpicture}
	\draw[name path=kreis] (0,0) circle(1);
	\draw[->] (1,0) -- (1,1); \draw[->] (-1,0) -- (-1,-1); \draw[->] (0,1) -- (-1,1); \draw[->] (0,-1) -- (1,-1);
	\path[name path=a] (-1,-1) -- (1,1);
	\path[name path=b] (-1,1) -- (1,-1);
	\path[name intersections={of= kreis and a}];
	\draw[->] (intersection-1) -- ($(intersection-1) + sqrt(0.5)*(-1,1)$); \draw[->] (intersection-2) -- ($(intersection-2) + sqrt(0.5)*(1,-1)$);
	\path[name intersections={of= kreis and b}];
	\draw[->] (intersection-1) -- ($(intersection-1) + sqrt(0.5)*(-1,-1)$); \draw[->] (intersection-2) -- ($(intersection-2) + sqrt(0.5)*(1,1)$);
\end{tikzpicture}\end{center}
\end{Loes}
%-_-_-_-_-_-_-_-_-_-_-_-_-_-_ Stichwortverzeichnis -_-_-_-_-_-_-_-_-_-_-_-_-_-_-_-_

\printindex

%-_-_-_-_-_-_-_-_-_-_-_-_-_-_ Symbolverzeichnis -_-_-_-_-_-_-_-_-_-_-_-_-_-_-_-_

\printglossaries

%-_-_-_-_-_-_-_-_-_-_-_-_-_-_ Literaturverzeichnis -_-_-_-_-_-_-_-_-_-_-_-_-_-_-_-_

\bibliographystyle{plain}
\bibliography{literature}

\end{document}
