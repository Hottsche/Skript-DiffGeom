%%
%% Skript Differentialgeometrie im Wintersemester 12/13
%% Zur Vorlesung von Dr. Grensing am KIT Karlsruhe
%%
%% Mitschrieb und Textsatz von Jan-Bernhard Kordaß.
%%

\documentclass[a4paper, twoside, 11pt]{scrartcl}

\usepackage[utf8x]{inputenc}
\usepackage[T1]{fontenc}
\usepackage{lmodern}

\usepackage[ngerman]{babel}
\usepackage[a4paper, top=2.5cm, bottom=3cm, left=2.5cm, right=5cm]{geometry}

\usepackage{fancyhdr} % erlaubt mehr Optionen in Kopf- und Fusszeile
\usepackage{marginnote} % Randnotizen
\usepackage{enumitem} % Fuer mehr Einstellungmoeglichkeiten bei Aufzaehlungen
\usepackage{xifthen} % Erlaubt die Verwendung von if-then-else Befehlen im Code
\usepackage{makeidx} % Fuer den Index
\usepackage{xspace} % intelligende Leerzeichen bei Macros

\usepackage{tikz} % Fuer Zeichnungen in TikZ
\usetikzlibrary{matrix,arrows,calc,intersections, positioning, patterns, decorations.text}

\usepackage[toc]{glossaries} % Symbolverzeichnis
\glossarystyle{treehypergroup}
\makeglossaries

% Mathe Pakete
\usepackage{amsmath}
\usepackage{amssymb}
\usepackage{amsthm}

% common mathematical operators and sets
\DeclareMathOperator{\aff}{aff} % affine Huelle
\DeclareMathOperator{\diam}{diam} % diameter
\DeclareMathOperator{\dist}{dist} % distance
\DeclareMathOperator{\ggT}{ggT} % goesster gemeinsamer Teiler
\DeclareMathOperator{\inh}{inh} % Inhalt
\DeclareMathOperator{\grad}{grad} % Gradient
\DeclareMathOperator{\kgV}{kgV} % kleinstes gemeinsames Vielfaches
\DeclareMathOperator{\mspan}{span} % Lineare Huelle
\DeclareMathOperator{\n}{n} % Umlaufzahl
\DeclareMathOperator{\offen}{offen}
\DeclareMathOperator{\res}{res} % Residuum
\DeclareMathOperator{\rg}{rg} % rank (i)
\DeclareMathOperator{\sgn}{sgn} % Signum
\DeclareMathOperator{\supp}{supp} % support
\DeclareMathOperator{\sternf}{sternf}

\DeclareMathOperator{\Abb}{Abb} % maps
\DeclareMathOperator{\Aut}{Aut} % automorphisms
\DeclareMathOperator{\Bild}{Bild}
\DeclareMathOperator{\Charakteristik}{char}
\DeclareMathOperator{\Charakt}{char}
\DeclareMathOperator{\Diff}{Diff}
\DeclareMathOperator{\End}{End} % endomorphisms
\DeclareMathOperator{\Graph}{Graph}
\DeclareMathOperator{\Hom}{Hom} % homomorphisms
\DeclareMathOperator{\Id}{Id} % identity
\DeclareMathOperator{\Inn}{Inn} % Untergruppe der inneren Automorphismen
\DeclareMathOperator{\Kern}{Kern}
\DeclareMathOperator{\Relation}{\scriptstyle\mathrm{R}} % custom Relation
\DeclareMathOperator{\Rang}{Rang} % rank (ii)
\DeclareMathOperator{\Stab}{Stab} % Stabilisator
\DeclareMathOperator{\Sym}{Sym} % symmetric group

\newcommand{\Zentrum}[1]{\ensuremath{\mathrm Z(#1)}} % Zentrum einer Gruppe
\newcommand{\Ordnung}[1][]{ % Ordnung einer Gruppe
  \ifthenelse{\isempty{#1}}{
    \#
  }{
    \left|#1\right|
  }
}

%Realteil und Imaginaerteil
\renewcommand{\Re}{\ensuremath{\operatorname{Re}}} % <-- sollte man da nicht besser \DeclareMathOperator verwenden?
\renewcommand{\Im}{\ensuremath{\operatorname{Im}}}

% \DeclareMathOperator{\Real}{Re} % real part
% \DeclareMathOperator{\Imag}{Im} % imaginary part

% canonic sets
\DeclareMathOperator{\C}{\mathbb{C}}
\DeclareMathOperator{\F}{\mathbb{F}}
\DeclareMathOperator{\K}{\mathbb{K}}
\DeclareMathOperator{\N}{\mathbb{N}}
\DeclareMathOperator{\Q}{\mathbb{Q}}
\DeclareMathOperator{\R}{\mathbb{R}}
\DeclareMathOperator{\RP}{\mathbb{RP}} % real projection plane
\DeclareMathOperator{\Tor}{\mathbb{T}} % torus
\DeclareMathOperator{\Z}{\mathbb{Z}}

% Redeclare \P (Prim or Propability) and put the old, reversed "breakline P" in \BreakLineP
\let\BreakLineP\P
\renewcommand{\P}{\ensuremath{\mathbb{P}}}

% canonic differentiation, 
\DeclareMathOperator{\T}{T} % tangent bundle
\DeclareMathOperator{\D}{D} % Jacobi matrix or derivative

\newcommand{\dop}{\mathrm{d}}	
\newcommand{\difffrac}[2][]{\dfrac{\dop #1}{\dop #2}}
\newcommand{\pdifffrac}[2][]{\dfrac{\partial #1}{\partial #2}}

% quotient space or group
\newcommand{\modulo}[1]{\ensuremath{/_{\displaystyle #1}}}

% declaring Index for group theory
\newcommand{\Index}[2]{\ensuremath{(#1 \SlimDdot #2)}}


% canonic environments
\newcounter{thmglobal}
\swapnumbers
\theoremstyle{plain}

%%%%%%%%%%%%%%%%%%%%%%%%%%%%%%%%%%%%%%%%%%%%%%%%%%%%%%%%%%%%%%%%%%%%%%%%%%%%%%%%%%%%%%%%%%%%%%%%%%%%%%%%%%%%%%%%%%%%%%%%%%%%%%%%%%%%%%%%%%%%%%%%%%%%%%%%%%%%%%%%%%%%%%%%

\makeatletter

% Options
\newboolean{enableDeepNumbering}
\setboolean{enableDeepNumbering}{false}

\DeclareOption{deepnum}{
  \setboolean{enableDeepNumbering}{true}
}

\newboolean{enableMarginThm}
\setboolean{enableMarginThm}{false}

\DeclareOption{marginthm}{
  \setboolean{enableMarginThm}{true}
}

\ProcessOptions\relax

% call makeindex for an index register
\makeindex

% Name language settings

\iflanguage{ngerman}{
  % theorem names, ngerman
  \newcommand{\cmLangThmSatz}{Satz\xspace}
  \newcommand{\cmLangThmLemma}{Lemma\xspace}
  \newcommand{\cmLangThmKor}{Korollar\xspace}
  \newcommand{\cmLangThmProp}{Proposition\xspace}

  \newcommand{\cmLangThmDfn}{Definition\xspace}
  \newcommand{\cmLangThmBsp}{Beispiel\xspace}

  \newcommand{\cmLangThmBem}{Bemerkung\xspace}

  % short forms, ngerman
  \newcommand{\cmLangThmShortSatz}{Satz\xspace}
  \newcommand{\cmLangThmShortLemma}{Lemma\xspace}
  \newcommand{\cmLangThmShortKor}{Kor\xspace}
  \newcommand{\cmLangThmShortProp}{Prop\xspace}

  \newcommand{\cmLangThmShortDfn}{Def\xspace}
  \newcommand{\cmLangThmShortBsp}{Bsp\xspace}

  \newcommand{\cmLangThmShortBem}{Bem\xspace}
}{
  % theorem names, english
  \newcommand{\cmLangThmSatz}{Theorem\xspace}
  \newcommand{\cmLangThmLemma}{Lemma\xspace}
  \newcommand{\cmLangThmKor}{Corollary\xspace}
  \newcommand{\cmLangThmProp}{Proposition\xspace}

  \newcommand{\cmLangThmDfn}{Definition\xspace}
  \newcommand{\cmLangThmBsp}{Example\xspace}

  \newcommand{\cmLangThmBem}{Remark\xspace}

  % short forms, ngerman
  \newcommand{\cmLangThmShortSatz}{Thm\xspace}
  \newcommand{\cmLangThmShortLemma}{Lemma\xspace}
  \newcommand{\cmLangThmShortKor}{Cor\xspace}
  \newcommand{\cmLangThmShortProp}{Prop\xspace}

  \newcommand{\cmLangThmShortDfn}{Def\xspace}
  \newcommand{\cmLangThmShortBsp}{Exp\xspace}

  \newcommand{\cmLangThmShortBem}{Rem\xspace}
}


\ifthenelse{\boolean{enableMarginThm}}{

  % use the especially for lecture scripts and summeries designed margin theorem style
  \RequirePackage{marginnote}

  \newcommand{\CmMarginThmHead}[1]{\textsf{\textbf{#1}}}
  \newcommand{\CmMarginThmNum}{\arabic{section}.\ifthenelse{\boolean{enableDeepNumbering}}{\arabic{subsection}.}{}\arabic{thmglobal}\xspace}
  \newcommand{\CmMarginThmDescription}[1][]{\ifthenelse{\isempty{#1}}{}{ \textit{#1}}}
  \newlength{\cmMarginThmSpaceBottom}
  \setlength{\cmMarginThmSpaceBottom}{0.3cm}
  \newlength{\cmMarginThmSpaceTop}
  \setlength{\cmMarginThmSpaceTop}{0.1cm}
  \newcommand{\CmMarginThmStandardLayoutBegin}[2]{
    \refstepcounter{thmglobal}
    \vspace{\cmMarginThmSpaceTop}
    \marginnote{\CmMarginThmHead{\CmMarginThmNum #2} \CmMarginThmDescription[#1]}
  }
  \newcommand{\CmMarginThmStandardLayoutEnd}{\vspace{\cmMarginThmSpaceBottom}}

  \newenvironment{satz}[1][]{
    \CmMarginThmStandardLayoutBegin{#1}{\cmLangThmShortSatz}
    \begin{itshape}
    }{\end{itshape}\CmMarginThmStandardLayoutEnd}
  \newenvironment{lemma}[1][]{
    \CmMarginThmStandardLayoutBegin{#1}{\cmLangThmShortLemma}
    \begin{itshape}
    }{\end{itshape}\CmMarginThmStandardLayoutEnd}
  \newenvironment{kor}[1][]{
    \CmMarginThmStandardLayoutBegin{#1}{\cmLangThmShortKor}
    \begin{itshape}
    }{\end{itshape}\CmMarginThmStandardLayoutEnd}
  \newenvironment{prop}[1][]{
    \CmMarginThmStandardLayoutBegin{#1}{\cmLangThmShortProp}
    \begin{itshape}
    }{\end{itshape}\CmMarginThmStandardLayoutEnd}

  \newenvironment{dfn}[1][]{
    \CmMarginThmStandardLayoutBegin{#1}{\cmLangThmShortDfn}
  }{\CmMarginThmStandardLayoutEnd}
  \newenvironment{bsp}[1][]{
    \CmMarginThmStandardLayoutBegin{#1}{\cmLangThmShortBsp}
  }{\CmMarginThmStandardLayoutEnd}

  \newenvironment{bem}[1][]{
    \CmMarginThmStandardLayoutBegin{#1}{\cmLangThmShortBem}
  }{\CmMarginThmStandardLayoutEnd}

}{% use classic amsthm theorems

  \newtheorem{satz}[thmglobal]{\cmLangThmSatz}
  \newtheorem{lemma}[thmglobal]{\cmLangThmLemma}
  \newtheorem{kor}[thmglobal]{\cmLangThmKor}
  \newtheorem{prop}[thmglobal]{\cmLangThmProp}

  \theoremstyle{definition}

  \newtheorem{dfn}[thmglobal]{\cmLangThmDfn}
  \newtheorem{bsp}[thmglobal]{\cmLangThmBsp}

  \theoremstyle{remark}

  \newtheorem{bem}[thmglobal]{\cmLangThmBem}
}

% Add unnumbered Theorems, use amsthm style in both style modes
\theoremstyle{plain}
\newtheorem*{satz*}{\cmLangThmSatz}
\newtheorem*{lemma*}{\cmLangThmLemma}
\newtheorem*{kor*}{\cmLangThmKor}
\newtheorem*{prop*}{\cmLangThmProp}

\theoremstyle{definition}
\newtheorem*{dfn*}{\cmLangThmDfn}
\newtheorem*{bsp*}{\cmLangThmBsp}

\theoremstyle{remark}
\newtheorem*{bem*}{\cmLangThmBem}


% 2-level numbering$
\numberwithin{thmglobal}{section}

% check if 3-level numbering is enabled
\ifthenelse{\boolean{enableDeepNumbering}}{
  \numberwithin{thmglobal}{subsection}
}{}


% some other customisations

% changing enumerations
\setlist[enumerate]{label=(\arabic*), itemsep=0cm, leftmargin=2cm}
\setlist[itemize]{itemsep=0cm, leftmargin=2cm}

% replace the slim emptyset symbol
\let\emptyset\varnothing

% set line distances
\linespread{1.1}

% Add a ':' for mathmode with tiny whitespaces around
\newcommand{\SlimDdot}{\ensuremath{\mathrm{:}}}


% headline and cover generation commands

% generate a simple headline
% usage: \CmHeadline[date]{title}{topic}{author}
\newcommand{\CmHeadline}[4][]{
  \begin{minipage}[t]{\textwidth}
    \huge{\textbf{#2}}\\
    \large{#3, #4}\relax
    \ifthenelse{\isempty{#1}}{}{\relax\large{, #1}}
  \end{minipage}
}

% generate a simple cover page
% usage: \CmCover[type(,skript)]{title}{subtitle}{date}
\newcommand{\CmCover}[4][]{
  \thispagestyle{empty}
  \begin{titlepage}
    \begin{center}
      \begin{minipage}[b]{0.8\textwidth}
	\vspace*{5cm}
        \ifthenelse{\isempty{#1}}{
          % Default cover arrangement
          \Huge{\textbf{#2}}\\[0.5cm]
          \huge{#3}\\[0.8cm]
          \Large{#4}
        }{
          \ifthenelse{\equal{#1}{skript}}{
            % Cover for lecture scripts
            \huge{#3}\\[0.5cm]
            \Huge{\textbf{#2}}\\[0.5cm]
            \Large{#4}
          }{}
        }
      \end{minipage}
    \end{center}
  \end{titlepage}
  \pagebreak
}


% indexing support

% Print and index given text
% usage: \CmIndex{[(optionally put another text for the index in here)]{(text to print and add to index)}
\newcommand{\CmIndex}[2][]{\ifthenelse{\isempty{#1}}{\index{#2}}{\index{#1}}#2}

% Highlight(bold) and index the given text
% usage: \CmMark[(optionally put another text for the index in here)]{(text to highlight and add to index)}
\newcommand{\CmMark}[2][]{\textbf{\CmIndex[#1]{#2}}}


% sectioning support

% Prints a description for a section in italic, bold. Most likely to use right under \section.
% usage: \CmSectionDescription{(short description of section contents)}
\newcommand{\CmSectionDescription}[1]{
  \vspace{-0.3cm}
  \hangindent=0.4cm
  \hangafter=0
  \begin{itshape}
    \textbf{#1}
  \end{itshape}
  \vspace{0.3cm}
}

% Starts a new paragraph inside of a theorem environment (as defined above)
% usage \CmSubThm[(paragraph title)]
\newenvironment{CmSubThm}[1]{
  \begin{itemize}[leftmargin=0.5cm,label=]
  \item
    \ifthenelse{\isempty{#1}}{}{
      \hspace{-0.5cm}(\textit{#1})\\[0.2cm]
    }
  }{
  \end{itemize}
}


% svg updater
% needs shell escape option

% Checks if the given image file has been modified and a custom command (if possible) via command line to generate something new.
\newcommand{\CmExecuteIfFileNewer}[3]{
  \ifnum\pdfstrcmp{\pdffilemoddate{#1}}
  {\pdffilemoddate{#2}}>0
  {\immediate\write18{#3}}\fi
}

% Tries to include an image file and checks if the given one has been modified. If so it calls inkscape (if possible) via command line to generate new pdf und pdf_tex files from the corresponding svg.
\newcommand{\CmIncludeSvg}[1]{
  \def\svg@filepath{}
  
  % check if corrosponding svg file exists
  \IfFileExists{#1.svg}
  {
    \def\svg@filepath{#1}
  }{
    % if it does not, search in the graphicspath for it
    \expandafter\@tfor\expandafter\currentsvgpath\expandafter:\expandafter=\Ginput@path\do{
      \IfFileExists{\currentsvgpath#1.svg}{
        \edef\svg@filepath{\currentsvgpath #1}
      }{}
    }
  }
  % if something was found, include the graphic, TODO: Make it work correctly
  \ifthenelse{\isundefined{\svg@filepath} \OR \isempty{\svg@filepath}}{
    \PackageError{canonicalmath}{Image file not found!}
  }{
    \PackageWarning{FilePath}{|\svg@filepath|}
    \CmExecuteIfFileNewer{\svg@filepath.svg}{\svg@filepath.pdf}{inkscape -z -D --file=\svg@filepath.svg --export-pdf=\svg@filepath.pdf --export-latex}
    \input{\svg@filepath.pdf_tex}
  }
}

% Tries to include an image file on the center of the margin at the current position.
\newcommand{\CmMarginSvg}[3][0cm]{
  \marginnote{
    \centering
    \def\svgwidth{#3}
    \CmIncludeSvg{#2}
  }[#1]
}

% Tries to include an image file centering it at the current position
\newcommand{\CmPutSvg}[3][0cm]{
  \begin{figure}[h!]
    \vspace{#1}
    \centering
    \def\svgwidth{#3}
    \CmIncludeSvg{#2}
  \end{figure}
}
\makeatother


%%%%%%%%%%%%%%%%%%%%%%%%%%%%%%%%%%%%%%%%%%%%%%%%%%%%%%%%%%%%%%%%%%%%%%%%%%%%%%%%%%%%%%%%%%%%%%%%%%%%%%%%%%%%%%%%%%%%%%%%%%%%%%%%%%%%%%%%%%%%%%%%%%%%%%%%%%%%%%%%%%%%%%%%

\usepackage{canonicalsync}
\CsUsePackage[/home/JB/Projects/tex-package-canonical-sync/]{canonicalsync}
\CsUsePackageWithOptions[/home/JB/Projects/tex-package-canonical-math/]{canonicalmath}{marginthm}

\usepackage{graphicx}
\usepackage{float}
\usepackage{xcolor}
\usepackage{transparent}
\usepackage{wrapfig}

\graphicspath{{img/}}

\parindent0pt

\setlist[enumerate]{label=(\arabic*), itemsep=0cm, leftmargin=1cm}

%--------------------------------------------------------------------
%-------------------- Eintraege fuer das Glossar --------------------
%--------------------------------------------------------------------

\newglossaryentry{topologischer Raum}{name=Topologischer Raum,description={Eine Menge \ensuremath{X} zusammen mit einer Topologie \ensuremath{T}, das hei\ss t einem Mengensystem das offene Teilmengen von \ensuremath{X} definiert, wobei die leere Menge, die Grundmenge, der Durchschnitt endlich vieler offener Mengen und die Vereinigung beliebig vieler offener Mengen offen sind},text={topologischer Raum}}

\newglossaryentry{Hausdorff-Raum}{name=Hausdorff-Raum,description={Ein topologischer Raum \ensuremath{M}, in dem es f"ur alle \ensuremath{x, y \in M, x \ne y} disjunkte offene Umgebungen \ensuremath{U(x)} und \ensuremath{U(y)} gibt, es werden also alle paarweise verschiedenen Punkte \ensuremath{x, y} durch Umgebungen getrennt.},text={Hausdorff-Raum}}

%--------------------------------------------------------------------
%-------------------- Hier beginnt das Skript -----------------------
%--------------------------------------------------------------------

\begin{document}

\CmHeadline{Vorlesung Differentialgeometrie}{Gehalten von Dr. Grensing}{Wintersemester 2012/13}

\vspace{0.5cm}

Version \textbf{0.01} \quad Build: \today

\paragraph{Wichtiger Hinweis:}
Dies ist eine Zusammenfassung der Vorlesung "`Differentialgeometrie"' von Dr. Grensing im Wintersemester 2012/13 am KIT und dient lediglich dazu die Inhalte für meine eigene Verwendung besser zusammenzufassen und aufzubereiten. Es besteht weder eine Garantie über Vollständigkeit, noch Korrektheit der enthaltenen Aussagen.\\

Bei Anmerkungen bzw. beim Auffinden von Fehlern schicken Sie bitte eine E-Mail an
\begin{center}
  jan-bernhard.kordass@student.kit.edu
\end{center}

%%
%% 1. Vorlesung 16.10.12
%% 
%% Skript Differentialgeometrie im Wintersemester 12/13
%% Zur Vorlesung von Dr. Grensing am KIT Karlsruhe
%%
%% Mitschrieb und Textsatz von Jan-Bernhard Kordaß.
%%

\section*{"Ubersicht}

\begin{itemize}
\item Mannigfaltigkeiten, Tangentialvektoren
\item Kovariante Ableitung
\item Riemannsche Metriken
\item Krümmung
\item Jacobifelder
\item Satz von Bonnet
\end{itemize}

\section{Differenzierbare Mannigfaltigkeiten}

\begin{dfn*}
  Eine $n$-dimensionale \CmMark{topologische Mannigfaltigkeit} $M$ ist ein topologischer Hausdorff-Raum mit einer abzählbaren Basis der Topologie in dem zu jedem Punkt $p \in M$ eine offene Menge $U$ mit $p \in U$ existiert und ein Hom"oomorphismus $\phi \colon U \to V$ auf eine offene Menge $V \subset \R^{n}$.

% Abbildung 1-1
%\CmPutSvg{1-1-topologische-mf}{8.5cm}
\begin{center}\begin{tikzpicture}[font=\scriptsize]
	\draw[->] (-1.5,0) to[out=20, in=160]node[above,font=\scriptsize]{$\varphi' \circ \varphi^{-1}$} (1.5,0);
	
	\draw[->] (-4,-0.5) -- (-2,-0.5); \draw[->] (-3.75,-0.75) -- (-3.75, 1.25); \node[font=\scriptsize] at (-4, 1.25) {$\R^n$};
	\draw[->] (2,-0.5) -- (4,-0.5); \draw[->] (2.25,-0.75) -- (2.25, 1.25); \node[font=\scriptsize] at (2, 1.25) {$\R^m$};
	
	\node[font=\scriptsize] at (0,2) {$U \cap U' \ne 0$};
	
	\draw (-4.25, 1.75) to[out=70,in=180] (-1.75,3) to[out=300,in=90] (-1.25, 1.25) to[out=180,in=340] (-4.25, 1.75) -- cycle; \node at (-1.25,3) {$M$};
	\filldraw[fill=gray!20] (-2.75,2) circle(0.4); \node[font=\scriptsize] at (-3.25,2.25) {$U$};
	\filldraw[fill=gray!20] (-3,0.25) circle (0.5); \node at (-2.25, 0.5) {$V$};
	\draw[->] (-2.75,1.5) to[out=280,in=80] node[right]{$\varphi$} (-2.75,0.75);
			
	\draw (1.75, 1.75) to[out=70,in=180] (4.25,3) to[out=300,in=90] (4.75, 1.25) to[out=180,in=340] (1.75, 1.75) -- cycle; \node at (4.75,3) {$M$};
	\filldraw[fill=gray!20] (3.55,2.25) circle(0.6); \node[font=\scriptsize] at (2.75,2.25) {$U'$};
		
	\coordinate (ctrl0up) at ($(2.5,-0.25) + 0.2*(0.5,2)$); \coordinate (ctrl0down) at ($(2.5,-0.25) + 0.2*(0,-1.5)$);
	\coordinate (ctrl1down) at ($(3,0.25) - 0.1*(0.5,1)$); \coordinate (ctrl1up) at ($(3,0.25) + 0.1*(0.5,1)$);
	\coordinate (ctrl2down) at ($(3,0.7) - 0.1*(0.5,1)$); \coordinate (ctrl2up) at ($(3,0.7) + 0.1*(0.5,1)$);
	\coordinate (ctrl3down) at ($(4,0.5) + 0.3*(-0.5,1)$); \coordinate (ctrl3up) at ($(4,0.5) - 0.3*(-0.25,1)$);
	\coordinate (ctrl4down) at ($(3.75,-0.3) + 0.2*(0.8,1)$); \coordinate (ctrl4up) at ($(3.75,-0.3) - 0.2*(0.7,0.75)$);
	\begin{scope}
		\fill[gray!20] (2.5,-0.25) ..controls(ctrl0up) and (ctrl1down).. (3,0.25) ..controls(ctrl1up) and (ctrl2down).. (3,0.7) ..controls(ctrl2up) and (ctrl3down).. (4,0.5) ..controls(ctrl3up) and (ctrl4down).. (3.75,-0.3) ..controls(ctrl4up) and (ctrl0down).. (2.5,-0.25); \node at (4.25, 0.5) {$V'$};
		\clip(2.5,-0.25) ..controls(ctrl0up) and (ctrl1down).. (3,0.25) ..controls(ctrl1up) and (ctrl2down).. (3,0.7) ..controls(ctrl2up) and (ctrl3down).. (4,0.5) ..controls(ctrl3up) and (ctrl4down).. (3.75,-0.3) ..controls(ctrl4up) and (ctrl0down).. (2.5,-0.25); \node at (4.25, 0.5) {$V'$};
		\fill[gray] (2,0) circle (1);
		 (2.5,-0.25) ..controls(ctrl0up) and (ctrl1down).. (3,0.25) ..controls(ctrl1up) and (ctrl2down).. (3,0.7) ..controls(ctrl2up) and (ctrl3down).. (4,0.5) ..controls(ctrl3up) and (ctrl4down).. (3.75,-0.3) ..controls(ctrl4up) and (ctrl0down).. (2.5,-0.25); \node at (4.25, 0.5) {$V'$};
	\end{scope}
	\draw  (2.5,-0.25) ..controls(ctrl0up) and (ctrl1down).. (3,0.25) ..controls(ctrl1up) and (ctrl2down).. (3,0.7) ..controls(ctrl2up) and (ctrl3down).. (4,0.5) ..controls(ctrl3up) and (ctrl4down).. (3.75,-0.3) ..controls(ctrl4up) and (ctrl0down).. (2.5,-0.25) -- cycle; \node at (4.25, 0.5) {$V'$};
	\draw[->] (3.5,1.5) to[out=280,in=80] node[right]{$\varphi'$} (3.5,0.75);
\end{tikzpicture}\end{center}

  $\varphi' \circ \varphi^{-1}$ ist ein Hom"oomorphismus offener Mengen des $\R^n$ bzw. $\R^m$. Nach dem Satz von Brouwer (1912) gilt dann $m = n$. Damit ist die Dimension einer zusammenh"angenden topologischen Mannigfaltigkeit eindeutig definiert.\\

  Die Abbildung $\varphi \colon U \to V \subset \R^n$ hei\ss t \CmMark{Karte} von $M$ um $p$, die Menge $U$ hei\ss t \CmMark{Kartengebiet}.\\

  Eine Menge von Karten $\mathcal A = \{(\varphi_{\alpha}, U_{\alpha}) \mid \alpha \in J \}$ hei\ss t \CmMark{Atlas} von $M$, falls $\bigcup_{\alpha \in J}U_{\alpha} = M$.\\

  Ein Atlas $\mathcal A$ von $M$ hei\ss t $C^k$-Atlas, wenn für alle $\alpha, \beta \in J$ mit $U_{\alpha} \cap U_{\beta} \neq \emptyset$ der sogenannte \CmMark{Kartenwechsel}:
  \begin{align*}
    \varphi_{\beta} \circ \varphi_{\alpha}^{-1}\colon \varphi_{\alpha}(U_{\alpha} \cap U_{\beta}) \to \varphi_{\beta}(U_{\alpha} \cap U_{\beta})
  \end{align*}
  ein $C^k$-Diffeomorphismus ist.\\

  % Abbildung 1-2
  %\CmPutSvg{1-2-kartenwechsel}{8cm}
  \begin{center}\begin{tikzpicture}[font=\scriptsize]
  	\draw[->] (-1.5,0) to[out=20, in=160]node[above,font=\scriptsize]{$\varphi_\beta \circ \varphi^{-1}_\alpha$} (1.5,0);
	
	\draw[->] (-4,-0.5) -- (-2,-0.5); \draw[->] (-3.75,-0.75) --node[left]{$\R^n$} (-3.75, 1.25);
	\draw[->] (2,-0.5) -- (4,-0.5); \draw[->] (2.25,-0.75) -- (2.25, 1.25);
	
	\draw[thick]  (-0.25, 3) to[out=0,in = 150] (2,2.5) -- (1.75, 1.5) to[out=190,in=350] (-1.75, 1.5) to[out=90,in=180] (-0.25, 3) -- cycle; \node at (2.25,2.75) {$M$};
	
	\begin{scope}
		\clip (0.25,2.25) circle(0.5);
		\clip (-0.25,2) circle(0.5);
		\fill[gray!20] (0,2) circle(1);
	\end{scope}
	\draw (0.25,2.25) circle(0.5) (-0.25,2) circle(0.5); \node at (-1, 2.25) {$U_\alpha$}; \node at (1,2.5) {$U_\beta$};
	
	\draw[->] (-0.5,2) to[out=180,in=75] node[left]{$\varphi_\alpha$} (-3,0.25);
	\draw[->] (0.5,2.25) to[out=0,in=105] node[right]{$\varphi_\beta$} (3,0.25);
  \end{tikzpicture}\end{center}


  Eine Karte $\psi \colon U \to V$ von $M$ hei\ss t \CmMark{verträglich} mit einem $C^k$-Atlas $\mathcal A = \{(\varphi_{\alpha},U_{\alpha}) \mid \alpha \in J\}$ wenn jeder Kartenwechsel
  \begin{align*}
    \varphi_{\alpha} \circ \psi(U \cap U_{\alpha}) \to \varphi_{\alpha}(U \cap U_{\alpha})
  \end{align*}
  ein $C^k$-Diffeomorphismus ist, i.e. $\mathcal A' = \mathcal A \cup \{(\psi, U)\}$ ist ebenfalls ein $C^k$-Atlas.\\

  Die Menge aller mit $\mathcal A$ verträglichen Karten ist ein \CmMark{maxmaler $C^k$-Atlas}. Jeder maximale Atlas enthält alle mit ihm verträglichen Karten. Ein maximaler $C^k$-Atlas hei\ss t auch \CmMark{$C^k$-differenzierbare Struktur}.

\end{dfn*}

\begin{dfn}[differenzierbare Mannigfaltigkeit]
  Eine \CmMark{differenzierbare Mannigfaltigkeit} der Klasse $C^k$ ist eine topologische Mannigfaltigkeit zusammen mit einer $C^{k}$-differenzierbaren Struktur.\\
\end{dfn}

\begin{bsp}
  Einige Beispiele f"ur glatte Mannigfaltigkeiten:
  \begin{enumerate}%[1)]
  \item $M = \R^n, \mathcal A = \{(\Id_{\R^n},\R^n)\}$
  \item $M \subset \R^n$ offen, $\mathcal A = \{(\imath_{M},M)\}$
  \item $S^1 \subset \R^2$ ist eine eindimensionale $C^{\infty}$-Mannigfaltigkeit:
    \begin{align*}
      U = \{(\sin t, \cos t) \mid t \in (0,2\pi)\}
    \end{align*}

    % Abbildung 1-3
    \marginnote{\begin{center}\begin{tikzpicture}[font=\footnotesize]
    		%\draw[step=0.25,gray!15] (-1,-1) grid (1,1); \draw[step=0.5,gray!30] (-1,-1) grid (1,1); \fill (0,0) circle(0.1); %Hilfsgitter
		\draw (0,0) circle (1); \draw[dashed] (0,0) circle (1.1); \draw[dotted] (0,0) circle (0.9); \node at (1,1) {$S^1$};
		\filldraw[fill=white] (-1,0) circle (0.1) (1,0) circle (0.1);
    \end{tikzpicture}\\
    \textcolor{gray}{$S^1$ Einheitskreis}
    \end{center}}[-2cm]
    % \CmMarginSvg[-2cm]{1-3-karten-der-s1}{3cm}

    ist offen in $S^1$ und die Kartenabbildung
    \begin{align*}
      \varphi \colon (\sin t, \cos t) \mapsto t
    \end{align*}
    ist ein Hom"oomorphismus.
    \begin{align*}
      \varphi' \colon U' = \{(\sin t, \cos t) \mid t \in (-\pi,\pi)\} \to (-\pi,\pi)
    \end{align*}
    ebenfalls. $\mathcal A = \{(\varphi, U), (\varphi',U')\}$ ist ein Atlas von $S^1$, denn $U \cup U' = S^1$.
    \begin{align*}
      & \varphi' \circ \varphi^{-1} \colon \varphi(U \cap U') \to \varphi'(U \cap U')\\
      & (0,\pi)\cup(\pi,2\pi) \to (-\pi,0)\cup(0,\pi), t \mapsto \begin{cases}
        t & 0 < t < \pi\\
        t-2\pi & \pi < t < 2\pi
      \end{cases}
    \end{align*}

  \item Jeder reelle Vektorraum endlicher Dimension ist in kanonischer Weise eine $C^{\infty}$-Mannigfaltigkeit.\\
    W"ahle eine Basis $\{v_1, \ldots, v_n\}$ von $V$. Diese definiert mit
    \begin{align*}
      \varphi\left(\sum\lambda_iv_i\right) = (\lambda_1, \ldots, \lambda_n)
    \end{align*}
    eine Bijektion auf $\R^n$. Damit erhält man eine globale Karte von $V$.
    Der zugehörige Atlas h"angt nicht von der Wahl der Basis ab, denn ist $\{w_1, \ldots, w_n\}$ eine weitere Basis von $V$ und $\psi(\sum \lambda_iw_i) = (\lambda_1, \ldots, \lambda_n)$ eine weitere Karte, so ist $\varphi \circ \psi^{-1}$ als Endomorphismus des $\R^n$ schon $C^{\infty}$.

  \item $S^n = \{(x^0, x^1, \ldots, x^n) \mid \sum_{i = 0}^n(x^{i})^2 = 1\}$.\\

    % Abbildung 1.4
    %\CmMarginSvg{1-4-s3-sphaere}{3.5cm}
    \marginnote{\begin{center}\begin{tikzpicture}[font=\scriptsize]
    		%\draw[step=0.25,gray!15] (-1,-1) grid (1,1); \draw[step=0.5,gray!30] (-1,-1) grid (1,1); \fill (0,0) circle(0.1); %Hilfsgitter
		% Koordinatenachsen mit Beschriftung
		\draw[->] (0,-1.25) -- (0,1.5) node[left]{$x^0$}; \draw[->] (-1.25,0) -- (1.5,0) node[below]{$x^1$}; \draw[->] (1,1) -- (-1.25,-1.25) node[right]{$x^2$}; \node at (1.25, 1.5) {$S^2 \subset \R^3$};
		% Kreis, Ellipse und Gerade (verwende Namen um Schnittpunkt bestimmen zu koennen)
		\path[draw, thick, name path=kreis] (0,0) circle (1) ellipse(1 and 0.5); \path[draw,name path=gerade] (0,1) -- (1,-1.25);
		% Punkte N und p
		\filldraw[fill=white] (0,1) circle (0.05) node[anchor=south west,xshift=-2,yshift=-1.5]{$N$} ($(0,1)+0.35*(1,-1.25)-0.35*(0,1)$) circle (0.05) node[right]{$p$};
		% Punkt phi(p) bei Schnittpunkt von Gerade und Kreis
		\path [name intersections={of=kreis and gerade}]; \filldraw[fill=white] (intersection-2) circle(0.05) node[right]{$\varphi(p)$};
	\end{tikzpicture}\end{center}}%[3.5cm]
    
    Betrachte den Nordpol $N = (1,0,\ldots,0)$ und den S"udpol $S = (-1,0,\ldots,0)$ und die Abbildung
    \begin{align*}
      & \varphi \colon U = S^{n}\setminus\{N\} \to \R^n, x \mapsto \left(\frac{x^1}{1-x^0}, \ldots, \frac{x^{n}}{1-x^0}\right),\\
      & \psi \colon U' = S^{n} \setminus \{S\} \to \R^n, x \mapsto \left(\frac{x^1}{1+x^0}, \ldots, \frac{x^n}{1+x^0}\right)
    \end{align*}

    Aufgabe: Zeige, dass $(\varphi, U), (\psi, U')$ einen $C^{\infty}$-Atlas auf $S^n$ definiert.

  \end{enumerate}
\end{bsp}

\begin{dfn}[Differenzierbare Abbildungen]
Eine stetige Abbildung $f \colon M \to N$ zwischen glatten Mannigfaltigkeiten $M$ und $N$ hei\ss t \CmMark{glatt} ($C^{\infty}$-differenzierbar), wenn es zu jedem $p \in M$ Karten $(\varphi, U)$ in $M$ um $p$ und geeignete $(\varphi', U')$ in $N$ um $f(p)$ gibt, so dass $\varphi' \circ f\circ\varphi^{-1}$ glatt ist.
% Abbildung 1-5
%\CmPutSvg{1-5-glatte-abb}{9cm}
\begin{center}\begin{tikzpicture}[font=\scriptsize]
	%\draw[step=0.25,gray!15] (-5,-1) grid (5,5); \draw[step=0.5,gray!30] (-5,-1) grid (5,5); \fill (0,0) circle(0.1); %Hilfsgitter
	
	% Die Abbildungspfeile
	\draw[->] (-1.5,0) to[out=20, in=160]node[above]{$\varphi' \circ f \circ \varphi$} (1.5,0);
	\draw[->] (-1,2) --node[above]{$f$} (1.5,2);
	
	% Die Achsen
	\draw[->] (-4.5,-0.5) -- (-2,-0.5); \draw[->] (-4.25,-0.75) --node[left]{$\R^n$} (-4.25, 1.25);
	\draw[->] (2,-0.5) -- (4.5,-0.5); \draw[->] (2.25,-0.75) --node[left]{$\R^m$} (2.25, 1.25);
	
	% Die Blasen
	\draw[thick] (-4.25, 1.75) to[out=70,in=180] (-1.75,3) to[out=300,in=90] (-1.25, 1.25) to[out=180,in=340] (-4.25, 1.75) -- cycle; \node[font=\normalfont] at (-1.25,3) {$M$};
	\draw[thick] (1.75, 1.75) to[out=70,in=180] (4.25,3) to[out=300,in=90] (4.75, 1.25) to[out=180,in=340] (1.75, 1.75) -- cycle; \node[font=\normalfont] at (4.75,3) {$N$};
	
	% Die linke Kartoffel (zuerst werden die Punkte definiert, dann die Richtungsvektoren der Splines, dann die Kartoffel selbst)
	\coordinate (kartoffel0l) at (-3.25,1.75); \coordinate (kartoffel1l) at (-3.25,2.5); \coordinate (kartoffel2l) at (-2.25,2.25); \coordinate (kartoffel3l) at (-2.5,1.75);
	\coordinate (ctrlk0l) at (-0.25,0.5); \coordinate (ctrlk1l) at (0.5,0.25); \coordinate (ctrlk2l) at (-0.25,1); \coordinate (ctrlk3l) at (2,0.25);
	\draw (kartoffel0l) ..controls($(kartoffel0l)+0.5*(ctrlk0l)$) and ($(kartoffel1l)-0.3*(ctrlk1l)$).. (kartoffel1l) ..controls($(kartoffel1l)+0.6*(ctrlk1l)$) and($(kartoffel2l)+0.45*(ctrlk2l)$).. (kartoffel2l) ..controls($(kartoffel2l)-0.25*(ctrlk2l)$) and ($(kartoffel3l)+0.15*(ctrlk3l)$).. (kartoffel3l)  ..controls($(kartoffel3l)-0.1*(ctrlk3l)$) and ($(kartoffel0l)-0.9*(ctrlk0l)$).. (kartoffel0l); \node at (-3.5,2.25) {$U$};
	% Der Punkt in der Kartoffel, der Pfeils raus und der Kreis
	\draw[->] (-2.75,2) node[right]{$p$} to[out=280,in=80] node[right]{$\varphi$} (-2.75,0); \fill (-2.75,2) circle (0.05);
	\draw (-3,0.25) circle(0.5); \node at (-3.5,0.75) {$V$};
	
	% Die rechte Kartoffel
	\coordinate (kartoffel0r) at (3.25,1.75); \coordinate (kartoffel1r) at (3.5,2.5); \coordinate (kartoffel2r) at (4.5,2.25); \coordinate (kartoffel3r) at (4.25,1.5);
	\coordinate (ctrlk0r) at (-0.25,0.5); \coordinate (ctrlk1r) at (-0.25,0.25); \coordinate (ctrlk2r) at (-0.25,0.5); \coordinate (ctrlk3r) at (0.25,0);
	\draw (kartoffel0r) ..controls($(kartoffel0r)+0.5*(ctrlk0r)$) and ($(kartoffel1r)-(ctrlk1r)$).. (kartoffel1r) ..controls($(kartoffel1r)+(ctrlk1r)$) and ($(kartoffel2r)+(ctrlk2r)$).. (kartoffel2r) ..controls($(kartoffel2r)-(ctrlk2r)$) and ($(kartoffel3r)+(ctrlk3r)$).. (kartoffel3r) ..controls($(kartoffel3r)-(ctrlk3r)$) and ($(kartoffel0r)-(ctrlk0r)$).. (kartoffel0r); \node at (3.25,2.25) {$U'$};
	
	\draw[->] (3.75,2) node[right]{$f(p)$} to[out=280,in=80] node[right]{$\varphi'$} (3.75,0); \fill (3.75,2) circle (0.05);
	\draw (3.5,0.25) circle(0.5); \node at (3,0.75) {$V'$};
\end{tikzpicture}

\textcolor{red}{Sollte das in der Zeichnung beim unteren Pfeil nicht $\varphi'\circ f \circ \varphi^{-1}$ hei\ss en?}\end{center}
Die Menge aller glatten Abbildungen von $M$ nach $N$ wird $C^{\infty}(M,N)$ genannt.

\end{dfn}

\textbf{Konvention}: Ab jetzt seien zunächst alle Mannigfaltigkeiten, wie auch alle Abbildungen als glatt vorrausgesetzt.

\begin{bem}
  Da Kartenwechsel $C^{\infty}$ sind, gilt obige Bedingung automatisch für alle Karten von $M$ und $N$ (evtl. nach Einschränkung).
\end{bem}

\begin{bsp}
  Es folgen zwei Beispiele für diffenrenzierbare Abbildungen:
  \begin{enumerate}
  \item $(\varphi,U) \in \mathcal A \Rightarrow \varphi \in C^{\infty}(U,\R^n)$, denn
    \begin{align*}
      \Id_{\R^n}\circ \varphi \circ \varphi^{-1} = \varphi \circ \varphi^{-1} \in C^{\infty}.
    \end{align*}
  \item $f \in C^{\infty}(M,N), \ g \in C^{\infty}(N,P) \Rightarrow g \circ f \in C^{\infty}(M,P)$, denn
    \begin{align*}
      \varphi_p \circ g \circ f \circ \varphi^{-1}_m = (\varphi_p \circ g \circ \varphi_n^{-1}) \circ (\varphi_n \circ f \circ \varphi_m^{-1}) \in C^{\infty}.
    \end{align*}
  \end{enumerate}
\end{bsp}

\begin{dfn}[Diffeomorphismus]
  Eine Abbildung $f \colon M \to N$ hei\ss t \CmMark{Diffeomorphismus}, wenn $f$ bijektiv ist und $f$, sowie $f^{-1}$ $C^{\infty}$-Abbildungen von $M$ nach $N$ sind. Insbesondere haben $M$ und $N$ in diesem Fall dieselbe Dimension.\\

Die Menge der Diffeomorphismen von $M$ nach $N$ wird mit $\Diff(M,N)$ bezeichnet. Die Menge der Diffeomorphismen von $M$ nach $M$ wird mit $\Diff(M)$ bezeichnet. $(\Diff(M), \circ)$ ist eine Gruppe.

\end{dfn}

%%% Local Variables: 
%%% mode: latex
%%% TeX-master: "../skript-diffgeom"
%%% End: 


%-_-_-_-_-_-_-_-_-_-_-_-_-_-_ Anhang -_-_-_-_-_-_-_-_-_-_-_-_-_-_-_-_

\appendix

%-_-_-_-_-_-_-_-_-_-_-_-_-_-_ Uebungen -_-_-_-_-_-_-_-_-_-_-_-_-_-_-_-_

%-_-_-_-_-_-_-_-_-_-_-_-_-_-_ Stichwortverzeichnis -_-_-_-_-_-_-_-_-_-_-_-_-_-_-_-_

\printindex

%-_-_-_-_-_-_-_-_-_-_-_-_-_-_ Symbolverzeichnis -_-_-_-_-_-_-_-_-_-_-_-_-_-_-_-_

\printglossaries

%-_-_-_-_-_-_-_-_-_-_-_-_-_-_ Literaturverzeichnis -_-_-_-_-_-_-_-_-_-_-_-_-_-_-_-_

\end{document}
