\newglossaryentry{Dualraum}{
	name=Dualraum,
	description={Die zu einem Vektorraum \ensuremath{V} "uber einem K"orper \ensuremath{K} geh"orende Menge aller linearen Abbildungen von \ensuremath{V} nach \ensuremath{K}. Der Dualraum selbst ist ebenfalls ein Vektorraum mit Skalarmultiplikation mit Elementen aus \ensuremath{K}},
	text={Dualraum}
}

\newglossaryentry{topologischer Raum}{
	name=Topologischer Raum,
	description={Eine Menge \ensuremath{X} zusammen mit einer Topologie \ensuremath{T}, das hei\ss t einem Mengensystem das offene Teilmengen von \ensuremath{X} definiert, wobei die leere Menge, die Grundmenge, der Durchschnitt endlich vieler offener Mengen und die Vereinigung beliebig vieler offener Mengen offen sind},
	text={topologischer Raum}
}

\newglossaryentry{Hausdorff-Raum}{
	name=Hausdorff-Raum,
	description={Ein topologischer Raum \ensuremath{M}, in dem es f"ur alle \ensuremath{x, y \in M, x \ne y} disjunkte offene Umgebungen \ensuremath{U(x)} und \ensuremath{U(y)} gibt, es werden also alle paarweise verschiedenen Punkte \ensuremath{x, y} durch Umgebungen getrennt},
	text={Hausdorff-Raum}
}

\newglossaryentry{GL}{
	name={GL\ensuremath{_n}},
	description={Allgemeine lineare Gruppe, Gruppe aller regul"aren $n \X n$-Matrizen mit Koeffizienten aus einem K"orper $K$},text={\ensuremath{\Gl}},
	symbol={\ensuremath{\Gl}},
	sort=GL
}

\newglossaryentry{Topologie}{
	name=Topologie,
	description={Ein Mengensystem das Teilmengen einer Grundmenge als offene Mengen definiert, wobei die leere Menge und die Grundmenge selbst offen sind und der Durchschnitt endlich vieler offener Mengen und die Vereinigung beliebig vieler offener Mengen wieder offen sind}
}

\newglossaryentry{Homoeomorphismus}{
	name={Hom"oomorphismus},
	description={Eine bijektive, stetig differenzierbare Abbildung zwischen zwei Objekten, deren Umkehrabbildung ebenfalls stetig differenzierbar ist},
	text={Hom{\"o}omorphsimus}
	sort={Homoomorphismus}
}

\newglossaryentry{Diffeomorphismus}{
	name={Diffeomorphismus},
	description={Eine bijektive, stetige Abbildung zwischen zwei Objekten, deren Umkehrabbildung ebenfalls stetig ist}
}