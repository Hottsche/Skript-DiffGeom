\usepackage{fancyhdr} % erlaubt mehr Optionen in Kopf- und Fusszeile

% header configuration
%\pagestyle{fancy}
%\fancyhf{
%\lhead[\thepage]{\rightmark}}
%\rhead[\nouppercase{\leftmark}]{\thepage}		

\usepackage{xcolor} % Farben
\usepackage{hyperref} % Verweise als Hyperlinks
\usepackage{marginnote} % Randnotizen
\usepackage{enumitem} % Fuer mehr Einstellungmoeglichkeiten bei Aufzaehlungen
\usepackage{xifthen} % Erlaubt die Verwendung von if-then-else Befehlen im Code
\usepackage{index} % Index erzeugen
\newindex{default}{idx}{ind}{Stichwortverzeichnis}
\usepackage{xspace} % intelligende Leerzeichen bei Macros
\usepackage[normalem]{ulem} % unterstreichen von Text
\usepackage{cancel} % schraeg durchstreichen von Text

\renewcommand{\CancelColor}{\color{gray}} % Farbe zum schraegen Druchstreichen in grau

\definecolor{rltred}{rgb}{0.75,0,0}
\definecolor{rltgreen}{rgb}{0,0.5,0}
\definecolor{rltblue}{rgb}{0,0,0.75}

%sichere Fraben, die sich auch bei einem SW-Druck unterscheiden lassen (Platzhalter momentan)
\definecolor{color1}{cmyk}{1,0,0,0} %cyan
\definecolor{color2}{rgb}{0,1,0} %green

\hypersetup{
  pdftitle={Differentialgeometrie Dr. Grensing},
  pdfsubject={Differentialgeometrie Geometrie},
  pdfkeywords={Differentialgeometrie Grensing},
  pdfproducer={pdfLaTeX},
  pdfpagemode={UseOutlines},
  colorlinks=true,
  bookmarksopen=true,
  bookmarksnumbered=true,
  urlcolor=rltblue,
  filecolor=rltgreen,
  linkcolor=rltblue,
  backref=true,
  pagebackref=true,
  pdfpagemode=None,
  citecolor=rltblue
}

\usepackage{tikz} % Fuer Zeichnungen in TikZ
\usetikzlibrary{matrix,arrows,calc,intersections, positioning, patterns, decorations.text, decorations.pathmorphing}

% neue Befehle fuer haeufig benutzte TikZ Formen; erstes Argument steht fuer die Position, Zweites fuer die Groesse
\newcommand{\tikzschnuller}[2][1]{
	% definiere die Knoten relativ zum ersten Knoten skaliert mit dem Faktor
	\coordinate (schnuller1) at #2; \coordinate (schnuller2) at ($(schnuller1)+#1*(-1.75,-0.75)$); \coordinate (schnuller3) at ($(schnuller1)+#1*(-2.5,-2.25)$); \coordinate (schnuller4) at ($(schnuller1)+#1*(0,-2)$); \coordinate (schnuller5) at ($(schnuller1)+#1*(1.75,-0.25)$);
    %\fill (schnuller1) circle (0.05) (schnuller2) circle (0.05) (schnuller3) circle (0.05) (schnuller4) circle (0.05) (schnuller5) circle (0.05);
    
    % die Richtungsvektoren der Bezier Tangenten fuer die einzelnen Knoten (der Erste und der letzte haben keine Tangente)
    \coordinate (ctrls1) at ($#1*(1.25,0.25)$); \coordinate (ctrls2) at ($-0.5*(ctrls1)$); \coordinate (ctrls4) at ($#1*(1,-1)$); \coordinate (ctrls3) at ($-0.5*(ctrls4)$); \coordinate (ctrls6) at ($#1*(1,1.5)$); \coordinate (ctrls5) at ($-0		.33*(ctrls6)$);
	% die eigentlichen Tangenten
    \coordinate (tang1) at ($(schnuller2)+(ctrls1)$); \coordinate (tang2) at ($(schnuller2)+(ctrls2)$); \coordinate (tang3) at ($(schnuller3)+(ctrls3)$); \coordinate (tang4) at ($(schnuller3)+(ctrls4)$); \coordinate (tang5) at ($(schnuller4)+(ctrls5)$); \coordinate (tang6) at ($(schnuller4)+(ctrls6)$);
    %\fill[red] (tang1) circle (0.05); \fill[red] (tang2) circle (0.05); \fill[red] (tang3) circle (0.05); \fill[red] (tang4) circle (0.05); \fill[red] (tang5) circle (0.05); \fill[red] (tang6) circle (0.05);
    %\draw[red] (tang1) -- (tang2); \draw[red] (tang3) -- (tang4); \draw[red] (tang5) -- (tang6);
	
	\draw (schnuller1) ..controls(schnuller1) and (tang1).. (schnuller2) ..controls(tang2) and (tang3).. (schnuller3) ..controls(tang4) and (tang5).. (schnuller4) ..controls(tang6) and (schnuller5).. (schnuller5);
	
	% zeichne nun das Loch in der Mitte
	\def\angle{20} % Rotationswinkel
	\coordinate (c) at ($#2+#1*(-1.25,-1.25)$); % Mittelpunkt der Ellipse die den unteren Bogen bildet
	\begin{scope}
		\clip[rotate=\angle] ($(c)-#1*(1,0.6)$) rectangle ($(c)+#1*(1,-0.1)$);
		\path[draw,rotate=\angle,name path=l] (c) ellipse(#1*1 and #1*0.5);
	\end{scope}
	\path[name path=u,rotate=\angle] ($(c)-#1*(0,0.5)$) ellipse(#1*0.75 and #1*0.5);
	\path[name intersections={of=u and l}];
	\begin{scope}
		\clip[rotate=\angle] (intersection-1) rectangle ($(intersection-2)+#1*(0,0.5)$);
		\draw[rotate=\angle] ($(c)-#1*(0,0.5)$) ellipse(#1*0.75 and #1*0.5);
	\end{scope}		
}
%\newcommand{\tikzkartoffel}[2][1]{}		
\newcommand{\tikzsegel}[2][1]{
	% definiere die Knoten relativ zum ersten Knoten skaliert mit dem Faktor
	\coordinate (segel1) at #2; \coordinate (segel2) at ($(segel1)+#1*(3.5,1.5)$); \coordinate (segel3) at ($(segel1)+(2.5,0)$);
	%\fill (segel1) circle (0.05) (segel2) circle (0.05) (segel3) circle (0.05);
	
	% die Richtungsvektoren der Bezier Tangenten fuer die einzelnen Knoten (der Erste und der letzte haben keine Tangente)
	\coordinate (ctrls1) at ($#1*(1,1.5)$); \coordinate (ctrls2) at ($#1*(-0.5,0.25)$); \coordinate (ctrls3) at ($#1*(-0.25,-0.25)$); \coordinate (ctrls4) at ($#1*(0,1)$); \coordinate (ctrls5) at ($#1*(-0.5,0.25)$); \coordinate (ctrls6) at ($#1*(1,0.25)$);
	% die eigentlichen Tangenten
	\coordinate (tang1) at ($(segel1)+(ctrls1)$); \coordinate (tang2) at ($(segel2)+(ctrls2)$); \coordinate (tang3) at ($(segel2)+(ctrls3)$); \coordinate (tang4) at ($(segel3)+(ctrls4)$); \coordinate (tang5) at ($(segel3)+(ctrls5)$); \coordinate (tang6) at ($(segel1)+(ctrls6)$);
	%\fill[red] (tang1) circle (0.05); \fill[red] (tang2) circle (0.05); \fill[red] (tang3) circle (0.05); \fill[red] (tang4) circle (0.05); \fill[red] (tang5) circle (0.05); \fill[red] (tang6) circle (0.05);
    %\draw[red] (tang1) -- (segel1) -- (tang6); \draw[red] (tang2) -- (segel2) -- (tang3); \draw[red] (tang4) -- (segel3) -- (tang5);
	
	\draw (segel1) ..controls(tang1) and (tang2).. (segel2) ..controls(tang3) and (tang4).. (segel3) ..controls(tang5) and (tang6).. (segel1) --cycle;
}
\newcommand{\tikztorus}[2][1]{
	% \draw[step=0.25,gray!15] (-6,-1) grid (6,5); \draw[step=0.5,gray!30] (-6,-1) grid (6,5); \fill (0,0) circle(0.1); %Hilfsgitter
	% zuerst die aeussere Ellips
	\draw[] #2  ellipse (#1*2 and #1*1);
	
	% dann das Loch
	\begin{scope}
      \clip ($#2 - #1*(1, 0.5)$) rectangle ($#2 + #1*(1, 1)$);
      \path[draw,name path=gkreis] ($#2 + #1*(0,0.75)$) ellipse (#1*1.25 and #1*1);
    \end{scope}
    \path[name path=kkreis] ($#2 - #1*(0,0.5)$) ellipse (#1*1 and #1*0.75);
    \path[name intersections={of=gkreis and kkreis}];
    \begin{scope}
      \clip (intersection-1) rectangle ($(intersection-2)+(0,0.5)$);
      \draw ($#2 - #1*(0,0.5)$) ellipse (#1*1 and #1*0.75);
    \end{scope}
    
    % definiere Werte auf die wir in der restlichen Zeichnung zurueckgreifen koennen
	\def\torusbreite{#1*2}
	\def\torushoehe{#1*1}
	\def\torusdicke{#1*0.75}
	\coordinate (torusUntenLoch) at ($#2 - #1*(0,0.25)$);
	\coordinate (torusUnten) at ($#2 - #1*(0,1)$);
}

\usepackage[toc]{glossaries} % Symbolverzeichnis
\glossarystyle{treehypergroup}
\makeglossaries

% Mathe Pakete
\usepackage{amsmath}
\usepackage{amssymb}
%\usepackage{amsthm}
\usepackage[hyperref,amsmath,thmmarks,thref, amsthm]{ntheorem}

% common mathematical operators and sets
\DeclareMathOperator{\aff}{aff} % affine Huelle
\DeclareMathOperator{\ddet}{det} % Determinante
\DeclareMathOperator{\diam}{diam} % diameter
\DeclareMathOperator{\dist}{dist} % distance
\DeclareMathOperator{\ddim}{dim} % dimension
\DeclareMathOperator{\ggT}{ggT} % goesster gemeinsamer Teiler
\DeclareMathOperator{\inh}{inh} % Inhalt
\DeclareMathOperator{\grad}{grad} % Gradient
\DeclareMathOperator{\kgV}{kgV} % kleinstes gemeinsames Vielfaches
\DeclareMathOperator{\mspan}{span} % Lineare Huelle
\DeclareMathOperator{\n}{n} % Umlaufzahl
\DeclareMathOperator{\offen}{offen}
\DeclareMathOperator{\pr}{pr}
\DeclareMathOperator{\res}{res} % Residuum
\DeclareMathOperator{\rg}{rg} % rank (i)
\DeclareMathOperator{\sgn}{sgn} % Signum
\DeclareMathOperator{\supp}{supp} % support
\DeclareMathOperator{\sternf}{sternf}

\DeclareMathOperator{\Abb}{Abb} % maps
\DeclareMathOperator{\Aut}{Aut} % automorphisms
\DeclareMathOperator{\Bild}{Bild}
\DeclareMathOperator{\Charakteristik}{char}
\DeclareMathOperator{\Charakt}{char}
\DeclareMathOperator{\Diff}{Diff}
\DeclareMathOperator{\End}{End} % endomorphisms
\DeclareMathOperator{\Gr}{Gr}
\DeclareMathOperator{\Graph}{Graph}
\DeclareMathOperator{\Hom}{Hom} % homomorphisms
\DeclareMathOperator{\Id}{Id} % identity
\DeclareMathOperator{\Inn}{Inn} % Untergruppe der inneren Automorphismen
\DeclareMathOperator{\Kern}{Kern}
\DeclareMathOperator{\Oo}{O} % Matrizen sie mit ihrer Transponierten multipiziert die Einheitsmatrix ergeben
\DeclareMathOperator{\Relation}{\scriptstyle\mathrm{R}} % custom Relation
\DeclareMathOperator{\Rang}{Rang} % rank (ii)
\DeclareMathOperator{\SL}{SL} % Matrizen mit Deteminante 1
\DeclareMathOperator{\Stab}{Stab} % Stabilisator
\DeclareMathOperator{\Sym}{Sym} % symmetric group

\newcommand{\Zentrum}[1]{\ensuremath{\mathrm Z(#1)}} % Zentrum einer Gruppe
\newcommand{\Ordnung}[1][]{ % Ordnung einer Gruppe
  \ifthenelse{\isempty{#1}}{
    \#
  }{
    \left|#1\right|
  }
}
\newcommand{\X}{\times}

%Realteil und Imaginaerteil
\renewcommand{\Re}{\ensuremath{\operatorname{Re}}} % <-- sollte man da nicht besser \DeclareMathOperator verwenden?
\renewcommand{\Im}{\ensuremath{\operatorname{Im}}}

% \DeclareMathOperator{\Real}{Re} % real part
% \DeclareMathOperator{\Imag}{Im} % imaginary part

% canonic sets
\DeclareMathOperator{\C}{\mathbb{C}}
\DeclareMathOperator{\F}{\mathbb{F}}
\DeclareMathOperator{\K}{\mathbb{K}}
\DeclareMathOperator{\N}{\mathbb{N}}
\DeclareMathOperator{\Q}{\mathbb{Q}}
\DeclareMathOperator{\R}{\mathbb{R}}
\DeclareMathOperator{\RP}{\mathbb{RP}} % real projection plane
\DeclareMathOperator{\Tor}{\mathbb{T}} % torus
\DeclareMathOperator{\Z}{\mathbb{Z}}

%  geschwungene Buchstaben
\DeclareMathOperator{\calD}{\mathcal{D}}
\DeclareMathOperator{\calI}{\mathcal{I}}
\DeclareMathOperator{\calJ}{\mathcal{J}}
\DeclareMathOperator{\calV}{\mathcal{V}}

% Redeclare \P (Prim or Propability) and put the old, reversed "breakline P" in \BreakLineP
\let\BreakLineP\P
\renewcommand{\P}{\ensuremath{\mathbb{P}}}

% canonic differentiation, 
\DeclareMathOperator{\T}{T} % tangent bundle
\DeclareMathOperator{\D}{D} % Jacobi matrix or derivative

% Differentialoperatoren als Brüche
\newcommand{\dop}{\mathrm{d}}	
%\newcommand{\difffrac}[3][]{\ifthenelse{\isempty{#1}}{\dfrac{\dop #2}{\dop #3}}{\left. \dfrac{\dop #2}{\dop #3} \right|_{#1}}}
\newcommand{\difffrac}[3][]{\ifthenelse{\isempty{#1}}{\frac{\dop #2}{\dop #3}}{\left. \frac{\dop #2}{\dop #3} \right|_{#1}}}
%\newcommand{\pdifffrac}[3][]{\ifthenelse{\isempty{#1}}{\dfrac{\partial #2}{\partial #3}}{\left. \dfrac{\partial #2}{\partial #3} \right|_{#1}}}
\newcommand{\pdifffrac}[3][]{\ifthenelse{\isempty{#1}}{\frac{\partial #2}{\partial #3}}{\left. \frac{\partial #2}{\partial #3} \right|_{#1}}}

% stellt einen großen vertikalen Strich an einen Term, nuetzlich in Bruechen
\newcommand{\bigvert}[1]{\left. #1 \right|}

% quotient space or group
\newcommand{\modulo}[1]{\ensuremath{/_{\displaystyle #1}}}

% declaring Index for group theory
\newcommand{\Index}[2]{\ensuremath{(#1 \SlimDdot #2)}}


% canonic environments
\newcounter{thmglobal}
%\swapnumbers
\theoremstyle{plain}

%%%%%%%%%%%%%%%%%%%%%%%%%%%%%%%%%%%%%%%%%%%%%%%%%%%%%%%%%%%%%%%%%%%%%%%%%%%%%%%%%%%%%%%%%%%%%%%%%%%%%%%%%%%%%%%%%%%%%%%%%%%%%%%%%%%%%%%%%%%%%%%%%%%%%%%%%%%%%%%%%%%%%%%%

\makeatletter

% Options
\newboolean{enableDeepNumbering}
\setboolean{enableDeepNumbering}{false}

\DeclareOption{deepnum}{
  \setboolean{enableDeepNumbering}{true}
}

\newboolean{enableMarginThm}
\setboolean{enableMarginThm}{false}

\DeclareOption{marginthm}{
  \setboolean{enableMarginThm}{true}
}

\ProcessOptions\relax

% call makeindex for an index register
%\makeindex

% Name language settings

% theorem names, ngerman
\newcommand{\cmLangThmSatz}{Satz\xspace}
\newcommand{\cmLangThmLemma}{Lemma\xspace}
\newcommand{\cmLangThmKor}{Korollar\xspace}
\newcommand{\cmLangThmProp}{Proposition\xspace}

\newcommand{\cmLangThmDfn}{Definition\xspace}
\newcommand{\cmLangThmBsp}{Beispiel\xspace}

\newcommand{\cmLangThmBem}{Bemerkung\xspace}

% short forms, ngerman
\newcommand{\cmLangThmShortSatz}{Satz\xspace}
\newcommand{\cmLangThmShortLemma}{Lemma\xspace}
\newcommand{\cmLangThmShortKor}{Kor\xspace}
\newcommand{\cmLangThmShortProp}{Prop\xspace}

\newcommand{\cmLangThmShortDfn}{Def\xspace}
\newcommand{\cmLangThmShortBsp}{Bsp\xspace}

\newcommand{\cmLangThmShortBem}{Bem\xspace}

% use classic amsthm theorems

%  \newtheorem{satz}[thmglobal]{\cmLangThmSatz}
%  \newtheorem{lemma}[thmglobal]{\cmLangThmLemma}
%  \newtheorem{kor}[thmglobal]{\cmLangThmKor}
%  \newtheorem{prop}[thmglobal]{\cmLangThmProp}

  \theoremstyle{definition}

%  \newtheorem{dfn}[thmglobal]{\cmLangThmDfn}
%  \newtheorem{bsp}[thmglobal]{\cmLangThmBsp}
  
  \newtheorem{Aufg}{Aufgabe}
  \newtheorem{Loes}{L\"osung}

  \theoremstyle{remark}

%  \newtheorem{bem}[thmglobal]{\cmLangThmBem}

\theoremstyle{plain}
\newtheorem{Dfn}{\cmLangThmDfn}[chapter]
\newtheorem{Satz}[Dfn]{Satz}
\newtheorem{Lemma}[Dfn]{\cmLangThmLemma}
\newtheorem{Kor}[Dfn]{\cmLangThmKor}
\newtheorem{Prop}[Dfn]{\cmLangThmProp}
\newtheorem{Bsp}[Dfn]{\cmLangThmBsp}
\newtheorem{Bem}[Dfn]{\cmLangThmBem}

\theoremstyle{nonumberplain}
\newtheorem{dfn}{\cmLangThmDfn}
\newtheorem{satz}{Satz}
\newtheorem{lemma}{\cmLangThmLemma}
\newtheorem{kor}{\cmLangThmKor}
\newtheorem{prop}{\cmLangThmProp}
\newtheorem{bsp}{\cmLangThmBsp}
\newtheorem{bem}{\cmLangThmBem}

\theoremstyle{break}

% Add unnumbered Theorems, use amsthm style in both style modes
\theoremstyle{plain}
%\newtheorem*{satz*}{\cmLangThmSatz}
%\newtheorem*{lemma}{\cmLangThmLemma}
%\newtheorem*{kor*}{\cmLangThmKor}
%\newtheorem*{prop*}{\cmLangThmProp}

\theoremstyle{definition}
%\newtheorem*{dfn}{\cmLangThmDfn}
%\newtheorem*{bsp*}{\cmLangThmBsp}

\theoremstyle{remark}
%\newtheorem*{bem*}{\cmLangThmBem}
\newtheorem*{beh*}{Behauptung}


% 2-level numbering$
\numberwithin{thmglobal}{section}

% check if 3-level numbering is enabled
\ifthenelse{\boolean{enableDeepNumbering}}{
  \numberwithin{thmglobal}{subsection}
}{}


% some other customisations

% changing enumerations
\setlist[enumerate]{label=(\arabic*), itemsep=0cm, leftmargin=2cm}
\setlist[itemize]{itemsep=0cm} %\setlist[itemize]{itemsep=0cm, leftmargin=2cm}

% replace the slim emptyset symbol
\let\emptyset\varnothing

% set line distances
\linespread{1.1}

% Add a ':' for mathmode with tiny whitespaces around
\newcommand{\SlimDdot}{\ensuremath{\mathrm{:}}}


% headline and cover generation commands

% generate a simple headline
% usage: \CmHeadline[date]{title}{topic}{author}
\newcommand{\CmHeadline}[4][]{
  \begin{minipage}[t]{\textwidth}
    \huge{\textbf{#2}}\\
    \large{#3, #4}\relax
    \ifthenelse{\isempty{#1}}{}{\relax\large{, #1}}
  \end{minipage}
}

% generate a simple cover page
% usage: \CmCover[type(,skript)]{title}{subtitle}{date}
\newcommand{\CmCover}[4][]{
  \thispagestyle{empty}
  \begin{titlepage}
    \begin{center}
      \begin{minipage}[b]{0.8\textwidth}
	\vspace*{5cm}
        \ifthenelse{\isempty{#1}}{
          % Default cover arrangement
          \Huge{\textbf{#2}}\\[0.5cm]
          \huge{#3}\\[0.8cm]
          \Large{#4}
        }{
          \ifthenelse{\equal{#1}{skript}}{
            % Cover for lecture scripts
            \huge{#3}\\[0.5cm]
            \Huge{\textbf{#2}}\\[0.5cm]
            \Large{#4}
          }{}
        }
      \end{minipage}
    \end{center}
  \end{titlepage}
  \pagebreak
}


% indexing support

% Print and index given text
% usage: \CmIndex{[(optionally put another text for the index in here)]{(text to print and add to index)}
\newcommand{\CmIndex}[2][]{\ifthenelse{\isempty{#1}}{\index{#2}}{\index{#1}}#2}

% Highlight(bold) and index the given text
% usage: \CmMark[(optionally put another text for the index in here)]{(text to highlight and add to index)}
\newcommand{\CmMark}[2][]{\textbf{\CmIndex[#1]{#2}}}


% sectioning support

% Prints a description for a section in italic, bold. Most likely to use right under \section.
% usage: \CmSectionDescription{(short description of section contents)}
\newcommand{\CmSectionDescription}[1]{
  \vspace{-0.3cm}
  \hangindent=0.4cm
  \hangafter=0
  \begin{itshape}
    \textbf{#1}
  \end{itshape}
  \vspace{0.3cm}
}

% Starts a new paragraph inside of a theorem environment (as defined above)
% usage \CmSubThm[(paragraph title)]
\newenvironment{CmSubThm}[1]{
  \begin{itemize}[leftmargin=0.5cm,label=]
  \item
    \ifthenelse{\isempty{#1}}{}{
      \hspace{-0.5cm}(\textit{#1})\\[0.2cm]
    }
  }{
  \end{itemize}
}


% svg updater
% needs shell escape option

% Checks if the given image file has been modified and a custom command (if possible) via command line to generate something new.
\newcommand{\CmExecuteIfFileNewer}[3]{
  \ifnum\pdfstrcmp{\pdffilemoddate{#1}}
  {\pdffilemoddate{#2}}>0
  {\immediate\write18{#3}}\fi
}

% Tries to include an image file and checks if the given one has been modified. If so it calls inkscape (if possible) via command line to generate new pdf und pdf_tex files from the corresponding svg.
\newcommand{\CmIncludeSvg}[1]{
  \def\svg@filepath{}
  
  % check if corrosponding svg file exists
  \IfFileExists{#1.svg}
  {
    \def\svg@filepath{#1}
  }{
    % if it does not, search in the graphicspath for it
    \expandafter\@tfor\expandafter\currentsvgpath\expandafter:\expandafter=\Ginput@path\do{
      \IfFileExists{\currentsvgpath#1.svg}{
        \edef\svg@filepath{\currentsvgpath #1}
      }{}
    }
  }
  % if something was found, include the graphic, TODO: Make it work correctly
  \ifthenelse{\isundefined{\svg@filepath} \OR \isempty{\svg@filepath}}{
    \PackageError{canonicalmath}{Image file not found!}
  }{
    \PackageWarning{FilePath}{|\svg@filepath|}
    \CmExecuteIfFileNewer{\svg@filepath.svg}{\svg@filepath.pdf}{inkscape -z -D --file=\svg@filepath.svg --export-pdf=\svg@filepath.pdf --export-latex}
    \input{\svg@filepath.pdf_tex}
  }
}

% Tries to include an image file on the center of the margin at the current position.
\newcommand{\CmMarginSvg}[3][0cm]{
  \marginnote{
    \centering
    \def\svgwidth{#3}
    \CmIncludeSvg{#2}
  }[#1]
}

% Tries to include an image file centering it at the current position
\newcommand{\CmPutSvg}[3][0cm]{
  \begin{figure}[h!]
    \vspace{#1}
    \centering
    \def\svgwidth{#3}
    \CmIncludeSvg{#2}
  \end{figure}
}
\makeatother


%%%%%%%%%%%%%%%%%%%%%%%%%%%%%%%%%%%%%%%%%%%%%%%%%%%%%%%%%%%%%%%%%%%%%%%%%%%%%%%%%%%%%%%%%%%%%%%%%%%%%%%%%%%%%%%%%%%%%%%%%%%%%%%%%%%%%%%%%%%%%%%%%%%%%%%%%%%%%%%%%%%%%%%%

%\usepackage{canonicalsync}
%\CsUsePackage[/home/JB/Projects/tex-package-canonical-sync/]{canonicalsync}
%\CsUsePackageWithOptions[/home/JB/Projects/tex-package-canonical-math/]{canonicalmath}{marginthm}

\usepackage{mathtools}

\usepackage{graphicx}
\usepackage{float}
\usepackage{transparent}
\usepackage{wrapfig}

\graphicspath{{img/}}

\parindent0pt

% Befehl fuer Anfuerungszeichen unten und oben
\newcommand{\quot}[1]{\textrm{\glqq}{#1}\textrm{\grqq}}

\setlist[enumerate]{label=(\arabic*), itemsep=0cm, leftmargin=1cm}