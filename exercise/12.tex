%%
%% Skript Differentialgeometrie im Wintersemester 12/13
%% Zur Vorlesung von Dr. Grensing am KIT Karlsruhe
%%
%% Uebung 12
%%

\section{28. Januar 2012}
\setcounter{Aufg}{0} %Damit die Aufgaben jedes Mal bei Aufgabe 1 anfangen
\setcounter{Loes}{0}

\begin{Aufg}
Es sei $\gamma:[0,a]\to M$ eine Geodätische mit $\gamma(0)=p$, $\gamma'(0)=v$, $\|v\|=1$ und $w\in \T_pM$ mit $\|w\|=1$ und $\langle v,w\rangle=0$.  Es sei $J$ das durch 
	\[J(t)=(\exp_p)_{*_{tv}}tw \in \T_{\gamma(t)}M\]
gegebene Jacobivektorfeld längs $\gamma$ und $\sigma=\mathrm{span}\{v,w\}$. Zeigen Sie, dass die Taylorentwicklung von $|J(t)|$ gegeben ist durch
	\[|J(t)|=t-\frac{1}{6}\sec(\sigma)t^3+o(t^3) \quad \text{für }t\to0.\]
{\footnotesize \textbf{Hinweis:} Berechnen Sie zunächst die Entwicklung von $|J(t)|^2$ und zeigen Sie hierfür die Identität $\nabla_{\gamma'}(R(\gamma',J)\gamma')(0)=R(\gamma',J')\gamma'(0)$.}
\end{Aufg}

\begin{Aufg}
Es seien $\gamma_1, \gamma_2:[0,a] \to M$ zwei Geodätische mit $\gamma_1(0)=\gamma_2=(0)=:p$, deren Ableitungen $v:=\dot{\gamma}_1(0)$ und $w:=\dot{\gamma}_2(0)$ normiert und linear unabhängig seien. Weiter sei $L(t)=d(\gamma_1(t), \gamma_2(t))$.

Zeigen Sie, dass
	\[L(t)=t\|v-w\|-\frac{1}{12}\sec(\mathrm{span}\{v,w\})\|v-w\|(1+\langle v,w\rangle)t^3 + o(t^3)\quad \text{für }t\to 0\]gilt.

{\footnotesize \textbf{Hinweis:} Betrachten Sie die Variation $(t,s)\mapsto\exp(s\exp^{-1}_{\gamma_1(t)}(\gamma_2(t)))$.}

Skizze:
\begin{center}\begin{tikzpicture}[font=\scriptsize]
	%\tikzgitter{(-2,-2)}{(4,4)}
	
	\coordinate (v) at (10:1); \coordinate (w) at (100:1);
	\coordinate (1) at (3.5,2.25); \coordinate (2) at (2.5,3.25);
	
	\fill (0,0) circle(0.05) node[below left]{$p$};
	\draw[->] (0,0) --node[below]{$v$} (v); \draw[->] (0,0) --node[left]{$w$} (w);
	\draw[decoration={markings,mark=at position 0.8 with{\arrow{>}}},postaction={decorate}] (0,0) ..controls (v) and ($(1) - (0.25,1)$).. coordinate[pos=0.6](a) (1);
	\draw[decoration={markings,mark=at position 0.8 with{\arrow{>}}},postaction={decorate}] (0,0) ..controls (w) and ($(2) - (1,0.25)$).. coordinate[pos=0.6](b) (2);
	
	\fill (a) circle(0.05) node[below right]{$\gamma_1(t)$} (b) circle(0.05) node[above left]{$\gamma_2(t)$};
	\draw (a) --node[right]{$L(t)$} (b);
\end{tikzpicture}\end{center}
\end{Aufg}

\begin{Loes}
Offensichtlich gilt $\calJ(0) = 0$.
Au"serdem gilt nach der Kettenregel f"ur die Ableitung
\begin{align*}
\difffrac[s=0]{}{s} (\exp_p(t(v+sw))) = (\exp_p)_{*tv}(tw) = \calJ(t)
\end{align*}
Wir erhalten dann
\begin{align*}
	\calJ'(0) &= \nabla_t (\exp_{p*tv}(tw))|_0 = \nabla_t (t \cdot \exp_{p*tv}(w))|_0 \\
	&= \exp_{p*tv}(w) + t \cdot \nabla_t (\exp_{p*tv}(w))|_0 \\
	&= w
\end{align*}
Ferner gilt $|\calJ(t)|^2 = \langle \calJ(t), \calJ(t) \rangle$. Wir betrachten nun die Ableitungen davon:
\begin{align*}
	\langle \calJ, \calJ \rangle' (0) &= 2 \langle \calJ', \calJ \rangle (0) = 0 \\
	\langle \calJ, \calJ \rangle'' (0) &= 2 \left( \langle \calJ'', \calJ \rangle + \langle \calJ', \calJ' \rangle \right) (0) \\
	&= 2 \|w\|^2 = 2 \\
	\langle \calJ, \calJ \rangle''' (0) &= 2 \left( \langle \calJ''', \calJ \rangle + \langle \calJ'', \calJ' \rangle + 2 \langle \calJ'', \calJ' \rangle \right) (0) \\
	&= 6 \langle \calJ''(0), \calJ'(0) \rangle = -6 \langle R(\dot\gamma(0), \underbrace{\calJ(0)}_{=0}) \dot\gamma(0), \calJ'(0) \rangle = 0 \\
	\calJ'''(0) &= \nabla_t(\calJ''(t))|_0 = \nabla_t (-R(\calJ, \dot\gamma) \dot\gamma)|_0 \\
	&= \nabla_{\dot\gamma} (-R(\calJ, \dot\gamma) \dot\gamma) \overset{\text{Hinweis}}{=} -R(\calJ', \dot\gamma) \dot\gamma|_0 \\
	&= -R(w,v) v \\
	\langle \calJ, \calJ \rangle^{(4)}(0) &= 2 \left( \langle \calJ^{(4)}, \calJ \rangle + \langle \calJ''', \calJ' \rangle + 3 \left( \langle \calJ''', \calJ' \rangle + \langle \calJ'', \calJ'' \rangle \right) \right) (0) \\
	&= 8 \langle \calJ'''(0), \calJ'(0) \rangle = -8 \langle R(w, v) v, w \rangle \\
	&= -8 \sec(\sigma)
\end{align*}
\begin{description}[leftmargin=*]\item[Beweis des Hinweises]
F"ur alle Vektorfelder $W$ l"angs $\gamma$ gilt:
\begin{align*}
	\langle \nabla_t (R(\dot\gamma, \calJ) \dot\gamma), w \rangle (0) &= \difffrac[t=0]{}{t} ( \langle R(\dot\gamma, \calJ) \dot\gamma, w \rangle) - \langle \underbrace{R(\dot\gamma, \calJ) \dot\gamma}_{= 0 \text{ in } 0}, \nabla_t w \rangle (0) \\
	&= \difffrac[t=0]{}{t} ( \langle R(\dot\gamma, w) \dot\gamma, \overbrace{\calJ}^{\mathclap{=0 \text{ in } 0}}) \\
	&= \langle \nabla_t ( R(\dot\gamma, w) \dot\gamma), \calJ \rangle (0) + \langle R(\dot\gamma, w) \dot\gamma, \calJ' \rangle (0) \\
	&= \langle R(\dot\gamma, \calJ') \dot\gamma, w \rangle (0)
\end{align*}
\hfill\ensuremath{\Box}
\end{description}
Aus den Ableitungen folgt nun
\begin{align*}
	|\calJ(t)|^2 &= 0 + 0 \cdot t + \frac{1}{2} \cdot 2 \cdot t^2 + 0 \cdot t^3 + \frac{1}{4} \cdot (-8) \cdot \sec(\sigma) \cdot t^4 + o(t^4) \\
	&= t^2 - \frac{1}{3} \sec(\sigma) t^4 + o(t^4) 
\end{align*}
Es sei nun
\begin{align*}
	|\calJ(t)| &= a_0 + a_1 t + a_2 t^2 + a_3 t^3 + o(t^3) \\
	|\calJ(t)|^2 &= a_0^2 + 2 a_0 a_1 t + (2 a_0 a_2 + a_1^2) t^2 + 2( a_0a_3 + a_1 a_2) t^3 + (2 a_3 a_1 + a_2^2) t^4 + o(t^4)
\end{align*}
Der Koeffizientenvergleich liefert:
\begin{align*}
	a_0 = 0, && a_1^2 = 1, && a_2 = 0, && a_3 = -\frac{1}{6} \sec(\sigma)
\end{align*}
und es gilt $a_1 = 1$ denn $|\calJ(t)| \ge 0$.
\end{Loes}

\begin{Loes}
Sei $\sigma = \mathop{\mathrm{span}}\{v,w\}$ und $V(t,s) := \exp_{\gamma_1(t)} (s \cdot \exp_{\gamma_1(t)}^{-1} (\gamma_2(t))$.
%\marginnote{\begin{tikzpicture}[font=\scriptsize,allow upside down] % Ich bin mir hier nicht sicher bei dem Bild was wie sein soll
%	\coordinate (v) at (10:1); \coordinate (w) at (100:1);
%	\coordinate (1) at (3.5,2.25); \coordinate (2) at (2.5,3.25);
%	\draw[decoration={markings,mark=at position 0.4 with{\arrow{>}}},postaction={decorate}] (0,0) ..controls (v) and ($(1) - (0.25,1)$).. node[below right,pos=0.4]{$\gamma_1$} coordinate[pos=0.7](a) (1);
%	\draw[decoration={markings,mark=at position 0.4 with{\arrow{>}}},postaction={decorate}] (0,0) ..controls (w) and ($(2) - (1,0.25)$).. node[above left,pos=0.4]{$\gamma_2$} coordinate[pos=0.7](b) (2);
%	\fill (a) circle(0.05) node[below right]{$\gamma_1(t)$} (b) circle(0.05);
%	\draw[decoration={markings,mark=at position 0.8 with{\arrow{>}}},postaction={decorate}] (a) to[out=110,in=340] node[pos=0.35,sloped,inner sep=0pt,outer sep=0pt,anchor=north]{\tikz \draw[->] (0,0) -- ++(0,-0.5);} node[above right,pos=0.8]{$V(t,s)$} (b);
%	\node at (3,2) {$T$};
%	\draw[->] (1.25,1.25) node[below]{Geod"atische} to[out=90,in=200] (2,2.25);
%\end{tikzpicture}}
Wir definieren desweiteren
\begin{align*}
	T(t,s) := \difffrac{}{t} (V(t,s)) && S(t,s) := \difffrac{}{s} (V(t,s))
\end{align*}
Die Abbildung $c_t := s \mapsto V(t,s)$ ist eine Geod"atische von $\gamma_1(t)$ nach $\gamma_2(t)$ und daraus folg dann $S(t,s) = \dot c_t$.
Desweiteren ist $t \mapsto c_t$ eine Variation durch Geod"atische und damit $s \mapsto T(t,s)$ ein Jacobifeld l"angs $c_t$.
Es gilt
\begin{align*}
	c_t(0) = \gamma_1(t) && c_t(1) = \gamma_2(t)
\end{align*}
und daraus folgt $L(c_t) = \| S(t,0) \|$. F"ur gen"ugend kleines $t$ gilt dann:
\begin{align*}
	L(t) = L(c_t) = \| S(t,0) \|
\end{align*}
Es gilt:\begin{itemize}
\item
	$\nabla_s \nabla_s T = -R(T,S)S$ ($T$ Jacobifeld l"angs $c_t$)
\item
	$\nabla_s S = 0$ ($c_t$ Geod"atische, $\dot c_t = S(t, \cdot)$)
\item
	$\nabla_t S = \nabla_s T$
\end{itemize}
Ferner gilt $L^2(t) = \langle S(t,0), S(t,0) \rangle$ f"ur kleines $t$ und daraus folgt dann
\begin{align*}
	(L^2)'(0) = 2 \langle \nabla_t S, S \rangle ((0,0))
\end{align*}
mit $S(0,0) = \dot c_0(0) = 0$ und $c_0 \equiv p$. Damit folgt also:
\begin{align*}
	(L^2)'' = 2 ( \langle \nabla_t \nabla_t S, S \rangle + \langle \nabla_t S, \nabla_t S \rangle ) (0,0) = 2 \langle \nabla_s T, \nabla_s T \rangle (0,0)
\end{align*}
Wir wissen bereits dass $T(0,0) = \dot\gamma_1(0) = v$, $T(0,1) = w$ und $c_0 \equiv p$, also $\nabla_s = \difffrac{}{s}$. Es gilt
\begin{align*}
	\nabla_s \nabla_s T(0,s) = -R ( T(0,s), \underbrace{S(0, s)}_{=0} ) \underbrace{S(0,s)}_{=0} = 0
\end{align*}
Damit ist $T(0,\cdot)$ linear, also $T(0,s) = v + S(w - v)$ und damit $\nabla_s T(0,s) = w - v$.
Daraus folgt dann $(L^2)'' = 2 \cdot \|w - v\|^2$.
Wir m"ochten nun dass folgendes gilt:
\begin{align*}
	(L^2)'''(0) = 6 ( \langle \nabla_t \nabla_t S, \nabla_t S \rangle (0,0) ) \overset{!}{=} 0
\end{align*}
Dazu zeigen wir zun"achst die folgenden drei Gleichungen:\begin{enumerate}[label=(\arabic*)]
	\item $\nabla_t T(0,s) = 0$
	\item $\nabla_s \nabla_t T(0,s) = 0$
	\item $\nabla_t \nabla_t S(0,s) = 0$
\end{enumerate}
Dass (2) aus (1) folgt ist klar. Dass (3) aus (2) folgt zeigt Folgendes (in $(0,s)$):
\begin{align*}
	\nabla_t \nabla_t S = \nabla_t \nabla_s T = \underbrace{\nabla_s \nabla_t T}_{=0} + \underbrace{R(T,S)T}_{=0}
\end{align*}
Um (1) zu zeigen gilt $(\nabla_t T)(0,0) = (\nabla_t T)(0,1)$ und dann bleibt noch zu zeigen, dass $(\nabla_t T)(0,s)$ linear ist.
Mit (3) folgt dann schlie"slich $(L^2)'''(0) = 0$.
F"ur die vierte Ableitung gilt dann
\begin{align*}
	(L^2)^{(4)}(0) = \underset{\mathclap{\substack{S(0,s)=0 \\ (\nabla_t \nabla_t S(0,0) = 0)}}}{\ldots} = 8 \langle \nabla_t\nabla_t\nabla_t S, \nabla_t S \rangle (0,0)
\end{align*}
Es gilt:
\begin{align*}
	\nabla_t\nabla_t\nabla_t S &= \nabla_t (\nabla_t \nabla_s T) = \nabla_t (R(T,S)T + \nabla_s \nabla_t T) \overset{\text{Hinweis}}{\underset{\text{A 1}}{=}} R(T, \nabla_t S) T + \nabla_t \nabla_s \nabla_t T
\end{align*}
In $(0,0)$ gilt:
\begin{align*}
	R (T, \underbrace{\nabla_t S}_{= \nabla_s T}) T (0,0) = R(\underbrace{\dot\gamma_1(0)}_{=v}, w - v) \underbrace{\dot\gamma_1(0)}_{=v}
\end{align*}
Damit folgt dann insgesamt:
\begin{align*}
	(L^2)^{(4)}(0) &= 8 \langle R(v, w - v) v, \overbrace{\nabla_t S(0,0)}^{=w-v} \rangle + 8 \underbrace{\langle \nabla_t\nabla_s\nabla_t T, \nabla_t S \rangle}_{= \ldots = 0} (0,0) \\
	&= 8 \langle R(v, w - v) v, w - v \rangle \\
	&= 8 \langle R(v, w) v, w \rangle \\
	&= -8 \langle R(v, w) w, v \rangle \\
	&= -8 \sec(\sigma) (\underbrace{\|v\|^2 \|w\|^2}_{=1} - \langle v, w \rangle^2)
\end{align*}

\end{Loes}