%%
%% Skript Differentialgeometrie im Wintersemester 12/13
%% Zur Vorlesung von Dr. Grensing am KIT Karlsruhe
%%
%% Uebung 9
%%

\section{7. Januar 2012}
\setcounter{Aufg}{0} %Damit die Aufgaben jedes Mal bei Aufgabe 1 anfangen
\setcounter{Loes}{0}

\begin{Aufg}
Es sei $(M,g)$ eine Riemannsche Mannigfaltigkeit und $p\in M$. Berechnen Sie $g_{ij}(p)$, $\frac{\partial g_{ij}}{\partial x^k}(p)$  und $\Gamma_{ij}^k(p)$ in Riemannschen Normalkoordinaten um $p$.

{\footnotesize \textbf{Hinweis:} Welche Form haben die Geodätischen durch $p$ in dieser Karte?}
\end{Aufg}

\begin{Aufg}
Bestimmen Sie die Schnittkrümmungen der Riemannschen Mannigfaltigkeiten $(\R^n,g_{\mathrm{eukl}})$ und $\mathbb{H}^n$ (siehe Blatt 6 Aufgaben 2 und 3).
\end{Aufg}

\begin{Aufg}
Bestimmen Sie die Schnittkrümmungen der $n$-Sphäre vom Radius $r>0$, also von $S^n$ mit der von $S^n=\{x \in \R^{n+1} | \|x\|=r\}$ induzierten Riemannschen Metrik.

{\footnotesize \textbf{Hinweis:} Benutzen Sie, dass der Levi-Civita Zusammenhang auf $S^n$  durch $(\nabla_X Y)_p=((\mathrm{D} Y)_p \cdot X_p)^{\T_p S^n}$ gegeben ist.}
\end{Aufg}

\begin{Loes}
Nach der Vorlesung gibt es einen Diffeomorphismus
\begin{align*}
	\exp_p: \underset{\subset \T_pM}{U(0)} \to \underset{\subset M}{U(p)}
\end{align*}
Seien $e_1,\ldots , e_n \in \T_pM$ eine Orthonormalbasis von $\T_pM$ und sei $\phi^{-1}(x_1,\ldots ,x_n) := \exp_p(x_1e_1,\ldots ,x_ne_n)$ definiert.

\textbf{Behauptung:}\begin{enumerate}[label=(\roman*)]
\item
	$g_{ij}(p) = \delta_{ij}$
\item
	$\difffrac{}{x^2}(g_{ij})(p) = 0$
\item
	$\Gamma_{ij}^k(p) = 0$
\end{enumerate}
\textbf{Beweis:}\begin{enumerate}[label=(\roman*),leftmargin=*,widest=iii]
\item
	\begin{align*}
		g_p \left( \smash{ \underset{\substack{=\exp_{p*}(e_i) \\ \equiv \id}}{\pdifffrac[p]{}{x^{i}}}, \underset{= \exp_{p*}(e_j)}{\pdifffrac[p]{}{x^{j}}} } \vphantom{\pdifffrac{}{}} \right) = g_p(e_i, e_j) = \delta_{ij} \vphantom{\underset{\substack{a\\a}}{\pdifffrac{}{}}}
	\end{align*}
\end{enumerate}
$\gamma_v(t) := \exp_p(tv)$ ist die Geod"atische die in $p$ in Richtung $v$ startet.

\emph{Beweis:} Es sei $c_v$ die Geod"atische mit $c_v(0) = p$, $\dot c_v(0) = v$, sowie $c_{\lambda v}$ die Ged"atische mit $c_{\lambda v}(0) = p$, $\dot c_{\lambda v}(0) = \lambda v$, wobei $\lambda \in \R$.
Definiere nun $\gamma(t) := c_v(\lambda t)$. Dann ist $\gamma(t) = c_{\lambda v}(t)$, denn:\begin{itemize}
\item
	$\gamma(0) = c_v(0) = p$
\item
	$\dot \gamma(0) = \lambda \cdot \dot c_v(0) = \lambda v$
\item
	$\ddot \gamma^k(t) + \sum_{i,j=1}^n \Gamma_{ij}^k(\gamma(t)) \dot\gamma^{i}(t) \dot\gamma^{j}(t)$\\
	$= \lambda^2 \left( \ddot c_v^k(\lambda t) + \sum_{ij} \Gamma_{ij}^k(c_v(\lambda t)) \dot c_v^{i}(\lambda t) \dot c_v^{j}(\lambda t) \right)$\\
	$=0$
\end{itemize}
Betrachte also
\begin{align*}
	c_v(t) = c_v(t \cdot 1) = c_{tv}(1) = \exp_p(t \cdot v)
\end{align*}
Die Geod"atischen durch $p$ werden also von den Normalkoordinaten auf die Ursprungsgeraden abgebildet. Setze $\gamma(t) = t \cdot e_{i_0}$, dann ist $\exp_p \circ \gamma$ eine Geod"atische und damit gilt f"ur alle $k$:
\begin{align*}
	0 = \ddot\gamma^k + \sum_{ij} \Gamma_{ij}^k \dot\gamma^{i} \dot\gamma^{j} = 0 + \sum_{ij} \Gamma_{ij}^k \circ \gamma \delta_{ii_0} \delta_{ji_0} = \Gamma_{i_0i_0}^k \circ \gamma
\end{align*}
\textbf{Beweis:}\begin{enumerate}[label=(\roman*),leftmargin=*,widest=iii]
\item[(iii)]
	Nun sei $i_0 \ne j_0$ und $\tilde\gamma(t) = t(e_{i_0} + e_{j_0})$, damit ist $\exp_p \circ \tilde\gamma$ eine Geod"atische und f"ur alle $k$ gilt:
	\begin{align*}
		0 &= \ddot{\tilde\gamma}^k + \sum_{ij} \Gamma_{ij}^k \dot{\tilde\gamma}^{i} \dot{\tilde\gamma}^{j} = 0 + \sum_{ij} \Gamma_{ij}^k (\delta_{ii_0} + \delta{ij_0}) (\delta_{ji_0} + \delta_{jj_0})\\
		&= \left( \Gamma_{i_0i_0}^k + \Gamma_{j_0j_0}^k + \Gamma_{i_0j_0}^k +\Gamma_{j_0i_0}^k \right) \circ \gamma
	\end{align*}
	In $0$ gilt $\gamma(0) = \tilde\gamma(0) = p$, also $\Gamma_{ij}^k(\gamma(0)) = 0$. Daraus folgt dann:
	\begin{align*}
		0 = \Gamma_{i_0j_0}^k (\tilde\gamma(0)) + \Gamma_{j_0i_0}^k (\tilde\gamma(0)) = 2 \Gamma_{i_0j_0}^2(p)
	\end{align*}
	Damit folgt schlie"slich $\Gamma_{ij}^k(p) = 0$ f"ur alle $i, j, k$.
\item[(ii)]
	\begin{align*}
		\pdifffrac[p]{g_{ij}}{x^k} ={}& \pdifffrac[p]{}{x^k} \left( g \left( \pdifffrac{}{x^{i}}, \pdifffrac{}{x^j} \right) \right)\\
		={}& g \left( \nabla_{\pdifffrac[p]{}{x^k}} \pdifffrac{}{x^{i}}, \pdifffrac[p]{}{x^{j}} \right) + g \left( \pdifffrac[p]{}{x^{i}}, \nabla_{\pdifffrac[p]{}{x^j}} \pdifffrac[p]{}{x^{j}} \right)\\
		={}& g \left( \sum_l \Gamma_{ki}^l(p) \cdot \pdifffrac[p]{}{x^l}, \pdifffrac[p]{}{x^j} \right) + g \left( \pdifffrac[p]{}{x^{i}}, \sum_l \smash{\underbrace{\Gamma_{kj}^l(p)}_{=0}} \pdifffrac[p]{}{x^l} \right)\\
		={}& 0
	\end{align*}
\end{enumerate}
\end{Loes}

\begin{Loes}\begin{enumerate}[label=\alph*), widest=b, leftmargin=*]\item
\emph{Behauptung:} $\sec(\R^n, g_{\eukl}) \equiv 0$

Es gilt $\nabla_XY = \D Y \cdot X$ und, wegen Torsionsfreiheit, $[X,Y] = \nabla_XY - \nabla_YX = \D Y \cdot X - \D X \cdot Y$. Nun gilt
\begin{align*}
	R(X,Y)Z &= \nabla_X(\nabla_YZ) - \nabla_Y(\nabla_XZ) - \nabla_{[X,Y]}Z
\end{align*}
Bevor wir fortfahren ben"otigen wir noch eine Nebenrechnung:
\begin{align*}
	&\left( \D(\D Z \cdot Y) \cdot X - \D(\D Z \cdot X) \cdot Y \right)_i\\
	&= \sum_l \left( \D(\D Z \cdot Y) \right)_{il} \cdot X_l - \sum_l \left( \D(\D Z \cdot X) \right)_{il} \cdot Y_l\\
	&= \sum_l \partial_l \left( (\D Z \cdot Y)_i \right) \cdot X_l - \sum_l \partial_l(\D Z \cdot X)_i \cdot Y_l\\
	&= \sum_l \partial_l \left( \sum_m \smash{\underbrace{(\D Z)_{im}}_{=\partial_mZ_i}} \cdot Y_m \right) \cdot X_l - \sum_l \partial_l \left( \sum_m (\D Z)_{im} \cdot X_m \right) \cdot Y_l \vphantom{\underbrace{A}_{A_A}}\\
	&= \sum_{lm} \left( \partial_l (\partial_m Z_i \cdot Y_m) \cdot X_l - \partial_l (\partial_m Z_i \cdot X_m) \cdot Y_l \right)\\
	&= \sum_{lm} ( \partial_l\partial_m Z_i Y_m X_l + \partial_m Z_i \partial_l Y_m X_l - \underbrace{\partial_l\partial_m Z_i}_{= \partial_m\partial_l Z_i} X_m Y_l + \partial_m Z_i \partial_l X_m Y_l )\\
	&= \sum_m \partial_m Z_i \cdot (\D Y \cdot X)_m - \sum_m \partial_m Z_i ( \D X \cdot Y )_m\\
	&= (\D Z \cdot (\D Y \cdot X) - \D Z \cdot \D X \cdot Y)_i\\
	&= (\D Z \cdot (\D Y \cdot X - \D X \cdot Y))_i
\end{align*}
Mit dieser Nebenrechnung folgern wir schlie"slich $R(X,Y)Z = 0$ und daraus folgt letztendlich
\begin{align*}
	\sec(\mspan\{X,Y\}) = \frac{g(R(X,Y)Y,X)}{\|X\|^2\{Y\|^2 - \langle X,Y \rangle^2} = 0
\end{align*}
\item
	Wir berechnen die Komponenten $R_{ijkl}=R(\pdifffrac{}{\xi^i},\pdifffrac{}{\xi^j},\pdifffrac{}{\xi^k},\pdifffrac{}{\xi^l})=g(R(\pdifffrac{}{\xi^i},\pdifffrac{}{\xi^j})\pdifffrac{}{\xi^k},\pdifffrac{}{\xi^l})$ in der Karte $\phi$ aus Aufgabe 6.3. In dieser Karte gilt $g=\frac{4}{(1-\|\xi\|^2)^2}\sum\limits_i d\xi^i\tensor d\xi^i$, also $g_{ij}=\delta_{ij} \frac{4}{(1-\|\xi\|^2)^2}$.
	
	Für die Ableitungen der Metrik gilt dann:
		\[g_{ij,k}=\frac{\partial g_{ij}}{\partial \xi^k}=16\delta_{ij} \frac{\xi^k}{(1-\|\xi\|^2)^3}.\]
	Die Koeffizienten der zu $(g_{ij})$ inversen Matrix sind $g^{kl}=\delta_{kl} \frac{(1-\|\xi\|^2)^2}{4}$. Damit gilt für die Christoffelsymbole
		\[\Gamma_{ij}^k=\frac{1}{2}\sum_l g^{kl}(g_{jl,i}-g_{ij,l} + g_{li,j})=\frac{2}{1-\|\xi\|^2}(\delta_{jk}\xi^i-\delta_{ij} \xi^k + \delta_{ki} \xi^j).\]
	Für die Ableitungen gilt also:
	\begin{align*}
		\pdifffrac{}{\xi^l}(\Gamma_{ij}^k)&=\pdifffrac{}{\xi^l} \left(\frac{2}{1-\|\xi\|^2}\right)(\delta_{jk}\xi^i-\delta_{ij} \xi^k + \delta_{ki} \xi^j)+ \frac{2}{1-\|\xi\|^2} \pdifffrac{}{\xi^l}(\delta_{jk}\xi^i-\delta_{ij} \xi^k + \delta_{ki} \xi^j)\\
		&=\frac{4\xi^l}{(1-\|\xi\|^2)^2}(\delta_{jk}\xi^i-\delta_{ij} \xi^k + \delta_{ki} \xi^j)+ \frac{2}{1-\|\xi\|^2}(\delta_{jk}\delta_{li}-\delta_{ij} \delta_{lk} + \delta_{ki} \delta_{lj}).
	\end{align*}
	
	Nun können wir die Koeffizienten des Krümmungstensors berechnen. Es gilt :
	\begin{align*}
		R(\pdifffrac{}{\xi^i}, \pdifffrac{}{\xi^j})\pdifffrac{}{\xi^k}&=\nabla_{\pdifffrac{}{\xi^i}}\nabla_{\pdifffrac{}{\xi^j}}\pdifffrac{}{\xi^k}-\nabla_{\pdifffrac{}{\xi^j}}\nabla_{\pdifffrac{}{\xi^i}}\pdifffrac{}{\xi^k}-\nabla_{[\pdifffrac{}{\xi^i},\pdifffrac{}{\xi^j}]=0}\pdifffrac{}{\xi^k}\\ 
		&=\nabla_{\pdifffrac{}{\xi^i}}\Big(\sum_l \Gamma_{jk}^l\pdifffrac{}{\xi^l}\Big)-\nabla_{\pdifffrac{}{\xi^j}}\Big(\sum_l \Gamma_{ik}^l\pdifffrac{}{\xi^l}\Big)\\
		&=\sum_l\left( \pdifffrac{}{\xi^i}(\Gamma_{jk}^l) \pdifffrac{}{\xi^l} + \Gamma_{jk}^l \nabla_{\pdifffrac{}{\xi^i}}\pdifffrac{}{\xi^l} - \pdifffrac{}{\xi^j}(\Gamma_{ik}^l) \pdifffrac{}{\xi^l} - \Gamma_{ik}^l \nabla_{\pdifffrac{}{\xi^j}}\pdifffrac{}{\xi^l}\right)\\
		&=\sum_l\left( \pdifffrac{}{\xi^i}(\Gamma_{jk}^l) \pdifffrac{}{\xi^l}  - \pdifffrac{}{\xi^j}(\Gamma_{ik}^l) \pdifffrac{}{\xi^l}\right)+\sum_{l,m}\left(\Gamma_{jk}^l\Gamma_{il}^m \pdifffrac{}{\xi^m}-\Gamma_{ik}^l \Gamma_{jl}^m \pdifffrac{} {\xi^m} \right)\\
		&=\sum_l \left(\pdifffrac{}{\xi^i}(\Gamma_{jk}^l)  - \pdifffrac{}{\xi^j}(\Gamma_{ik}^l)+\sum_{\alpha}\left(\Gamma_{jk}^\alpha\Gamma_{i\alpha}^l-\Gamma_{ik}^\alpha \Gamma_{j\alpha}^l \right)\right)  \pdifffrac{} {\xi^l}=: \sum_l R_{ijk}^{\phantom{ijk}l} \;\pdifffrac{}{\xi^l}
	\end{align*}
	und
	\begin{align*}
		R_{ijk}^{\phantom{ijk}l}&=\tfrac{4}{(1-\|\xi\|^2)^2}\Big(\xi^i(\delta_{kl} \xi^j- \delta_{jk} \xi^l + \delta_{lj}\xi^k)-\xi^j(\delta_{kl} \xi^i- \delta_{ik} \xi^l + \delta_{li}\xi^k)\\
		&\qquad\qquad + \tfrac{1-\|\xi\|^2}{2}\big((\delta_{kl} \delta_{ij} - \delta_{jk} \delta_{il}+\delta_{lj}\delta_{ik})-(\delta_{kl} \delta_{ij}-\delta_{ik}\delta_{jl}+\delta_{li}\delta_{jk})\big)\\
		&\qquad \qquad + \sum_\alpha \big( ( \delta_{k\alpha} \xi^j-\delta_{jk}\xi^{\alpha}+\delta_{\alpha j} \xi^k)(\delta_{\alpha l} \xi^i-\delta_{i \alpha}\xi^l+\delta_{li}\xi^\alpha)\\[-0.8em]
		&\qquad \qquad \qquad -  ( \delta_{k\alpha} \xi^i-\delta_{ik}\xi^{\alpha}+\delta_{\alpha i} \xi^k)(\delta_{\alpha l} \xi^j-\delta_{j \alpha}\xi^l+\delta_{lj}\xi^\alpha)\big)\Big)\\
		&= \tfrac{4}{(1-\|\xi\|^2)^2}\Big(- \delta_{jk}\xi^i \xi^l + \delta_{lj}\xi^i\xi^k+ \delta_{ik}\xi^j \xi^l - \delta_{li}\xi^j\xi^k\\
		&\qquad\qquad + \tfrac{1-\|\xi\|^2}{2}\big( - \delta_{jk} \delta_{il}+\delta_{lj}\delta_{ik}+\delta_{ik}\delta_{jl}-\delta_{li}\delta_{jk}\big)\\
		&\qquad \qquad + \sum_\alpha \big( ( \delta_{k\alpha} \delta_{\alpha l} \xi^i \xi^j-\delta_{k\alpha} \delta_{i \alpha}  \xi^j\xi^l+\delta_{k\alpha} \delta_{li} \xi^j \xi^\alpha\\
		&\qquad \qquad \qquad -\delta_{jk}\delta_{\alpha l} \xi^i\xi^{\alpha}+\delta_{jk}\delta_{i \alpha}\xi^l\xi^{\alpha} -\delta_{jk}\delta_{li}(\xi^{\alpha})^2\\
		&\qquad \qquad \qquad+ \delta_{\alpha j}\delta_{\alpha l} \xi^i \xi^k - \delta_{\alpha j}\delta_{i \alpha}\xi^k \xi^l + \delta_{\alpha j} \delta_{li}   \xi^k \xi^\alpha)\\
		& \qquad \qquad - (  \delta_{k\alpha} \delta_{\alpha l} \xi^i \xi^j-\delta_{k\alpha} \delta_{j \alpha}  \xi^i\xi^l+\delta_{k\alpha} \delta_{lj} \xi^i \xi^\alpha\\
		&\qquad \qquad \qquad -\delta_{ik}\delta_{\alpha l} \xi^j\xi^{\alpha}+\delta_{ik}\delta_{j \alpha}\xi^l\xi^{\alpha} -\delta_{ik}\delta_{lj}(\xi^{\alpha})^2\\
		&\qquad \qquad \qquad+ \delta_{\alpha i}\delta_{\alpha l} \xi^j \xi^k - \delta_{\alpha i}\delta_{j \alpha}\xi^k \xi^l + \delta_{\alpha i} \delta_{lj}  \xi^k \xi^\alpha)\big)\Big)\\
	\end{align*}
	\begin{align*}
		&= \tfrac{4}{(1-\|\xi\|^2)^2}\Big(- \delta_{jk}\xi^i \xi^l + \delta_{lj}\xi^i\xi^k+ \delta_{ik}\xi^j \xi^l - \delta_{li}\xi^j\xi^k + (1-\|\xi\|^2)\big(\delta_{ik}\delta_{jl}-\delta_{li}\delta_{jk}\big)\\
		&\qquad \qquad + \sum_\alpha \big( ( \delta_{k l} \xi^i \xi^j- \delta_{i k}  \xi^j\xi^l+ \delta_{li} \xi^j \xi^k -\delta_{jk} \xi^i\xi^{l}+\delta_{jk}\xi^l\xi^{i} -\delta_{jk}\delta_{li}\|\xi\|^2+ \delta_{j l} \xi^i \xi^k - \delta_{i j}\xi^k \xi^l + \delta_{li} \xi^k \xi^j)\\
		&\qquad \qquad \qquad - (  \delta_{k l} \xi^i \xi^j- \delta_{j k}  \xi^i\xi^l+ \delta_{lj} \xi^i \xi^k -\delta_{ik} \xi^j\xi^{l}+\delta_{ik}\xi^l\xi^{j} -\delta_{ik}\delta_{lj}\|\xi\|^2 + \delta_{i l} \xi^j \xi^k - \delta_{ij}\xi^k \xi^l + \delta_{lj} \xi^k \xi^i)\big)\Big)\\
		&=  \tfrac{4}{(1-\|\xi\|^2)^2}\Big( (1-\|\xi\|^2)\big(\delta_{ik}\delta_{jl}-\delta_{li}\delta_{jk}\big) +  \|\xi\|^2 \big(\delta_{ik}\delta_{jl}-\delta_{li}\delta_{jk}\big)\\
		&=  \tfrac{4}{(1-\|\xi\|^2)^2}\big(\delta_{ik}\delta_{jl}-\delta_{li}\delta_{jk}\big)
	\end{align*}
	Und somit 
		\[R_{ijkl}=g(\sum_m R_{ijk}^{\phantom{ijk}m} \pdifffrac{}{\xi^m},\pdifffrac{}{\xi^l})=\sum_m  R_{ijk}^{\phantom{ijk}m} g(\pdifffrac{}{\xi^m},\pdifffrac{}{\xi^l})=\frac{4}{(1-\|\xi\|^2)^2} R_{ijk}^{\phantom{ijk}l}=\frac{16}{(1-\|\xi\|^2)^4}\big(\delta_{ik}\delta_{jl}-\delta_{li}\delta_{jk}\big).\]
	Für linear unabhängige $X=\sum X_i \pdifffrac{}{\xi^i}$, $Y=\sum Y_j \pdifffrac{}{\xi^j} \in \T_p\mathbb{H}^n$ gilt dann
	\begin{align*}
		R(X,Y,Y,X)&= \sum_{i,j,k,l} X_iX_lY_j Y_k \;R(\pdifffrac{}{\xi^i},\pdifffrac{}{\xi^j},\pdifffrac{}{\xi^k},\pdifffrac{}{\xi^l}) \\
		{}=& \sum_{i,j,k,l} X_iX_lY_j Y_k \;\tfrac{16}{(1-\|\xi\|^2)^4}\big(\delta_{ik}\delta_{jl}-\delta_{li}\delta_{jk}\big)\\
		{}=& \sum_{i,j} X_iX_jY_jY_i \; \left(\tfrac{4}{(1-\|\xi\|^2)^2}\right)^2- \sum_{i,j} X_i^2 Y_j^2 \; \left(\tfrac{4}{(1-\|\xi\|^2)^2}\right)^2\\
		{}=& \sum_{i,j} X_iX_jY_jY_i g(\pdifffrac{}{\xi^i},\pdifffrac{}{\xi^i})g(\pdifffrac{}{\xi^j},\pdifffrac{}{\xi^j})-  \sum_{i,j} X_i^2 Y_j^2 \;g(\pdifffrac{}{\xi^i},\pdifffrac{}{\xi^i})g(\pdifffrac{}{\xi^j},\pdifffrac{}{\xi^j})\\
		{}=& \Big(\sum_ig(X_i\pdifffrac{}{\xi^i}, Y_i\pdifffrac{}{\xi^i})\Big)\Big(\sum_jg(X_j\pdifffrac{}{\xi^j}, Y_j\pdifffrac{}{\xi^j})\Big) \\
		&- \Big(\sum_ig(X_i\pdifffrac{}{\xi^i}, X_i\pdifffrac{}{\xi^i})\Big)\Big(\sum_jg(Y_j\pdifffrac{}{\xi^j}, Y_j\pdifffrac{}{\xi^j})\Big)\\
		{}=& g(X,Y)^2 - \|X\|^2 \|Y\|^2=-(\|X\|^2 \|Y\|^2 - g(X,Y)^2).
	\end{align*}
	Hieraus folgt
		\[\sec(\mathrm{span}\{X,Y\})=\frac{R(X,Y,Y,X)}{\|X\|^2 \|Y\|^2 - g(X,Y)^2}=-1.\]
\end{enumerate}\end{Loes}

\begin{Loes}
Sei $S^n(r) := \{x \in \R^{n+1} | \|x\| = r\}$, \emph{Behauptung:} $\sec_{S^n(r)} \equiv \frac{1}{r^2}$

Sei $\nabla = \nabla S^n(r)$ der Levi-Civita Zusammenhang von $(S^n(r), g_{\text{ind}})$. Daraus folgt:
\begin{align*}
	(\nabla_X Y)_p &= \left( (\nabla_X^{\R^{n+1}} Y)_p \right) \T_p S^n(r) \\
	&= (\nabla_X^{\R^{n+1}} Y)_p - \langle (\nabla_X^{\R^{n+1}} Y)_p, N(p) \rangle \cdot N(p)
\end{align*}
wobei $N(p) = \frac{1}{r} \cdot p$ das Normaleneinheitsvektorfeld an $S^n(r)$ ist. Betrachte nun
\begin{align*}
	\langle (\nabla_X^{\R^{n+1}} Y)_p, N(p) \rangle &= \underbrace{X_p ( \overbrace{\langle Y, N \rangle}^{\equiv 0} )}_{=0} - \langle Y_p, (\nabla_X^{\R^{n+1}} N)_p \rangle\\
	&= - \langle Y_p, \underbrace{(\D N)_p}_{\frac{1}{r}\cdot\id} \cdot X_p \rangle\\
	&= - \frac{1}{r} \langle Y_p, X_p \rangle
\end{align*}
Daraus folgt dann $\nabla_XY = \nabla_X^{R^{n+1}} Y + \frac{1}{r} \langle X, Y \rangle N$. Als n"achstes betrachten wir nun:
\begin{align*}
R^{S^n(r)}(X,Y)Z ={}& \nabla_X \nabla_Y Z - \nabla_Y \nabla_X - \nabla_{[X,Y]}Z\\
={}& \nabla_X^{\R^{n+1}} (\nabla_Y Z) + \frac{1}{r} \langle X, \nabla_Y Z \rangle \cdot N\\
   & - \nabla_Y^{\R^{n+1}} (\nabla_X Z) - \frac{1}{r} \langle Y, \nabla_X Z \rangle \cdot N\\
   & - \nabla_{[X,Y]}^{\R^{n+1}} Z - \frac{1}{r} \langle [X,Y], Z \rangle \cdot N\\
={}& \overbrace{R^{\R^{n+1}} (X,Y)Z}^{=0} + \nabla_X^{\R^{n+1}} (\frac{1}{r} \langle Y, Z \rangle \cdot N) + \frac{1}{r} \langle X, \nabla_Y^{\R^{n+1}} Z + \frac{1}{r} \langle Y, Z \rangle \cdot \overset{\textcolor{gray}{\perp X}}{N} \rangle N\\
   & - \nabla_Y^{\R^{n+1}} ( \frac{1}{r} \langle X, Y \rangle N) - \frac{1}{r} \langle Y, \nabla_X^{\R^{n+1}} Z + \frac{1}{r} \langle Y, Z \rangle \cdot \overset{\textcolor{gray}{\perp Y}}{N} \rangle \cdot N\\
   & - \frac{1}{r} \langle \underbrace{[X,Y]}_{\mathclap{=\D Y \cdot X - \D X \cdot Y}} Z \rangle \cdot N\\
={}& X(\frac{1}{r} \langle Y, Z \rangle) N + \frac{1}{r} \langle Y, Z \rangle \nabla_X^{\R^{n+1}} N + \frac{1}{r} \langle X, \D Z \cdot Y \rangle \cdot N\\
   & - Y ( \frac{1}{r} \langle X, Z \rangle ) \cdot N - \frac{1}{r} \langle X, Z \rangle \nabla_Y^{\R^{n+1}} N - \frac{1}{r} \langle Y, \D Z \cdot X \rangle \cdot N - \frac{1}{r} \langle \D Y \cdot X\\
   & - \D X \cdot Y, Z \rangle \cdot N\\
={}& \frac{1}{r} ( \langle \D Y \cdot X, Z \rangle + \langle Y, \D Z \cdot X \rangle ) \cdot N + \frac{1}{r} \langle Y, Z \rangle \overbrace{\D N}^{= \frac{1}{r} \cdot \id} \cdot X + \langle X, \D Z \cdot Y \rangle \cdot N\\
   & - \frac{1}{r} ( \langle \D X \cdot Y, Z \rangle + \langle X, \D Z \cdot Y \rangle ) \cdot N - \frac{1}{r^2} \langle X Z \rangle \cdot Y + \langle Y, \D Z \cdot X \rangle \cdot N\\
   & - \frac{1}{r} \langle \D Y \cdot X - \D X \cdot Y, Z \rangle \cdot N\\
={}& \frac{1}{r^2} ( \langle Y, Z \rangle \cdot X - \langle X, Z \rangle \cdot Y)
\end{align*}
Daraus folgt dann
\begin{align*}
	\langle \D(X,Y) Y, X \rangle = \frac{1}{r^2} ( \langle Y, Y, \rangle \langle X, X \rangle - \langle X, Y \rangle^2)
\end{align*}
und damit folgt dann schlie"slich
\begin{align*}
\sec_{S^n(r)}(\span\{X,Y\}) = \frac{1}{r^2}
\end{align*}
\end{Loes}