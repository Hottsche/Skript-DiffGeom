%%
%% Skript Differentialgeometrie im Wintersemester 12/13
%% Zur Vorlesung von Dr. Grensing am KIT Karlsruhe
%%
%% Uebung 7
%%

\section{10. Dezember 2012}
\setcounter{Aufg}{0} %Damit die Aufgaben jedes Mal bei Aufgabe 1 anfangen
\setcounter{Loes}{0}

\begin{Loes}
$c: [0,1] \to (S^2, g_{\text{std}})$ k"urzeste $C^1$-Kurve zwischen $c(0) = N = (0,0,1)$ und $c(1)$. \emph{Behauptung:} $\Bild c$ ist im Gro"skreis enthalten.

W"ahle ein geeignetes Intervall der L"ange $2 \pi$, sodass
	\[\begin{array}{cccc} f: & I \X \left( -\frac{\pi}{2}, \frac{\pi}{2} \right) &\to& f \left( I \X \left( -\frac{\pi}{2}, \frac{\pi}{2} \right) \right)\\
		& (\phi, \theta) &\mapsto& (\cos\theta \cos\phi, \cos\theta\sin\phi, \sin\theta)\end{array}\]
bijektiv ist und $c(1) \in U$ (falls $c(1) \ne N,S$). Daraus folgt dass $f^{-1}$ eine Karte von $S^2$ ist. Sei nun ohne Einschr"ankung $c$ in keiner Umgebung von $0$ konstant. Weiter sei $\gamma := f^{-1} \circ c$ (eventuell nur in einer Umgebung von $0$ ohne $\{0\}$ definiert). Es bleibt nun zu zeigen, dass $\gamma_1$ konstant ist.

Bestimme $g_{ij}$ bez"uglich $f^{-1}$. Es ist
\begin{align*}
	\partial_1 f = & \pdifffrac{f}{\phi} = \left( -\cos\theta \sin\phi, \cos\theta \cos\phi, 0 \right)\\
	\partial_2 f = & \pdifffrac{f}{\theta} = \left( -\sin\theta \cos\phi, -\sin\theta \sin\phi, \cos\theta \right)\\
\end{align*}
Daraus folgt
\begin{align*}
	g_{11} = & \left\langle \partial_1 f, \partial_1 f \right\rangle = \cos^2 \theta\\
	g_{22} = & \left\langle \partial_2 f, \partial_2 f \right\rangle = 1\\
	g_{12} = & \left\langle \partial_1 f, \partial_2 f \right\rangle = 0 = g_{21}
\end{align*}
Damit ist dann
\begin{align*}
	L(c|_{(0, \epsilon)}) &= \int_0^\epsilon \sqrt{g(\dot c, \dot c)} = \int_0^\epsilon \sqrt{\vphantom{g(\dot c, \dot c)} \smash{\underbrace{g_{11}(c(t)) \dot\gamma_1(t)^2 + g_{22}(c(t)) \dot\gamma_2(t)^2}_{= (\dot\gamma_1, \dot\gamma_2) \left(\begin{smallmatrix} g_{11} & g_{12} \\ g_{21} & g_{22} \end{smallmatrix}\right) \left(\begin{smallmatrix} \dot\gamma_1 \\ \dot\gamma_2 \end{smallmatrix}\right)}}} \dop t\\
	&= \int_0^\epsilon \sqrt{ \smash{\underbrace{\cos^2(\gamma_2(t))}_{\ge 0}} \dot\gamma_1(t)^2 + \dot\gamma_2(t)^2} \dop t \rule{0pt}{33pt}\\ % \smash macht dass die Groesse nicht von \sqrt beruecksichtig wird
	&\ge \int_0^\epsilon \left| \dot\gamma_2(t) \right| \dop t =L(\tilde c) \rule{0pt}{25pt}% \rule fuegt ein wenig vertikalen Abstand nach oben ein
\end{align*}
f"ur $\tilde c(t) = f(\gamma_1(\epsilon) \gamma_2(t))$. Dann ist $c$ die K"urzeste. Daraus folgt $L(c|_{(0,\epsilon]}) = L(\tilde c|_{(0,\epsilon]}$ und somit ist f"ur alle $t \in (0,\epsilon]$ stets $\cos^2(\gamma_2(t)) \dot\gamma_1(t) = 0$. Da $\cos^2(\gamma_2(t)) > 0$ muss $\dot\gamma_1(t) = 0$ gelten. Da sich $c$ nicht aus $\Bild f$ heraus bewegt, au"ser eventuell f"ur $c(1)$, ist $\dot\gamma_1(t) = 0$ f"ur alle $t \in (0,1)$.
\end{Loes}

\begin{Loes}
Sei $M$ eine Untermannigfaltigkeit von $\R^k$ und die Abbildung $\nabla^M: \calV(M) \X \calV(M) \to \calV(M)$ mit $(\nabla_X^MY)_p :=((\nabla_{\tilde X} \tilde Y)_p)^{\T_pM}$, wobei $\tilde X, \tilde Y$ Fortsetzungen von $X, Y$ sind. Wir haben zu zeigen:
\begin{enumerate}[label=(\arabic*.\arabic*)]
\item[(0)]
	Unabh"angigkeit von der Wahl der Fortsetzungen
\item[(1.1)]
	$\nabla_{X_1+X_2}^M Y = \nabla_{X_1}^M Y + \nabla_{X_2}^M Y$
\item[(1.2)]
	$\nabla_{fX}^M Y = f \nabla_{X}^M Y$
\item[(2.1)]
	$\nabla_X^M (Y_1 + \lambda Y_2) = \nabla_X^M Y_1 + \lambda \nabla_X^M Y_2$
\item[(2.2)]
	$\nabla_X^M (fY) = X(f) \cdot Y + f \nabla_X^M Y$
\item[(3)]
	$\nabla_X^M Y \in \calV(M)$ (klar ist, dass auf $\T_pM$ projiziert wird)
\end{enumerate}
Wir werden nun diese einzelnen Behauptungen beweisen, wobei der Beweis von (2.1) in (1.1) enthalten ist.
\begin{enumerate}[label=(\arabic*.\arabic*)]
\item[(0)]
	Es sei $c: (-\epsilon, \epsilon) \to M$ glatt mit $c(0) = p$ und $\dot c(0) = X_p$. Damit erh"alt man folgendes:
	\begin{align*}
		((\nabla_{\tilde X} \tilde Y))^{\T_pM} &= ((\D \tilde Y)_p \tilde X_p)^{\T_pM}&\\
		&= ((\D \tilde Y)_p X_p)^{\T_pM} & \Rightarrow \text{unabh. von der Wahl von }\tilde X\\
		&= \left( \difffrac[t=0]{}{t}\left( \tilde Y \circ c(t) \right) \right)^{\T_pM}&\\
		&= \left( \difffrac[t=0]{}{t}\left( Y \circ c(t) \right) \right)^{\T_pM} & \Rightarrow \text{unabh. von der Wahl von }\tilde Y\\
	\end{align*}
\item[(1.1)]
	W"ahle $\tilde X_1 + \tilde X_2$, beziehungsweise $\tilde Y_1 + \lambda \tilde Y_2$, als Fortsetzung entsprechend der Regeln f"ur $\nabla$. Daraus folgen dann die Behauptungen.
\item[(1.2)]
	Es gilt:
	\begin{align*}
		(\nabla_{fX}^MY)_p &= \left( (\nabla_{f\tilde X} \tilde Y)_p \right)^{\T_pM} = \left( (D \tilde Y)_p \smash{\underbrace{(f \tilde X)_p}_{=f(p)X_p}} \right)^{\T_pM} \vphantom{\underbrace{(f \tilde X)_p}_{=f(p)X_p}}\\
		&= \left( (\D \tilde Y)_p (f(p)X_p) \right)^{\T_pM}\\
		&= f(p) \left( (\D \tilde Y)_p X_p \right)^{\T_pM}\\
		&= f(p) \left( \nabla_X^M Y)_p \right)
	\end{align*}
\item[(2.2)]
	Es gilt:
	\begin{align*}
		\left( \nabla_X^M(fX) \right)_p &= \left( \big(\D\widetilde{(fY)}\big)_p \cdot X_p \right)^{\T_pM}\\
		&= \left( \difffrac[t=0]{}{t} \Big( \widetilde{(fY)} \circ c \Big) (t) \right)^{\T_pM}\\
		&= \left( \difffrac[t=0]{}{t} \Big( (fY) \circ c \Big) (t) \right)^{\T_pM}\\
		&= \left( \difffrac[t=0]{}{t} \Big( f\big(c(t)\big) \cdot Y\big(c(t)\big) \Big) (t) \right)^{\T_pM}\\
		&= \left( \difffrac[t=0]{}{t} \Big( f\big(c(t)\big) \Big) Y\big(c(0)\big) + f\big(c(0)\big) \cdot \difffrac[t=0]{}{t} \Big( Y\big(c(t)\big) \Big) \right)^{\T_pM}\\
		&= \left( X_P(f) \cdot Y_p + f(p)(\nabla_{\tilde X} \tilde Y)_p \right)^{\T_pM}\\
		&= X_p(f) \cdot Y_p^{\T_pM} + f(p)\left((\nabla_{\tilde X} \tilde Y)_p \right)^{\T_pM}\\
		&= X_p(f) Y_p + f(p)(\nabla_X^MY)_p
	\end{align*}
\item[(3)]
	Es gilt $(\nabla_X^MY)_p \in \T_pM$ (klar wegen Projektion) und h"angt glatt von $p$ ab. Sind $e_1, \ldots ,e_n$ um $p$, definiere glatte Vektorfelder, so dass f"ur $q \in \mathrm{Umg}(p)$ $e_1|_q, \ldots ,e_n|q$ (erhalte mit Gram-Schmitt (glatt!)) eine Orthonormalbasis ist, so ist (lokal):
		\[ \nabla_X^M Y = \sum_{i=1}^n \langle \D \tilde Y \cdot \tilde X, e_i \rangle \cdot e_i \Rightarrow \text{glatt} \]
	(Erhalte $e_1,\ldots ,e_n$ aus beliebigen lokalen Basisvektoren durch Gram-Schmitt.)
\end{enumerate}
\end{Loes}

\begin{bem}\begin{itemize}
\item
	$\nabla^M$ ist der \CmMark[Zusammenhang!Levi-Civita-]{Levi-Civita Zusammenhang} von $(M,g)$ mit $g(X,Y) = \langle X, Y \rangle$.
\item
	Die Projektion auf $\T_pM$ ist n"otig, zum Beispiel $M = S^1$, $\tilde X = \tilde Y = \left(\begin{smallmatrix} -Y \\ X \end{smallmatrix}\right) \in \calV(\R^2) \Rightarrow X = Y = \tilde X|_{S^1} \in \calV(S^1)$
		\[ \D \tilde Y \cdot \tilde X = \begin{pmatrix} 0 & -1 \\ 1 & 0 \end{pmatrix} \cdot \begin{pmatrix} -Y \\ X \end{pmatrix} = \begin{pmatrix} -X \\ -Y \end{pmatrix} = -\begin{pmatrix} X \\ Y \end{pmatrix} \in \left( \T_{(X,Y)}S^1 \right)^\perp \]
\end{itemize}\end{bem}

\begin{Loes}\begin{enumerate}[label=\alph*),leftmargin=*,widest=b]
\item
	Es sei $E$ ein Vektorb"undel "uber $M$ und $\nabla$ ein Zusammenhang auf $E$. Betrachte $(\nabla_X^*s^*)_p(v) := X_p(s^*\tilde v) - s_p^*((\nabla_X \tilde v)_p)$ mit $X \in \calV(M)$, $s^* \in \Gamma(E^*)$, $v \in E_p$, $\tilde v \in \Gamma(E)$ und $\tilde v_p = v$.
	Warum dieses $\nabla^*$ betrachten? F"ur $s^* \in \Gamma(E^*)$, $s \in \Gamma(E)$ sei
		\[ \langle s^*, s \rangle := s^*(s) \in C^\infty(M) \]
	Damit ist
	\begin{align*}
		X(\langle s^*, s \rangle) &= \langle \nabla_X^* s^*, s \rangle + \langle s^*, \nabla_X s \rangle\\
		&= (\nabla_X^* s^*)(s) + s^*(\nabla_X s)
	\end{align*}
	Das f"uhrt zu
		\[ (\nabla_X^* s^*)(s) = X(s^*(s)) - s^*(\nabla_Xs) \]
\item
	Seien $E_1$ und $E_2$ Vektorb"undel mit Zusammenh"angen $\nabla^1$ und $\nabla^2$. Wir haben zu zeigen, dass es genau einen Zusammenhang $\nabla$ auf $E_1 \otimes E_2$ gibt mit $\nabla_X(S_1 \otimes S_2) = (\nabla_X^1 s_1) \otimes s_2 + s_1 \otimes (\nabla_X^2 s_2)$.
	\begin{description}[font=\normalfont\itshape]
	\item[Eindeutigkeit:]
		Seien $s \in \Gamma(E_1 \otimes E_2)$ und seien $e_1^1,\ldots ,e_m^1, e_1^2,\ldots ,e_n^2 : U \to E_1$ beziehungsweise $E_2$ lokale Basisschnitte von $E_1$ und $E_2$. Daraus folgt $s|_U = \sum \sigma_{ij} e_i^1 \otimes e_j^2$.
		\begin{align*}
			\nabla_x(s)|_U &= \nabla_X \left( \smash{\sum_{ij}} \sigma_{ij} e_i^1 \otimes e_j^2 \right) \vphantom{\sum_{ij}}\\
			&= \sum_{ij} \left( X(\sigma_{ij})\cdot e_i^1 \otimes e_j^2 + \sigma_{ij} \nabla_X (e_i^1 \otimes e_j^2) \right)\\
			&= \sum_{ij} \left( X(\sigma_{ij}) e_i^1 \otimes e_j^2 + \sigma_{ij}\big((\nabla_X^1 e_i^1) \otimes e_j^2 + e_i^1 \otimes (\nabla_X^2 e_j^2)\big) \right)
		\end{align*}
	\item[Existenz:]
		Zeige die Unabh"angigkeit von der Wahl der $e_i^k$.
	\end{description}
\end{enumerate}\end{Loes}