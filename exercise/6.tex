%%
%% Skript Differentialgeometrie im Wintersemester 12/13
%% Zur Vorlesung von Dr. Grensing am KIT Karlsruhe
%%
%% Uebung 6
%%

\section{3. Dezember 2012}
\setcounter{Aufg}{0} %Damit die Aufgaben jedes Mal bei Aufgabe 1 anfangen
\setcounter{Loes}{0}

\begin{Aufg}
Es sei $(M,g)$ eine zusammenhängende Riemannsche Mannigfaltigkeit. Zeigen Sie:
\begin{enumerate}[label=\alph*),leftmargin=*,widest=b]
\item
	Für je zwei Punkte in $M$ existiert eine stückweise glatte Kurve, die diese verbindet.
\item
	Die Abstandsfunktion 
		\[d(p,q)=\inf\{\mathcal{L}(c)|\text{ $c:[0,1] \to M$ ist stückweise glatt, } c(0)=p, c(1)=q\}\]
	ist eine Metrik, welche die ursprüngliche Topologie erzeugt.
\end{enumerate}\end{Aufg}

\begin{Aufg}
Es sei für $x,y\in \R^{n+1}$ 
	\[\langle x,y \rangle:= - x^0 y^0 + x^1 y^1 +\dots + x^n y^n,\]
sowie 
	\[\mathbb{H}^n:= \{x\in \R^{n+1}| \langle x,x \rangle =-1, x^0 > 0\}.\]
Zeigen Sie, dass $\mathbb{H}^n$ eine glatte Mannigfaltigkeit ist, und $\langle. \,, . \rangle$ für alle $p\in \mathbb{H}^n$ ein Skalarprodukt auf $\T_p \mathbb{H}^n \subset \T_p \R^{n+1}$   definiert und die Gesamtheit dieser Skalarprodukte eine Riemannsche Metrik $g$ auf $\mathbb{H}^n$ ist.

Die Riemannsche Mannigfaltigkeit $(\mathbb{H}^n, g)$ heißt \emph{$n$-dimensionaler hyperbolischer Raum}.
\end{Aufg}

\begin{Aufg}
Es sei $s=(-1,0,\dots,0) \in \R^{n+1}$. Zeigen Sie:
\begin{enumerate}[label=\alph*),leftmargin=*,widest=b]
\item
	Die Abbildung $\phi$ mit 
		\[\phi(x):=s-\frac{2 (x-s)}{\langle x-s,x-s\rangle}, \quad x\in \mathbb{H}^n, \]
	ist ein Diffeomorphismus von $\mathbb{H}^n$ auf $\{\xi \in \R^n \cong \{0\} \times \R^n\subset \R^{n+1}\;|\; \| \xi\|<1\}$.
\item
	In der Karte $\phi$ hat die Riemannsche Metrik auf $\mathbb{H}^n$ die Form \[ \frac{4}{(1-\|\xi \|^2)^2} \sum_{i=1}^nd\xi^i \tensor d\xi^i.\] 
\end{enumerate}\end{Aufg}

\begin{Loes}\begin{enumerate}[label=\alph*),leftmargin=*,widest=b]
\item
	asdf
\item
	asdf
\end{enumerate}\end{Loes}

\begin{Loes}
asdf
\end{Loes}

\begin{Loes}\begin{enumerate}[label=\alph*),leftmargin=*,widest=b]
\item
	asdf
\item
	asdf
\end{enumerate}\end{Loes}