%%
%% Skript Differentialgeometrie im Wintersemester 12/13
%% Zur Vorlesung von Dr. Grensing am KIT Karlsruhe
%%
%% Uebung 3
%%

\section{12. November 2012}
\setcounter{Aufg}{0} %Damit die Aufgaben jedes Mal bei Aufgabe 1 anfangen
\setcounter{Loes}{0}

\begin{Aufg}
Es sei $M$ eine glatte Mannigfaltigkeit. Zeigen Sie:
\begin{enumerate}[label=\alph*),leftmargin=*,widest=b]
\item
	Die kanonische Projektion $\pi:\mathrm{T}M \to M$ ist ein Submersion.
\item
	Der Nullschnitt $\sigma:M\to \mathrm{T}M$, $p\mapsto 0 \in \mathrm{T}_p M$ ist eine Einbettung.
\item
	Ist $N$ eine weitere glatte Mannigfaltigkeit und $\Phi:M \to N$ glatt, so ist $\Phi_*:\mathrm{T}M \to \mathrm{T}N$ glatt.
\end{enumerate}\end{Aufg}

\begin{Aufg}
Es sei $M$ eine glatte $n$-dimensionale Mannigfaltigkeit und $X,Y \in \mathcal{V}(M)$.
\begin{enumerate}[label=\alph*),leftmargin=*,widest=b]
\item
	Zeigen Sie, dass die Lieklammer im Allgemeinen nicht $C^{\infty}(M)$-bilinear ist.
\item
	Zeigen Sie, dass $XY$ mit $XY(p)(f):=X_p(Y(f))$ für $p \in M$ und $f\in C^\infty(M)$ im Allgemeinen kein Vektorfeld ist.
\end{enumerate}
Es sei ferner $(\phi,U)$  eine Karte von $M$ und $X|_U=\sum\limits_{i=1}^n \xi^i \pdifffrac{}{x^i}$, $Y|_U=\sum\limits_{i=1}^n \eta^i \pdifffrac{}{x^i}$, sowie $[X,Y]|_U=\sum\limits_{i=1}^n \zeta^i \pdifffrac{}{x^i}$ die lokalen Darstellungen von $X$, $Y$ und $[X,Y]$ bezüglich $\phi$.
\begin{enumerate}[label=\alph*),leftmargin=*,widest=b]
\item[c)]
	Zeigen Sie, dass gilt: 
		\[\zeta^j=\sum_{i=1}^n \left(\xi^i \frac{\partial \eta^j}{\partial x^i} - \eta^i \frac{\partial \xi^j}{\partial x^i}\right).\]
\end{enumerate}\end{Aufg}

\begin{Aufg}\begin{enumerate}[label=\alph*),leftmargin=*,widest=b]
\item
	Es seien auf $\R^2$ die beiden Vektorfelder $X=-y \pdifffrac{}{x}+x \pdifffrac{}{y}$ und $Y=-2 y \pdifffrac{}{x}+\tfrac{1}{2} x \pdifffrac{}{y}$ gegeben. Skizzieren Sie die Vektorfelder und bestimmen Sie die Flüsse von $X$ und $Y$.
\item
	Auf dem Torus $\mathrm{T}^2=\mathrm{S}^1 \times \mathrm{S}^1=\{(e^{i \theta^1}, e^{i \theta^2})\in \C^2 | \theta^1, \theta^2 \in \R\}$ betrachten wir für $k \in \N$  das Vektorfeld $X_k=\pdifffrac{}{\theta^1}+ \tfrac{1}{k}\pdifffrac{}{\theta^2}$. Bestimmen Sie die Integralkurve von $X_k$ durch den Punkt $(1,1) \in \mathrm{T}^2$.
\end{enumerate}\end{Aufg}

\begin{Loes}
Sei $M$ eine glatte Mannigfaltigkeit. Sei desweiteren f"ur alle drei Teilaufgaben $p \in M$, $\phi$ eine Karte um $p$ mit Kartengebiet $U$ und
	\[ \overline \phi: \left\{ \begin{array}{ccc}  \T M|_U &\to& \R^{2n} \\
		\sum_i \xi^i\pdifffrac[q]{}{x^i} &\mapsto& (\phi(q), \xi) \end{array} \right.\]
eine Karte von $\T M$. Alle diese Karten bilden dann einen Atlas von $\T M$.
\begin{enumerate}[label=\alph*),widest=a,leftmargin=*]
\item
	\emph{Zeige:} $\pi: \T M \to M$, $\T_p M \ni x \mapsto p$ ist eine Submersion.
	
	Es ist
	\begin{align*} \phi \circ \pi \circ \overline \phi^{-1} \underbrace{(y, \xi)}_{\mathclap{\in \phi(U) \X \R^n}} &= \phi\left(\pi\left(\sum_i \xi^i \pdifffrac[\phi^{-1}(y)]{}{x^i}\right)\right)\\
		&= \phi(\phi^{-1}(y)) = y
	\end{align*}
	also ist $\pi$ glatt. Desweiteren ist
		\[ \D(\phi \circ \pi \circ \overline \phi^{-1})|_{(y, \xi)} = \left(\begin{smallmatrix}
        1 &  & \\
        & \ddots & & \\
         & & 1
      \end{smallmatrix} 0\right) \]
     surjektiv und damit auch $\pi_{*\overline\phi^{-1}(y, \xi)}$. Offensichtlich ist $\pi$ surjektiv und damit eine Submersion.
\item
	\emph{Zeige:} $\sigma: M \to \T M$, $p \mapsto 0_{\T_pM}$ ist eine Einbettung.
	
	Es gilt
		\[\overline \phi \circ \sigma \circ \phi^{-1} (y) = \overline \phi(0_{\T_{\phi^{-1}(y)}M}) = (\phi(\phi^{-1}(y)),0) = (y,0) \]
	Daraus folgt folgt dass $D(\overline \phi \circ \sigma \circ \phi^{-1})|_y = \left( \begin{smallmatrix} 1 & &  \\ & \ddots & \\ & & 1 \\ & 0 & \end{smallmatrix} \right)$ injektiv ist und damit auch $\sigma_{* \phi^{-1}(y)}$. $\sigma$ ist injektiv und stetig, $\pi \circ \sigma = \Id_m$, also ist $\sigma^{-1} = \pi|_{\Bild(\sigma)}$ stetig.
\item
	\emph{Zeige:} Ist $\Phi: M \to N$ eine glatte Abbildung, so auch $\Phi_*: \T M \to \T N$.
	
	Sei $\psi$ eine Karte um $\Phi(p)$ und $\overline \psi$ die zugeh"orige Karte von $\T N$.
	\begin{align*}
		(\overline \psi \circ \Phi_* \circ \overline \phi^{-1})\underbrace{(y, \xi)}_{\mathclap{\in \phi(U) \X \R^n}} &= (\overline \psi \circ \Phi_*)\left(\sum \xi^{i} \pdifffrac[\phi^{-1}(y)]{}{x^{i}}\right)\\
		&= \overline \psi\left(\sum_i\left(\sum_j \pdifffrac[\phi^{-1}(y)]{\Phi^{i}}{x^j} \xi^j\right) \pdifffrac[\Phi(\phi^{-1}(y))]{}{\psi^{i}}\right)\\
		&= \left( \underbrace{(\psi \circ \Phi \circ \phi^{-1})}_{\text{glatt}}(y), \underbrace{\left( \sum_j \underbrace{\pdifffrac[_\phi^{-1}(y)]{\Phi^{i}}{x^j}}_{\text{glatt in }y} \overbrace{\xi^j}^{\substack{\text{glatt}\\ \text{in }\xi}} \right)_{i = 1,\ldots ,n}}_{\text{glatt}} \right)
	\end{align*}
	Daraus folgt dass $\Phi_*$ glatt ist.
\end{enumerate}\end{Loes}

\begin{Loes}\begin{enumerate}[label=\alph*),leftmargin=*,widest=a]
\item
	\emph{Zu zeigen:} $[\cdot,\cdot]$ ist im Allgemeinen nicht $C^{\infty}(M)$-bilinear.
	
	$M = \R$, $\pdifffrac{}{x} = X = Y$, $f = \Id \quot{= x}$
		\[\begin{array}{rcl} [X,\underbrace{fY}_{=x \pdifffrac{}{x}}] &\overset{\text{c)}}{=}& \left( 1 \underbrace{\pdifffrac{x}{x}}_{=1} - x \underbrace{\pdifffrac{1}{x}}_{=0} \right) \pdifffrac{}{x} = \pdifffrac{}{x} \\
			f[X,Y] &\overset{\text{c)}}{=}& f \left( 1 \underbrace{\pdifffrac{1}{x}}_{=0} - 1 \underbrace{\pdifffrac{1}{x}}_{=0} \right) \pdifffrac{}{x} = 0 \end{array}\]
\item
	\emph{Zu zeigen:} f"ur $X, Y \in \calV(M)$ ist $XY$ mit $(XY)|_p(f) = X_p(Y(f))$ im Allgemeinen keine Derivation.
	\begin{align*}
		(XY)_p(fg) &= X_p(Y(fg))\\
		&= X_p(q \mapsto Y_q(fg))\\
		&= X_p(q \mapsto f(q) Y_q(g) + g(q) Y_q(f))\\
		&= X_p(fY(g) + gY(f))\\
		&= X_p(fY(g)) + X_p(gY(f))\\
		&= f(p) \cdot X_p(Y(g)) + Y_p(g) \cdot X_p(f) + g(p) \cdot X_p(Y(f)) + Y_p(f)X_p(g)\\
		&= f(p) \cdot (XY)|_p(g) + g(p)(XY)|_p(f) + \underbrace{Y_p(g)X_p(f) + Y_p(f)X_p(g)}_{\ne 0}
	\end{align*}
	$M = \R$, $X = Y = \pdifffrac{}{x}$, $f = g = \Id$ $\Rightarrow $ Leibnitz-Regel gilt nicht.
\item
	\emph{Bemerkung:} Ist $X|_U = \sum_{i=1}^{n}\xi^{i}\pdifffrac{}{x^{i}}$ lokale Darstellung bez"uglich $\phi$ von $X \in \calV(M)$, so ist
		\[ \xi^{i} = X(\phi^{i}) \]
	Seien $X|_U = \sum_i \xi^i \pdifffrac{}{x^{i}}$, $Y|_U = \sum_i \eta^{i} \pdifffrac{}{x^{i}}$. Damit gilt dann
	\begin{align*}
		[X,Y](x^j) &= (XY - YX)(x^j)\\
		&= X(Y(x^j)) - Y(X(x^j))\\
		&= X\left(\sum_i \eta^{i}\underbrace{\pdifffrac{}{x^{i}}(x^j)}_{\delta_{ij}}\right) - Y\left(\sum_i \xi^{i}\underbrace{\pdifffrac{}{x^{i}}(x^j)}_{\delta_{ij}}\right)\\
		&= X(\eta^j) - Y(\xi^j)\\
		&= \sum_i \left( \xi^{i} \pdifffrac{}{x^{i}}(\eta^j) - \eta^{i} \pdifffrac{}{x^{i}}(\xi^j) \right)
	\end{align*}
\end{enumerate}\end{Loes}

\begin{Loes}\begin{enumerate}[label=\alph*),widest=a,leftmargin=*]
\item
	Es seien $X = -y\pdifffrac{}{x} + x\pdifffrac{}{y}, Y = -2y\pdifffrac{}{x} + \frac{1}{2}x\pdifffrac{}{y} \in \calV(\R)$, bestimme $\gamma_x^t$ und $\gamma_y^t$.\begin{description}[font=\normalfont\itshape]
	\item[F"ur $X$:]
		$t \mapsto \gamma_*^t(p)$ ist Integralkurve von $X$ mit $\gamma_x^0(p) = p$. \emph{Gesucht:} Kurve mit $\gamma(0) = p$, $\gamma_{*t}\pdifffrac{}{t} = X(\gamma(t)) \Leftrightarrow \gamma_{*t}\pdifffrac{}{t}(x) = X(\gamma(t))(x)$ und $\gamma_{*t}\pdifffrac{}{t}(y) = X(\gamma(t))(y) \Leftrightarrow \gamma_1'(t) = - \gamma_2(t)$ und $\gamma_2'(t) = \gamma_1(t)$. Das Anfangswertproblem
			\[ \begin{pmatrix} \gamma_1 \\ \gamma_2 \end{pmatrix}' (t) = \begin{pmatrix} 0 & -1 \\ 1 & 0 \end{pmatrix} \begin{pmatrix} \gamma_1(t) \\ \gamma_2(t) \end{pmatrix} \text{ und } \gamma(0) = p \]
		hat als L"osung $t \mapsto \exp(t \left( \begin{smallmatrix} 0 & -1 \\ 1 & 0 \end{smallmatrix} \right) ) \cdot p$. Es gilt:
			\begin{align*}
				\begin{pmatrix} 0 & -1 \\ 1 & 0 \end{pmatrix}^{2n} = \begin{pmatrix} (-1)^n & 0 \\ 0 & (-1)^n \end{pmatrix} &&\text{und}&&
				\begin{pmatrix} 0 & -1 \\ 1 & 0 \end{pmatrix}^{2n+1} = (-1)^n \begin{pmatrix} 0 & -1 \\ 1 & 0 \end{pmatrix}
			\end{align*}
		Damit gilt:
		\begin{align*}
			\exp\left(t \begin{pmatrix} 0 & -1 \\ 1 & 0 \end{pmatrix} \right) &= \sum_{k=0}^{\infty} \frac{1}{k!} t^k (\ldots)^k\\
			&= \begin{pmatrix} \sum\limits_{k=0}^{\infty} \frac{(-1)^k}{2k!} t^{2k} & -\sum\limits_{k=0}^{\infty} \frac{t^{2k+1}}{(2k+1)!} (-1)^{k} \\
				-\sum\limits_{k=0}^{\infty} \frac{t^{2k+1}}{(2k+1)!} (-1)^{k} & \sum\limits_{k=0}^{\infty} \frac{(-1)^k}{2k!} t^{2k} \end{pmatrix} \\
			&= \begin{pmatrix} \cos(t) & -\sin(t) \\ \sin(t) & \cos(t) \end{pmatrix}
		\end{align*}
	Daraus folgt $\gamma(t) = \left( \begin{smallmatrix} \cos(t) & -\sin(t) \\ \sin(t) & \cos(t) \end{smallmatrix} \right) \cdot p = \gamma_x^t(p)$
	\item[F"ur $Y$:]
		$\gamma_{*t}\pdifffrac{}{t} = Y(\gamma(t))$, $\gamma(0) = p \xLeftrightarrow{\text{analog}} \gamma'(t) = \left( \begin{smallmatrix} 0 & -2 \\ \frac{1}{2} & 0 \end{smallmatrix} \right) \gamma(t)$
			\[ \begin{pmatrix} 0 & -2 \\ \frac{1}{2} & 0 \end{pmatrix}^{2n} = \begin{pmatrix} (-1)^n & 0 \\ 0 & (-1)^n \end{pmatrix} \]
		Daraus folgt $\underbrace{\gamma(t)}_{\mathclap{= \gamma_x^t(p)}} = \exp(t \left( \begin{smallmatrix} 0 & -2 \\ \frac{1}{2} & 0 \end{smallmatrix} \right) )\cdot p = \left( \begin{smallmatrix} \cos(t) & -2\sin(t) \\ \frac{1}{2}\sin(t) & \cos(t) \end{smallmatrix} \right) \cdot p$
	\end{description}
\item
	asdf
\end{enumerate}\end{Loes}