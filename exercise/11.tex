%%
%% Skript Differentialgeometrie im Wintersemester 12/13
%% Zur Vorlesung von Dr. Grensing am KIT Karlsruhe
%%
%% Uebung 11
%%
\section{21. Januar 2012}
\setcounter{Aufg}{0} %Damit die Aufgaben jedes Mal bei Aufgabe 1 anfangen
\setcounter{Loes}{0}

Allgemein sei $\kappa \in \R$
\begin{Loes}
\emph{Erinnerung:} Definition von Jacobifeldern
\begin{align*}
	\text{Jacobi-Gleichung: } \ddot{\calJ}(t) + (R(\calJ, \dot\gamma) \dot\gamma) (t) = 0
\end{align*}
(Vektorfelder von Variationen durch Geod"atische)
\begin{align*}
	\langle \ddot\calJ, \calJ \rangle & = -R(\calJ, \dot\gamma, \ddot\gamma, \calJ) \\
	& = -\sec(\span\{\calJ, \dot\gamma\}) \cdot (\|\calJ\|^2 \underbrace{\|\dot\gamma\|}_{=0}^2 - \langle\calJ, j \rangle^2) \\
	& = - \kappa \langle \calJ, \calJ \rangle
\end{align*}
Wir m"ochten nun zeigen dass $\ddot\calJ = -\kappa \calJ$ gilt. Es sei $e_1, \ldots ,e_{n-1}, e_n = \dot\gamma(t)$ eine Orthonormalbasis von $\T_{\gamma(t)}M$.
Daraus folgt $0 = R(\calJ, \dot\gamma, \dot\gamma, e_n)$ und $0 = \langle \calJ, e_n \rangle$. F"ur $i < n$ gilt:
\begin{align*}
	R(\calJ + e_i, \dot\gamma, \dot\gamma, \calJ + e_i) = \kappa \cdot ( \| \calJ + e_i \|^2 \cdot \underbrace{\| \gamma \|^2}_{=1} - \underbrace{\langle \calJ + e_i, \dot\gamma \rangle^2}_{=0}) = \kappa \cdot \| \calJ + e_i \|^2
\end{align*}
und
\begin{align*}
	R(\calJ + e_i, \dot\gamma, \dot\gamma, \calJ + e_i) &= R(\calJ, \dot\gamma, \dot\gamma, \calJ) + R(e_i, \dot\gamma, \dot\gamma, e_i) + 2R(\calJ, \dot\gamma, \dot\gamma, e_i) \\
	&= \kappa \cdot ( ||\calJ\|^2 + \|e_i\|^2) + 2R(\calJ, \dot\gamma, \dot\gamma, e_i)
\end{align*}
Daraus folgt
\begin{align*}
	R(\calJ, \dot\gamma, \dot\gamma, e_i) = \frac{1}{2}\kappa ( \|\calJ + e_i\|^2 - \|\calJ\|^2 - \|e_i\|^2 ) = \kappa \langle \calJ, e_i \rangle
\end{align*}
und damit gilt $R(\calJ, \dot\gamma, \dot\gamma) = \kappa \cdot \calJ$ und damit wird die Jacobi-Gleichung zu $\ddot\calJ = -\kappa \calJ$.
Setze $A(0) = \calJ(0)$ und $B(0) = \dot\calJ(0)$ parallel fort zu $A(t)$ und $B(t)$ und definiere $\tilde\calJ(t) = C_\kappa(t) \cdot A(t) + S_\kappa(t) \cdot B(t)$.
\marginnote{$\frac{\D}{\dop t} \hat{=} \nabla_t$}Betrachte nun:
\begin{align*}
	\frac{\D}{\dop t} \tilde\calJ &= C_\kappa' A + C_\kappa \underbrace{\frac{\D}{\dop t} A}_{=0} + S_\kappa' B + S_\kappa \underbrace{\frac{\D}{\dop t} B}_{=0} = C_\kappa' A + S_\kappa' B \\
	\ddot{\tilde{\calJ}} &= C_\kappa'' A + S_\kappa'' B
\end{align*}
Es gilt:
\begin{align*}
	C_\kappa'' = -\kappa C_\kappa \text{ und } S_\kappa'' = -\kappa S_\kappa
\end{align*}
Daraus folgt $\ddot{\tilde{\calJ}} = -\kappa \tilde\calJ$ und damit $\tilde\calJ = \calJ$ (eindeutige L"osung zu gegebenen $\calJ(0)$, $\dot{\calJ}(0)$).
Der Beweis zeigt, dass parallele $A, B \perp \dot\gamma$ ein Jacobifeld definieren.
\end{Loes}

\begin{Loes}\begin{description}[leftmargin=*]
\item[$\bm{k > 0}$:]
	Sei $v \in \T_pM$ mit $\|v|| = 1$ und $\gamma(t) = \exp_p(tv)$.
	Da $M$ vollst"andig ist, ist $\gamma$ auf ganz $\R$ definiert.
	Es sei $B(0) \in v^\perp$ und $B$ die parallele Fortsetzung l"angs $\gamma$.
	Setze $\calJ(t) = S_\kappa(t) \cdot B(t)$.
	Daraus folg dass $\calJ$ ein Jacobifeld ist mit $\calJ(0) = \underbrace{S_\kappa(0)}_{=0} B(0) = 0$ und
	\begin{align*}
		\calJ \left( \frac{\pi}{\sqrt\kappa} \right) = \frac{1}{\kappa} \sin(\pi) \cdot B \left( \frac{\pi}{\sqrt\kappa} \right) = 0
	\end{align*}
	Daraus folgt dass $p$ und $\gamma(\frac{\pi}{\sqrt\kappa})$ konjugiert sind.
\item[$\bm{k \le 0}$:]
	Angenommen $p$ und $q$ sind konjugiert l"angs $\gamma$ (mit $\|\dot\gamma\| = 1$).
	Es sei $\gamma(0) = p$ und $\gamma(t_0) = q$ mit $t_0 \ge 0$.
	Dann gibt es ein Jacobifeld $\calJ \ne 0$ l"angs $\gamma$ mit $\calJ(0) = 0 = \calJ(t_0)$ und damit ist $\calJ$ orthogonal.
	Nach Aufgabe 1 gilt $\calJ = C_\kappa \cdot A + S_\kappa \cdot B$. Es gilt:
	\begin{align*}
		0 = \calJ(0) = C_\kappa(0) \cdot A(0) + S_\kappa(0) \cdot B(0) = A(0)
	\end{align*}
	Da $A$ parallel ist gilt $A \equiv 0$ und daraus folgt
	\begin{align*}
		0 = \calJ(t_0) = \underbrace{S_\kappa(t_0)}_{\mathclap{= \begin{cases} \scriptstyle{t_0} & \scriptsize{\text{falls }} \scriptstyle{k = 0} \\ \scriptstyle{\frac{1}{\sqrt{|\kappa|}} \sinh\left(\sqrt{|\kappa|}t_0\right)} & \scriptsize{\text{falls }} \scriptstyle{k < 0} \end{cases}}} \cdot B(t_0) > 0
	\end{align*}
	Daraus folt $B(t_0) = 0$ und, da $B$ parallel ist, $B \equiv 0$.
	Also ist $\calJ \equiv 0$, was einen Widerspruch darstellt. $\lightning$
\end{description}\end{Loes}

\begin{bem}
Der Beweis von Aufgabe 2 zeigt:
Eine vollst"andige Riemannsche Mannigfaltigkeit mit $\sec \equiv \kappa > 0$ hat Durchmesser $\le \frac{\pi}{\sqrt\kappa} = \diam(S^n(\sqrt\kappa) = \{ x \in \R^{n+1} \mid \|x\| = \sqrt\kappa\})$
\end{bem}

\begin{bsp}\begin{enumerate}[label=(\arabic*),leftmargin=*]
\item
	$S^n, \sec \equiv 1$; $p$ und $-p$ sind konjugiert:
	\begin{center}\begin{tikzpicture}[font=\scriptsize]
		%\tikzgitter{(-3,-3)}{(3,3)}
		\def\radius{1.5}
		\coordinate (p) at (220:\radius); \coordinate (q) at (80:\radius);
		
		\draw (0,0) circle (\radius);
		\begin{scope}
			\clip (-\radius,0) rectangle (\radius,\radius);
			\draw[dashed] (0,0) ellipse[x radius=\radius, y radius=0.5];
		\end{scope}\begin{scope}
			\clip (-\radius,0) rectangle (\radius,-\radius);
			\draw (0,0) ellipse[x radius=\radius, y radius=0.5];
		\end{scope}
		
		\def\angleLow{90}
		\def\angleHigh{210}
		
		\foreach \angle in {20,-30,-50,-70}{
			\draw[gray] (p) ..controls($(p) + (\angleLow+\angle:1)$) and ($(q) + (\angleHigh-\angle:1)$).. (q);
		}
		\draw[very thick] (p) ..controls($(p) + (\angleLow:1)$) and ($(q) + (\angleHigh:1)$)..coordinate[pos=0.7] (pkt) (q);
		
		\draw[->] (-2,1.5) node[left]{$\gamma$} to[out=0,in=140] (pkt);
		
		\fill (p) circle(0.05) node[below left]{$p$} (q) circle(0.05) node[above right]{$-p = \gamma(\pi)$};
		\node at (\radius,-1.25) {$S^{n-1}$};
	\end{tikzpicture}\end{center}
	und $p$ ist zudem zu sich selbst konjugiert ($\calJ(2\pi) = 0$), und dies sind alle zu $p$ konjugierten Punkte.
\item
	$\underline{\R\P^n} = \FakRaum{S^n}{q \sim (-q)}$ mit der von $S^n$ induzierten Metrik $\leadsto [p] = [-p]$. Daraus folgt dass $[p]$ der einzige zu $p$ konjugierte Punkt ist.
	\begin{center}\begin{tikzpicture}[font=\scriptsize]
		%\tikzgitter{(-4,-4)}{(4,4)}
		\def\radius{2.5}
		\coordinate (vec) at (-1.25,-0.25);

		\begin{scope}
			\clip (-\radius-1,0) rectangle (\radius+1,\radius+1);
			\draw (0,0) circle(\radius);
		\end{scope}\begin{scope}
			\clip (-\radius,0) rectangle (\radius,\radius);
			\draw[dashed] (0,0) ellipse[x radius=\radius,y radius=0.75];
		\end{scope}\begin{scope}
			\clip (-\radius,0) rectangle (\radius,-\radius);
			\draw (0,0) ellipse[x radius=\radius,y radius=0.75];
		\end{scope}
		
		\coordinate (p) at (0,\radius); \coordinate (a) at (240:0.7*\radius); \coordinate (b) at ($-1*(a)$); \coordinate (pktA) at (235:2.5 and 0.75); \coordinate (pktB) at (70:2.5 and 0.75);
		\fill (p) circle(0.05) node[above]{$p$} (a) circle(0.05) (b) circle(0.05);
		
		\draw[decoration={markings,mark=at position 0.5 with{\arrow{>}}},postaction={decorate}] (a) -- (b);
		\draw (pktA) ..controls(pktA) and ($(p) + (vec)$).. (p) ..controls($(p) - 0.5*(vec)$) and (b).. (b);
		\draw[dashed] (a) -- (pktA) (b) -- (pktB);
	\end{tikzpicture}\end{center}
 \item
	$(\R^n, g_{\text{eukl}})$ parallele Vektorfelder $=$ konstante Vektorfelder.
	Damit haben die Jacobifelder die Form $\calJ(t) = A + t B$. (mit $A, B \in \R^n$)
\item
	$\H^2$
	\begin{center}\begin{tikzpicture}[font=\scriptsize]
		\fill[gray!20] (-2.5,0) rectangle (2.5,1.75);
		\draw[->] (-3,0) -- (3,0);
		\draw[->,decoration={markings,mark=at position 0.75 with{\arrow{>}}},postaction={decorate}] (0,0) --node[right,pos=0.75]{$\gamma$} (0,2);
		\fill (0,1) circle(0.05) node[right]{$i$};
	\end{tikzpicture}\end{center}
	\begin{align*}
		\gamma(t) &= i e^t \text{ Geod"atische (letzte "Ubung)}\\
		\dot\gamma(t) &= e^t \pdifffrac{}{y} = e^t \begin{pmatrix}0 \\ 1 \end{pmatrix}
	\end{align*}
	Es sei $A$ eine paralleles Vektorfeld l"angs $\gamma$. Dann gilt:
	\begin{align*}
		0 &\overset{!}{=} \nabla_t A = \nabla_t \begin{pmatrix} A_1 \\ A_2 \end{pmatrix} = \nabla_t \begin{pmatrix} A_1 \\ 0 \end{pmatrix} + \nabla_t \begin{pmatrix} 0 \\ A_2 \end{pmatrix} \\
		&= A_1' \begin{pmatrix} 1 \\ 0 \end{pmatrix} + A_1 \cdot \nabla_t \begin{pmatrix} 1 \\ 0 \end{pmatrix} + A_2' \begin{pmatrix} 0 \\ 1 \end{pmatrix} + A_2 \cdot \nabla_t \begin{pmatrix} 1 \\ 0 \end{pmatrix} \\
		&= \begin{pmatrix} A_1' \\ A_2' \end{pmatrix} + A_1 \cdot \nabla_{\dot\gamma} \begin{pmatrix} 1 \\ 0 \end{pmatrix} + A_2 \cdot \nabla_{\dot\gamma} \begin{pmatrix} 0 \\ 1 \end{pmatrix} \qquad \text{wobei } \nabla_{\dot\gamma}= e^t \begin{pmatrix}0\\1\end{pmatrix} \\
		&= \begin{pmatrix} A_1' \\ A_2' \end{pmatrix} + e^tA_1 \begin{pmatrix} \overbrace{\Gamma_{21}^1(\gamma(t))}^{= - \frac{1}{\gamma^2(t)} = \frac{-1}{e^t}} \\ \underbrace{\Gamma_{21}^2(\gamma(t))}_{=0} \end{pmatrix} + e^t A_2 \begin{pmatrix} \overbrace{\Gamma_{22}^1(\gamma(t))}^{=0} \\ \underbrace{\Gamma_{22}^2(\gamma(t))}_{= \frac{1}{e^t}} \end{pmatrix} \\
		&= \begin{pmatrix} A_1' - A_1 \\ A_2' + A_2 \end{pmatrix}
	\end{align*}
	Daraus folgt $A_1(t) = A_1(0) \cdot e^t$ und $A_2(t) = A_2(0) \cdot e^{-t}$.
\end{enumerate}\end{bsp}