%% 
%% Vorlesung <2012-12-14 Fri>, Fortsetzung
%%

\chapter{Geod\"atische und die Exponentialabbildung}

\begin{emptythm}[Heuristik:] Geodätische sind Minimalstellen des Energiefunktionals $\gamma \mapsto E(\gamma) = \int \|\dot\gamma\|^2$. 
Was sind kritische Punkte dieser Abbildung? Für $f \in C^{\infty}(M)$ ist $p$ kritischer Punkt, wenn alle Richtungsableitungen verschwinden, das hei"st $0 = X(f) = \difffrac[t=0]{}{t}(f(c(t)))$.
\end{emptythm}

\begin{center}\begin{tikzpicture}[font=\scriptsize]
%	\draw[step=0.25,gray!15] (-6,-4) grid (6,4); \draw[step=0.5,gray!30] (-6,-4) grid (6,4); \fill (0,0) circle(0.1); %Hilfsgitter
	
	\coordinate (p) at (-2,-1); \coordinate (q) at (2,1); \coordinate (wirbel) at (-0.5,-0.5);
	\coordinate (ctrl1) at (1,0);
	\def\left{0.75}
	\def\right{1.75}
	\fill (p) circle(0.05)node[below]{$p$}; \fill (q) circle(0.05)node[right]{$q$};
	
	\draw[name path=kurve] (p) ..controls(p) and ($(wirbel) - \left*(ctrl1)$).. (wirbel)node[below]{$\gamma(t)$} ..controls($(wirbel) + \right*(ctrl1)$) and (q)..node[below]{$\gamma$} (q);
	\fill (wirbel) circle(0.05);
	
	\coordinate (vec) at (-0.5,1);
	\foreach \shift in {0.2,0.4,...,1}{
		\coordinate (neuwirbel) at ($(wirbel) + \shift*(vec)$);
		\draw[name path=obere kurve] (p) ..controls(p) and ($(neuwirbel) - \left*(ctrl1)$).. (neuwirbel) ..controls($(neuwirbel) + \right*(ctrl1)$) and (q).. (q);
	}
	\draw[->] (wirbel) -- ($(wirbel) + 1.3*(vec)$);
	
	\path[name path=vert] (0,-1) -- (0,1);
	\path[name intersections={of={kurve and vert}}];
	\fill (intersection-1) circle(0.05);
	\draw[->] (intersection-1) -- ($(intersection-1) + (0.5,1)$);
	\path[name intersections={of={obere kurve and vert}}];
	\node[above] at (intersection-1) {$h_s$};
\end{tikzpicture}\end{center}

Eine \quot{Kurve} durch $\gamma$ ist eine sogenannte \CmMark[Variation!glatte]{glatte Variation} $h\colon[0,1]\times[0,1] \to M$, $h(s,t) = h_s(t)$ mit $h_0 = \gamma$ und $h_s(0) = p$, sowie $h_s(1) = q$ f"ur alle $s \in [0,1]$. Dann ist
\begin{align*}
  X(t) = \difffrac[s=0]{}{s}h_s(t)
\end{align*}
ein glattes Vektorfeld entlang $\gamma$.
Ferner gilt $X(0) = 0$ und $X(1) = 0$.
Nun betrachte
\begin{align*}
	0  = \difffrac[s=0]{}{s}E(h_s) &= \int_{0}^{1} \difffrac[s=0]{}{s} \left<\difffrac{}{t} h_s(t),\difffrac{}{t}h_s(t)\right>\\
	& = \int_0^1 2 \left<\nabla_s \difffrac{}{t}h_s(t), \difffrac{}{t}h_s(t)\right>\\
	& = \int_0^1 2 \left<\nabla_t\smash{\underbrace{\difffrac{}{s}h_s(t)}_{=X(t)}}, \difffrac{}{t}h_s(t)\right> \vphantom{\underbrace{\difffrac{}{s}h_s(t)}_{=X(t)}}\\
	& = \int_0^1 2 \left<\nabla_tX,\difffrac{}{t}h_s(t)\right>\\
	& = 2 \int_0^1 \difffrac{}{t}\left<X,\difffrac{}{t}h_s(t)\right> - \left<X,\nabla_t\difffrac{}{t}h_s(t)\right>\\
	& = \underbrace{2 \int_0^1 \difffrac{}{t}\left<X,\difffrac{}{t}h_s(t)\right>}_{=0} - 2 \int_0^1 \left<X,\nabla_t\difffrac{}{t}h_s(t)\right>\\
	& = -2 \int_0^1 \left<X(t),\nabla_t\dot\gamma(t)\right>\dop t
\end{align*}

% Definition 8.1
\begin{Dfn}\label{dfn-8-1}
  Eine glatte Kurve $c$ in $M$ heißt \CmMark{Geod\"atische}\footnote{Die Äquivalenz zur bereits bekannten Definition wird in Kürze gezeigt.}, wenn $\nabla_t\dot c \equiv 0$ gilt.
\end{Dfn}

Ist $c$ Geodätische, so ist $c$ proportional zur Bogenlängenparametrisierung, das hei"st $\|\dot c\| = $const, denn $\difffrac{}{t}\|\dot c(t)\|^{2} = \difffrac{}{t}\left<\dot c(t),\dot c(t) \right> = 2\left<\nabla_t\dot c(t), \dot c(t)\right> = 0$.
Mit $c$ ist auch jede affine Umparametrisierung $t \mapsto c(at + b)$ eine Geodätische.

% Proposition 8.2
\begin{Prop}
Für jedes $p \in M$ und $v \in \T_pM$ existiert genau eine Geodätische $\gamma_{p,v}\colon[0,\epsilon] \to M$ mit $\gamma_{p,v}(0) = p$ und $\dot \gamma_{p,v}(0) = v$.
Zudem hängt $\gamma_{p,v}$ glatt von $p$ und $v$ ab.
\end{Prop}

\begin{bew}\begin{enumerate}[label=(\Alph*),leftmargin=*,widest=B]
\item
	Es sei $(\phi, U)$ eine Karte um $p$, $\gamma^i(t) = \phi^i(\gamma(t))$. Dann besitzt das folgende Anfangswertproblem
	\begin{align*}
		\begin{cases}
			0 = \nabla_t\dot \gamma|_t = \sum_k\left(\ddot \gamma^k(t) + \sum_{ij}\Gamma_{ij}^k\big(\gamma(t)\big)\dot \gamma^i(t)\gamma^j(t)\right) \pdifffrac[\gamma(t)]{}{x^k}\\
			\gamma^i(0) = \phi^i(p)\\
			\dot\gamma^i(0) = \xi_p^i, \quad v = \sum \xi^i_p\pdifffrac[p]{}{x^i}
		\end{cases}
	\end{align*}
	eine eindeutige Lösung (lokal), wleche glatt von den Startwerten $p$ und $v$ abhängt.
\item
	(Alternativ) Ist $(\phi, U)$ eine Karte von $M$ um $p$, dann ist
	\[ \overline \phi \colon \left\{\begin{array}{cccl}
		\T M|_U &\to& \R^{2m}&\\
		X_p=\sum \xi_p^i\pdifffrac[p]{}{x^i} &\mapsto& \overline\phi(X_p) &= (\phi^1(p), \ldots, \phi^m(p), \xi_p^1, \ldots, \xi_p^m)\\
		&&& =: (y^1, \ldots, y^{2m})
	\end{array}\right.\]
	eine Karte von $\T M$.	
	Es sei $S$ das durch
	\[ S \colon \left\{ \begin{array}{ccc}
		\T M &\to& \T\T M\\
		X = \sum \xi^i \pdifffrac{}{x^i} &\mapsto& \sum_i^m \xi^i \pdifffrac{}{y^i} - \sum_{i,j,k=1}^{m} \Gamma_{ij}^k \xi^i\xi^j\pdifffrac{}{y^{m+k}}
	\end{array}\right.\]
	defninierte glatte Vektorfeld auf $\T M$.	
	$g^t$ ist genau dann Integralkurve von $S$ durch $X_p = \sum \xi_p^i\pdifffrac[p]{}{x^i}$, wenn
	\begin{align*}
		\difffrac{}{t}g^t = \dot g^t = S(g^t) \text{ und } g^0 = X_p.
	\end{align*}
	Setzt man $\overline \phi(g^t) = (\gamma^1(t), \ldots, \gamma^m(t),\eta^1(t), \ldots, \eta^m(t))$, so ist dies genau dann der Fall, wenn gilt:
	\begin{align*}
		& (\dot\gamma^1,\ldots, \dot\gamma^m,\dot\eta^1,\ldots, \dot\eta^m) = \left(\eta^1, \ldots, \eta^m, -\sum_{i,j}\Gamma_{ij}^1\eta^i\eta^j, \ldots, -\sum_{i,j}\Gamma_{ij}^m\eta^i\eta^j\right)\\
		& \rightsquigarrow \eta^i = \dot\gamma^i \text{ und } \ddot\gamma = -\sum_{i,j}\Gamma_{ij}^k\dot\gamma^i \dot\gamma^j
	\end{align*}
	und 
	\begin{align*}
		(\gamma^1(0), \ldots, \gamma^m(0), \eta^1(0), \ldots, \eta^m(0) = \overline\phi(X_p) = (\phi^1(p), \ldots, \phi^m(p), \xi_p^1, \ldots, \xi_p^m)
	\end{align*}
	also genau dann, wenn
	\begin{align*}
		\gamma(t) = \overline\phi^{-1}(\gamma^1(t), \ldots, \gamma^m(t))
	\end{align*}
	eine Geodätische durch $p$ mit $\dot \gamma(0) = X_p$ ist.	
	Der maximale Fluss $g^t$ von $S$ heißt \CmMark[Fluss!geod\"atischer]{geod"atischer Fluss}.
	Mit Satz \ref{satz-4-9} folgt die Aussage der Proposition.
\end{enumerate}\end{bew}

\begin{center}\begin{tikzpicture}[font=\scriptsize]
%	\draw[step=0.25,gray!15] (-6,-1) grid (6,5); \draw[step=0.5,gray!30] (-6,-1) grid (6,5); \fill (0,0) circle(0.1); %Hilfsgitter
	
	\def\breite{2.5}
	\def\hoehe{2}
	\def\shift{1}
	\def\vert{2.5}
	\draw (-\breite, \vert) -- (-\breite+\shift, \vert+\hoehe) -- (\breite+\shift, \vert+\hoehe) -- (\breite, \vert) -- cycle;
	\fill (-0.5,3.25) circle(0.05) node[left]{$0_p$};
	\draw[->] (-0.5,3.25) --node[above]{$v$} (0.75,3.75);
	\node at (3.5,3.5) {$\T_pM$};
	
	\coordinate (segel) at (-2.5,-0.75); \node at ($(segel) + (4.75,1.25)$) {$M$};
	\tikzsegel[1.5]{(segel)}
	\coordinate (pkt) at ($0.75*(-0.25,0.5)+(segel3)$);
	\coordinate (ctrl1) at (1,1); \coordinate (ctrl2) at (-1,1); \coordinate (ctrl3) at (0,1); \coordinate (ctrl4) at (-1.25,-0.25);
	\draw[dashed] (segel1) ..controls($(segel1) + 0.5*(ctrl1)$) and ($(pkt) + 0.25*(ctrl2)$).. (pkt) ..controls($(pkt) + 0.5*(ctrl3)$) and ($(segel2) + (ctrl4)$).. (segel2);
	
	\coordinate(pkt1) at ($(segel) + (1.25,0.75)$); \coordinate(pkt2) at ($(segel) + (2.5,1)$); \coordinate(pkt3) at ($(segel) + (3.75,1.75)$);
	\fill(pkt1) circle(0.05)node[anchor=north east,font=\tiny]{$p$}; \fill(pkt2) circle(0.05) node[below,font=\tiny]{$\gamma_v(1)= \exp_p(v)$};
	\coordinate (ctrl1) at (1.5,1); \coordinate (ctrl2) at (-1,-0.25);
	\draw[->](pkt1) --node[above,sloped,font=\tiny]{$\dot\gamma_v(0)=v$} ($(pkt1) + 0.75*(ctrl1)$);
	\draw (pkt1) ..controls($(pkt1) + 0.25*(ctrl1)$) and ($(pkt2) + 0.5*(ctrl2)$).. (pkt2) ..controls($(pkt2) - 0.5*(ctrl2)$) and (pkt3).. (pkt3);
\end{tikzpicture}\end{center}

Für $v \in \T_pM$ sei $\gamma_v(t) = \pi(g^t(v))$ die eindeutige Geodätische mit $\gamma_v(0) = p$ und $\dot \gamma_v(0) = v$.
Ist $\delta \in \R$ und $c(t) = \gamma_v(\delta t)$, so ist $c$ eine Geodätische durch $p$ mit $\dot c(0) = \delta v$, das hei"st $c = \gamma_{\delta v}$, beziehungsweise $\gamma_{\delta v}(t) = \gamma_v(\delta t)$.

Der Definitionsbereich $\mathcal D_S$ des geodätischen Flusses ist eine offene Menge in $\R \X \T_pM$ und somit sind sowohl $\mathcal D = \{v \in \T M \mid (1,v) \in \mathcal D_S\}$, als auch $\mathcal D_p = \mathcal D \cap T_pM$ offen für alle $p \in M$ (in $\T M$, beziehungsweise $\T_pM$). Weiterhin gilt $0_p \in \mathcal D_p$.

% Definition 8.3
\begin{Dfn}
  Die Abbildung $\exp_p\colon\mathcal D_p \to M$, $v \mapsto \gamma_v(1)$ heißt \CmMark{Exponentialabbildung}.
\end{Dfn}


%% 
%% Vorlesung <2012-12-18 Tue>
%%

Es wurde bereits gezeigt, dass $\nabla_t \dot \gamma_v \equiv 0$ ist (Geodätische Differentialgelichung).
Die Exponentialabbildung is nach Satz \ref{satz-4-6} glatt.
Es gilt $\exp_p(0_p) = p$.
Zur Berechnung des Differentiales von $\exp_p$ in $0_p$
\begin{align*}
  \exp_{p*0_p} \colon \T_{0_p}\T_pM \to \T_pM
\end{align*}
identifiziert man $\T_{0_p}\T_pM$ mit $\T_pM$.
Es gilt
\begin{align*}
  \exp_{p*0_p}(v) = \difffrac[t=0]{}{t}\exp_p(tv) = \difffrac[t=0]{}{t}\gamma_{tv}(1) = \difffrac[t=0]{}{t}\gamma_v(t) = \dot \gamma_v(0) = v,
\end{align*}
also $\exp_{p*0_p} = 1 \dop_{\T_pM}$.
Es existiert für alle $p \in M$ eine Umgebung $V$ von $0_p \in \T_pM$ und $U$ von $p$, so dass $\exp_p \colon V \to U$ ein Diffeomorphismus ist.
Wählt man eine Orthonormalbasis $e_1, \ldots, e_m$ von $\T_pM$ und setzt
\begin{align*}
  \psi \colon \T_pM \to \R^m, v = \sum_i b^ie_i \mapsto (b^1, \ldots, b^m),
\end{align*}
so ist $(\psi \circ \exp_p|_U^{-1}, U)$ eine Karte von $M$ um $p$.
Im Allgemeinen ist dies keine Isometrie!

% Definition 8.4
\begin{Dfn}
Diese Karte bezeichnet man als \CmMark[Normalkoordinaten!Riemannsche]{Riemannsche Normalkoordinaten}.
\end{Dfn}

% Proposition 8.5
\begin{Prop}
In Riemannschen Normalkoordinaten gilt für alle $i,j,k \leq m$:
\begin{enumerate}[label=(\roman*)]
\item
	$g_{ij}(0) = \delta_{ij}$
\item
	$\Gamma^k_{ij}(0) = 0$
\item
	$\partial_k g_{ij}(0) = \pdifffrac[0]{g_{ij}}{x^k} = 0$
\end{enumerate}\end{Prop}

Der Beweis sei zur "Ubung "uberlassen.

\section{Polarkoordinaten}

Es ist $\phi = (r, \theta^1, \ldots, \theta^{m-1})$ die Hintereinanderausführung von Riemannschen Normalkoordinaten des $\R^m$.

\begin{center}\begin{tikzpicture}[font=\scriptsize]
%	\draw[step=0.25,gray!15] (-3,-6) grid (9,6); \draw[step=0.5,gray!30] (-3,-6) grid (9,6); \fill (0,0) circle(0.1); %Hilfsgitter
	
	\def\breite{2.5}
	\def\hoehe{2}
	\def\shift{1}
	\def\vert{2.5}
	\draw (-\breite, \vert) -- (-\breite+\shift, \vert+\hoehe) -- (\breite+\shift, \vert+\hoehe) -- (\breite, \vert) -- cycle;
	
	\coordinate (0p) at (-1,3);
	\draw[->] (0p) --node[below]{$e_1$} +(2,0); \draw[->] (0p) --node[left]{$e_2$} +(0.5,1); \draw[->] (0p) -- +(25:2) node[below]{$v$}; \fill (0p) circle(0.05) node[anchor=north east]{$0_p$};
	%\draw[->] ($(0p) + (1.25,0)$) arc (0:25:1.25) node[right]{$\theta$};
	\draw[->] ($(0p) + (1.25,0)$) arc (0:12.5:1.25) node[right]{$\theta$} arc(12.5:25:1.25);
	
	\draw[->] (-0.5,1) to[out=120,in=240]node[right]{$\exp_p^{-1}$} +(0,1.25);
	
	\def\scl{1.5}
	\tikzsegel[\scl]{(-2.5,-1)};
	\coordinate (pkt) at ($(segel3) + 0.5*(-0.25,0.75)$);
	\draw[dashed] (segel1) ..controls($(segel1) + (ctrls6)$) and ($(pkt) + (ctrls5)$).. (pkt) ..controls($(pkt) + (ctrls4)$) and ($(segel2) + (ctrls3)$).. (segel2);
	
	\coordinate (p) at ($(segel) + (1.5,0.75)$); \coordinate (q) at ($(segel) + (3.25,0.75)$); \fill (p) circle(0.05)node[left]{$p$}; \fill (q) circle(0.05)node[above right]{$q = \exp_p(v) = \gamma_v(1)$};
	\draw[->] (p) -- +(1,0)node[below]{$\gamma_v$}; \draw (p) -- (q);
	
	\node (q) at (5,0) {$q= \exp_p(v)$}; \node (v) at (5,3.5) {$v$}; \node[align=flush left,text depth=-18pt,anchor=west] (v2) at (7.4,3.5) {$(\| v \|, \theta)$\\ $= (r,\theta)$ \\ $=\phi(q) \in (0,\epsilon) \X S^1$}; \node[anchor=west] (R) at (7.4,4) {$\R^2$};
	\draw[|->] (q) -- (v); \draw[|->] (v) -- (v2);
	\draw[->] ($(v) + (0,0.5)$) -- (R);
\end{tikzpicture}
\end{center}

Die Umkehrabbildung ist ein Diffeomorphismus
\begin{align*}
  f \colon (0, \epsilon) \times S^{m-1} \to U \subseteq M, \ 
  (t,v) \mapsto \exp_p(tv) = \gamma_v(t).
\end{align*}
Für jedes $v \in S^{m-1}$ ist $t \mapsto f(t,v) = \gamma_v(t)$ eine Geodätische in $M$. Wir bezeichnen solche Geod"atischen im Folgenden als \CmMark[Geod\"atische!radiale]{radiale Geod\"atische}.

% Lemma 8.6
\begin{Lemma}[Gauß-Lemma]
  Jede radiale Geodätische $\gamma_v$ ist orthogonal zu der geodätischen Sphäre
  \begin{align*}
    S_r = \{q \in M \mid \ \exists v \in \T_pM: \|v\| = r \text{ und } q = \exp_p(v) \}.
  \end{align*}
\end{Lemma}

\begin{bew}
Man zeigt das Folgende:
Ist $X$ ein Vektorfeld auf $S^{m-1}$ und bezeichnet man seine Fortsetzung auf $(0,\epsilon) \times S^{m-1}$ \quot{$\subseteq$} $\B_{\epsilon}(0)\setminus\{0\}$ bzw. $\B_{\epsilon}(0_p)\setminus\{0_p\} \subseteq \T_pM$ mit $X_{rv} = X_v$, so ist
\begin{align*}
	Y_q = Y_{f(r,v)} = f_{*(r,v)}(0,X_v) = \exp_{p*}(r X_v)
\end{align*}
orthogonal zu 
\begin{align*}
	\pdifffrac[q]{}{r} = \difffrac[t=r]{}{t}\exp_p(tv) = \dot \gamma_v(r)
\end{align*}

\begin{center}\begin{tikzpicture}[font=\scriptsize,normal/.style={above,sloped, inner sep=0pt,outer sep=0pt,allow upside down}]
%	\draw[step=0.25,gray!15] (-6,-1) grid (6,5); \draw[step=0.5,gray!30] (-6,-1) grid (6,5); \fill (0,0) circle(0.1); %Hilfsgitter
	
	\def\breite{2.75}
	\def\hoehe{2.5}
	\def\shift{1}
	\def\vert{2.25}
	\draw (-\breite, \vert) -- (-\breite+\shift, \vert+\hoehe) -- (\breite+\shift, \vert+\hoehe) -- (\breite, \vert) -- cycle;
	\draw[->] (-2,2.75)node[above]{$0_p$} -- (2.5,2.75); \draw[dashed, name path=strich] (-2,2.75) -- (1.5,4.75); \fill (-2,2.75) circle(0.05);
	
	\foreach \x in {-1.25, -0.25, 0.75}{
		\path[name path=senkrecht] (\x,2) -- (\x,5);
		\path[name intersections={of={strich and senkrecht}},draw];
		\draw[->] (\x,2.75) -- (intersection-1);
		\draw (\x,2.875) to[out=180,in=90] (\x-0.125,2.75); \fill ($(\x,2.75)+(135:0.125/2)$) circle(0.02);
	}
	\node at (0,3.25) {$X_v$}; \node at (1.5,3.75) {$X_{rv} = rX_v$};
	\path (-0.25,2.75)node[below]{$v$}; \path (0.75,2.75)node[below]{$rv$};
	
	\draw[->] (-0.5,2) to[out=240,in=120]node[right]{$\exp_p$} (-0.5,1);
	
	\coordinate (segel) at (-2.5,-1);
	\tikzsegel[1.5]{(segel)}
	
	\coordinate (p) at ($(segel) + (0.75,0.5)$); \coordinate (q) at ($(segel) + (2.75,-0.25)$); \fill (p) circle(0.05) node[anchor=north east]{$p$};
	\draw[->] (p) ..controls($(p) + (0.5,0.25)$) and ($(q) + (-0.5,0.75)$)..node[below]{$\gamma_v$} (q) node[pos=0.4,normal]{\tikz \draw[->] (0,0) -- ++(0,0.7);} node[pos=0.8,normal]{\tikz \draw[->] (0,0) -- ++(0,1.5);};
	\node at ($(segel) + (2,1)$) {$Y_1$}; \node at ($(segel) + (3,1.25)$) {$Y_r$};
\end{tikzpicture}\end{center}

$Y(t) = Y_{\gamma_v(t)}$ als Vektorfeld entlang $\gamma_v$.
Dann gilt:
\begin{align*}
	\difffrac[t=r]{}{t} \left<Y,\pdifffrac{}{r}\right>_{\gamma_v(t)} & = \left<\nabla_tY|_r, \dot \gamma_v(r)\right> + \left<Y(r), \smash{\underbrace{\nabla_t\dot\gamma_v|_r}_{=0}}\right> \vphantom{\underbrace{\gamma_v}_{0}}\\
	& = \left<\nabla_{Y(r)}\dot\gamma_v(r), \dot \gamma_v(r)\right> + \left< \smash{\underbrace{[\dot\gamma_v(r),Y(r)]}_{\mathclap{\begin{subarray}{l}= [f_{*}(\pdifffrac{}{r}),f_{*}(0,X_v)]\\ = f_{*}[\pdifffrac{}{r},X] = 0\end{subarray}}}}, \dot \gamma_v(r) \vphantom{\nabla_{Y(r)}} \right>  \vphantom{\underbrace{\gamma_v(r)}_{\pdifffrac{}{r}} }\\
	&= \frac{1}{2}Y(t) \|\dot\gamma_v\|^2 = 0.
\end{align*}
Ferner gilt
\begin{align*}
	\left<Y(r),\pdifffrac{}{r}\right>_{\gamma_v(r)} = \left<\exp_{p*}(rX_v),\dot\gamma_v(r)\right> \xrightarrow{r \to 0}\left<\exp_{p*}(0_p),v\right> = 0,
\end{align*}
also $\left<Y,\pdifffrac{}{r}\right> \equiv 0$.
\end{bew}

\begin{bem}
  Insbesondere gilt für alle $i \leq m-1$:
  \begin{align*}
    \left<\pdifffrac{}{r}, \pdifffrac{}{\theta^i}\right> = 0.
  \end{align*}
\end{bem}

% Satz 8.7
\begin{Satz}\label{satz-8-7}
Für jedes $p \in M$ existiert ein $\epsilon > 0$, so dass für alle $q \in \B_{\epsilon}(p)$ genau eine minimierende Geodätische von $p$ nach $q$ existiert, das hei"st eine Geodätische $\gamma$ im Sinne der Definition \ref{dfn-8-1} mit $\mathcal L(\gamma) = \dop(p,q)$.
Ist $q \notin \exp_p(\B_{\epsilon}(0_p)) = \B_{\epsilon}(p)$, so existiert ein $q' \in \partial \B_{\epsilon}(p)$ mit
\begin{align*}
	\dop(p,q) = \epsilon + \dop(q',q).
\end{align*}
Ferner, ist $\delta < \epsilon$ und $q \notin \B_\delta(p)$, so existiert ein $q' \in \B_\delta(q)$ mit
\begin{align*}
	\dop(p,q) = \delta + \dop(q',q)
\end{align*}
\end{Satz}

\begin{bew}
\marginnote{\begin{tikzpicture}[font=\scriptsize,scale=0.7]
%	\draw[step=0.25,gray!15] (-6,-3) grid (6,3); \draw[step=0.5,gray!30] (-6,-3) grid (6,3); \fill (0,0) circle(0.1); %Hilfsgitter
	\coordinate (p) at (0,0); \coordinate (q) at (3.5,1.5);
	\coordinate (a) at (1.5,1);
	\coordinate (ctrlp) at (0.25,1.5); \coordinate (ctrla) at (0.75,0.25); \coordinate (ctrlq) at (-1.75,1);
	\draw[dashed] (p) circle(1.75);
	\draw[name path=kreis] (p) circle(1);
	\draw[name path=kurve] (p) ..controls($(p) + (ctrlp)$) and ($(a) - (ctrla)$).. (a) ..controls($(a) + (ctrla)$) and ($(q) + (ctrlq)$).. (q) node[pos=0.6,above]{$c$};
	\begin{scope}
		\path[clip] (p) circle(1);
		\draw[very thick] (p) ..controls($(p) + (ctrlp)$) and ($(a) - (ctrla)$).. (a) ..controls($(a) + (ctrla)$) and ($(q) + (ctrlq)$).. (q);
	\end{scope}
	\fill (p) circle(0.1)node[below left]{$p$} (q) circle(0.1)node[above]{$q$};
	\draw[name path=strich] (p) -- (q);
	\path[name intersections={of={kreis and kurve}}];
	\coordinate (ct0) at (intersection-1);
	\path[name intersections={of={kreis and strich}}];
	\coordinate (q') at (intersection-1);
	\fill (ct0) circle(0.1)node[above]{$c(t_0)$} (q') circle(0.1)node[below right]{$q'$};
\end{tikzpicture}}
Es sei $\epsilon > 0$ so, dass auf $\B_{\epsilon}(p)$ Polorkoordinaten $\phi = (r,\theta^1, \ldots, \theta^{m-1})$ existieren. Sei weiter $c \colon [0,1]$ eine beliebige glatte Kurve von $p$ nach $q$ mit Koordinaten $\phi(c(t)) = (r(t), \theta^1(t), \ldots, \theta^{m-1}(t))$.
\begin{center}\begin{tikzpicture}[font=\scriptsize]
%	\draw[step=0.25,gray!15] (-6,-6) grid (6,6); \draw[step=0.5,gray!30] (-6,-6) grid (6,6); \fill (0,0) circle(0.1); %Hilfsgitter
	
	\coordinate (segel) at (-3,-1);
	\tikzsegel[2]{(segel)};
	\coordinate (p) at ($(segel) + (2.5,1.25)$); \coordinate (q) at ($(segel) + (3.25,0.75)$);
	\coordinate (pkt1) at ($(segel) + (3.25,2)$); \coordinate (pkt2) at ($(segel) + (3.75,1.75)$);
	\coordinate (ctrl1) at (1,0); \coordinate (ctrl2) at (1,1); \coordinate (ctrl3) at (0,1); \coordinate (ctrl4) at (-1,-0.25);

	\fill (p) circle(0.05)node[left]{$p$}; \fill (q) circle(0.05)node[right]{$q$};
	\draw (p) ..controls($(p) + 0.75*(ctrl1)$) and ($(pkt1) - 0.25*(ctrl2)$).. (pkt1) ..controls($(pkt1) + 0.25*(ctrl2)$) and($(pkt2) + 0.25*(ctrl3)$).. (pkt2)node[right]{$c$} ..controls($(pkt2) - 0.5*(ctrl3)$) and($(q) + (ctrl4)$).. (q);
	
	\draw[clip] ($(segel) + (2.825,1)$) ellipse(1 and 0.5);
	\draw[very thick] (p) ..controls($(p) + 0.75*(ctrl1)$) and ($(pkt1) - 0.25*(ctrl2)$).. (pkt1);
\end{tikzpicture}\\
Das Bild von $c$ ist nicht notwendig in $\B_\epsilon(0)$ enthalten
\end{center}
Für $t_0 = \inf\{t \in [0,1] \mid c(t) \notin \B_{\epsilon}(p)\}$ ist $c|_{[0,t_0]}$ eine Kurve zu $\B_{\epsilon}(p)$.
Es gilt
\begin{align*}
	\left\|\pdifffrac[t]{}{r}\right\| = \|\dot\gamma_w(t)\| = \|w\| = 1.
\end{align*}
Aus der Cauchy-Schwarz-Ungleichung folgt
\begin{align*}
	\|\dot c(t)\| & = \|\dot c(t)\|\left\|\pdifffrac[c(t)]{}{r}\right\|\\
	& \geq \left|\left< \dot c(t), \pdifffrac[c(t)]{}{r}\right>\right|\\
	& = \left|\left<\dot r(t) \pdifffrac{}{r} + \sum_{i=1}^{m-1}\dot\theta^i(t)\pdifffrac{}{\theta^i},\pdifffrac[c(t)]{}{r}\right>\right|\\
	& = \left|\left<\dot r(t)\pdifffrac[c(t)]{}{r}, \pdifffrac[c(t)]{}{r}\right>\right|\\
	& = \left|\dot r(t)\right|,
\end{align*}
wobei die Gleichheit genau dann gilt, wenn $\dot c(t)$ und $\pdifffrac[c(t)]{}{r}$ linear abhängig sind.
\begin{align*}
	\mathcal L(c) = \int_0^{t_0}\|\dot c\| + \int_{t_0}^T\|\dot c\| \geq \int_0^{t_0} \left|\left<\dot c,\pdifffrac{}{r}\right>\right| = \int_0^{t_0}|\dot r| = r(t_0)
\end{align*}
Gleichheit gilt genau dann, wenn $\theta^1(t), \ldots, \theta^{m-1}(t)$ konstant sind und $\dot r(t) \geq 0$ gilt, also genau dann, wenn $c$ eine monotone Umparametrisierung von $t \mapsto \exp_p(tv)$ f"ur $v \in S^{m-1}$ ist.

F"ur den zweiten Teil sei $\epsilon$ so, dass Polarkoordinaten $\phi=(r, \theta^1,\ldots ,\theta^{m-1})$ um $p$ existieren. Es sei $q \in \B_\delta(p)$ und $c$ sei eine glatte Kurve von $p$ nach $q$.
F"ur $t_0 = \inf \{ t \in [0,1] | c(t) \notin \B_\delta(p) \}$ gilt dann:
\begin{align*}
	\calL(c) \ge \delta + \dop(c(t_0), q) \ge \delta + \dop(\partial \B_\delta(p), q),
\end{align*}
also $d(p,q) = \inf_c \calL(c) \ge \delta + \dop(\partial \B_\delta(p), q)$. Da $\partial \B_\delta(p)$ kompakt ist, die Abstandsfunktion $\dop(\cdot, q)$ auf $\partial \B_\delta(p)$ ihr Minimum in $q'$ an. Damit gilt
\begin{align*}
	\dop(q',q) &= \dop(\partial \B_\delta(p), q) \qquad \text{und}\\
	\dop(p,q) &= \dop(p,q') + \dop(q',q) = \delta + \dop(q',q)
\end{align*}
somit gilt dann die Behauptung.
\end{bew}

\begin{Kor}\label{kor-8-8}
F"ur alle $p \in M$ existiert ein $\rho > 0$, so dass f"ur alle $q, q' \in \B_\rho(p=$ genau eine minimierende Geod"atische von $q$ nach $q'$ existiert.
\end{Kor}

\begin{bew}
F"ur $q \in M$ existiert ein $\rho = \rho(q) > 0$, so dass $\exp$ auf $\B_\rho(q)$ ein Diffeomorphismus ist. Da $\exp: \calD \to \calD$ glatt und $\calD$ offen ist, existiert eine Umgebung $U_q$ von $q$, so dass $\exp_p: \B_{\frac{\rho}{2}}(0_q) \to \B_{\frac{\rho}{2}}(q')$ ein Diffeomorphismus ist f"ur alle $q' \in U_q$. F"ur $p \in ;$ existiert nach Satz \ref{satz-8-7} ein $\epsilon > 0$, so dass $\overline \B_\epsilon(p)$ kompakt ist. Die "Uberdeckung $\bigcup_{q \in \overline \B_\epsilon(q)} \B_{\frac{\rho(q)}{2}}(q)$ besitzt eine endliche Teil"uberdeckung. F"ur $\rho = \min_{i \le k} \{ \frac{\rho(q_i)}{4} \}$ existieren auf jedem $\B_{2\rho}(q)$, $q \in \B_\epsilon(p)$, Polarkoordinaten; insbesondere existiert f"ur $q, q' \in \B_\rho(p)$ eine eindeutige minimierende Geod"atische von $q$ nach $q'$.
\end{bew}

\begin{bem}
Die Geod"atischen im obigen Korollar h"angen stetig von ihren Endpunkten ab.
\end{bem}

\begin{Kor}\label{kor-8-9}
Es seien $p, q \in M$ und $c: [0,1] \to M$ st"uckweise glatte Kurven von $p$ nach $q$, so dass $\calL(c) = \dop(p,q)$. Damit ist $c$ eine unparametrisierte Geod"atische im Sinne von Definition \ref{dfn-8-1}.
\end{Kor}

\begin{bew}
\marginnote{\begin{tikzpicture}[font=\scriptsize,scale=0.5]
%	\draw[step=0.25,gray!15] (-6,-1) grid (6,5); \draw[step=0.5,gray!30] (-6,-1) grid (6,5); \fill (0,0) circle(0.1); %Hilfsgitter
	\draw[name path=kurve] (-3,0) ..controls(-3,0) and (-2.5,1).. (-1,1.5)node[above]{$c$} ..controls(0.5,2) and (2.5,2).. (2.5,2);
	\path[name path=links] (-2.5,0) -- (-2.5,3); \path[name path=right] (1,0) -- (1,3);
	\path[name intersections={of={kurve and links}}]; \coordinate (cs) at (intersection-1);
	\path[name intersections={of={kurve and right}}]; \coordinate (ct) at (intersection-1);
	\fill (cs) circle(0.1); \fill (ct) circle(0.1);
	\draw (cs)node[left]{$c(s)$} --node[below]{$\overline c$} (ct)node[above]{$c(t)$};
\end{tikzpicture}}
Die Kurve ist lokal l"angenminimierend, denn ist $\overline c$ eine Kurve von $c(s)$ nach $c(t)$ mit $\calL(\overline c) < \calL(c|_{[s,t]})$, so w"are $c|_{[0,s]} \cup \overline c \cup c|_{[t,1]}$ eine Kurve k"urzer als $c$.
Da $c$ kompaktes Bild hat, exisitert ein minimales $\rho > 0$ f"ur alle $c(t)$ wie in Korollar \ref{kor-8-8}. Dann findet man eine Partition $0 = t_0 < t_1 < \ldots < t_k = 1$ mit $\dop(c(t_{i-1}), c(t_{i})) < \frac{\rho}{2}$, so dass $c|_{[t_{i-1}, t_{i+1}]}$ glatt ist.
\begin{center}\begin{tikzpicture}[font=\scriptsize]
%	\draw[step=0.25,gray!15] (-6,-5) grid (6,5); \draw[step=0.5,gray!30] (-6,-5) grid (6,5); \fill (0,0) circle(0.1); %Hilfsgitter
	
	\draw[dashed] (0,0) circle(1.5);
	\begin{scope}
		\path[name path=kreis,clip] (0,0) circle (1.0);
		\draw[very thick] (-2.5,-1) ..controls(-2.5,-1) and (-0.75,-0.5).. (0,0) ..controls(0.75,0.5) and (1.5,2).. (1.5,2);
	\end{scope}
	
	\draw[name path=kurve] (-2.5,-1) ..controls(-2.5,-1) and (-0.75,-0.5).. (0,0)node[right]{$c(t_i)$} ..controls(0.75,0.5) and (1.5,2).. (1.5,2) node[right,pos=0.7]{$c$};
	
	\path[name intersections={of={kreis and kurve}}];
	\fill (intersection-2) circle(0.05) node[right]{$c(t_{i-1})$}; \fill (intersection-1) circle(0.05) node[right]{$c(t_{i+1})$}; \fill (0,0) circle(0.05);
\end{tikzpicture}\end{center}
Dann stimmt $c|_{[t_{i-1}, t_{i+1}]}$ f"ur jedes $i < k$ mit der nach Korollar \ref{kor-8-8} eindeutigen Geod"atischen (bis auf Umparametrisierung) "uberein.
\end{bew}

\begin{Dfn}
Eine Riemannsche Mannigfaltigkeit hei"st \CmMark[vollst\"andig!geo\"atisch]{geod"atisch vollst"andig}, wenn jede Geod"atische auf ganz $\R$ fortgesetzt werden kann.
\end{Dfn}

\begin{Satz}[Satz von Hopf-Rinow]
F"ur eine Riemannsche Mannigfaltigkeit sind die folgenden Aussagen "aquivalent:
\begin{enumerate}[label=(\roman*)]
\item
	$M$ ist ged"atisch vollst"andig
\item
	F"ur alle $p \in M$ gilt $\calD_p = \T_pM$
\item
	Es existiert ein $p \in M$ mit $\calD_p = \T_pM$
\item
	Abgeschlosse und beschr"ankte Mengen sind kompakt
\item
	$M$ ist vollst"andig (als metrischer Raum)
\end{enumerate}
Jede dieser Eigenschaften impliziert, dass je zwei Punkte in $M$ durch eine minimierende Geod"atische verbunden werden k"onnen.
\end{Satz}

\begin{bew}
Man zeigt zun"achst, dass falls (iii) f"ur $p \in M$ gilt, es zu jedem  $q \in M$ eine minimierende Geod"atische von $p$ nach $q$ gibt. Es gelte $\calD_p = \T_pM$ und es sei $q \in M$.
\begin{center}\begin{tikzpicture}
%	\draw[step=0.25,gray!15] (-6,-1) grid (6,5); \draw[step=0.5,gray!30] (-6,-1) grid (6,5); \fill (0,0) circle(0.1); %Hilfsgitter

	\draw (-2,4.5) -- (-3,2.5) -- (2,2.5);
	\coordinate (0p) at (-1.75,3.25);
	\fill(0p) circle(0.05) node[below left]{$0_p$}; \draw[->] (0p) -- +(1.5,0.75);
	
	\coordinate (p) at ($(0p) - (0,3.25)$); \coordinate (q) at ($(p) + (7,3)$);
	\fill (p) circle(0.05)node[left]{$p$} (q) circle(0.05)node[right]{$p$};
	\draw[dashed,name path=strich] (p) -- (q);
	
	\draw[name path=kreis] (p) circle(1.25);
	\path[name intersections={of={strich and kreis}}];
	\draw[->] (p) -- (intersection-1);
\end{tikzpicture}\end{center}
F"ur $\epsilon > 0$ wie in Satz \ref{satz-8-7} ist um $\partial \B_{\frac{\epsilon}{2}}(p)$ kompakt; es sei $\overline q \in \partial \B_{\frac{\epsilon}{2}}(p)$ ein Punkt minimalen Abstandes zu $q$. Dann gilt $\overline q = \exp_p(\frac{\epsilon}{2}v)$ f"ur ein $v \in \T_pM$ mit $\|v\| = 1$.
\begin{description}[font=\normalfont\itshape]
\item[Behauptung:] Dann ist $\gamma_v|_{[0,R]}: t \mapsto \exp_p(tv)$ minimierende Geod"atische nach $q$ f"ur $R = \dop(p,q)$.
\end{description}
Es sei $\calI = \{t \in [0,R] | \dop(\gamma_v(t),q) = R - t\}$. Dann ist $\calI$ nichtleer und abgeschlossen, denn $t \mapsto \dop(\gamma_v(t),q) + t$ ist stetig.
\begin{center}\begin{tikzpicture}[font=\scriptsize]
%	\draw[step=0.25,gray!15] (-6,-1) grid (6,5); \draw[step=0.5,gray!30] (-6,-1) grid (6,5); \fill (0,0) circle(0.1); %Hilfsgitter
	
	\coordinate (p) at (-3,0); \coordinate (p0) at (3.5,2.5);
	\coordinate(ctrlp) at (0.5,1.5); \coordinate(ctrlp0) at (-2.5,-0.25);
	\coordinate (q) at ($(p0) - 1.25*(ctrlp0)$);
	
	\draw (p) ..controls($(p) + (ctrlp)$) and ($(p0) + (ctrlp0)$)..node[below]{$\gamma_v$} (p0);
	\fill (p) circle(0.05) node[below]{$p$} (p0) circle(0.05) node[below]{$p_0= \gamma_v(t_0)$} (q) circle(0.05) node[right]{$q$};

	\path[name path=strich] (p0) -- (q);
		
	\draw[dashed,name path=kreis] (p0) circle(1.25);
	
	\path[name intersections={of={kreis and strich}}];
	\draw (p0) -- (intersection-1); \draw[dashed] (intersection-1) -- (q);
	\fill (intersection-1) circle(0.05) node[below right]{$q'$};
\end{tikzpicture}\end{center}
F"ur $t_0 \in \calI$ und $0 < \rho < \epsilon_0$ sei $q' \in \partial \B_\rho(\gamma_v(t_0))$ wie in Satz \ref{satz-8-7} angewandt auf $p_0 = \gamma_v(t_0)$. Dann gilt $\dop(p_0, q) = \rho + \dop(q',q)$ und es folgt:
\begin{align*}
	\dop(p,q') &\ge \dop(p,q) - \dop (q',q)\\
	&= \dop(p,q) - \dop(p_0,q) + \rho\\
	&= R - (R - t) + \rho = t_0 + \rho
\end{align*}
Damit ist die Verkettung von $\gamma_v|_{[0,t_v]}$ und der minimalen Geod"atischen von $p_0$ nach $q'$ nach Korollar \ref{kor-8-9} eine Geod"atische. Aus der Eindeutigkeit von kurzen Geod"atischen folgt, dass diese Zusamensetzung mit $\gamma_v$ "ubereinstimmt. Es gilt also $q' = \gamma_v(t_0 + \rho)$ und mit $\dop(\gamma_v(t_0 + p), q) = \dop(p_0,q) - \rho = R - (t_0 + \rho)$ gilt $t_0 + \rho \in \calI$.

Wir k"onnen nun die einzelnen Implikationen zeigen. Dabei gelten (i) $\Rightarrow$ (ii) $\Rightarrow$ (iii) offensichtlich.
\begin{description}[font=\normalfont]
\item[(iii) $\Rightarrow$ (iv):]
	Es gelte $\calD_p = \T_pM$ und es sei $K \subseteq M$ abgeschlossen und beschr"ankt. Dann existiert $R$ mit $K \subseteq \overline\B_R(p)$. Da $\overline\B_R(0_p)$ kompakt ist, ist auch $K$ kompakt.
\item[(iv) $\Rightarrow$ (v):]
	gilt offensichtlich
\item[(v) $\Rightarrow$ (i):]
	Es sei $c$ eine nach Bogenl"ange parametrisierte Geod"atische mit maximalem Definitionsintervall $\calI$. $\calI$ ist nichtleer und offen. Ist $(t_n)$ eine Folge in $\calI$ mit Grenzwert $t$. Dann ist $q_n = c(t_n)$ wegen
		\[ \dop(c(t_n), c(t_m)) \le |t_m - t_n| \]
	eine Cauchy-Folge und konvergiert somit gegen ein $p \in M$.
	\begin{center}\begin{tikzpicture}[font=\scriptsize,scale=0.9]
%		\draw[step=0.25,gray!15] (-6,-3) grid (6,3); \draw[step=0.5,gray!30] (-6,-3) grid (6,3); \fill (0,0) circle(0.1); %Hilfsgitter
		
		\coordinate (ctn) at (0,0); \coordinate (ctrl) at (0,1); 
		\fill (ctn) circle(0.05)node[above right]{$c(t_n)$};
		
		% ich muss an ddieser Stelle schummeln, auskommentiert ist der richtige Code, aber das Kompilieren dauert zu lange mit Fixed Point Arithmetic -Aleks
		\draw[clip] (ctn) circle(2); \draw[dashed] (ctn) circle(1);
%		\draw[fixed point arithmetic,decoration={markings, mark=at position 0.8 with {\arrow{>};}},postaction={decoration={markings, mark=at position 0.2 with {\arrow{>};}},decorate}] (-1,-2.5) .. controls(-1,-2.5) and ($(ctn) - (ctrl)$).. node[right,pos=0.6]{$c$} (ctn) ..controls($(ctn) + (ctrl)$) and (-0.5,2.5).. (-0.5,2.5);
		\draw (-1,-2.5) .. controls(-1,-2.5) and ($(ctn) - (ctrl)$).. node[right,pos=0.6]{$c$}node[sloped]{>} (ctn) ..controls($(ctn) + (ctrl)$) and (-0.5,2.5)..node[sloped]{<} (-0.5,2.5);
	\end{tikzpicture}\end{center}
	Es sei $\rho > 0$ wie in Korollar \ref{kor-8-8}. F"ur hinreichend gro"ses $n$ gilt dann $|t_n - t| < \frac{\rho}{2}$. Die nach Bogenl"ange parametrisierte Geod"atische von $q_n = c(t_n)$ mit Startvektor $\dot c(t_n)$ existiert auf $(-\frac{\rho}{2}, \frac{\rho}{2})$, setzt also $c$ bis zum Zeitpunkt $|t_n| + \frac{\rho}{2} > |t|$ fort.
\end{description}
\end{bew}


%%% Local Variables: 
%%% mode: latex
%%% TeX-master: "../skript-diffgeom"
%%% End: 
