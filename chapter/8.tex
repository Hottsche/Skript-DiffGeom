%% 
%% Vorlesung <2012-12-14 Fri>, Fortsetzung
%%

\chapter{Geod\"atische und die Exponentialabbildung}

\begin{emptythm}[Heuristik:] Geodätische sind Minimalstellen des Energiefunktionals $\gamma \mapsto E(\gamma) = \int \|\dot\gamma\|^2$. 
Was sind kritische Punkte dieser Abbildung? Für $f \in C^{\infty}(M)$ ist $p$ kritischer Punkt, wenn alle Richtungsableitungen verschwinden, das hei"st $0 = X(f) = \difffrac[t=0]{}{t}(f(c(t)))$.
\end{emptythm}

\begin{center}
  \textcolor{red}{Abbildung: \quot{Kurve von Kurven}}
\end{center}

Eine \quot{Kurve} durch $\gamma$ ist eine sogenannte \CmMark[Variation!glatte]{glatte Variation} $h\colon[0,1]\times[0,1] \to M$, $h(s,t) = h_s(t)$ mit $h_0 = \gamma$ und $h_s(0) = p$, sowie $h_s(1) = q$ f"ur alle $s \in [0,1]$. Dann ist
\begin{align*}
  X(t) = \difffrac[s=0]{}{s}h_s(t)
\end{align*}
ein glattes Vektorfeld entlang $\gamma$.
Ferner gilt $X(0) = 0$ und $X(1) = 0$.
Nun betrachte
\begin{align*}
	0  = \difffrac[s=0]{}{s}E(h_s) &= \int_{0}^{1} \difffrac[s=0]{}{s} \left<\difffrac{}{t} h_s(t),\difffrac{}{t}h_s(t)\right>\\
	& = \int_0^1 2 \left<\nabla_s \difffrac{}{t}h_s(t), \difffrac{}{t}h_s(t)\right>\\
	& = \int_0^1 2 \left<\nabla_t\smash{\underbrace{\difffrac{}{s}h_s(t)}_{=X(t)}}, \difffrac{}{t}h_s(t)\right> \vphantom{\underbrace{\difffrac{}{s}h_s(t)}_{=X(t)}}\\
	& = \int_0^1 2 \left<\nabla_tX,\difffrac{}{t}h_s(t)\right>\\
	& = 2 \int_0^1 \difffrac{}{t}\left<X,\difffrac{}{t}h_s(t)\right> - \left<X,\nabla_t\difffrac{}{t}h_s(t)\right>\\
	& = \underbrace{2 \int_0^1 \difffrac{}{t}\left<X,\difffrac{}{t}h_s(t)\right>}_{=0} - 2 \int_0^1 \left<X,\nabla_t\difffrac{}{t}h_s(t)\right>\\
	& = -2 \int_0^1 \left<X(t),\nabla_t\dot\gamma(t)\right>\dop t
\end{align*}

% Definition 8.1
\begin{Dfn}
  Eine glatte Kurve $c$ in $M$ heißt \CmMark{Geod\"atische}\footnote{Die Äquivalenz zur bereits bekannten Definition wird in Kürze gezeigt.}, wenn $\nabla_t\dot c \equiv 0$ gilt.
\end{Dfn}

Ist $c$ Geodätische, so ist $c$ proportional zur Bogenlängenparametrisierung, das hei"st $\|\dot c\| = $const, denn $\difffrac{}{t}\|\dot c(t)\|^{2} = \difffrac{}{t}\left<\dot c(t),\dot c(t) \right> = 2\left<\nabla_t\dot c(t), \dot c(t)\right> = 0$.
Mit $c$ ist auch jede affine Umparametrisierung $t \mapsto c(at + b)$ eine Geodätische.

% Proposition 8.2
\begin{Prop}
Für jedes $p \in M$ und $v \in \T_pM$ existiert genau eine Geodätische $\gamma_{p,v}\colon[0,\epsilon] \to M$ mit $\gamma_{p,v}(0) = p$ und $\dot \gamma_{p,v}(0) = v$.
Zudem hängt $\gamma_{p,v}$ glatt von $p$ und $v$ ab.
\end{Prop}

\begin{bew}\begin{enumerate}[label=(\Alph*),leftmargin=*,widest=B]
\item
	Es sei $(\varphi, U)$ eine Karte um $p$, $\gamma^i(t) = \varphi^i(\gamma(t))$. Dann besitzt das folgende Anfangswertproblem
	\begin{align*}
		\begin{cases}
			0 = \nabla_t\dot \gamma|_t = \sum_k\left(\ddot \gamma^k(t) + \sum_{ij}\Gamma_{ij}^k\big(\gamma(t)\big)\dot \gamma^i(t)\gamma^j(t)\right) \pdifffrac[\gamma(t)]{}{x^k}\\
			\gamma^i(0) = \varphi^i(p)\\
			\dot\gamma^i(0) = \xi_p^i, \quad v = \sum \xi^i_p\pdifffrac[p]{}{x^i}
		\end{cases}
	\end{align*}
	eine eindeutige Lösung (lokal), wleche glatt von den Startwerten $p$ und $v$ abhängt.
\item
	(Alternativ) Ist $(\varphi, U)$ eine Karte von $M$ um $p$, dann ist
	\[ \overline \varphi \colon \left\{\begin{array}{cccl}
		\T M|_U &\to& \R^{2m}&\\
		X_p=\sum \xi_p^i\pdifffrac[p]{}{x^i} &\mapsto& \overline\varphi(X_p) &= (\varphi^1(p), \ldots, \varphi^m(p), \xi_p^1, \ldots, \xi_p^m)\\
		&&& =: (y^1, \ldots, y^{2m})
	\end{array}\right.\]
	eine Karte von $\T M$.	
	Es sei $S$ das durch
	\[ S \colon \left\{ \begin{array}{ccc}
		\T M &\to& \T\T M\\
		X = \sum \xi^i \pdifffrac{}{x^i} &\mapsto& \sum_i^m \xi^i \pdifffrac{}{y^i} - \sum_{i,j,k=1}^{m} \Gamma_{ij}^k \xi^i\xi^j\pdifffrac{}{y^{m+k}}
	\end{array}\right.\]
	defninierte glatte Vektorfeld auf $\T M$.	
	$g^t$ ist genau dann Integralkurve von $S$ durch $X_p = \sum \xi_p^i\pdifffrac[p]{}{x^i}$, wenn
	\begin{align*}
		\difffrac{}{t}g^t = \dot g^t = S(g^t) \text{ und } g^0 = X_p.
	\end{align*}
	Setzt man $\overline \varphi(g^t) = (\gamma^1(t), \ldots, \gamma^m(t),\eta^1(t), \ldots, \eta^m(t))$, so ist dies genau dann der Fall, wenn gilt:
	\begin{align*}
		& (\dot\gamma^1,\ldots, \dot\gamma^m,\dot\eta^1,\ldots, \dot\eta^m) = \left(\eta^1, \ldots, \eta^m, -\sum_{i,j}\Gamma_{ij}^1\eta^i\eta^j, \ldots, -\sum_{i,j}\Gamma_{ij}^m\eta^i\eta^j\right)\\
		& \rightsquigarrow \eta^i = \dot\gamma^i \text{ und } \ddot\gamma = -\sum_{i,j}\Gamma_{ij}^k\dot\gamma^i \dot\gamma^j
	\end{align*}
	und 
	\begin{align*}
		(\gamma^1(0), \ldots, \gamma^m(0), \eta^1(0), \ldots, \eta^m(0) = \overline\varphi(X_p) = (\varphi^1(p), \ldots, \varphi^m(p), \xi_p^1, \ldots, \xi_p^m)
	\end{align*}
	also genau dann, wenn
	\begin{align*}
		\gamma(t) = \overline\varphi^{-1}(\gamma^1(t), \ldots, \gamma^m(t))
	\end{align*}
	eine Geodätische durch $p$ mit $\dot \gamma(0) = X_p$ ist.	
	Der maximale Fluss $g^t$ von $S$ heißt \CmMark[Fluss!geod\"atischer]{geod"atischer Fluss}.
	Mit Satz \ref{satz-4-9} folgt die Aussage der Proposition.
\end{enumerate}\end{bew}

\begin{center}
  \textcolor{red}{Abbildung: geodätischer Fluss}
\end{center}

Für $v \in \T_pM$ sei $\gamma_v(t) = \pi(g^t(v))$ die eindeutige Geodätische mit $\gamma_v(0) = p$ und $\dot \gamma_v(0) = v$.
Ist $\delta \in \R$ und $c(t) = \gamma_v(\delta t)$, so ist $c$ eine Geodätische durch $p$ mit $\dot c(0) = \delta v$, das hei"st $c = \gamma_{\delta v}$, beziehungsweise $\gamma_{\delta v}(t) = \gamma_v(\delta t)$.

Der Definitionsbereich $\mathcal D_S$ des geodätischen Flusses ist eine offene Menge in $\R \X \T_pM$ und somit sind sowohl $\mathcal D = \{v \in \T M \mid (1,v) \in \mathcal D_S\}$, als auch $\mathcal D_p = \mathcal D \cap T_pM$ offen für alle $p \in M$ (in $\T M$, beziehungsweise $\T_pM$). Weiterhin gilt $0_p \in \mathcal D_p$.

% Definition 8.3
\begin{Dfn}
  Die Abbildung $\exp_p\colon\mathcal D_p \to M$, $v \mapsto \gamma_v(1)$ heißt \CmMark{Exponentialabbildung}.
\end{Dfn}

%%% Local Variables: 
%%% mode: latex
%%% TeX-master: "../skript-diffgeom"
%%% End: 
