%% 
%% Vorlesung <2012-12-14 Fri>, Fortsetzung
%%

\chapter{Geod\"atische und die Exponentialabbildung}

\begin{emptythm}[Heuristik:] Geodätische sind Minimalstellen des Energiefunktionals $\gamma \mapsto E(\gamma) = \int \|\dot\gamma\|^2$. 
Was sind kritische Punkte dieser Abbildung? Für $f \in C^{\infty}(M)$ ist $p$ kritischer Punkt, wenn alle Richtungsableitungen verschwinden, das hei"st $0 = X(f) = \difffrac[t=0]{}{t}(f(c(t)))$.
\end{emptythm}

\begin{center}\begin{tikzpicture}[font=\scriptsize]
%	\draw[step=0.25,gray!15] (-6,-4) grid (6,4); \draw[step=0.5,gray!30] (-6,-4) grid (6,4); \fill (0,0) circle(0.1); %Hilfsgitter
	
	\coordinate (p) at (-2,-1); \coordinate (q) at (2,1); \coordinate (wirbel) at (-0.5,-0.5);
	\coordinate (ctrl1) at (1,0);
	\def\left{0.75}
	\def\right{1.75}
	\fill (p) circle(0.05)node[below]{$p$}; \fill (q) circle(0.05)node[right]{$q$};
	
	\draw[name path=kurve] (p) ..controls(p) and ($(wirbel) - \left*(ctrl1)$).. (wirbel)node[below]{$\gamma(t)$} ..controls($(wirbel) + \right*(ctrl1)$) and (q)..node[below]{$\gamma$} (q);
	\fill (wirbel) circle(0.05);
	
	\coordinate (vec) at (-0.5,1);
	\foreach \shift in {0.2,0.4,...,1}{
		\coordinate (neuwirbel) at ($(wirbel) + \shift*(vec)$);
		\draw[name path=obere kurve] (p) ..controls(p) and ($(neuwirbel) - \left*(ctrl1)$).. (neuwirbel) ..controls($(neuwirbel) + \right*(ctrl1)$) and (q).. (q);
	}
	\draw[->] (wirbel) -- ($(wirbel) + 1.3*(vec)$);
	
	\path[name path=vert] (0,-1) -- (0,1);
	\path[name intersections={of={kurve and vert}}];
	\fill (intersection-1) circle(0.05);
	\draw[->] (intersection-1) -- ($(intersection-1) + (0.5,1)$);
	\path[name intersections={of={obere kurve and vert}}];
	\node[above] at (intersection-1) {$h_s$};
\end{tikzpicture}\end{center}

Eine \quot{Kurve} durch $\gamma$ ist eine sogenannte \CmMark[Variation!glatte]{glatte Variation} $h\colon[0,1]\times[0,1] \to M$, $h(s,t) = h_s(t)$ mit $h_0 = \gamma$ und $h_s(0) = p$, sowie $h_s(1) = q$ f"ur alle $s \in [0,1]$. Dann ist
\begin{align*}
  X(t) = \difffrac[s=0]{}{s}h_s(t)
\end{align*}
ein glattes Vektorfeld entlang $\gamma$.
Ferner gilt $X(0) = 0$ und $X(1) = 0$.
Nun betrachte
\begin{align*}
	0  = \difffrac[s=0]{}{s}E(h_s) &= \int_{0}^{1} \difffrac[s=0]{}{s} \left<\difffrac{}{t} h_s(t),\difffrac{}{t}h_s(t)\right>\\
	& = \int_0^1 2 \left<\nabla_s \difffrac{}{t}h_s(t), \difffrac{}{t}h_s(t)\right>\\
	& = \int_0^1 2 \left<\nabla_t\smash{\underbrace{\difffrac{}{s}h_s(t)}_{=X(t)}}, \difffrac{}{t}h_s(t)\right> \vphantom{\underbrace{\difffrac{}{s}h_s(t)}_{=X(t)}}\\
	& = \int_0^1 2 \left<\nabla_tX,\difffrac{}{t}h_s(t)\right>\\
	& = 2 \int_0^1 \difffrac{}{t}\left<X,\difffrac{}{t}h_s(t)\right> - \left<X,\nabla_t\difffrac{}{t}h_s(t)\right>\\
	& = \underbrace{2 \int_0^1 \difffrac{}{t}\left<X,\difffrac{}{t}h_s(t)\right>}_{=0} - 2 \int_0^1 \left<X,\nabla_t\difffrac{}{t}h_s(t)\right>\\
	& = -2 \int_0^1 \left<X(t),\nabla_t\dot\gamma(t)\right>\dop t
\end{align*}

% Definition 8.1
\begin{Dfn}\label{dfn-8-1}
  Eine glatte Kurve $c$ in $M$ heißt \CmMark{Geod\"atische}\footnote{Die Äquivalenz zur bereits bekannten Definition wird in Kürze gezeigt.}, wenn $\nabla_t\dot c \equiv 0$ gilt.
\end{Dfn}

Ist $c$ Geodätische, so ist $c$ proportional zur Bogenlängenparametrisierung, das hei"st $\|\dot c\| = $const, denn $\difffrac{}{t}\|\dot c(t)\|^{2} = \difffrac{}{t}\left<\dot c(t),\dot c(t) \right> = 2\left<\nabla_t\dot c(t), \dot c(t)\right> = 0$.
Mit $c$ ist auch jede affine Umparametrisierung $t \mapsto c(at + b)$ eine Geodätische.

% Proposition 8.2
\begin{Prop}
Für jedes $p \in M$ und $v \in \T_pM$ existiert genau eine Geodätische $\gamma_{p,v}\colon[0,\epsilon] \to M$ mit $\gamma_{p,v}(0) = p$ und $\dot \gamma_{p,v}(0) = v$.
Zudem hängt $\gamma_{p,v}$ glatt von $p$ und $v$ ab.
\end{Prop}

\begin{bew}\begin{enumerate}[label=(\Alph*),leftmargin=*,widest=B]
\item
	Es sei $(\varphi, U)$ eine Karte um $p$, $\gamma^i(t) = \varphi^i(\gamma(t))$. Dann besitzt das folgende Anfangswertproblem
	\begin{align*}
		\begin{cases}
			0 = \nabla_t\dot \gamma|_t = \sum_k\left(\ddot \gamma^k(t) + \sum_{ij}\Gamma_{ij}^k\big(\gamma(t)\big)\dot \gamma^i(t)\gamma^j(t)\right) \pdifffrac[\gamma(t)]{}{x^k}\\
			\gamma^i(0) = \varphi^i(p)\\
			\dot\gamma^i(0) = \xi_p^i, \quad v = \sum \xi^i_p\pdifffrac[p]{}{x^i}
		\end{cases}
	\end{align*}
	eine eindeutige Lösung (lokal), wleche glatt von den Startwerten $p$ und $v$ abhängt.
\item
	(Alternativ) Ist $(\varphi, U)$ eine Karte von $M$ um $p$, dann ist
	\[ \overline \varphi \colon \left\{\begin{array}{cccl}
		\T M|_U &\to& \R^{2m}&\\
		X_p=\sum \xi_p^i\pdifffrac[p]{}{x^i} &\mapsto& \overline\varphi(X_p) &= (\varphi^1(p), \ldots, \varphi^m(p), \xi_p^1, \ldots, \xi_p^m)\\
		&&& =: (y^1, \ldots, y^{2m})
	\end{array}\right.\]
	eine Karte von $\T M$.	
	Es sei $S$ das durch
	\[ S \colon \left\{ \begin{array}{ccc}
		\T M &\to& \T\T M\\
		X = \sum \xi^i \pdifffrac{}{x^i} &\mapsto& \sum_i^m \xi^i \pdifffrac{}{y^i} - \sum_{i,j,k=1}^{m} \Gamma_{ij}^k \xi^i\xi^j\pdifffrac{}{y^{m+k}}
	\end{array}\right.\]
	defninierte glatte Vektorfeld auf $\T M$.	
	$g^t$ ist genau dann Integralkurve von $S$ durch $X_p = \sum \xi_p^i\pdifffrac[p]{}{x^i}$, wenn
	\begin{align*}
		\difffrac{}{t}g^t = \dot g^t = S(g^t) \text{ und } g^0 = X_p.
	\end{align*}
	Setzt man $\overline \varphi(g^t) = (\gamma^1(t), \ldots, \gamma^m(t),\eta^1(t), \ldots, \eta^m(t))$, so ist dies genau dann der Fall, wenn gilt:
	\begin{align*}
		& (\dot\gamma^1,\ldots, \dot\gamma^m,\dot\eta^1,\ldots, \dot\eta^m) = \left(\eta^1, \ldots, \eta^m, -\sum_{i,j}\Gamma_{ij}^1\eta^i\eta^j, \ldots, -\sum_{i,j}\Gamma_{ij}^m\eta^i\eta^j\right)\\
		& \rightsquigarrow \eta^i = \dot\gamma^i \text{ und } \ddot\gamma = -\sum_{i,j}\Gamma_{ij}^k\dot\gamma^i \dot\gamma^j
	\end{align*}
	und 
	\begin{align*}
		(\gamma^1(0), \ldots, \gamma^m(0), \eta^1(0), \ldots, \eta^m(0) = \overline\varphi(X_p) = (\varphi^1(p), \ldots, \varphi^m(p), \xi_p^1, \ldots, \xi_p^m)
	\end{align*}
	also genau dann, wenn
	\begin{align*}
		\gamma(t) = \overline\varphi^{-1}(\gamma^1(t), \ldots, \gamma^m(t))
	\end{align*}
	eine Geodätische durch $p$ mit $\dot \gamma(0) = X_p$ ist.	
	Der maximale Fluss $g^t$ von $S$ heißt \CmMark[Fluss!geod\"atischer]{geod"atischer Fluss}.
	Mit Satz \ref{satz-4-9} folgt die Aussage der Proposition.
\end{enumerate}\end{bew}

\begin{center}\begin{tikzpicture}[font=\scriptsize]
%	\draw[step=0.25,gray!15] (-6,-1) grid (6,5); \draw[step=0.5,gray!30] (-6,-1) grid (6,5); \fill (0,0) circle(0.1); %Hilfsgitter
	
	\def\breite{2.5}
	\def\hoehe{2}
	\def\shift{1}
	\def\vert{2.5}
	\draw (-\breite, \vert) -- (-\breite+\shift, \vert+\hoehe) -- (\breite+\shift, \vert+\hoehe) -- (\breite, \vert) -- cycle;
	\fill (-0.5,3.25) circle(0.05) node[left]{$0_p$};
	\draw[->] (-0.5,3.25) --node[above]{$v$} (0.75,3.75);
	\node at (3.5,3.5) {$\T_pM$};
	
	\coordinate (segel) at (-2.25,-0.5); \node at ($(segel) + (4.5,1.25)$) {$M$};
	\tikzsegel[1.5]{(segel)}
	\coordinate (pkt) at ($0.75*(-0.5,0.5)+(segel3)$);
	\coordinate (ctrl1) at (1,1); \coordinate (ctrl2) at (-1,1); \coordinate (ctrl3) at (0,1); \coordinate (ctrl4) at (-1,0);
	\draw[dashed] (segel1) ..controls($(segel1) + 0.5*(ctrl1)$) and ($(pkt) + 0.25*(ctrl2)$).. (pkt) ..controls($(pkt) + 0.5*(ctrl3)$) and ($(segel2) + 0.75*(ctrl4)$).. (segel2);
	
	\coordinate(pkt1) at ($(segel) + (1.25,0.75)$); \coordinate(pkt2) at ($(segel) + (2.5,1)$); \coordinate(pkt3) at ($(segel) + (3.75,1.75)$);
	\fill(pkt1) circle(0.05)node[anchor=north east,font=\tiny]{$p$}; \fill(pkt2) circle(0.05) node[below,font=\tiny]{$\gamma_v(1)= \exp_p(v)$};
	\coordinate (ctrl1) at (1.5,1); \coordinate (ctrl2) at (-1,-0.25);
	\draw[->](pkt1) --node[above,sloped,font=\tiny]{$\dot\gamma_v(0)=v$} ($(pkt1) + 0.75*(ctrl1)$);
	\draw (pkt1) ..controls($(pkt1) + 0.25*(ctrl1)$) and ($(pkt2) + 0.5*(ctrl2)$).. (pkt2) ..controls($(pkt2) - 0.5*(ctrl2)$) and (pkt3).. (pkt3);
\end{tikzpicture}\end{center}

Für $v \in \T_pM$ sei $\gamma_v(t) = \pi(g^t(v))$ die eindeutige Geodätische mit $\gamma_v(0) = p$ und $\dot \gamma_v(0) = v$.
Ist $\delta \in \R$ und $c(t) = \gamma_v(\delta t)$, so ist $c$ eine Geodätische durch $p$ mit $\dot c(0) = \delta v$, das hei"st $c = \gamma_{\delta v}$, beziehungsweise $\gamma_{\delta v}(t) = \gamma_v(\delta t)$.

Der Definitionsbereich $\mathcal D_S$ des geodätischen Flusses ist eine offene Menge in $\R \X \T_pM$ und somit sind sowohl $\mathcal D = \{v \in \T M \mid (1,v) \in \mathcal D_S\}$, als auch $\mathcal D_p = \mathcal D \cap T_pM$ offen für alle $p \in M$ (in $\T M$, beziehungsweise $\T_pM$). Weiterhin gilt $0_p \in \mathcal D_p$.

% Definition 8.3
\begin{Dfn}
  Die Abbildung $\exp_p\colon\mathcal D_p \to M$, $v \mapsto \gamma_v(1)$ heißt \CmMark{Exponentialabbildung}.
\end{Dfn}


%% 
%% Vorlesung <2012-12-18 Tue>
%%

Es wurde bereits gezeigt, dass $\nabla_t \dot \gamma_v \equiv 0$ (Geodätische Differentialgelichung).

Die Exponentialabbildung is nach Satz \textcolor{red}{4.7 (?!)} glatt.
Es gilt $\exp_p(0_p) = p$.
Zur Berechnung des Differentiales von $\exp_p$ in $0_p$
\begin{align*}
  \exp_{p*0_p} \colon \T_{0_p}\T_pM \to \T_pM
\end{align*}
identifiziert man $\T_{0_p}\T_pM$ mit $\T_pM$.
Es gilt
\begin{align*}
  \exp_{p*0_p}(v) = \difffrac[t=0]{}{t}\exp_p(tv) = \difffrac[t=0]{}{t}\gamma_{tv}(1) = \difffrac[t=0]{}{t}\gamma_v(t) = \dot \gamma_v(0) = v.
\end{align*}

Es existiert für alle $p \in M$ eine Umgebung $V$ von $0_p \in \T_pM$ und $U$ von $p$, so dass $\exp_p \colon V \to U$ ein Diffeomorphismus ist.

Wählt man eine Orthonormalbasis $e_1, \ldots, e_m$ von $\T_pM$ und setzt
\begin{align*}
  \psi \colon \T_pM \to \R^m, v = \sum_i b^ie_i \mapsto (b^1, \ldots, b^m),
\end{align*}
so ist $(\psi \circ \exp_p|_U^{-1}, U)$ eine Karte von $M$ um $p$.

Im Allgemeinen ist dies keine Isometrie!

% Definition 8.4
\begin{Dfn}
  Diese Karte bezeichnet man als \CmMark{Riemannsche Normalkoordinaten}.
\end{Dfn}

% Proposition 8.5
\begin{Prop}
  In Riemannschen Normalkoordinatn gilt für alle $i,j,k \leq m$:
  \begin{enumerate}[label=(\roman*)]
  \item $g_{ij}(0) = \delta_{ij}$
  \item $\Gamma^k_{ij}(0) = 0$
  \item $\partial_k g_{ij}(0) = \pdifffrac[0]{g_{ij}}{x^k} = 0$
  \end{enumerate}
\end{Prop}

Beweis zur Übung.

\section{Polarkoordinaten}

Es ist $\varphi = (r, \vartheta^1, \ldots, \vartheta^{m-1})$ die Hintereinanderausführung von Riemannschen Normalkoordinaten des $\R^m$.

\begin{center}
  \textcolor{red}{Abbildung: Zurückziehen von Polarkoordinaten via Riemannscher Normalkoordinaten.}
\end{center}

Die Umkehrabbildung ist ein Diffeomorphismus
\begin{align*}
  f \colon (0, \varepsilon) \times S^{m-1} \to U \subset M, \ 
  (t,v) \mapsto \exp_p(tv) = \gamma_v(t).
\end{align*}
Für jedes $v \in S^{m-1}$ ist $t \mapsto f(t,v) = \gamma_v(t)$ eine Geodätische in $M$.

% Lemma 8.6
\begin{Lemma}[Gauß-Lemma]
  Jede radiale Geodätische $\gamma_v$ ist orthogonal zu der geodätischen Sphäre
  \begin{align*}
    S_r = \{q \in M \mid \ \exists v \in M: \|v\| = r \text{ und } q = \exp_p(v) \}.
  \end{align*}
\end{Lemma}

\begin{bew}
  Man zeigt das Folgende:
  Ist $X$ ein Vektorfeld auf $S^{m-1}$ und bezeichnet man seine Fortsetzung auf $(0,\varepsilon) \times S^{m-1}$ \quot{$\subset$} $\B_{\varepsilon}(0)\setminus\{0\}$ bzw. $\B_{\varepsilon}(0_p)\setminus\{0_p\} \subset \T_pM$ mit $X_{rv} = X_v$, so ist
  \begin{align*}
    Y_q = Y_{f(r,v)} = f_{*(r,v)}(0,X_v) = \exp_{p*}(r X_v)
  \end{align*}
  orthogonal zu 
  \begin{align*}
    \pdifffrac[q]{}{r} = \difffrac[t=r]{}{t}\exp_p(t,v) = \dot \gamma_v(r)
  \end{align*}

  \begin{center}
    \textcolor{red}{Abbildung: Erhalten der Orthogonalität durch $\exp_p$.}
  \end{center}

  $Y(t) = Y_{\gamma_v(t)}$ als Vektorfeld entlang $\gamma_v$.
  Dann gilt:
  \begin{align*}
    \difffrac[t=r]{}{t} \left<Y,\pdifffrac{}{r}\right>_{\gamma_v(t)} & = \left<\nabla_tY|_r, \dot \gamma_v(r)\right> + \left<Y(r), \underbrace{\nabla_t\dot\gamma_v|_r}_{=0}\right>\\
    & = \left<\nabla_{Y(r)}\dot\gamma_v(r), \dot \gamma_v(r)\right> + \left<\underbrace{[\dot\gamma_v(r),Y(r)]}_{\mathclap{= [f_{*}(\pdifffrac{}{r},f_{*}(0,X_v)] = f_{*}[\pdifffrac{}{r},X] = 0}}, \dot \gamma_v(r)\right>\\
    = \frac{1}{2}Y(t) \|\dot\gamma_v\|^2.
  \end{align*}
  Ferner gilt
  \begin{align*}
    \left<Y(r),\pdifffrac{}{r}\right>_{\gamma_v(r)} = \left<\exp_{p*}(rX_v),\dot\gamma_v(r)\right> \xrightarrow{r \to 0}\left<\exp_{p*}(0_p),v\right> = 0,
  \end{align*}
  also $\left<Y,\pdifffrac{}{r}\right> \equiv 0$.
\end{bew}

\begin{bem}
  Insbesondere gilt für alle $i \leq m-1$:
  \begin{align*}
    \left<\pdifffrac{}{r}, \pdifffrac{}{\vartheta^i}\right> = 0.
  \end{align*}
\end{bem}

% Satz 8.7
\begin{Satz}
  Für jedes $p \in M$ existiert ein $\varepsilon > 0$, so dass für alle $q \in \B_{\varepsilon}(p)$ genau eine minimierende Geodätische von $p$ nach $q$ existiert, d.h. eine Geodätische $\gamma$ im Sinne der Definition \ref{dfn-8-1} mit $\mathcal L(\gamma) = \dop(p,q)$.

  Ist $q \notin \exp_p(\B_{\varepsilon}(0_p)) = \B_{\varepsilon}(p)$, so existiert ein $q' \in \partial \B_{\varepsilon}(p)$ mit
  \begin{align*}
    \dop(p,q) = \varepsilon + \dop(q',q).
  \end{align*}
\end{Satz}

\begin{bew}
  Es sei $\varepsilon > 0$ so, dass auf $\B_{\varepsilon}(p)$ Polorkoordinaten $\varphi = (r,\vartheta^1, \ldots, \vartheta^{m-1})$ existieren.

  Es sei $c \colon [0,1]$ eine beliebige glatte Kurve von $p$ nach $q$ mit Koordinaten $\varphi(c(t)) = (r(t), \vartheta^1(t), \ldots, \vartheta^{m-1}(t))$.

  \begin{center}
    \textcolor{red}{Abbildung: Bild von $c$.}
  \end{center}

  Für $t_0 = \inf\{t \in [0,1] \mid c(t) \notin \B_{\varepsilon}(p)\}$ ist $c|_{[0,t_0]}$ eine Kurve zu $\B_{\varepsilon}(p)$.
  Es gilt
  \begin{align*}
    \left\|\pdifffrac[t]{}{r}\right\| = \|\dot\gamma_w(t)\| = \|w\| = 1.
  \end{align*}

  Aus der Cauchy-Schwarz-Ungleichung folgt
  \begin{align*}
    \|\dot c(t)\| & = \|\dot c(t)\|\left\|\pdifffrac[c(t)]{}{r}\right\|\\
    & \geq \left|\left< \dot c(t), \pdifffrac[c(t)]{}{r}\right>\right|\\
    & = \left|\left<\dot r(t) \pdifffrac{}{r} + \sum_{i=1}^{m-1}\dot\vartheta(t)\pdifffrac{}{\vartheta^i},\pdifffrac[c(t)]{}{r}\right>\right|\\
    & = \left|\left<\dot r(t)\pdifffrac[c(t)]{}{r}, \pdifffrac[c(t)]{}{r}\right>\right|\\
    & = \left|\dot r(t)\right|,
  \end{align*}
  wobei Gleichheit genau dann gilt, wenn $\dot c(t)$ und $\pdifffrac[c(t)]{}{r}$ linear abhängig sind.

  \begin{align*}
    \mathcal L(c) = \int_0^{t_0}\|\dot c\| + \int_{t_0}^T\|\dot c\| \geq \int_0^{t_0} \left|\left<\dot c,\pdifffrac{}{r}\right>\right| = \int_0^{t_0}|\dot r| = r(t_0)
  \end{align*}
  Gleichheit gilt genau dann, wenn $\vartheta^1(t), \ldots, \vartheta^{m-1}(t)$ konstant sind und $\dot r(t) \geq 0$ gilt, also genau dann, wenn $c$ eine monotone Umparametrisierung von $t \mapsto \exp_p(tv)$ ist.
\end{bew}


%%% Local Variables: 
%%% mode: latex
%%% TeX-master: "../skript-diffgeom"
%%% End: 
