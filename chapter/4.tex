%% 
%% 5. Vorlesung <2012-10-30 Tue>, Fortsetzung
%% 
%% Skript Differentialgeometrie im Wintersemester 12/13
%% Zur Vorlesung von Dr. Grensing am KIT Karlsruhe
%% 
%% Mitschrieb und Textsatz von Jan-Bernhard Kordaß.
%% 

\section{Tangentialbündel und Vektorfelder}

\begin{dfn}[Tangentialbündel]
  Es se $M$ eine glatte Mannigfaltigkeit. Die Menge $\T M = \dot \bigcup_{p \in M}\T_pM$ zusammen mit der sogenannten kanonischen Projektion $\pi \colon \T M \to M, \T_pM \ni X_p \mapsto p$ heißt das \CmMark{Tangentialbündel} von $M$.
\end{dfn}

\subsection{Das Tangentialbündel als glatte Mannigfaltigkeit}

Es sei $(\varphi, U)$ eine Karte von $M$. Setzt man $\T M|_U  \pi^{-1}(U) = \dot \bigcup_{p\in U}\T_pM$, so ist nach \textcolor{red}{Satz 2.8} % REFERENZ!
die Abbildung
\begin{align*}
  & \overline \varphi \colon \T M|_U \to \varphi(U) \times \R^m \subset \R^{2m}\\
&\T_p M \ni X_p = \sum \xi^i\pdifffrac[p]{}{x^i} \mapsto (\varphi(p),\xi)
\end{align*}
bijektiv.
Es sei eine Topologie auf $\T M$ dadurch erklärt, dass eine Menge $V \subset \T M$ genau dann offen ist, wenn für alle Karten $(\varphi, U)$ die Menge $\overline \varphi(V \cap \T M|_U$ offen in $\R^{2m}$ ist.\\
Diese Topologie ist hausdorffsch und besitzt eine abzählbare Basis, da dies für $M$ und $\R^m$ gilt.
Nach Konstruktion sind alle $\overline \varphi$ Homöomorphismen.\\
Ist $\mathcal A = \{(\varphi, U\}$ ein $C^{\infty}$-Atlas von $M$, so definiert
\begin{align*}
  \overline A = \{(\overline \varphi, \T M|_U) \mid (\varphi, U) \in \mathcal A\}
\end{align*}
eine glatte Struktur auf $\T M$.\\
Für Karten $(\varphi, U), (\psi, V)$ von $M$ ist der Kartenwechsel
\begin{align*}
  \overline \psi \circ \overline \phi^{-1} \colon \varphi(U \cap V) \times \R^m \to \psi(U \cap V) \times \R^m, \ (x, \xi) \mapsto (\psi \circ \varphi^{-1}(x),\D(\psi \circ \varphi^{-1}|_x\xi),
\end{align*}
$\left(X_p = \sum \xi \pdifffrac[p]{}{x^1}\right)$ ist glatt.\\
Damit trägt $\T M$ in kanonischer Weise eine glatte Struktur.
Darüber hinaus ist die kanonische Projektion $\pi \colon \T M \to M$ bezüglich dieser glatten Struktur eine Submersion. (Beweis: Übungsaufgabe)\\

Ist $N$ eine weitere glatte Mannigfaltigkeit und $\Phi \colon M \to N$ glatt, so ist $\Phi_{*} \colon \T M \to \T N, \ X_p \mapsto \Phi_{*p}X_p$ eine glatte Abbildung (ebensfalls Übungsaufgabe).\\


\begin{dfn}
  Eine stetige Abbildung $X \colon M \to \T M$ mit $\psi \circ X = \Id_M$ heißt \CmMark{Vektorfeld} auf $M$.
Ist $X$ glatt (als Abbildung zwischen glatten Mannigfaltigkeiten), so heißt $X$ ein \CmMark{glattes Vektorfeld.}
\end{dfn}

\begin{bem}
  Ist $(\varphi, U)$ eine Karte von $M$, so sind die Abbildungen $U \colon \T M|_U, \ p \mapsto \pdifffrac[p]{}{x^{i}}$ glatte Vektorfelder (in der Karte $\overline \varphi$ sind diese genau die Abbildungen $(x,e_i)$).\\
Ist $X$ ein glattes Vektorfeld, so gilt für jedes $u \in U$:
\begin{align*}
  X_U = \sum \xi^i(u) \pdifffrac[U]{}{x^i},
\end{align*}
wobei $\xi(u) = (\xi^1(u), \ldots, \xi^m(u))$ eine glatte Abbildung $U \to \R^m$ ist.\\

Ein Vektorfeld ist genau dann glatte, wenn für eine Karte $(\varphi, U)$ die Koeffizientenfunktionen $\xi^i(u)$ von $X_U = \sum \xi^i(u) \pdifffrac[U]{}{x^i}$ glatte Funktionen sind.
\end{bem}

%%% Local Variables: 
%%% mode: latex
%%% TeX-master: "../skript-diffgeom"
%%% End: 
