%% 
%% 5. Vorlesung <2012-10-30 Tue>, Fortsetzung
%% 
%% Skript Differentialgeometrie im Wintersemester 12/13
%% Zur Vorlesung von Dr. Grensing am KIT Karlsruhe
%% 
%% Mitschrieb und Textsatz von Jan-Bernhard Kordaß.
%% 

\section{Tangentialbündel und Vektorfelder}

\begin{dfn}[Tangentialbündel]
  Es se $M$ eine glatte Mannigfaltigkeit. Die Menge $\T M = \dot \bigcup_{p \in M}\T_pM$ zusammen mit der sogenannten kanonischen Projektion $\pi \colon \T M \to M, \T_pM \ni X_p \mapsto p$ heißt das \CmMark{Tangentialbündel} von $M$.
\end{dfn}

\subsection{Das Tangentialbündel als glatte Mannigfaltigkeit}

Es sei $(\varphi, U)$ eine Karte von $M$. Setzt man $\T M|_U  \pi^{-1}(U) = \dot \bigcup_{p\in U}\T_pM$, so ist nach \textcolor{red}{Satz 2.8} % REFERENZ!
die Abbildung
\begin{align*}
  & \overline \varphi \colon \T M|_U \to \varphi(U) \times \R^m \subset \R^{2m}\\
  &\T_p M \ni X_p = \sum \xi^i\pdifffrac[p]{}{x^i} \mapsto (\varphi(p),\xi)
\end{align*}
bijektiv.
Es sei eine Topologie auf $\T M$ dadurch erklärt, dass eine Menge $V \subset \T M$ genau dann offen ist, wenn für alle Karten $(\varphi, U)$ die Menge $\overline \varphi(V \cap \T M|_U$ offen in $\R^{2m}$ ist.\\
Diese Topologie ist hausdorffsch und besitzt eine abzählbare Basis, da dies für $M$ und $\R^m$ gilt.
Nach Konstruktion sind alle $\overline \varphi$ Homöomorphismen.\\
Ist $\mathcal A = \{(\varphi, U\}$ ein $C^{\infty}$-Atlas von $M$, so definiert
\begin{align*}
  \overline A = \{(\overline \varphi, \T M|_U) \mid (\varphi, U) \in \mathcal A\}
\end{align*}
eine glatte Struktur auf $\T M$.\\
Für Karten $(\varphi, U), (\psi, V)$ von $M$ ist der Kartenwechsel
\begin{align*}
  \overline \psi \circ \overline \phi^{-1} \colon \varphi(U \cap V) \times \R^m \to \psi(U \cap V) \times \R^m, \ (x, \xi) \mapsto (\psi \circ \varphi^{-1}(x),\D(\psi \circ \varphi^{-1}|_x\xi),
\end{align*}
$\left(X_p = \sum \xi \pdifffrac[p]{}{x^1}\right)$ ist glatt.\\
Damit trägt $\T M$ in kanonischer Weise eine glatte Struktur.
Darüber hinaus ist die kanonische Projektion $\pi \colon \T M \to M$ bezüglich dieser glatten Struktur eine Submersion. (Beweis: Übungsaufgabe)\\

Ist $N$ eine weitere glatte Mannigfaltigkeit und $\Phi \colon M \to N$ glatt, so ist $\Phi_{*} \colon \T M \to \T N, \ X_p \mapsto \Phi_{*p}X_p$ eine glatte Abbildung (ebensfalls Übungsaufgabe).\\


\begin{dfn}
  Eine stetige Abbildung $X \colon M \to \T M$ mit $\psi \circ X = \Id_M$ heißt \CmMark{Vektorfeld} auf $M$.
  Ist $X$ glatt (als Abbildung zwischen glatten Mannigfaltigkeiten), so heißt $X$ ein \CmMark{glattes Vektorfeld.}
\end{dfn}

\begin{bem}
  Ist $(\varphi, U)$ eine Karte von $M$, so sind die Abbildungen $U \colon \T M|_U, \ p \mapsto \pdifffrac[p]{}{x^{i}}$ glatte Vektorfelder (in der Karte $\overline \varphi$ sind diese genau die Abbildungen $(x,e_i)$).\\
  Ist $X$ ein glattes Vektorfeld, so gilt für jedes $u \in U$:
  \begin{align*}
    X_U = \sum \xi^i(u) \pdifffrac[U]{}{x^i},
  \end{align*}
  wobei $\xi(u) = (\xi^1(u), \ldots, \xi^m(u))$ eine glatte Abbildung $U \to \R^m$ ist.\\

  Ein Vektorfeld ist genau dann glatte, wenn für eine Karte $(\varphi, U)$ die Koeffizientenfunktionen $\xi^i(u)$ von $X_U = \sum \xi^i(u) \pdifffrac[U]{}{x^i}$ glatte Funktionen sind.
\end{bem}


%%% 
%%% 6. Vorlesung <2012-11-2 Fri>
%%% 

\begin{bsp*}
  Betrachte die $n$-Sphäre $S^n \subset \R^{n+1}$ und deren Tangentialraum $\T_pS^n = p^{\perp}$.
  Ein glattes Vektorfeld auf $S^n$ ist also eine glatte Abbildung $X \colon S^n \to \R^{nü1}$ mit $X_p \perp p$.\\
  Es sei $n=2k-1$, dann ist
  \begin{align*}
    X \colon S^n \to \R^{n+1}, (x^1,y^1, \ldots, x^k,y^k) \mapsto (-y^1,x^1, \ldots, -y^k,y^k)
  \end{align*}
  ein glattes Vektorfeld auf $S^n$ ohne eine Nullstelle.\\
\end{bsp*}

\begin{bem*}
  Der \CmMark{Satz vom Igel} besagt gerade: Jedes glatte Vektorfeld auf einer Sphäre gerader Dimension hat eine Nullstelle.
\end{bem*}

\begin{bem*}
  Es bezeichne $\mathcal V(M)$ die Menge aller glatten Vektorfelder auf der Mannigfaltigkeit $M$. Der sogenannte \CmMark{Nullschnitt}:
  \begin{align*}
    - \colon M \to \T M, p \mapsto 0_p \in \T_pM
  \end{align*}
  ist ein glattes Vektorfeld auf $M$.\\

  Übungsaufgabe: Zeige: Der Nullschnitt ist eine Einbettung.
\end{bem*}

\begin{bem*}
  Sind $X,Y \in \mathcal V(M)$ und ist $g \in \C^{\infty}$, so sind die punktweise Summe $X+Y$ und das Produkt $gX$ wieder ein glattes Vektorfeld auf $M$.
  Damit ist $\mathcal V(M)$ ein $\R$-Vektorraum bzw. $C^{\infty}(M)$-Modul.\\

  Jedes Vektorfeld $X$ ist eine Derivation von $C^{\infty}$:
  \begin{align*}
    X(fg)(p) = X(f)(p)g(p) + f(p) X(g)(p) = \left(gX(f) + fX(g)\right)(p).
  \end{align*}
  Es seien $X,Y \in \mathcal V(M)$ glatte Vektorfelder. Die \CmMark{Lieklammer} $[X,Y]$ von $X$ und $Y$ ist dann durch den folgenden Ausdruck definiert:
  \begin{align*}
    [X,Y](f)(p) = X_p(Yf)-Y_p(Xf).
  \end{align*}
\end{bem*}

% Lemma 4.3
\begin{lemma}
  Die Lieklammer ist eine schiefsymmetrische $\R$-bilineare Abbildung $\mathcal V \times \mathcal V(M) \to \mathcal V(M)$.\\
  Es gitl die sogenannte \CmMark{Jacobiidentität}:
  \begin{align*}
    [X,[Y,z]] + [Y,[Z,X]] + [Z,[X,Y]] = 0.
  \end{align*}
\end{lemma}

\begin{proof}
  Es seien $X,Y \in \mathcal V(M)$, $f,g \in \C^{\infty}$ und $p \in M$. Dann gilt:
  \begin{align*}
    [X,Y](fg)  = & X_p(Y(f)g + fY(g)) - Y_p(X(f)g+fX(g))\\
    = & X_p(Y(f))g(p) + Y_p(f)X_p(g) + X_p(f)Y(g)(p) + f(p)X_p(Y(g))\\
    & - Y_p(X(f))g(p) - X_p(f)Y_p(g) - Y_p(f)X(g)(p)-f(p)Y_p(X(g))\\
    = & (X_p(Y(f))-Y_p(X(f))g(p) + f(p)(X_p(Y(g))-Y_p(X(g))\\
    = & [X,Y]_p (f)g(p) + f(p)[X,Y]_p(g).
  \end{align*}
  Damit gilt $[X,Y] \in \mathcal V(M)$. Schiefsymmetrie und $\R$-Linearität gelten offensichtlich. Die Jacobiidentität sie als Übungsaufgabe überlassen (Rechnen!).
\end{proof}


% Lemma 4.4
\begin{lemma}
  Es seien $X,Y \in \mathcal V(M)$ glatte Vektorfelder und $(\varphi,U)$ eine Karte von $M$.
  Sind dann $X|_U = \sum \xi^i\pdifffrac{}{x^i}, \ Y|_U = \sum \eta^ipdifffrac{}{x^i}$ und $[X,Y]|_U = \sum \zeta^i \pdifffrac{}{x^i}$ die ensprechenden lokalen Darstellungen, so gilt:
  \begin{align*}
    \zeta^j = \sum (\xi^i\pdifffrac{\eta^j}{x^i} - \eta^i \pdifffrac{\xi^j}{x^i}.
  \end{align*}
\end{lemma} 

Der Beweis als Übungs überlassen.


\subsection{Flüsse}

Was haben Vektorfelder mit Differentialgleichungen zu tun?\\

%%% 
%%% Abbildung 6-1
%%% 
\textcolor{red}{Abbildung 6-1}\\

Jedes glatte Vektorfeld $X$ definiert ein Anfangswertproblem
\begin{align*}
  \begin{cases}
    \dot \gamma(t) = X_{\gamma(t)}\\
    \gamma(0) = p
  \end{cases},
\end{align*}
oder in lokalen Koordinaten:
\begin{align*}
\begin{cases}
  \dot \gamma(t) = \xi(\tilde \gamma(t))\\
  \tilde \gamma(0) = 0, \text{ falls } \varphi(p) = 0,
\end{cases}
\end{align*}
mit $\tilde \gamma = \varphi \circ \gamma$ für eine Karte $(\varphi,U)$ und $X|_U = \sum \xi^i \pdifffrac{}{x^i}$.

% Definition 4.5
\begin{dfn}
  Es sei $X \in \mathcal V(M)$ ein glattes Vektorfeld und $p \in M$, sowie $J \subset \R$ ein offenes, zusammenhängendes Intervall um $0$. Eine glatte Kurve $\gamma \colon J \to M$ mit
  \begin{align*}
    \dot \gamma(t) = X_{\gamma(t)}, \ \gamma(0) = p
  \end{align*}
heißt \CmMark{Integralkurve} oder \CmMark{Trajektorie} von $X$ durch $p$.
\end{dfn}

\begin{bem*}
  Eine Kurve $\gamma$ ist genau dann Integralkurve von $X$ durch $p$, wenn für jede Karte $(\varphi,U)$ die Kurve $\tilde \gamma = \varphi \circ \gamma$ eine Lösung des (autonomen) Anfangswertproblems
  \begin{align*}
    \dot{\tilde \gamma} = \xi(\gamma(t)),  \ \tilde \gamma(0) = \varphi(p)
  \end{align*}
  ist, wobei $X_U = \sum \xi^i \pdifffrac{}{x^i}$ gelte.\\

  Für jedes $p \in M$ ist somit (lokal) ein Anfangswertproblem gestellt. Gesucht ist eine "`simultane"' Lösung all dieser Anfangswertprobleme, also eine Abbildung $(t,p) \mapsto \gamma(t,p) = \gamma^i(p)$ mit 
  \begin{align*}
    \begin{cases}
      \dot \gamma^t(p) = X_{\gamma^t(p)}\\
      \gamma^0(p) = p
    \end{cases}.
  \end{align*}
\end{bem*}

% Satz 4.6
\begin{satz}[Lokale Existenz und Eindeutigkeit]
  Es sei $U \subset \R^n$ offen und $J_{\varepsilon}=(-\varepsilon,\varepsilon)$ und $F \colon J_{\varepsilon} \times \R^n \to \R^n$ $C^k$-differenzierbar.
  Dann existiert für alle $x \in U$ eine Umgebung $V$ von $x$ in $U$ und ein $\delta > 0$, so dass
  \begin{enumerate}
  \item Für alle $x \in V$ existiert eine $C^{k+1}$-Lösung $y_x\colon I_{\varepsilon} \to V$, von $y_x'(t)=F(t,y(t))$ und $y_x(0) = x$.
  \item Diese Lösung ist lokal eindeutig, d.h. falls $\tilde y_x$ eine weitere Lösung auf $I_{\tilde\delta}$ ist, so gilt
    \begin{align*}
      y_x(t) = \tilde y_x(t) \text{ für alle } |t| \leq \min\{\delta, \tilde\delta\}.
    \end{align*}
    :\item Die Abbildung 
    \begin{align*}
      y\colon J_{\delta} \times V \to U, \ (t,x) \mapsto y_x(t)
    \end{align*}
    ist $C^k$-differenzierbar.
  \end{enumerate}
\end{satz}

\begin{proof}
  Siehe Lang: "`Differential and Riemanian Manifolds"', 3. Auflage, 1995, Chapter IV.1, P.65 
\end{proof}

% Korollar 4.7
\begin{kor}
  Es sei $X \in \mathcal V(M)$ und $p \in M$. Dann existiert eine offene Umgebung $U$ von $p$ und ein $\varepsilon > 0$ und eine glatte Abbildung:
  \begin{align*}
    \gamma\colon(-\varepsilon,\varepsilon) \times U \to M
  \end{align*}
so dass $t \mapsto \gamma^t(p)$ eine Integralkurve von $X$ durch $p$ ist.\\
(Setze dann $F(t,x) = \xi(\varphi^{-1}(x))$.)
\end{kor}

% Korollar 4.8
\begin{kor}
Sind $\gamma_1 \colon J_1 \to M, \ \gamma_2 \colon J_2 \to M$ Integralkurven eines Vektorfeldes $X \in \mathcal V(M)$ durch $p$, dann gilt $0 \in J_1 \cap J_2$ und $\gamma_1(0)= p = \gamma_2(0)$.
Nach \textcolor{red}{Satz 4.6 (ii)} % TODO: Referenz setzen
gilt dann $\gamma_1(t) = \gamma_2(t)$ für alle $t \in J_1 \cap J_2$. Damit ist
\begin{align*}
  \gamma \colon J_1 \cap J_2, \ t \mapsto 
  \begin{cases}
    \gamma_1(t) & t \in J_1\\
\gamma_2(t) & t \in J_2
  \end{cases}
\end{align*}
eine Integralkurve von $X$ durch $p$.
Damit existiert für jedes $p \in M$ ein maximaler Definitionsbereich $I_p$ für Integralkurven von $X$ durch $p$; dieser ist offen.
\end{kor}

\begin{dfn*}
  Für $X \in \mathcal V(M)$ heißt die, wie im vorigen Korollar definierte, Familie maximaler Integralkurven
  \begin{align*}
    \gamma(t,p) = \gamma^t(p),\ t \in I_p
  \end{align*}
der \CmMark{Fluss} des Vektorfeldes $X$.
Seinen Definitionsbereich notiert man mit:
\begin{align*}
  \mathcal D_X = \{(t,p) \in \R \times M \mid t \in I_p\}.
\end{align*}
\end{dfn*}

\begin{bem*}
  Es gilt: $\gamma^0(\cdot) = \Id_M$.\\
Ist $(s,p) \in \mathcal D_{X}$ gilt $\difffrac[t=0]{}{t}(t \mapsto \gamma^{t+s}(p)) = X_{\gamma^s(p)}$, also ist $t \mapsto \gamma^{t+s}(p)$ eine Integralkurve von $X$ durch $q = \gamma^s(p)$.\\
Aus der Eindeutigkeit folgt damit:
\begin{align*}
  \gamma^{t+s}(p) = \gamma^t(\gamma^s(p)) \text{ für alle } s,t s+t \in I_p.
\end{align*}
Ferner gilt $I_q = I_p - s$.\\

Kurz: $\gamma(t+s) = \gamma^t \circ \gamma^s$.\\
Also definiert $\gamma$ einen "`lokalen Gruppenhomomorphismus"' von $\R$ in $M^M$.
\end{bem*}


% Satz 4.9
\begin{satz}
  Ist $X \in \mathcal V(M)$ ein glattes Vektorfeld, so ist $\mathcal D_X$ eine offene Menge und sein Fluss glatt.
\end{satz}

%%% Local Variables: 
%%% mode: latex
%%% TeX-master: "../skript-diffgeom"
%%% End: 
