%% 
%% 14. Vorlesung <2012-11-30 Fri>, Fortsetzung
%% 

\chapter{Kovariante Ableitungen}

\paragraph{Frage:} Was ist eine \quot{gute} Differentialrechnung für Vektorfelder?

Das gewöhnliche Differential im $\R^n$ für $Y \colon \R^n \to \R^n$ ist gerade die lineare Abbildung $\D Y|_p \cdot v = \lim \frac{1}t \left(Y(p+tr) -Y(p)\right) = \difffrac[t=0]{}{t} Y(p+tr)$

Betrachte im euklidischen Fall einen Punkt $p$, sowie einen Tangentialvektor $Y_p$.

\textcolor{red}{Abbildung 14.2: Motivation Differential im Euklidischen, evtl. auch noch die Idee der Parallelverschiebung erklären.}

Nun gehe zur Betrachtung von Vektorfeldern $X \colon \R^n \to \R^n$ über und setze $\D_XY|_p = \D Y|_p\cdot X_p$. Hierfür gilt:
\begin{itemize}
\item $\D$ ist $\R$-linear in $Y$: $\D(Y + \tilde Y) = \D Y + \D \tilde Y$.
\item Es gilt die Leibnizregel: $\D fY = \D f \cdot Y + f\D Y$.
\item $\D$ ist $C^{\infty}(\R^n)$-linear in $X$:
  \begin{align*}
    \D_{fX}Y|_p = \D Y|_p\cdot(fX)_p = \D Y|_p \cdot fp)X_p = f(p) \D Y|_p \cdot X_p = (f \D_XY)(p).
  \end{align*}
\end{itemize}

Erinnerung: Die Lieableitung $\mathcal L_{(\cdot)}Y$ ist nicht $C^{\infty}$-linear.

% Definition 7.1
\begin{Dfn}
  Es seien $M$ eine glatte Mannigfaltigkeit und $E$ ein Vektorbündel über $M$.
  Ein \CmMark{kovariante Ableitung} (oder \CmMark{Zusammenhang} ([engl.] \quot{connection}) auf $E$ ist eine Abbildung
  \begin{align*}
    \nabla \colon \mathcal V(M) \times \Gamma(E) \to \Gamma(E), \quad \nabla(X,S) = \nabla_XS
  \end{align*}
  mit den folgenden Eigenschaften:
  \begin{enumerate}[label=(\roman*)]
  \item $\nabla S$ ist $C^{\infty}(M)$-linear, d.h.
    \begin{align*}
      \nabla_{X+Y}S = \nabla_XS+\nabla_YS \text{ und } \nabla_{fX}S = f\nabla_XS \; \forall X,Y \in \mathcal V(M), f \in C^{\infty}(M).
    \end{align*}
  \item $\nabla_X$ ist $\R$-linear, d.h.
    \begin{align*}
      \nabla_X(\mu S + \nu T) = \mu\nabla_XS + \nu\nabla_XT.
    \end{align*}
  \item $\nabla_X$ erfüllt die folgende Leibnizregel:
    \begin{align*}
      \nabla_X(fS) = \dop f(X) \cdot S + f\cdot \nabla_XS = X(f)\cdot S + f \cdot \nabla_XS.
    \end{align*}
  \end{enumerate}
  Kurzform: $\nabla \colon \Gamma(E) = \Gamma(\T^{*}M \oplus E), S \mapsto \nabla_{(.)}S$ ist eine $C^{\infty}(M)$-Modulderivation.
\end{Dfn}

\begin{bsp}
  \begin{enumerate}[label=(\arabic*)]

  \item Das gewöhnlich Differential $\D$ definiert in kanonischer Weise eine kovariante Ableitung auf $\T\R^n$.
    \begin{align*}
      X \in \mathcal V(\R^n), \X \colon \R^n \to \T\R^n \cong \R^n \times \R^n \text{ via } \calI\colon X_p \mapsto (p,\underbrace{\calI_p(X_p)}_{=:\overline X_p}).
    \end{align*}
    Nun ist wie folgt eine kovariante Ableitung gegeben: $(\nabla_XY)_p = \calI^{-1}(p,\D_{\overline X_p}\overline Y)$.

  \item $E = M \times \R^n$, ein Schnitt $S$ von $E$ ist von der Form $S_p = (p,s(p)), s \colon M \to \R^n$ glatt.

    Hier definiert man die kovariante Ableitung:
    \begin{align*}
      & \nabla_XS = (p,\calI_{s(p)}(s_{*p},X_p)\\
      &  s_{*p}\colon \T_{*p}M \to \T_{*p}\R^n, s_{*p}\colon X_p \in \T_{*p}\R^n \xrightarrow{\calI_{s(p)}} \R^n.
    \end{align*}

  \item $E = M \times \R^n, \omega = (\omega_j^i)$ eine $(n\times n)$-Matrix
    von $1$-Formen auf $M$, d.h. $\omega(x)_p \in \mathfrak M^{n\times n}(\R)$.
    Für einen Schnitt $S = (\Id, s)$ und $\mathcal V(M)$ ist 
    \begin{align*}
      \nabla_XS = (p,\calI_{s(p)}(s_{*p}X_p) + \sum \omega(X)_p^i\cdot \xi^i).
    \end{align*}

  \item $\dop \colon \Omega^0(M) = C^{\infty}(M) = \Gamma(M\times \R) \to \Omega^1(M) = \Gamma(\T^{*}M) = \Gamma(\underbrace{\T^{*}M \oplus (M \times \R)}_{\mathclap{\text{Fasern: } \T_p^{*}M\oplus\R \cong \T_p^{*}M}})$ mit $f \mapsto [\dop f \colon X \mapsto \dop f(X) = X(f)]$.

Dann ist
\begin{align*}
& \dop \colon \mathcal V(M) \times C^{\infty}(M) \to \C^{\infty}(M),\\
& \nabla_Xf = \dop (X,f) \mapsto X(f)
\end{align*}
eine kovariante Ableitung.

\item Es sei $M \subseteq \R^k$ eine glatte Untermannigfaltigkeit und $\nabla$ die kanonische kovariante Ableitung auf $\T \R^k$. 

Erster Ansatz für eine kovariante Ableitung:
  \begin{align*}
    \tilde \nabla_XY = \nabla_{\tilde X}\tilde Y|_M \text{ das funktioniert noch nicht.}
  \end{align*}

Für $X,Y \in \mathcal V(M)$ seien $\tilde X, \tilde Y$ Fortsetzungen, d.h. $\tilde X|_M = X$ und $\tilde Y|_M = Y$.

\begin{align*}
  (\nabla_{\tilde X}\tilde Y)_p \in \T_p\R^k \supseteq \T_pM.
\end{align*}

Nächster Ansatz, der tasächlich eine kovariante Ableitung definiert.
\begin{align*}
  \tilde \nabla_XY = (\nabla_{\tilde X}\tilde Y|_M)^{proj\T_pM},
\end{align*}
wobei $X^{proj.\T_pM}$ die orthogonale Projektion von X auf den Tangentialraum $\T_pM$ bzgl. des Standardskalarproduktes ist.

  \end{enumerate}
\end{bsp}

%%% Local Variables: 
%%% mode: latex
%%% TeX-master: "../skript-diffgeom"
%%% End: 
