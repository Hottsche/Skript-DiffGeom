%% 
%% 14. Vorlesung <2012-11-30 Fri>, Fortsetzung
%% 

\chapter{Kovariante Ableitungen}

\paragraph{Frage:} Was ist eine \quot{gute} Differentialrechnung für Vektorfelder?

Das gewöhnliche Differential im $\R^n$ für $Y \colon \R^n \to \R^n$ ist gerade die lineare Abbildung $\D Y|_p \cdot v = \lim \frac{1}t \left(Y(p+tv) -Y(p)\right) = \difffrac[t=0]{}{t} Y(p+tr)$.
Betrachte im euklidischen Fall einen Punkt $p$, sowie einen Tangentialvektor $Y_p$.
\begin{center}\begin{tikzpicture}[font=\scriptsize]
	\coordinate (end) at (1.5,0.75); %Endrichtung
	\draw (0,0) -- ($(0,0) + 4*(end)$);
	% die Punkte
	\coordinate (p) at ($(0,0) + 0.5*(end)$); \coordinate (q) at ($(0,0) + 2.75*(end)$);
	\fill (p) circle(0.1)node[anchor=north west]{$p$}; \fill (q) circle(0.1)node[anchor=north west]{$p + tv$};
	% die Pfeile
	\draw[->] (p) --node[left]{$Y_p$} ($(p) + 1.25*(-0.75,2)$);
	\coordinate (dir) at (0.15,1);
	\def\scl{1.5}
	\draw[->] (q) --node[right]{$Y_{p+tv}$} ($(q) + \scl*(dir)$); \draw[->] (p) -- ($(p) + \scl*(dir)$);
	% Parallelverschiebung
	\draw[->,dashed] ($(q) + 0.5*\scl*(dir)$) --node[above,sloped]{Parallelverschiebung} ($(p) + 0.5*\scl*(dir)$);
\end{tikzpicture}\\
\textcolor{red}{evtl. auch noch die Idee der Parallelverschiebung erklären.}\end{center}

Nun gehe zur Betrachtung von Vektorfeldern $X \colon \R^n \to \R^n$ über und setze $\D_XY|_p = \D Y|_p\cdot X_p$. Hierfür gilt:
\begin{itemize}
\item $\D$ ist $\R$-linear in $Y$: $\D(Y + \tilde Y) = \D Y + \D \tilde Y$.
\item Es gilt die Leibnizregel: $\D(fY) = \D f \cdot Y + f\D Y$.
\item $\D$ ist $C^{\infty}(\R^n)$-linear in $X$:
  \begin{align*}
    \D_{fX}Y|_p = \D Y|_p\cdot(fX)_p = \D Y|_p \cdot f(p)X_p = f(p) \D Y|_p \cdot X_p = (f \D_XY)(p).
  \end{align*}
\end{itemize}

\emph{Erinnerung:} Die Lieableitung $\mathcal L_{(\cdot)}Y$ ist \emph{nicht} $C^{\infty}$-linear.

% Definition 7.1
\begin{Dfn}
  Es seien $M$ eine glatte Mannigfaltigkeit und $E$ ein Vektorbündel über $M$.
  Eine \CmMark{kovariante Ableitung} (oder \CmMark{Zusammenhang} ([engl.] \quot{connection}) auf $E$ ist eine Abbildung
  \begin{align*}
    \nabla \colon \mathcal V(M) \times \Gamma(E) \to \Gamma(E), \quad \nabla(X,S) = \nabla_XS
  \end{align*}
  mit den folgenden Eigenschaften:
  \begin{enumerate}[label=(\roman*),widest=iii]
  \item $\nabla S$ ist $C^{\infty}(M)$-linear, das hei"st
    \begin{align*}
      \nabla_{X+Y}S = \nabla_XS+\nabla_YS \text{ und } \nabla_{fX}S = f\nabla_XS
    \end{align*}
    f"ur alle $X, Y \in \calV(M)$ und $f \in C^{\infty}(M)$.
  \item $\nabla_X$ ist $\R$-linear, das hei"st
    \begin{align*}
      \nabla_X(\mu S + \nu T) = \mu\nabla_XS + \nu\nabla_XT.
    \end{align*}
  \item $\nabla_X$ erfüllt die folgende Leibnizregel:
    \begin{align*}
      \nabla_X(fS) = \dop f(X) \cdot S + f\cdot \nabla_XS = X(f)\cdot S + f \cdot \nabla_XS.
    \end{align*}
  \end{enumerate}
  Kurzform: $\nabla \colon \Gamma(E) \to \Gamma(\T^{*}M \otimes E), S \mapsto \nabla_{(.)}S$ ist eine $C^{\infty}(M)$-Modulderivation.
\end{Dfn}

\begin{bsp}\begin{enumerate}[label=\arabic*),leftmargin=*]
\item
	Das gewöhnliche Differential $\D$ definiert in kanonischer Weise eine kovariante Ableitung auf $\T\R^n$.
	\begin{align*}
		X \in \mathcal V(\R^n), X \colon \R^n \to \T\R^n \cong \R^n \times \R^n \text{ via } \calI\colon X_p \mapsto (p,\underbrace{\calI_p(X_p)}_{=:\overline X_p}).
	\end{align*}
	Nun ist wie folgt eine kovariante Ableitung gegeben: $(\nabla_XY)_p = \calI^{-1}(p,\D_{\overline X_p}\overline Y)$.
\item
	$E = M \times \R^n$, ein Schnitt $S$ von $E$ ist von der Form $S_p = (p,s(p))$, $s \colon M \to \R^n$ glatt.

	Hier definiert man die kovariante Ableitung:
	\begin{align*}
		& \nabla_XS = (p,\calI_{s(p)}^{-1}(s_{*p},X_p))\\
		&  s_{*p}\colon \T_{*p}M \to \T_{*p}\R^n, s_{*p}\colon X_p \in \T_{*p}\R^n \xrightarrow{\calI_{s(p)}} \R^n.
	\end{align*}
\item
	Sei $E = M \times \R^n$, ein Schnitt $S = (\Id, \sigma)$, $\sigma: M \to \R^n$. Dann ist $(\nabla_X S)_p = (p, \calI_p(\sigma_{*p}(X_p))$, $\sigma_{*p}: \T_pM \to \T_{\sigma(p)}\R^n$. Sei $ \omega = (\omega_j^k)_{j,k \le n}$ eine $(n\times n)$-Matrix von 1-Formen auf $M$, das hei"st $\omega(X)|_p \in \mathfrak M^{n\times n}(\R)$.
	Für einen Schnitt $S = (\Id, \sigma)$ und sei dann
		\[ (\nabla_XS)_p = (\Id, \calI_p(\sigma_{*p}(X_p)) + \omega(X)|_p \cdot \sigma(p). \]
	Dies definiert eine kovariante Ableitung auf $E = M \X \R^n$.
\item
	$\dop \colon \Omega^0(M) = C^{\infty}(M) = \Gamma(M\times \R) \to \Omega^1(M) = \Gamma(\T^{*}M) = \Gamma(\underbrace{\T^{*}M \otimes (M \times \R)}_{\mathclap{\text{Fasern: } \T_p^{*}M\otimes\R \cong \T_p^{*}M}})$ mit $f \mapsto [\dop f \colon X \mapsto \dop f(X) = X(f)]$.

	Dann ist
	\begin{align*}
		& \dop \colon \mathcal V(M) \times C^{\infty}(M) \to C^{\infty}(M),\\
		& \nabla_Xf = \dop (X,f) \mapsto X(f)
	\end{align*}
	eine kovariante Ableitung auf $C^{\infty}(M)$.
\item
	Es sei $M \subseteq \R^k$ eine glatte Untermannigfaltigkeit und $\nabla$ die kanonische kovariante Ableitung auf $\T \R^k$.

	Erster Ansatz für eine kovariante Ableitung:
	\begin{align*}
		\tilde \nabla_XY = \nabla_{\tilde X}\tilde Y|_M \text{ das funktioniert noch nicht.}
	\end{align*}
	Für $X,Y \in \mathcal V(M)$ seien $\tilde X, \tilde Y$ Fortsetzungen, das hei"st $\tilde X|_M = X$ und $\tilde Y|_M = Y$.
	\begin{align*}
		(\nabla_{\tilde X}\tilde Y)_p \in \T_p\R^k \supseteq \T_pM.
	\end{align*}

	Nächster Ansatz, der tasächlich eine kovariante Ableitung definiert.
	\begin{align*}
		\tilde \nabla_XY = (\nabla_{\tilde X}\tilde Y|_M)^{\text{proj}\T_pM},
	\end{align*}
	wobei $X^{\text{proj}\T_pM}$ die orthogonale Projektion von X auf den Tangentialraum $\T_pM$ bzgl. des Standardskalarproduktes ist.
\end{enumerate}\end{bsp}

Schreibt man in Beispiel 3) $\sigma = ( \sigma^1, \ldots ,\sigma^n)$, so kann man $\dop \sigma = (\dop \sigma^1,\ldots ,\dop\sigma^n)$ als 1-Form auf $M$ mit Werten in $\R^n$ auffassen:
	\begin{align*}
		\dop \sigma(X)_p &= (\dop \sigma^1(X)_p,\ldots , \dop \sigma^n(X)_p)\\
		&= (X(\sigma^1)_p,\ldots ,X(\sigma^n)_p)\\
		&= \calI_p(\sum X(\sigma^{i}) \partial_i),
	\end{align*}
	wobei $\partial_i$ das $i$-te Koordinatenfeld in der Karte $(\Id, \R^n)$ ist. Identifiziert man $E = M \X \R^n$ mit $C^{\infty}(M, \R^n)$, so gilt $\nabla_X S = \dop \sigma(X) \omega(X) \sigma$ (Kurzschreibweise f"ur die zweite Komponente von $S$). Lokal ist \emph{jede} kovariante Ableitung von dieser Form.

\begin{Lemma}
Die kovariante Ableitung $(\nabla_XS)_p$ h"angt nur von den Werten von $X$ und $S$ in einer Umgebung von $p$ ab.
\end{Lemma}

\begin{bew}
Es seien $p \in M$ und $X_1, X_2 \in \calV(M)$ sowie $S_1, S_2 \in \Gamma(E)$ und $U$ eine Umgebung von $p$ mit $X_1|_U = X_2|_U$ und $S_1|_U = S_2|_U$. W"ahle nun ein $\sigma \in C^{\infty}(M)$ mit dem Tr"ager $\supp \sigma \subseteq U$ und $\sigma|_V \equiv 1$ auf einer Umgebung $V$ von $p$. Dann gilt: $\sigma X_1 = \sigma X_2$ und $\sigma S_1 = \sigma S_2$. F"ur $q \in V$ folgt dann:
\begin{align*}
	(\nabla_{\sigma X_i} \sigma S_i)_q &= \sigma(q)(\nabla_{X_i} \sigma S_i)|_q\\
	&= \sigma(q)(\underbrace{X_i(\sigma)|_q}_{=0} S_i + \underbrace{\sigma(q)}_{=1}\nabla_{X_i} S_i|_q)\\
	&= \nabla_{X_i} S_i|_q
\end{align*}
Damit folgt $\nabla_{X_1} S_1 = \nabla_{X_2} S_2$
\end{bew}

\section{Lokale Koordinaten}

Es sei $(\varphi, U)$ eine Karte von $M$ um $p \in M$ und $E|_U \overset{\tau}{\to} U \X \R^n$. Dann ist $s_i(p) = \tau^{-1}(p, e_i)$ eine lokale Basis. Jeder Schnitt $S$ ist also lokal von der Form $S|_U = \sum_i \sigma^{i} s_i$. Somit existieren glatte Funktionen $\Gamma_{ij}^k$, die sogenannten \CmMark{Christoffelsymbole} mit 
	\[ \nabla_{\pdifffrac{}{x^{i}}} s^j = \sum_k \Gamma_{ij}^k s^k. \]
F"ur $S = \sum \sigma^j s_j$ und $X = \sum \xi^{i} \pdifffrac{}{x^{i}}$ folgt dann:
\begin{align*}
	(\nabla_XS)_p &= \sum_{i,j} \xi_p^{i} \nabla_{\pdifffrac{}{x^{i}}} \left(\sigma^j s_j\right)\\
	&= \sum_{i,j} \xi_p^{i} \left(\pdifffrac{\sigma^j}{x^{i}} \cdot s_j(p) \nabla_{\pdifffrac{}{x^{i}}} s_j|_p\right)\\
	&= \sum_{i,j} \xi_p^{i} \left(\pdifffrac[p]{\sigma^j}{x^{i}} s_j(p) + \sigma^j(p) \sum_k \Gamma_{ij}^k(p) s_k(p)\right)\\
	&= \sum_k \Bigg(\underbrace{\sum_i \xi_p^{i} \pdifffrac[p]{\sigma^k}{x^{i}}}_{\mathclap{= X(\sigma^k)|_p = \difffrac[t=0]{}{t} (\sigma^k \circ \gamma) \text{ mit } \dot\gamma(0) = X_p}} + \sum_{i,j} \xi_p^{i} \sigma^j(p) \Gamma_{ij}^k(p)\Bigg) s_k(p)
\end{align*}

\begin{bem}\begin{enumerate}[label=\arabic*),leftmargin=*]
\item
	$X \mapsto (\nabla_XS)_p$ h"angt nur von dem Wert $X_p$ von  $X$ in $p$ ab, Schreibweise $(\nabla_XS)_p = \nabla_{X_p}S$.
\item
	$S \mapsto \nabla_{X_p}S$ h"angt nur von den Werten von $S$ entlang einer Kurve $\gamma$ mit $\dot\gamma(0) = X_p$ ab. Es gilt
		\[ \nabla_XS = \sum_k X(\sigma^k)S_k + \sum_k \sum_j\left(\left(\sum_i \Gamma_{ij}^k \xi^{i}\right) \sigma^j\right) s_k. \]
	Schreibt man $\sigma = (\sigma^1,\ldots ,\sigma^n)$ und fasst $\dop \sigma = (\dop \sigma^1,\ldots ,\dop \sigma^n)$ also lokale 1-Form mit Werten in $\R^n$ auf, so ist f"ur $s=(s_1,\ldots ,s_n)$ $\dop\sigma \cdot s = \sum \dop \sigma^j s^j$ eine lokale 1-Form mit Werten in $E$. Es gilt: $\dop \sigma \cdot s(X) = \D_X \sigma \cdot s$. Analog definiert $\omega(X) = (\omega_j^k(X))_{k,j}$ eine lokale 1-Form mit Werten in den reellen $(n\X n)$-Matrizen. Dann ist 
	\[ \omega \sigma : X \mapsto \omega(X) \sigma = \left( \sum_{i,j} \Gamma_{ij}^k \xi^{i} \sigma^j \right)^k \]
eine lokale 1-Form mit Werten in $\R^n$ und $\omega\sigma \cdot s$ eine lokale 1-Form mit Werten in $E$. Damit gilt
	\[ \nabla_XS = (\dop \sigma(X) + \omega(X) \sigma) \cdot s \]
oder kurz
	\[ \nabla = \dop + \omega. \]
\end{enumerate}\end{bem}

\section{Transformationsverhalten}

asdf

\begin{bem}
asdf
\end{bem}

\begin{Prop}
asdf
\end{Prop}

\begin{Prop}
asdf
\end{Prop}


%%% Local Variables: 
%%% mode: latex
%%% TeX-master: "../skript-diffgeom"
%%% End: 
