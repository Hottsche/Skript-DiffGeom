%% 
%% 14. Vorlesung <2012-11-30 Fri>, Fortsetzung
%% 

\chapter{Kovariante Ableitungen}

\paragraph{Frage:} Was ist eine \quot{gute} Differentialrechnung für Vektorfelder?

Das gewöhnliche Differential im $\R^n$ für $Y \colon \R^n \to \R^n$ ist gerade die lineare Abbildung $\D Y|_p \cdot v = \lim \frac{1}t \left(Y(p+tv) -Y(p)\right) = \difffrac[t=0]{}{t} Y(p+tr)$.
Betrachte im euklidischen Fall einen Punkt $p$, sowie einen Tangentialvektor $Y_p$.
\begin{center}\begin{tikzpicture}[font=\scriptsize]
	\coordinate (end) at (1.5,0.75); %Endrichtung
	\draw (0,0) -- ($(0,0) + 4*(end)$);
	% die Punkte
	\coordinate (p) at ($(0,0) + 0.5*(end)$); \coordinate (q) at ($(0,0) + 2.75*(end)$);
	\fill (p) circle(0.1)node[anchor=north west]{$p$}; \fill (q) circle(0.1)node[anchor=north west]{$p + tv$};
	% die Pfeile
	\draw[->] (p) --node[left]{$Y_p$} ($(p) + 1.25*(-0.75,2)$);
	\coordinate (dir) at (0.15,1);
	\def\scl{1.5}
	\draw[->] (q) --node[right]{$Y_{p+tv}$} ($(q) + \scl*(dir)$); \draw[->] (p) -- ($(p) + \scl*(dir)$);
	% Parallelverschiebung
	\draw[->,dashed] ($(q) + 0.5*\scl*(dir)$) --node[above,sloped]{Parallelverschiebung} ($(p) + 0.5*\scl*(dir)$);
\end{tikzpicture}\\
\textcolor{red}{evtl. auch noch die Idee der Parallelverschiebung erklären.}\end{center}

Nun gehe zur Betrachtung von Vektorfeldern $X \colon \R^n \to \R^n$ über und setze $\D_XY|_p = \D Y|_p\cdot X_p$. Hierfür gilt:
\begin{itemize}
\item $\D$ ist $\R$-linear in $Y$: $\D(Y + \tilde Y) = \D Y + \D \tilde Y$.
\item Es gilt die Leibnizregel: $\D(fY) = \D f \cdot Y + f\D Y$.
\item $\D$ ist $C^{\infty}(\R^n)$-linear in $X$:
  \begin{align*}
    \D_{fX}Y|_p = \D Y|_p\cdot(fX)_p = \D Y|_p \cdot f(p)X_p = f(p) \D Y|_p \cdot X_p = (f \D_XY)(p).
  \end{align*}
\end{itemize}

\emph{Erinnerung:} Die Lieableitung $\mathcal L_{(\cdot)}Y$ ist \emph{nicht} $C^{\infty}$-linear.

% Definition 7.1
\begin{Dfn}
  Es seien $M$ eine glatte Mannigfaltigkeit und $E$ ein Vektorbündel über $M$.
  Eine \CmMark{kovariante Ableitung} (oder \CmMark{Zusammenhang} ([engl.] \quot{connection}) auf $E$ ist eine Abbildung
  \begin{align*}
    \nabla \colon \mathcal V(M) \times \Gamma(E) \to \Gamma(E), \quad \nabla(X,S) = \nabla_XS
  \end{align*}
  mit den folgenden Eigenschaften:
  \begin{enumerate}[label=(\roman*),widest=iii]
  \item $\nabla S$ ist $C^{\infty}(M)$-linear, das hei"st
    \begin{align*}
      \nabla_{X+Y}S = \nabla_XS+\nabla_YS \text{ und } \nabla_{fX}S = f\nabla_XS
    \end{align*}
    f"ur alle $X, Y \in \calV(M)$ und $f \in C^{\infty}(M)$.
  \item $\nabla_X$ ist $\R$-linear, das hei"st
    \begin{align*}
      \nabla_X(\mu S + \nu T) = \mu\nabla_XS + \nu\nabla_XT.
    \end{align*}
  \item $\nabla_X$ erfüllt die folgende Leibnizregel:
    \begin{align*}
      \nabla_X(fS) = \dop f(X) \cdot S + f\cdot \nabla_XS = X(f)\cdot S + f \cdot \nabla_XS.
    \end{align*}
  \end{enumerate}
  Kurzform: $\nabla \colon \Gamma(E) \to \Gamma(\T^{*}M \otimes E), S \mapsto \nabla_{(.)}S$ ist eine $C^{\infty}(M)$-Modulderivation.
\end{Dfn}

\begin{bsp}\begin{enumerate}[label=\arabic*),leftmargin=*]
\item
	Das gewöhnliche Differential $\D$ definiert in kanonischer Weise eine kovariante Ableitung auf $\T\R^n$.
	\begin{align*}
		X \in \mathcal V(\R^n), X \colon \R^n \to \T\R^n \cong \R^n \times \R^n \text{ via } \calI\colon X_p \mapsto (p,\underbrace{\calI_p(X_p)}_{=:\overline X_p}).
	\end{align*}
	Nun ist wie folgt eine kovariante Ableitung gegeben: $(\nabla_XY)_p = \calI^{-1}(p,\D_{\overline X_p}\overline Y)$.
\item
	$E = M \times \R^n$, ein Schnitt $S$ von $E$ ist von der Form $S_p = (p,s(p))$, $s \colon M \to \R^n$ glatt.

	Hier definiert man die kovariante Ableitung:
	\begin{align*}
		& \nabla_XS = (p,\calI_{s(p)}^{-1}(s_{*p},X_p))\\
		&  s_{*p}\colon \T_{*p}M \to \T_{*p}\R^n, s_{*p}\colon X_p \in \T_{*p}\R^n \xrightarrow{\calI_{s(p)}} \R^n.
	\end{align*}
\item
	Sei $E = M \times \R^n$, ein Schnitt $S = (\Id, \sigma)$, $\sigma: M \to \R^n$. Dann ist $(\nabla_X S)_p = (p, \calI_p(\sigma_{*p}(X_p))$, $\sigma_{*p}: \T_pM \to \T_{\sigma(p)}\R^n$. Sei $ \omega = (\omega_j^k)_{j,k \le n}$ eine $(n\times n)$-Matrix von 1-Formen auf $M$, das hei"st $\omega(X)|_p \in \mathfrak M^{n\times n}(\R)$.
	Für einen Schnitt $S = (\Id, \sigma)$ und sei dann
		\[ (\nabla_XS)_p = (\Id, \calI_p(\sigma_{*p}(X_p)) + \omega(X)|_p \cdot \sigma(p). \]
	Dies definiert eine kovariante Ableitung auf $E = M \X \R^n$.
\item
	$\dop \colon \Omega^0(M) = C^{\infty}(M) = \Gamma(M\times \R) \to \Omega^1(M) = \Gamma(\T^{*}M) = \Gamma(\underbrace{\T^{*}M \otimes (M \times \R)}_{\mathclap{\text{Fasern: } \T_p^{*}M\otimes\R \cong \T_p^{*}M}})$ mit $f \mapsto [\dop f \colon X \mapsto \dop f(X) = X(f)]$.

	Dann ist
	\begin{align*}
		& \dop \colon \mathcal V(M) \times C^{\infty}(M) \to C^{\infty}(M),\\
		& \nabla_Xf = \dop (X,f) \mapsto X(f)
	\end{align*}
	eine kovariante Ableitung auf $C^{\infty}(M)$.
\item
	Es sei $M \subseteq \R^k$ eine glatte Untermannigfaltigkeit und $\nabla$ die kanonische kovariante Ableitung auf $\T \R^k$.

	Erster Ansatz für eine kovariante Ableitung:
	\begin{align*}
		\tilde \nabla_XY = \nabla_{\tilde X}\tilde Y|_M \text{ das funktioniert noch nicht.}
	\end{align*}
	Für $X,Y \in \mathcal V(M)$ seien $\tilde X, \tilde Y$ Fortsetzungen, das hei"st $\tilde X|_M = X$ und $\tilde Y|_M = Y$.
	\begin{align*}
		(\nabla_{\tilde X}\tilde Y)_p \in \T_p\R^k \supseteq \T_pM.
	\end{align*}

	Nächster Ansatz, der tasächlich eine kovariante Ableitung definiert.
	\begin{align*}
		\tilde \nabla_XY = (\nabla_{\tilde X}\tilde Y|_M)^{\text{proj}\T_pM},
	\end{align*}
	wobei $X^{\text{proj}\T_pM}$ die orthogonale Projektion von X auf den Tangentialraum $\T_pM$ bzgl. des Standardskalarproduktes ist.
\end{enumerate}\end{bsp}

Schreibt man in Beispiel 3) $\sigma = ( \sigma^1, \ldots ,\sigma^n)$, so kann man $\dop \sigma = (\dop \sigma^1,\ldots ,\dop\sigma^n)$ als 1-Form auf $M$ mit Werten in $\R^n$ auffassen:
	\begin{align*}
		\dop \sigma(X)_p &= (\dop \sigma^1(X)_p,\ldots , \dop \sigma^n(X)_p)\\
		&= (X(\sigma^1)_p,\ldots ,X(\sigma^n)_p)\\
		&= \calI_p(\sum X(\sigma^{i}) \partial_i),
	\end{align*}
	wobei $\partial_i$ das $i$-te Koordinatenfeld in der Karte $(\Id, \R^n)$ ist. Identifiziert man $E = M \X \R^n$ mit $C^{\infty}(M, \R^n)$, so gilt $\nabla_X S = \dop \sigma(X) \omega(X) \sigma$ (Kurzschreibweise f"ur die zweite Komponente von $S$). Lokal ist \emph{jede} kovariante Ableitung von dieser Form.

\begin{Lemma}
Die kovariante Ableitung $(\nabla_XS)_p$ h"angt nur von den Werten von $X$ und $S$ in einer Umgebung von $p$ ab.
\end{Lemma}

\begin{bew}
Es seien $p \in M$ und $X_1, X_2 \in \calV(M)$ sowie $S_1, S_2 \in \Gamma(E)$ und $U$ eine Umgebung von $p$ mit $X_1|_U = X_2|_U$ und $S_1|_U = S_2|_U$. W"ahle nun ein $\sigma \in C^{\infty}(M)$ mit dem Tr"ager $\supp \sigma \subseteq U$ und $\sigma|_V \equiv 1$ auf einer Umgebung $V$ von $p$. Dann gilt: $\sigma X_1 = \sigma X_2$ und $\sigma S_1 = \sigma S_2$. F"ur $q \in V$ folgt dann:
\begin{align*}
	(\nabla_{\sigma X_i} \sigma S_i)_q &= \sigma(q)(\nabla_{X_i} \sigma S_i)|_q\\
	&= \sigma(q)(\underbrace{X_i(\sigma)|_q}_{=0} S_i + \underbrace{\sigma(q)}_{=1}\nabla_{X_i} S_i|_q)\\
	&= \nabla_{X_i} S_i|_q
\end{align*}
Damit folgt $\nabla_{X_1} S_1 = \nabla_{X_2} S_2$
\end{bew}

\section{Lokale Koordinaten}

Es sei $(\varphi, U)$ eine Karte von $M$ um $p \in M$ und $E|_U \overset{\tau}{\to} U \X \R^n$. Dann ist $s_i(p) = \tau^{-1}(p, e_i)$ eine lokale Basis. Jeder Schnitt $S$ ist also lokal von der Form $S|_U = \sum_i \sigma^{i} s_i$. Somit existieren glatte Funktionen $\Gamma_{ij}^k$, die sogenannten \CmMark{Christoffelsymbole} mit 
	\[ \nabla_{\pdifffrac{}{x^{i}}} s^j = \sum_k \Gamma_{ij}^k s^k. \]
F"ur $S = \sum \sigma^j s_j$ und $X = \sum \xi^{i} \pdifffrac{}{x^{i}}$ folgt dann:
\begin{align*}
	(\nabla_XS)_p &= \sum_{i,j} \xi_p^{i} \nabla_{\pdifffrac{}{x^{i}}} \left(\sigma^j s_j\right)\\
	&= \sum_{i,j} \xi_p^{i} \left(\pdifffrac{\sigma^j}{x^{i}} \cdot s_j(p) \nabla_{\pdifffrac{}{x^{i}}} s_j|_p\right)\\
	&= \sum_{i,j} \xi_p^{i} \left(\pdifffrac[p]{\sigma^j}{x^{i}} s_j(p) + \sigma^j(p) \sum_k \Gamma_{ij}^k(p) s_k(p)\right)\\
	&= \sum_k \Bigg(\underbrace{\sum_i \xi_p^{i} \pdifffrac[p]{\sigma^k}{x^{i}}}_{\mathclap{= X(\sigma^k)|_p = \difffrac[t=0]{}{t} (\sigma^k \circ \gamma) \text{ mit } \dot\gamma(0) = X_p}} + \sum_{i,j} \xi_p^{i} \sigma^j(p) \Gamma_{ij}^k(p)\Bigg) s_k(p)
\end{align*}

\begin{bem}\begin{enumerate}[label=\arabic*),leftmargin=*]
\item
	$X \mapsto (\nabla_XS)_p$ h"angt nur von dem Wert $X_p$ von  $X$ in $p$ ab, Schreibweise $(\nabla_XS)_p = \nabla_{X_p}S$.
\item
	$S \mapsto \nabla_{X_p}S$ h"angt nur von den Werten von $S$ entlang einer Kurve $\gamma$ mit $\dot\gamma(0) = X_p$ ab. Es gilt
		\[ \nabla_XS = \sum_k X(\sigma^k)S_k + \sum_k \sum_j\left(\left(\sum_i \Gamma_{ij}^k \xi^{i}\right) \sigma^j\right) s_k. \]
	Schreibt man $\sigma = (\sigma^1,\ldots ,\sigma^n)$ und fasst $\dop \sigma = (\dop \sigma^1,\ldots ,\dop \sigma^n)$ also lokale 1-Form mit Werten in $\R^n$ auf, so ist f"ur $s=(s_1,\ldots ,s_n)$ $\dop\sigma \cdot s = \sum \dop \sigma^j s^j$ eine lokale 1-Form mit Werten in $E$. Es gilt: $\dop \sigma \cdot s(X) = \D_X \sigma \cdot s$. Analog definiert $\omega(X) = (\omega_j^k(X))_{k,j}$ eine lokale 1-Form mit Werten in den reellen $(n\X n)$-Matrizen. Dann ist 
	\[ \omega \sigma : X \mapsto \omega(X) \sigma = \left( \sum_{i,j} \Gamma_{ij}^k \xi^{i} \sigma^j \right)^k \]
eine lokale 1-Form mit Werten in $\R^n$ und $\omega\sigma \cdot s$ eine lokale 1-Form mit Werten in $E$. Damit gilt
	\[ \nabla_XS = (\dop \sigma(X) + \omega(X) \sigma) \cdot s \]
oder kurz
	\[ \nabla = \dop + \omega. \]
\end{enumerate}\end{bem}

\section{Transformationsverhalten}

Es seien $E|_{U_\alpha} \overset{\tau_\alpha}{\to} U_\alpha \X \R^n$ und $E|_{U_\beta} \overset{\tau_\beta}{\to} U_{\beta} \X \R^n$ lokale Trivialisierungen. Die "Ubergangsfunktion
	\[ \psi = \psi_{\beta\alpha}: U_\alpha \cap U_\beta \to \GL_n(\R) \]
war durch
	\[ \tau_\beta \circ \tau_\alpha^{-1} (p,x) = (p, \psi x) \]
definiert. F"ur die lokalen Darstellungen $S = \sum \sigma^j s_j = \sum \tilde\sigma^j s_j$ in $\tau_\alpha$ beziehungsweise $\tau_\beta$ gilt damit $\tilde\sigma^{i} = \sum_k \psi_k^{i} \sigma^k$, kurz $\tilde\sigma = \psi \sigma$. Es folgt daraus:
	\[ (\dop \sigma(X) + \omega(X) \sigma) S = \nabla_X S = (\dop \tilde\sigma(X) + \tilde\omega(X)\tilde\sigma) \tilde S \]
also
\begin{align*}
	\dop \sigma(X) + \omega(X) \sigma &= \psi^{-1}(\dop \tilde\sigma(X) + \tilde\omega(X) \tilde\sigma)\\
	&= \psi^{-1} (\dop(\psi\sigma)(X) + \tilde\omega(X) \psi \sigma)\\
	&= \psi^{-1} ((D_X f) \sigma + \psi \dop \sigma(X) + \tilde\omega(X) \psi \sigma)\\
	&= \dop \sigma(X) + (\underbrace{\psi^{-1}(D_X \psi) + \psi^{-1} \tilde\omega(X) \psi}_{=\omega(X)}) \sigma.
\end{align*}
Damit gilt
	\[ \tilde\omega(X) = \psi \omega(X) \psi^{-1} - D_X \psi \cdot \psi^{-1}. \qquad (*) \]
Daher definiert $\omega(X)$ \emph{keinen} Schnitt in $\Hom(E, E)$, denn in Kapitel \ref{kapitel-5} wurde gezeigt, dass die "Ubergangsfunktion von $\Hom(E, E)$ gegeben ist durch
	\[ (p, \eta) \to (p, \psi \circ \eta \circ \psi^{-1}). \]

\begin{bem}
Der zweite Summand in (*) h"angt \emph{nur} von der "Ubergangsfunktion $\psi$ und $X$ ab, und somit \emph{nicht} von $\nabla$. Das hei"st sind $\nabla$ und $\tilde\nabla$ kovariante Ableitungen auf $E$, so ist ihre Differenz $\nabla - \tilde\nabla$ eine globale 1-Form mit Werten in $\Hom(E,E)$.
\end{bem}

Durch eine kovariante Ableitung auf einem Vektorb"undel $E$ erhalten wir kovariante Ableitungen auf dem dualen Vektorb"undel $E^*$ und Tensorprodukten von Vektorb"undeln wie folgt:

\begin{Prop}
Die f"ur $X \in \calV(M)$, $S^* \in \Gamma(E^*)$ und $v \in E_p$ sowie eine Fortsetzung $\tilde v \in \Gamma(E)$ von $v_p$ durch
	\[ (\nabla_X^* S^*)_p(v) = X_p(S^*(\tilde v)) - S^*|_p (\nabla_X \tilde v) \]
definierte Abbildung ist eine kovariante Ableitung auf $E^*$. Dass $S^*(\tilde v) = (S^*, \tilde v)$ ist führt zu $X_p(S^*, \tilde v) = (\nabla_X^* S^*, \tilde v) + (S^*, \nabla_X \tilde v)$.
\end{Prop}
Der Beweis sei zur "Ubung "uberlassen.

\begin{Prop}
Es seien $E_1$ und $E_2$ Vektorb"undel mit kovarianten Ableitungen  $\nabla^1$ und $\nabla^2$ "uber $M$. Dann definiert f"ur $X \in \calV(M)$, $S_i \in \Gamma(E_i)$
	\[ \nabla_X (S_1 \otimes S_2) = \nabla_X^1 S_1 \otimes S_2 + \nabla_X^2 S_1 \otimes S_2 \]
durch lineare Fortsetzungen eine kovarainte Abbildung auf $E_1 \otimes E_2$.
\end{Prop}

\begin{Dfn}
Die Abbildung
\begin{align*}
	R^\nabla &= R : \calV(M) \X \calV(M) \X \Gamma(E) \to \Gamma(E)\\
	R(X,Y)S &= \nabla_X \nabla_Y S - \nabla_Y \nabla_X S - \nabla_{[X,Y]} S
\end{align*}
hei"st der \CmMark[Tensor!Kr"ummungs-]{Kr"ummungstensor} der Abbildung $\nabla$.
\end{Dfn}

\begin{bem}
F"ur alle $X, Y \in \calV(M)$ gilt $R(Y,X) = - R(X,Y)$.
\end{bem}

\begin{Prop}
$R$ ist $C^\infty(M)$-linear in allen Argumenten.
\end{Prop}
Der Beweis sei zur "Ubung "uberlassen.

\section{Schnitte entlang von Ableitungen}

\begin{center}\begin{tikzpicture}
%	\draw[step=0.25,gray!15] (-6,-5) grid (6,5); \draw[step=0.5,gray!30] (-6,-5) grid (6,5); \fill (0,0) circle(0.1); %Hilfsgitter
	
	\node (1) at (-5,0.5) {$\calI$}; \node (2) at (-5,2) {$M$}; \node (3) at (-3,2) {$\T M$};
	\draw[->] (1) --node[left,font=\scriptsize]{$c$} (2); \draw[->] (2) -- (3);
	
	\node at (5,1) {$\nabla_{\dot c} \tilde X = \gamma$};
	
	\draw[name path=kurve] (-2,-0.5) ..controls(-2,-0.5) and (-1.5,0).. (0.5,0)node[above,font=\scriptsize]{$c$} ..controls(2.5,0) and (3,1.5).. (3,1.5);
	
	\path[name path=pkt] (-1.5,-1) -- (-1.5,2);
	\path[name intersections={of=kurve and pkt}];
	\coordinate (a) at (intersection-1);
	\draw[->] (a) -- ($(a) + 0.8*(0.7,1.5)$);
	
	\path[name path=pkt] (-0.5,-1) -- (-0.5,2);
	\path[name intersections={of=kurve and pkt}];
	\coordinate (a) at (intersection-1);
	\draw[->] (a) -- ($(a) + 0.8*(1,0.8)$);
	
	\path[name path=pkt] (0.75,-1) -- (0.75,2);
	\path[name intersections={of=kurve and pkt}];
	\coordinate (a) at (intersection-1);
	\draw[->] (a) -- ($(a) + 0.5*(0.7,-1.5)$);
	
	\path[name path=pkt] (1.75,-1) -- (1.75,2);
	\path[name intersections={of=kurve and pkt}];
	\coordinate (a) at (intersection-1);
	\draw[->] (a) -- ($(a) + (1,-0.5)$);
	
	\path[name path=pkt] (2.5,-1) -- (2.5,2);
	\path[name intersections={of=kurve and pkt}];
	\coordinate (a) at (intersection-1);
	\draw[->] (a) -- ($(a) + 0.5*(0.7,2.5)$);
\end{tikzpicture}\\
\textcolor{red}{Anmerkung: die Zeichnung ist wom"oglich unvollst"andig}\end{center}

\begin{Dfn}
Es seien $E$ ein Vektorb"undel "uber $M$ mit kovarianter Ableitung $\nabla$ und $\Phi: N \to M$. Ein \CmMark{Schnitt} entlang $\Phi$ ist eine glatte Abbildung $S: N \to E$, so dass $S(p) \in E_{\Phi(p)}$ gilt, dies entspricht genau dem Schnitt in das l"angs $\Phi$ zur"uckgezogene B"undel $\Phi^*$.
\begin{center}\begin{tikzpicture}
	\def\hor{1.75}
	\def\vert{1}
	\node (1) at (-\hor,\vert) {$\Phi^*E$}; \node (2) at (\hor,\vert) {$E$}; \node (3) at (-\hor,-\vert) {$N$}; \node (4) at (\hor,-\vert) {$M$};
	
	\draw[right hook->] (1) -- (2); \draw[->] (1) -- (3); \draw[->] (2) --node[right,font=\scriptsize]{$\pi$} (4); \draw[->] (3) --node[above,font=\scriptsize]{$\Phi$} (4); \draw[->] (3) --node[above,font=\scriptsize]{$S$} (2);
	\draw[->,dashed] (3) to[out=120,in=240]node[left,font=\scriptsize]{$\tilde S$} (1);
\end{tikzpicture}\end{center}
\end{Dfn}
F"ur einen Schnitt $S$ in $E$ l"angs $\Phi$ und $X_p \in \T_pN$ ist die kovariante Abbildung $\nabla_{X_p}S$ von $S$ in Richtung $X_p$ wie folgt definiert:

Es sei $s_1,\ldots ,s_n$ eine lokale Basis "uber einer Trivialisierungsumgebung $U \subseteq M$. Dann ist $S$ lokal gegeben durch
	\[ S_p = \sum_j \sigma^j(p) s_j(\Phi(p)) \]
f"ur $p \in V = \Phi^{-1}(U) \subseteq N$, und damit
	\[ \nabla_{X_p}S = \left( \dop \sigma(X_p) + \omega(\Phi_{*p}X_p) \sigma(p) \right) S(\Phi(p)).\]
Dies H"angt nicht von der Wahl der Trivialisierung ab, denn ist $U'$ ein weiteres Trivialisierungsgebiet mit lokaler Basis $\tilde s_1, \ldots , \tilde s_n$ und "Ubergangsfunktion $\psi: C \cap U' \to \GL_n(\R)$, so gilt
\begin{align*}
	\tilde\sigma &= (\psi \circ \Phi) \sigma \qquad \text{und}\\
	\tilde\omega &= (\psi \circ \Phi) \omega(\psi \circ \Phi)^{-1} - \dop (\psi \circ \Phi)(\psi \circ \Phi)^{-1}
\end{align*}
damit folgt
\begin{align*}
	\dop \tilde\sigma(X_p) &+ \tilde\omega(\Phi_{*p}X_p) \tilde\sigma(p) - \dop((\psi \circ \Phi)\sigma) (X_p)\\
	&+ (\psi \circ \Phi) \omega (\Phi_{*p}X_p)(\psi \circ \Phi)^{-1}(\psi \circ \Phi)\sigma\\
	&- \dop(\psi \circ \Phi)(X_p)(\psi \circ \Phi)^{-1}(\psi \circ \Phi)\sigma
\end{align*}\begin{align*}
	&= \dop(\psi \circ \Phi)(X_p)\sigma + (\psi \circ \Phi)\omega(\Phi_{*p}X_p)\sigma - \dop(\psi \circ \Phi)(X_p)\sigma\\
	&= (\psi \circ \Phi)(\dop\sigma(X_p) + \omega(\Phi_{*p}X_p)\sigma)
\end{align*}
Damit ist $p \mapsto \nabla_{X_p}S$ als Schnitt entlang $\Phi$ wohldefiniert.
\begin{bem}
Dies definiert eine kovariante Ableitung auf $\Phi^*E \subseteq N \X E$. Sind umgekehrt $S \in \Gamma(E)$ und $X_p \in \T_pN$, so ist $S \circ \Phi$ ein Schnitt entlang $\Phi$ und es gilt
	\[ \nabla_{X_p}(S \circ \Phi) = \nabla_{\Phi_{*p}X_p}S \]
\end{bem}

\begin{emptythm}[Spezialfall:]
Sei $\Phi = c : \calI = [a,b] \to M$. Ein Schnitt in $E$ entlang $c$ ist eine glatte Abbildung $S: \calI \to E$ mit $S(t) \in E_{c(t)}$. Die kovariante Abbildung $\nabla_{\pdifffrac[t]{}{t}} S$ wird kurz als $\nabla_tS$ oder $S'(t)$ geschrieben. In lokalen Koordinaten gilt
\begin{align*}
	S' &= \left( \dop \sigma \left( \pdifffrac{}{t} \right) + \omega\left( c_* \left( \pdifffrac{}{t} \right) \right) \sigma \right) S \circ c\\
	&= \left( \sigma' + \omega(\dot c) \sigma \right) S \circ c
\end{align*}
\end{emptythm}

\begin{Dfn}
Ein Schnitt $S \in \Gamma(E)$ hei"st \CmMark[Schnitt!paralleler]{parallel} (oder \CmMark[Schnitt!konstanter]{konstant}), wenn $\nabla S \equiv 0$. Ein Schnitt $S$ entlang $c$ hei"st \CmMark[Schnitt!paralleler]{parallel}, wenn $S' \equiv 0$ gilt.
\end{Dfn}

\begin{Prop}\label{prop-7-9}
Es sei $c: \calI \to M$ eine (st"uckweise) glatte Kurve und $\xi \in E_{c(s)}$. Dann existiert genau ein entlang $c$ paralleler Schnitt $S_\xi$ in $E$ mit $S_\xi(s) = \xi$.
\end{Prop}

\begin{bew}
in lokalen Koordinaten definiert
	\[ 0 = S_\xi'(t) = (\sigma'(t) + \omega(\dot c(t) \sigma(t)) S(\dop t)) \]
ein lineares Differentialgleichungssystem:
	\[ \sigma'(t)  = A(t) \cdot \sigma(t) \]
wobei $A(t) = -\omega(\dot c(t))$. Ist $[t, T]$ \textcolor{red}{ein kompaktes} Teilintervall in $\calI$ mit $s \in [t, T]$, so existiert eine Partition $t = t_0 < \ldots < t_k = T$, so dass $c([t_i, t_{i+1}])$ in einer Trivialisierungsumgebung liegt. Man findet so sukzessive eindeutige L"osungen auf den Teilintervallen (\emph{lineares} System), welche durch Fortsetzungen eine eindeutige L"osung auf $[t, T]$ definieren. Erneut folgt aus der Eindeutigkeit, dass ein f"ur alle Zeiten definierter paralleler Schnitt $S_\xi$ existiert.
\end{bew}

\begin{Dfn}
Es sei $c$ eine glatte Kurve in $M$. Die lineare Abbildung
	\[\begin{array}{cccc} P_{s,t}^c: & E_{c(s)} &\to& E_{c(t)} \\
		& \xi &\mapsto& S_\xi(t), \end{array}\]
wobei $S_\xi$ den nach Proposition \ref{prop-7-9} eindeutigen parallelen Schnitt einlang $c$ mit $S_\xi(s) = \xi$ bezeichnet, hei"st \CmMark{Paralleltransport} entlang $c$.
\end{Dfn}

\begin{bem}\begin{enumerate}[label=\arabic*),leftmargin=*]
\item
	$P_{s,t}^c$ ist invertierbar mit Inversen $(P_{s,t}^c)^{-1} = P_{t,s}^c = P_{s,t}^{\overline c}$, wobei $\overline c = (s+t-\tau)$.
\item
	Die Abbildung $P_{s,t}^c$ ist im Allgemeinen \emph{nicht} unabh"angig von der Wahl von $c$.
\end{enumerate}\end{bem}

\begin{bsp}
In $\R^n$ ist ein Vektorfeld $X$ genau dann parallel, wenn $X$ (beziehungsweise $\overline X_p \in \calI_p(X_p)$) konstant im \quot{"ublichen} Sinne ist: Paralleltransport entlang einer Kurve entspricht der gew"ahlten Parallelverschiebung.
\begin{center}\begin{tikzpicture}[font=\scriptsize]
%	\draw[step=0.25,gray!15] (-6,-1) grid (6,5); \draw[step=0.5,gray!30] (-6,-1) grid (6,5); \fill (0,0) circle(0.1); %Hilfsgitter
	
	\draw[name path=kurve] (-4.5,-0.5) ..controls(-4.5,-0.5) and (-4,0.5).. (0,0.5)node[below]{$c$} ..controls(4,0.5) and (4.5,3).. (4.5,3);
	
	\foreach \x in {3.5, 1.5, ..., -1.5}{% Schleife von rechts nach links
		\path[name path=strich] (\x,-1) -- (\x,5);
		\path[name intersections={of=strich and kurve}];
		\coordinate (a) at (intersection-1);
		\draw[->] (a) -- +(0,3);
		\draw[dashed] ($(a) + (0,1)$) -- +(-1,0);
		
		\path[name path=strich] (\x - 1,-1) -- (\x - 1,5);
		\path[name intersections={of=strich and kurve}];
		\coordinate (a) at (intersection-1);
		\draw[->] (a) -- +(0,3);
		\draw[dashed] ($(a) + (0,2)$) -- +(-1,0);
	}
	\path[name path=strich] (-2.5,-1) -- (-2.5,5);
	\path[name intersections={of=strich and kurve}];
	\coordinate (a) at (intersection-1);
	\draw[->] (a) -- +(0,3);
	\draw[dashed] ($(a) + (0,1)$) -- +(-1,0);
	
	\path[name path=strich] (-3.5,-1) -- (-3.5,5);
	\path[name intersections={of=strich and kurve}];
	\coordinate (a) at (intersection-1);
	\draw[->] (a) -- +(0,3);

	\fill (a) circle(0.05) node[anchor=north west]{$c(s)$};% nur am letzten Strich
\end{tikzpicture}\end{center}
\end{bsp}

Es seien $S \in \Gamma(E)$ und $X_p \in \T_pM$. Ist $c$ eine Integralkurve von $X_p$, das hei"st $c(0) = p$ und $\dot c(0) = X_p$, so ist $\tilde S = S \circ c$ ein Schnitt entlang $c$ und es gilt $\tilde S'(0) = \nabla_{X_p} S$. Nun sei ferner $\xi_1,\ldots ,\xi_n$ eine Basis von $E_p$ und es bezeichnen $s_1,\ldots ,s_n$ die parallelen Schnitte  entlang $c$ mit $s_i(0) = \xi_i$. Dann gilt $\tilde S(t) = \sum \sigma^j(t)s_j(t)$ und es folgt
\begin{align*}
	\nabla_{X_p}S &= \tilde S'(0) = \nabla_t \big( \sum_j \sigma^j s_j \big)(0)\\
	&= \sum_j \big( (\sigma^j)' s_j + \sigma^j \underbrace{\nabla_t s_j}_{\equiv 0} \big) (0)\\
	&= \sum_j \lim_{t \to 0} \left( \frac{\sigma^j(t) - \sigma^j(0)}{t} \right) s_j(0)\\
	&= \lim_{t \to 0} \frac{1}{t} \left( \sum_j \sigma^j(t) s_j(0) - \sigma^j(0) s_j(0) \right)\\
	&= \lim_{t \to 0} \frac{1}{t} \left( \sum_j \sigma^j \big(P_{t,0}^c (s_j(t) \big) - \sigma^j(0) s_j(0) \right)\\
	&= \lim_{t \to 0} \frac{1}{t} \left( P_{t,0}^c \big( \sum \sigma^j s_j(t) \big) - \sum \sigma^j s_j(0) \right)\\
	&= \lim_{t \to 0} \frac{1}{t} \left( P_{t,0}^c \big( \tilde S(t) - \tilde S(0) \big) \right)
\end{align*}
%%% Local Variables: 
%%% mode: latex
%%% TeX-master: "../skript-diffgeom"
%%% End: 
