%% 
%% Skript Differentialgeometrie im Wintersemester 12/13
%% Zur Vorlesung von Dr. Grensing am KIT Karlsruhe
%% 
%% Kapitel 9
%% 
\chapter{Jacobifelder}

F"ur $p,q \in M$ sei $\Omega_{pq}$ der Raum aller glatten Kurven
$c:[0,1] \to M$ mit $c(0)=p$ und $c(1)=q$.
\begin{center}\begin{tikzpicture}[font=\scriptsize]
    %	\tikzgitter{(-5,-5)}{(5,5)} % Hilfsgitter
	
    \coordinate (p) at (-3.5,-1); \coordinate (q) at ($-1*(p)$); \fill
    (p) circle(0.05)node[left]{$p$} (q) circle(0.05)node[right]{$q$}
    (0,0) circle(0.05)node[below right]{$c$};
	
    \coordinate (ctrl) at (2,-0.25); \draw (p) ..controls(p) and
    ($-1*(ctrl)$).. (0,0) ..controls(ctrl) and (q).. (q);
	
    \coordinate (a) at (0,-1); \coordinate (b) at ($-1*(a)$);
    \coordinate (ctrl2) at (0.25,0.5); \draw (a) ..controls(a) and
    ($-1*(ctrl2)$).. (0,0) ..controls(ctrl2) and (b).. (b); \draw[->]
    (0,0) -- ($2*(ctrl2)$) node[right]{$\difffrac{}{s}c_s$};
	
    \def\faktor{0.2} \foreach \y in {1, 2}{ \path (a) ..controls(a)
      and ($-1*(ctrl2)$).. (0,0) ..controls(ctrl2) and
      (b).. coordinate[pos=\y * \faktor] (coord) (b);
      % \draw[red] (coord) circle(0.05);
      \draw[dashed] (p) ..controls(p) and ($(coord)
      -1*(ctrl)$).. (coord) ..controls($(coord) + (ctrl)$) and
      (q).. (q); } \foreach \y in {1, 2, 3}{ \path (a) ..controls(a)
      and ($-1*(ctrl2)$).. coordinate[pos=1 - \y * \faktor] (coord)
      (0,0) ..controls(ctrl2) and (b).. (b);
      % \draw[red] (coord) circle(0.05);
      \draw[dashed] (p) ..controls(p) and ($(coord)
      -1*(ctrl)$).. (coord) ..controls($(coord) + (ctrl)$) and
      (q).. (q); }
  \end{tikzpicture}\end{center}

\begin{Dfn}
  Eine \CmMark[Variation]{(glatte) Variation} einer glatten Kurve
  $c:[a,b] \to M$ ist eine glatte Abbildung
  \begin{align*}
    h:(-\epsilon, \epsilon) \X [a,b] \to M && h_s(t) = h(s,t)
  \end{align*}
  mit $h_0 = c$. Gilt $h(\cdot, a) \equiv c(a)$ und $h(\cdot, b)
  \equiv c(b)$, so hei"st $h$ eine \CmMark[Variation!mit festen
  Endpunkten]{Variation mit festen Endpunkten} oder
  \CmMark[Variation!eigentliche]{eigentliche Variation}. Man schreibt
  $c_s$ f"ur eine Variation $h$ von $c$.
\end{Dfn}

Ist $c_s$ eine glatte Variation von $c$, so ist
\begin{align*}
  Y(t) &= \difffrac[s=0]{}{s} c_s(t)\\
  &= \difffrac[s=0]{}{s} h(s,t) =
  h_{*(0,t)}\left(\pdifffrac{}{s}\right)
\end{align*}
ein Vektorfeld entlang $c$. ist $c_s$ eigentlich, so gilt $Y(a) = 0
\in \T_{c(a)}M$ und $Y(b) = 0 \in \T_{c(b)}M$.
\begin{center}\begin{tikzpicture}[font=\scriptsize]
    \coordinate (1) at (-2,-1); \coordinate (2) at (0,0); \coordinate
    (3) at (2,1.5); \coordinate (ctrl) at (1.5,0.25); \draw (1)
    ..controls(1) and ($-1*(ctrl)$)..node[pos=0.6,below]{$c$}
    coordinate[pos=0.9] (a) (2) ..controls(ctrl) and (3)..  (3); \fill
    (a) circle(0.05); \draw[->] (a) -- ($(a) + (0.5,1)$)
    node[left]{$Y(t)$};
  \end{tikzpicture}\end{center}
Tats"achlich ist jedes Vektorfeld ein solches Variationsfeld einer
Variation von $c$: Ist $Y$ ein Vektorfeld entlang $c$, so definiert
$h(s,t) = \exp_{c(t)}(s Y(t))$ eine Variation von $c$ und es gilt:
\begin{align*}
  \difffrac[s=0]{}{s} h(s,t) &= \exp_{c(t)*0}(Y(t))\\
  &= \id_{\T_{c(t)}M}(Y(t)) = Y(t).
\end{align*}
Falls $Y$ in den Endpunkten von $c$ verschwindet, so ist die so
definierte Variation eigentlich. Bestimme $\difffrac[s=0]{}{s} E(c_s)$
und $\difffrac[s=0]{}{s}\calL(c_s)$:
\begin{align*}
  \frac{1}{2} \difffrac[s=0]{}{s} \langle \dot c_s, \dot c_s \rangle &= \langle \nabla_s \dot c(s), \dot c(s) \rangle\\
  &= \left\langle \nabla_s \difffrac{}{t} c_s, \dot c(s) \right\rangle = \left\langle \nabla_t \difffrac{}{s} c_s, \dot c_s \right\rangle \\
  &= \left\langle \difffrac[s=0]{}{s} c_s, \dot c_s \right\rangle' - \left\langle \difffrac[s=0]{}{s} c_s, \nabla_t \dot c_s \right\rangle \\
  &= \left\langle Y, \dot c \right\rangle' - \left\langle Y, \nabla_t \dot c \right\rangle\\
\end{align*}
\begin{align*}
  \difffrac[s=0]{}{t} \|\dot c_s\| &= \frac{1}{2 \|c_s\|} \difffrac[s=0]{}{s} \langle \dot c_s, \dot c_s \rangle\\
  &= \frac{\langle Y, \dot c \rangle' - \langle Y, \nabla_t \dot c
    \rangle}{\|\dot c\|}
\end{align*}
Damit folgt:
\begin{align*}
  \difffrac[s=0]{}{s} E(c_s) = \difffrac[s=0]{}{s} \int_a^b
  \frac{1}{2} \|\dot c_s\| = \left. \langle Y, \dot c \rangle
  \right|_a^b - \int_a^b \langle Y, \nabla_t \dot c \rangle
\end{align*}
Betrachte $E: \Omega_{pq} \to \R$. Dann ist $c \in \Omega_{pq}$ genau
dann eine Geod"atische, wenn $c$ ein kritischer Punkt von $E$ ist, das
hei"st $\difffrac[s=0]{}{s}E(c_s) = 0$ f"ur jede eigentliche Veriation
von $c$.  Ist $c$ ein kritischer Punkt von $E$, so sei $c_s$ die von
$Y = f \nabla_t \dot c$ mit $f(0) = 0 = f(1)$ erzeugte Variation.
Dann ist $c_s$ eigentlich und es gilt
\begin{align*}
  0 = \difffrac[s=0]{}{s} E(c_s) = - \int_a^b f \|\nabla_t \dot c\|^2
\end{align*}
also $\nabla_t \dot c = 0$.  Ist $c$ nach der Bogenl"ange
parametrisiert, so gilt
\begin{align*}
  \difffrac[s=0]{}{s} \calL(c_s) = \difffrac[s=0]{}{s} E(c_s)
\end{align*}
%% 
%% Vorlesung <2013-1-11 Fri>
%% 
Eine kurve $c \in \Omega_{pq}$ ist genau dann ein kritischer Punkt von
$\calL$, wenn $c$ eine umparametrisierte Geod"atische ist.

\section{Ausblick: Hesse \& Morse - Theorie}

Sei $f \in C^\infty(M)$, sei nach Konvention $\nabla_X f = X(f) = \dop
f(X)$, und $\nabla f = \dop f \in \Omega^1(M) = \Gamma(\T M^*)$. F"ur
die Hessesche $\Hh_f = \nabla^2 f$ gilt nach Proposition
\ref{prop-7-3}:
\begin{align*}
  \nabla^2 f (X,Y) &= (\nabla_X \dop f)(Y) = X(\dop f(Y)) - \dop f(\nabla_X Y)\\
  &= X(Yf) - (\nabla_X Y)(f) \qquad (= \nabla_{X,Y}^2 \text{ in Kapitel 7})\\
  &= [X,Y]f + Y(Xf) - (\underbrace{\nabla_X Y - \nabla_Y X}_{\mathclap{[X,Y] \text{ Torsionsfreiheit}}}) f - (\nabla_Y X) f\\
  &= Y(Xf) - (\nabla_YX)(f) = \nabla^2 f(Y,X) = \Hh_f(Y,X)
\end{align*}
Die Hessesche ist also eine symmetrische $\R$-Bilinearform $\Hh_f:
\calV(M) \X \calV(M) \to C^\infty(M)$. Sie ist im Allgemeinen
\emph{nicht} $\C^\infty(M)$-bilinear. Ist $p \in M$ ein kritischer
Punkt von $f$, das hei"st $\dop f|_p = 0$, dann h"angt $\Hh_f|_p$ nur
von $\xi = X_p$ und $\eta = Y_p$ ab: Ist $\tilde X$ ein Vektorfeld mit
$\tilde X_p = \xi = X_p$, so gilt:
\begin{align*}
  \Hh_f|_p(\tilde X,Y) &= \tilde X_p(Yf) - \underbrace{\dop f|_p(\nabla_{\tilde X}Y)}_{=0} = \tilde X_p(Yf) = \xi(Yf)\\
  &= X_p(Yf) = \ldots = \Hh_f|_p(X,Y)
\end{align*}
$\Hh_f|_p$ ist eine Bilinearform auf $\T_pM$. Insbesondere h"angt
$\Hh_f|_p$ nur von der differenzierbaren Struktur von $M$ und
\emph{nicht} von der Riemannschen Struktur ab.  Ist $\Hh_f$ nicht
ausgeartet, so hei"st die Anzahl der negativen Eigenwerte der
\CmMark{Index} von $f$ in $p$.  Ist $v \in \T_pM$ der Eigenvektor zu
einem negativen Eigenwert $k$ und $\gamma$ eine Kurve mit
$\gamma(0)=p$ und $\dot\gamma(0)=v$. Dann gilt
\begin{align*}
  0 > \lambda || v ||^2 = \Hh_f|_p (v,v) = v((f \circ \gamma)') =
  \difffrac[t=0]{^2}{t^2} f(\gamma(t))
\end{align*}
Entlang der Kurve $\gamma$ nimmt $f$ also ein striktes Maximum an.
\begin{center}\begin{tikzpicture}[font=\scriptsize,normal/.style={above,sloped,
      inner sep=0pt,outer sep=0pt,allow upside down,below}]
    % \tikzgitter{(-7,-7)}{(7,7)}
    
    % neuer Befehl f"ur die kleine Parabel (1. Parameter Scheite,
    % 2. Parameter linker Linenstil, 3. Parameter rechter Linienstil,
    % optionaler Parameter Rotation)
    \newcommand\tikzKleineParabel[4][0]{ \def\kbreite{1}
      \def\khoehe{2} \def\kstretch{0.75} \coordinate[rotate
      around={#1:#2}] (links) at ($#2 + (-\kbreite,-\khoehe)$);
      \coordinate[rotate around={#1:#2}] (rechts) at ($#2 +
      (\kbreite,-\khoehe)$);
      
      \draw[#3,rotate around={#1:#2}] (links) ..controls (links) and
      ($#2 - (\kstretch,0)$).. #2; \draw[#4,rotate around={#1:#2}] #2
      ..controls($#2 + (\kstretch,0)$) and (rechts).. (rechts); }
    
    % grosse Parabel mit den kleinen Parabeln drauf
    \def\stretch{2.5} \def\gbreite{4} \def\ghoehe{4} \coordinate
    (lende) at (-\gbreite,\ghoehe); \coordinate (rende) at
    (\gbreite,\ghoehe); \coordinate (scheitel) at (0,0); \draw (lende)
    ..controls(lende) and ($(scheitel) - (\stretch,0)$)..
    node[pos=0.4,normal]{\tikz \tikzKleineParabel{(0,0)}{dashed}{};}
    node[pos=0.7,normal]{\tikz \tikzKleineParabel{(0,0)}{dashed}{};}
    (scheitel) ..controls($(scheitel) + (\stretch,0)$) and (rende)..
    node[pos=0.3,normal]{\tikz \tikzKleineParabel{(0,0)}{}{dashed};}
    node[pos=0.6,normal]{\tikz \tikzKleineParabel{(0,0)}{}{dashed};}
    (rende);
    
    % Beschriftung
    \fill (scheitel) circle(0.05) node[above]{$p$}; \node at
    (0.5*\gbreite,0.75*\ghoehe) {$M$}; \node at
    (\gbreite,-0.5*\ghoehe) {$f$ H"ohenfunktion};
    
    % mittlere kleine Parabel und die beiden Waagrechten kleinen
    % Parabeln links und rechts
    \tikzKleineParabel{(scheitel)}{}{dashed}
    \tikzKleineParabel[270]{(lende)}{}{}
    \tikzKleineParabel[90]{(rende)}{}{}
  \end{tikzpicture}\end{center}
Tats"achlich ist jeder nicht ausgeartete kritische Punkt von solcher
Gestalt.

\begin{emptythm}[Morse-Lemma]
  Es sei $p \in M$ ein nicht ausgearteter kritischer Punkt von $f \in
  C^\infty(M)$ mit Index $\alpha$. Dann existiert eine Karte ($\phi,
  U)$ um $p$ mit $\phi(p) = 0$ und $f = f(p) - (\phi^1)^2 - (\phi^2)^2
  - \ldots -(\phi^\alpha)^2 + (\phi^{\alpha+1})^2 + \ldots +
  (\phi^m)^2$.
\end{emptythm}

\paragraph{Morse-Theorie}
\begin{center}\begin{tikzpicture}[font=\scriptsize]
    %	\tikzgitter{(-5,-5)}{(5,5)}
    
    % die grosse Ellipse
    \def\hoehe{2} \def\breite{1.25} \coordinate (mitte) at (0,0);
    \draw (mitte) ellipse({\breite} and \hoehe);
    
    % die kleinen Ellipse
    \def\kbreite{0.5} \def\khoehe{0.125} \coordinate (a) at ($(mitte)
    - (\breite,0)$); \coordinate (b) at ($(mitte) - (0.25,0)$);
    \coordinate (d) at ($(mitte) + (\breite,0)$); \coordinate (c) at
    ($(mitte) + (0.25,0)$); \coordinate (cntrl) at
    ($0.5*(a)+0.5*(b)$); \coordinate (cntrr) at ($0.5*(c)+0.5*(d)$);
    \begin{scope}
      \clip ($(mitte) - 1.1*(\breite,0)$) rectangle ($(mitte) +
      1.1*(\breite,-2*\khoehe)$); \draw (cntrl) ellipse({\kbreite} and
      {\khoehe}); \draw (cntrr) ellipse({\kbreite} and {\khoehe});
    \end{scope}
    \begin{scope}
      \clip ($(mitte) - 1.1*(\breite,0)$) rectangle ($(mitte) +
      1.1*(\breite,2*\khoehe)$); \draw[dashed] (cntrl)
      ellipse({\kbreite} and {\khoehe}); \draw[dashed] (cntrr)
      ellipse({\kbreite} and {\khoehe});
    \end{scope}
    
    % das Loch in der Mitte
    \begin{scope}
      \clip ($(mitte) + (0.5,1)$) rectangle ($(mitte) - (1, 1)$);
      \path[draw,name path=gkreis] ($(mitte) - (0.75,0)$) ellipse (1
      and 1.25);
    \end{scope}
    \path[name path=kkreis] ($(mitte) + (0.5,0)$) ellipse (0.75 and
    1); \path[name intersections={of=gkreis and kkreis}];
    \begin{scope}
      \clip (intersection-1) rectangle ($(intersection-2)-(0.5,0)$);
      \draw ($(mitte) + (0.5,0)$) ellipse (0.75 and 1);
    \end{scope}
    
    % die Punkte
    \coordinate (p) at ($(mitte) - (0,\hoehe)$); \coordinate (q) at
    (intersection-2); \coordinate (r) at (intersection-1); \coordinate
    (s) at ($(mitte) + (0,\hoehe)$); \fill (p)
    circle(0.05)node[below]{$p=0$} (q) circle(0.05)node[below
    right]{$q$} (r) circle(0.05)node[above right]{$r$} (s)
    circle(0.05)node[above]{$s$};
    
    % der Strich
    \def\strichentfernung{2.5} \draw ($(d) +
    (\strichentfernung,-\hoehe - 0.5)$) node[below left]{$\R$} --
    ++(0,2*\hoehe+1);
    
    % der Pfeil
    \draw[->] ($(d) + 0.5*(\strichentfernung,0) - (0.5,0)$)
    --node[above]{$f$} ++(1,0);
    
    % ich bastle einen neuen Befehl f"ur die Striche mit der
    % Beschriftung
    \newcommand\beschriftung[2]{ \coordinate (pos) at ($#1 +
      (\strichentfernung+\breite,0)$); \def\weite{0.1} \draw ($(pos) -
      (\weite,0)$) -- ++(2*\weite,0) node[right,align=left]{#2}; }
    \beschriftung{(s)}{$f(s)$ Index 2\\ (glob. Max. d. H"ohenfktn.)}
    \beschriftung{(r)}{$f(r)$ Index 1}
    \beschriftung{(mitte)}{$a=f(x)$} \beschriftung{(q)}{$f(q)$ Index
      1} \beschriftung{(p)}{$f(p)=0$ Index 0}
    
    \node at ($(mitte) + (-1.75*\breite,0.75*\hoehe)$) {$M = T^2
      \subseteq \R^3$}; \node (txt) at ($(mitte) +
    (-1.75*\breite,-1*\hoehe)$) {$M^{a} = \{ f \le a \}$}; \draw[->]
    (txt) to[in=180] ($(mitte) - 0.5*(\breite,\hoehe)$);
    
  \end{tikzpicture}\end{center}
Die Topologien von $M^a$ und $M^b$ sind identisch, wenn zwischen $a$
und $b$ keine kritischen Werte auftreten.
\emph{\quot{Rekonstruktion}:} Klebe sukzessive f"ur die nicht
ausgearteten kritischen Punkte $p$ Zellen der Dimension $\Ind_f(p)$,
das hei"st $\B_1(0) \subseteq \R^{\Ind_f(p)}$.
\begin{center}\textcolor{red}{[BILD]}\end{center}
Auf jeder glatten Mannigfaltigkeit existiert einee sogenannte
\CmMark{Morse-Funktion}, das hei"st eine Funktion mit isolierten
kritschen Punkten, alle nicht entartet und $f^{-1}([a,b])$
kompakt. Ist $f(p) = a$ ein kritischer Wert, so unterscheiden sich
$M^{\alpha - \epsilon}$ und $M^{\alpha + \epsilon}$ durch das Ankleben
einer $\Ind_f(p)$-Zelle.

Weitere Informationen zu diesem Thema lassen sich im Buch \quot{Morse
  Theory} von J. Milnor \cite{milnor1963morsetheo} finden.

\section{Zweite Ableitung des Energiefunktionals (in kritischen
  Punkten)}
Es sei $c$ eine nach Bogen"ange parametrisierte Geod"atische, $c_s$
eine Variation von $c$ und $Y(t) = \difffrac[s=0]{}{s} c_s(t)$. Dann
gelten die folgenden Gleichungen:
\begin{align*}
  E(c_s) = \frac{1}{2} \int_0^{\calL} \| \dot c_s \|^2
\end{align*}
\begin{align*}
  \difffrac{}{s} \langle \dot c_s, \dot c_s \rangle &= 2 \langle \nabla_s \dot c_s, \dot c_s \rangle \\
  &= 2 \left\langle \nabla_t \difffrac{}{s} c_s, \dot c_s \right\rangle \\
  \difffrac{^2}{s^2} \langle \dot c_s, \dot c_s \rangle &= 2 \left\langle \nabla_s \nabla_t \difffrac{}{s} c_s, \dot c_s \right\rangle + 2 \left\langle \nabla_t \difffrac{}{s} c_s, \nabla_s \dot c_s \right\rangle\\
  &= 2 \left\langle \nabla_s \nabla_t \difffrac{}{s} c_s, \dot c_s
  \right\rangle + 2 \left\| \nabla_t \difffrac{}{s} c_s \right\|^2
\end{align*}
\begin{align*}
  \nabla_s \nabla_t \difffrac{}{s} c_s = \nabla_t \nabla_s
  \difffrac{}{s} c_s + R \left( \smash{\underbrace{\difffrac{}{s}
        c_s}_{\mathclap{s=0:\, Y(t)}}}, \difffrac{}{t} c_s \right)
  \difffrac{}{s} c_s \vphantom{\underbrace{\difffrac{}{a}}_{A}}
\end{align*}
Zur "Ubersichtlichkeit setzen wir nun $\nabla_t Y =: Y'$
\begin{align*}
  \frac{1}{2} \difffrac[s=0]{^2}{s^2} \langle \dot c_s, \dot c_s \rangle &= \left\langle \nabla_t \nabla_s \difffrac{}{s} c_s, \dot c_s \right\rangle + \left\langle R(Y,\dot c)Y, \dot c \right\rangle + \| \nabla_t Y \|^2\\
  &= \left\langle \nabla_s \difffrac{}{s} c_s, \dot c \right\rangle' -
  \left\langle R(Y, \dot c) \dot c, Y \right\rangle + \| Y' \|^2
\end{align*}
\begin{align*}
  \difffrac{^2}{s^2} E(c_s) = \left. \left\langle \nabla_s
      \difffrac{}{s} c_s, \dot c_s \right\rangle \right|_{0}^{\calL} +
  \int_0^{\calL} \| Y' \|^2 - \left\langle R(Y, \dot c) \dot c, Y
  \right\rangle
\end{align*}
\begin{align*}
  \difffrac[s=0]{^2}{s^2} \| \dot c_s \| &= \difffrac[s=0]{}{s} \left( \frac{1}{2 \| \dot c_s \|} \difffrac{}{s} \| \dot c \|^2 \right) \\
  &= - \frac{1}{4} \left( \difffrac[s=0]{}{s} \| c_s \|^2 \right)^2 +
  \frac{1}{2} \difffrac[s=0]{^2}{s^2} \| \dot c_s \|^2
\end{align*}
\begin{align*}
  \difffrac[s=0]{^2}{s^2} \calL(c_s) &= \difffrac[s=0]{^2}{s^2} E(c_s) - \frac{1}{4} \int \left( \difffrac[s=0]{}{s} \| \dot c_s \|^2 \right)^2 \\
  &= \left.\left\langle \nabla_s \difffrac{}{s} c_s, \dot c_s
    \right\rangle\right|_{0}^{\calL} + \int_0^{\calL} \| Y' \|^2 -
  \langle R(Y, \dot c) \dot c, Y \rangle - ( \langle Y', \dot c
  \rangle )^2
\end{align*}
Bezeichnet $Y^\perp = Y - \langle \dot c, Y \rangle \dot c$ den
Normalenanteil von $Y$ bez"uglich $\dot c$, so gilt:
\begin{align*}
  {Y^\perp}' &= Y' - \langle \nabla_t \dot c, Y \rangle \dot c - \langle \dot c, Y' \rangle \dot c - \langle \dot c, Y \rangle \nabla_t \dot c \\
  &= Y' - \langle \dot c, Y' \rangle \dot c = (Y')^\perp
\end{align*}
\begin{align*}
  \| {Y'}^\perp \| - \langle R(Y^\perp, \dot c) \dot c, Y^\perp \rangle ={}& \langle Y' - \langle \dot c, Y' \rangle \dot c, Y' - \langle \dot c, Y' \rangle \dot c \rangle - \langle R(Y, \dot c) \dot c, Y \rangle \\
  & + \langle R(Y, \dot C) \dot c, \langle \dot c, Y \rangle \dot c \rangle \\
  & + \langle R ( \langle \dot c, Y \rangle \dot c, \dot c) \dot c, Y - \langle \dot c, Y' \rangle \dot c \rangle \\
  ={}& \| Y' \|^2 - \langle R(Y, \dot c) \dot c, Y \rangle - ( \langle
  Y', \dot c \rangle )^2
\end{align*}
Es gilt:
\begin{align*}
  \difffrac[s=0]{^2}{s^2} \calL(c_s) = \left.\left\langle \nabla_s
      \difffrac{}{s} c_s, \dot c \right\rangle\right|_{0}^{\calL} +
  \int_0^{\calL} \| {Y'}^\perp \|^2 - \langle R(Y^\perp, \dot c) \dot
  c, Y^\perp \rangle
\end{align*}

%% 
%% Vorlesung <2013-1-15 Tue>
%% 

\begin{emptythm}[Erinnerung]
  Für eine glatte Funktion $f$ auf $M$ gilt in kritischen Punkten $p$:
  \begin{align*}
    H_f(v,v) = \difffrac[t=0]{^2}{t^2}f(\gamma(t))
  \end{align*}
  mit $\gamma(0) = p, \dot\gamma(0) = v$.

  Diese Eigenschaft verwenden wir in der folgenden Definition als
  Ausgangspunkt.
\end{emptythm}

\begin{Dfn}
  Es sei $Y$ ein Vektorfeld entlang einer nach Bogenlänge
  parametrisierte Kurve $c$ und $c_s$ die von $Y$ erzeugte
  Variation. Die durch
  \begin{align*}
    \calI(Y,Y) = \difffrac[s=0]{^2}{s^2} E(c_s)
  \end{align*}
  auf dem Vektorraum der Vektorfelder entlang $c$ definierte
  symmetrische Bilinearform hei"st die \CmMark{Indexform} von $c$.
\end{Dfn}

Sind $X,Y$ Vektorfelder entlang $c$, welche in den Endpunkten
verschwinden, so gilt
\begin{align*}
  \calI(X,Y) = -\int_0^{\mathcal L}\left<X'' + R(X,\dot c)\dot
    c,Y\right>
\end{align*}
denn bezeichnet $c_s$ die von $Y$ erzeugte eigentliche Variation, so
gilt
\begin{align*}
  \difffrac[s=0]{^2}{s^2}E(c_s) & =
  \left.\left<\nabla_s\difffrac{}{s}c_s,\dot
      c\right>\right|_0^{\mathcal L} + \int_0^{\mathcal L}\|Y'\|^2 -
  \left<R(Y,\dot c)\dot c,Y\right>\\
  & = \int_0^{\mathcal L}\left<Y',Y\right>' - \left<Y'',Y\right> -
  \left<R(Y,\dot c)\dot c,Y\right>\\
  & = \left.\left<Y',Y\right>\right|_0^{\mathcal L} - \int_0^{\mathcal
    L} \left<Y'',Y\right> + \left<R(Y,\dot c)\dot c,Y\right>\\
  & = - \int_0^{\mathcal L}\left<Y'' + R(Y,\dot c) \dot c, Y\right>.
\end{align*}

Die Indexform um eine Geodätische $c$ ist genau dann ausgeartet, wenn
ein zu den Endpunkten verschwindendes Vektorfeld entlang $c$ existiert
mit
\begin{align*}
  Y'' + R(Y,\dot c) \dot c \equiv 0. \label{eq:jacobi-field-deq}
\end{align*}

% Definition 9.3
\begin{Dfn}
  Ein Vektorfeld entlang einer Geodätischen $c$ heißt
  \CmMark{Jacobifeld}, wenn es obige Differentialgleichung
  (\ref{eq:jacobi-field-deq}) erfüllt.
\end{Dfn}

% Lemma 9.4
\begin{Lemma}\label{thm:lemma-9-4}
  Es sei $c \colon [0,1] \to M$ eine Geödätische, $p = c(0)$. Dann
  existiert für alle $v,w \in \T_pM$ genau ein Jacobifeld $J$ entlang
  $c$ mit $J(c) = v, \ J'(0) = w$.
\end{Lemma}

\begin{bew}
  Es sei $e_1, \ldots, e_m \in \T_pM$ eine Orthonormalbasis des
  Tangentialraums in $p$ und es bezeichne n $E_1, \ldots, E_m$ die
  entlang $c$ parallelen Vektorfelder mit $E_i(0) = e_i$. Dann ist
  jedes Vektorfeld $Y$ entlang $c$ von der Form $Y = \sum_i
  \eta^iE_i$. Dann gilt:
  \begin{align*}
    Y' = \sum_i (\dot \eta^i E_i + \eta^i\nabla_tE_i) = \sum_i \dot
    \eta^i E_i
  \end{align*}
  und $Y'' = \sum \ddot \eta^i E_i$. Setzt man $R(E_j,\dot c)\dot c =
  \sum_i\rho_j^i E_i$, so ist (\ref{eq:jacobi-field-deq}) äquivalent
  zum System linearer Differentialgleichungen zweiter Ordnung
  \begin{align*}
    \ddot \eta^i + \sum_i \eta^i\rho_j^i = 0.
  \end{align*}
  Existens und Eindeutigkeit folgen mit der Lösungstheorie
  gewöhnlicher Differentialgleichungen.
\end{bew}

\begin{bsp}[Jacobifelder des $\R^n$]
  Die Geodätischen des $\R^n$ sind genau die Geraden. Ein Vektorfeld
  $Y$ entlang einer Geraden ist genau dann ein Jacobifeld, wenn $Y'' =
  0$ gilt; jedes solche ist der Form $Y(t) = v + tw$.

  \begin{center}
    \textcolor{red}{Abbildung 1, <2013-1-15 Tue>}
  \end{center}
\end{bsp}

Sind die Startwerde eines Jacobifeldes tangential an $c$, etwa $J(0) =
\lambda \dot c(0)$ und $J'(0) = \mu \dot c(0)$, so gilt
\begin{align*}
  J(t) = (\lambda + t\mu)\dot c(t),
\end{align*}
denn
\begin{align*}
  & J''(t) = \nabla_t(\mu \dot c(t) + (\lambda + t
  \mu)\underbrace{\nabla_t\dot c(t)}_{=0}) = \mu\nabla_t\dot c = 0,\\
  & \left.R(J,\dot c)\dot c\right|_t = (\lambda + t\mu)R(\dot c, \dot
  c)\dot c = 0.
\end{align*}
Zu $c$ tangentiale Jacobifelder tragen keine geometrischen
Informationen; vgl. zweite Ableitung des Längenfunktionals.

Gilt für die Startwerte eines Jacobifeldes $J(0),\ J'(0) = \dot
c(0)^{\perp}$,
\begin{align*}
  \left<J',\dot c\right> = \left<J'',\dot c\right> + \left<J',
    \nabla_t\dot c\right> = - \left<R(J,\dot c)\dot c,\dot c\right> =
  0,
\end{align*}
also $J'(t) \perp \dot c(t)$ für alle $t$ und $\left<J,\dot c\right>'
= \left<J',\dot c\right> = 0$, somit $J(t) \perp \dot c(t)$ für alle
$t$.

Der $\R$-Vektorraum der Jacobifelder entlang einer Geodätischen $c$
hat Dimension $2 \dim(M)$ und die zu $c$ normalen Jacobifelder bilden
einen Vektorraum der Dimension $2 \dim(M) - 2$.


% Satz 9.5
\begin{Satz}
  Es sei $c \colon [0,1] \to M$ eine Geodätische und $c_s$ eine
  Variation von $c$, so dass alle Kurven $c_s$ Geodätische sind. Dann
  ist das zugehörige Variationsfeld ein Jacobifeld entlang $c$.

  Jedes Jacobifeld in von dieser Gestalt.
\end{Satz}

\begin{bew}
  Es sei $c_s$ eine Variation von $c$ und alle $c_s$ seien
  Geodätische. Dann gilt:
  \begin{align*}
    Y'' & = \left.\nabla_t(\nabla_t\difffrac{}{s}c_s)\right|_{s=0} =
    \left.\nabla_t(\nabla_s\difffrac{}{t}c_s)\right|_{s=0}\\
    & = \nabla_s\underbrace{\nabla_t \difffrac{}{t} c_s}_{=0} +
    \left.R(\underbrace{\difffrac{}{t}c_s}_{=\dot
        c},\underbrace{\difffrac{}{s}c_s}_{=Y})\underbrace{\difffrac{}{t}c_s}_{=\dot
        c}\right|_{s=0}\\
    & = -R(Y,\dot c)\dot c
  \end{align*}
  Es sei umgekehrt $J$ ein Jacobifeld entlang $c$, $\gamma$ die durch
  $\gamma(0) = c(0), \ \dot \gamma(0) = J(0)$ definierte Geodätische,
  sowie $V$ und $W$ die entlang $\gamma$ parallelen Vektorfelder mit
  $V(0) = \dot c(0)$ und $W(0) = J'(0)$. Dann ist
  \begin{align*}
    c_s(t) = \exp_{\gamma(s)}(t(V(s) + sW(s)))
  \end{align*}
  eine Variation von $c$ und alle Kurven $c_s$ sind Geodätische.

  Das zugehörige Variationsfeld $Y = \difffrac[t=0]{}{s} c_s$ ist nach
  dem oberen Bemerkungen ein Jacobifeld. Es gilt
  \begin{align*}
    Y(0) = \difffrac[s=0]{}{s}\exp_{\gamma(s)}(0) =
    \difffrac[s=0]{}{s}\gamma(s) = J(0).
  \end{align*}
  und 
  \begin{align*}
    Y'(0) & = \left.\nabla_t\difffrac[s=0]{}{s}c_s\right|_{t=0}\\
    & = \nabla_s\difffrac[t=0]{}{t}\left.\exp_{\gamma(s)}(t(V(s) +
      sW(s))\right|_{s=0}\\
    & = \left.\nabla_s(V(s) + sW(s))\right|_{s=0}\\
    & = V'(0) + W(0) + 0W'(0)\\
    & = W(0) = J'(0)
  \end{align*}
  Nach Lemma \ref{thm:lemma-9-4} stimmen $J$ und $V$ überein.
\end{bew}

\begin{emptythm}[Erinnerung]
  Korollar \ref{thm:kor-8-12}(iii).
  \begin{center}
    \textcolor{red}{Abbildung 2 <2013-1-15>}
  \end{center}
\end{emptythm}

% Definition 9.6
\begin{Dfn}
  Ein Punkt $p \in M$ heißt zu $q$ \CmMark{konjugiert}, wenn $q$ ein singulärer
  Wert von $\exp_p$ ist.

  $p$ heißt \CmMark[konjugiert!entlang einer Geodätischen]{konjugiert
  entlang der Geodätischen} $c$, wenn $\exp_{p*\dot c(0)}$ singulär
  ist, d.h. $\Kern \exp_{p*\dot c(0)} \neq \{0\}$.
\end{Dfn}

% Proposition 9.7
\begin{Prop}
  Ein Punkt $p$ ist gnau dann konjugiert zu $q$ entlang einer
  Geodätischen $c$, wenn es ein nichttriviales Jacobifeld entlang $c$
  gibt, welches in den Endpunkten verschwindet.
\end{Prop}

\begin{center}
  \textcolor{red}{Abbildung 3 <2013-1-15>: Variationsfeld aus
    Geodätischen auf der Sphäre für das Übungsblatt}
\end{center}

%%% Local Variables: 
%%% mode: latex
%%% TeX-master: "../skript-diffgeom"
%%% End: 
