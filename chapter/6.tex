\chapter{Riemannsche Metriken}

\begin{emptythm}[Was ist Geometrie?]
Vereinfacht ausgedr"uckt suchen wir eine M"oglichkeit um Distanzen und Winkel auszudrücken. Betrachte im Folgenden die Einheitssph"are, auf der wir den eine Reise von $x$ nach $y$ unternehmen m"ochten.
\begin{center}\begin{tikzpicture}
	%\draw[step=0.25,gray!15] (-4,-4) grid (4,4); \draw[step=0.5,gray!30] (-4,-4) grid (4,4); \fill (0,0) circle(0.1); %Hilfsgitter
	
	\draw (0,0) circle (3);
	\begin{scope}
		\clip (-3,0) rectangle (3,1.6);
		\draw[dashed] (0,0) ellipse (3 and 1);
	\end{scope}
	\begin{scope}
		\clip (-3,0) rectangle (3,-1.6);
		\draw (0,0) ellipse (3 and 1);
	\end{scope}
	
	\fill (0,3) circle(0.1) node[anchor=south west]{$N$};
	\fill (0,-3) circle(0.1) node[anchor=north west]{$S$};
	
	\coordinate (x) at (-0.5,2); \coordinate (y) at (0.75,-2);
	\fill (x) circle(0.1)node[above]{$x$};
	\fill (y) circle(0.1)node[below]{$y$};
	\draw (x) -- (y);
	\coordinate (a) at (1.25,0.75); \coordinate (b) at (1,0.25); \coordinate (c) at (1.5,-0.25);
	\draw (x) ..controls(x) and ($(a) + (-1.25,1.25)$).. (a) ..controls($(a) + (0.25,-0.25)$) and ($(b) + (0,0.25)$).. (b) ..controls($(b) + (0,-0.25)$) and ($(c) + (0,0.25)$).. (c) ..controls($(c) + (0,-0.75)$) and (y).. (y);
	
	\node at (2.75,2.75) {$S^2 \subset \R^3$};
	\draw[->] (1.5,-2)node[anchor=north]{$c$} to[out=90,in=310] (1.25,-1.25);
	
	\node at (-3.25,-2.5) {$c: [0,1] \to S^2$};
	\node at (-3.25,-3.25) {$c(0) = x,\quad c(1) = y$};
\end{tikzpicture}\end{center}
Wir definieren mit $\calL(c) = \int_0^1 \|\dot c\| \dop t$ die \CmMark[Metrik!Riemann-]{Riemann-Metrik}, also das Skalarprodukt mit allen $\T_pM$. Damit folgt dass wenn $c: [0,1] \to M$ glatt ist, dass $\calL(c) = \int_0^1 \sqrt{\langle \dot c, \dot c \rangle}$ und der Abstand auf $M$ kann ausgedr"uckt werden durch $d_M(x,y) = \inf \{\calL(c) | c$ von $x$ nach $y\}$.

Das wirft Fragen auf nach der Existenz k"urzester Abst"ande, Unterschieden zwischen lokal K"urzestem und global K"urzesten und der Eindeutigkeit.
\end{emptythm}

\begin{Dfn}
Es sei $M$ eine glatte Mannigfaltigkeit. Eine \CmMark[Metrik!Riemann-]{Riemannsche Metrik} $g$ auf $M$ ist gegeben durch ein Skalarprodukt auf jedem $\T_pM$, welches glatt von $p$ abh"angt, das hei"st $g \in \calT_2^0(M)$, so dass $g_p = \langle \cdot, \cdot \rangle_p : \T_pM \X \T_pM \ \to \R$ symmetrisch und positiv definit ist. Ist $g$ eine Riemann-Metrik auf M, so hei"st $(M,g)$ eine \CmMark[Mannigfaltigkeit!Riemannsche]{Riemannsche Mannigfaltigkeit}.
\end{Dfn}

Ist $(M,g)$ eine Riemannsche Mannigfaltigkeit, $X, Y \in \calV(M)$, $X = \sum \xi^{i} \pdifffrac{}{x^{i}}$, $Y = \sum \eta^j \pdifffrac{}{y^j}$, dann ist
\begin{align*}
	g(X, Y) &= g\left(\sum \xi^{i} \pdifffrac{}{x^{i}}, \sum \eta^j \pdifffrac{}{y^j}\right)\\
	&= \sum_{i,j}\xi^{i} \eta^j g\left(\pdifffrac{}{x^{i}}, \pdifffrac{}{y^j}\right)\\
	&= \sum_{i,j} \xi^{i} \eta^j g_{ij} \qquad (g_{ij} \text{ glatt, } g_{ij} = g_{ji})
\end{align*}

\begin{bsp}\begin{enumerate}[label=\arabic*),leftmargin=*]
\item
	$\R^m$ tr"agt eine nat"urliche Riemannsche Metrik: F"ur $x \in \R^m$ ist $\calI_x: \T_x\R^m \to \R^n$ ein (nat"urlicher) Isomorphismus. Damit definiert
		\[ g_x(\cdot,\cdot) = \langle \calI_x(\cdot,\cdot), \calI_x(\cdot,\cdot) \rangle \]
	eine Riemannsche Metrik auf $\R^m$. Bez"uglich der Karte $(\Id, \R^m)$ gilt
		\[ g_{ij} = \sum_{ij} \delta_{ij} \dop x^{i} \otimes \dop x^j = \sum_i \dop x^{i} \otimes \dop x^j \]
\item
	Betrachtet man Polarkoordinaten auf $\R^2(r, \vartheta)$:
	\begin{align*}
		\pdifffrac[(r,\vartheta)]{}{r} &= (\cos \vartheta, \sin \vartheta)\\
		\pdifffrac[(r,\vartheta)]{}{\vartheta} &= r(-\sin \vartheta, \cos \vartheta)
	\end{align*}
	\begin{align*}
		g_{rr} &= g\left( \pdifffrac{}{r}, \pdifffrac{}{r} \right) = 1\\
		g_{\vartheta\vartheta} &= g\left( \pdifffrac{}{\vartheta}, \pdifffrac{}{\vartheta} \right) = r^2\\
		g_{r\vartheta} &= g_{\vartheta r} = 0
	\end{align*}
\item
	Sei $M \subseteq \R^n$ $m$-dimensionale glatte Untermannigfaltigkeit. $M$ tr"agt eine nat"urliche Riemann-Metrik:
	\begin{center}\begin{tikzpicture}
		%\draw[step=0.25,gray!15] (-4,-4) grid (4,4); \draw[step=0.5,gray!30] (-4,-4) grid (4,4); \fill (0,0) circle(0.1); %Hilfsgitter
		
		\draw (0,0) circle(2);
		\begin{scope}
			\clip (-2,0) rectangle (2,2);
			\draw[dashed] (0,0) ellipse(2 and 0.75);
		\end{scope}
		\begin{scope}
			\clip (-2,0) rectangle (2,-2);
			\draw (0,0) ellipse(2 and 0.75);
		\end{scope}
		
		\path[name path=laenge] (0,2) to[out=335,in=33] (0,-2);
		\path[name path=breite] (0,2) ellipse(2 and 0.75);
		\path[name path=aequator] (0,0) ellipse(2 and 0.75);
		\path[name intersections={of=laenge and breite, by=p}];
		\path[name intersections={of=laenge and aequator, by={grenze1, grenze2}}];
		
		\coordinate (a) at (1,2.25); \coordinate (b) at (-1,0.5); \coordinate (c) at (1.75,-1.5); \coordinate (d) at ($(a) + (c) - (b)$);
		\path[draw,name path=raute] (a) -- (b) -- (c) -- (d) -- cycle;
		
		\begin{scope}
			\clip (0,2) rectangle ($(grenze2) + (0.5,0)$);
			\draw (0,2) to[out=335,in=33] (0,-2);
			\clip (0,0) circle(2);
			\draw (0,2) ellipse(2 and 0.75);
		\end{scope}
		\begin{scope}
			\clip (0,0) circle(2);
			\draw (0,2) ellipse(2 and 0.75);
		\end{scope}
		
		\node at (-2,2) {$S^2 \subset \R^3$};
		\fill (p) circle(0.05)node[anchor=south west,font=\scriptsize] at (p) {$p$};
		\draw[->] (p) -- ($(p) + 0.6*(1,-1.7)$);
		\draw[->] (p) -- ($(p) + 1.1*(1,0.15)$);
		
		\draw[->] (2.25,2.25)node[right,font=\scriptsize]{$\pdifffrac{}{\varphi}$} to[out=210,in=80] (1.65,1.5);
		\draw[->] (3.25,1.25)node[right,font=\scriptsize]{$\pdifffrac{}{\vartheta}$} to[out=180,in=30] (1.2, 0.75);
	\end{tikzpicture}\end{center}
	F"ur jedes $p \in M$ ist $\T_pM$ kanonisch isomorph zum von partiellen Ableitungen $\partial_1F|_p, \ldots ,\partial_mF|_p$ einer lokalen Parametrisierung $F$ aufgespannten Untervektorraum $\R^m$. Mit diesem (lokalen) isomorphismus definiert
		\[ g_{ij} = \langle \partial_i F, \partial_j F \rangle \]
	eine Riemann-Metrik auf $M$.
\end{enumerate}\end{bsp}

\begin{bem}
Sind $\varphi$ und $\psi$ Karten einer Riemannschen Mannigfaltigkeit $(M,g)$ um $p$ und sind $g = \sum g_{ij} \dop x^{i} \otimes \dop x^j$ und $h = \sum h_{ij} \dop y^{i} \otimes \dop y^j$ die lokalen Darstellungen bez"uglich $\varphi$ beziehungsweise $\psi$, so gilt
	\[ h_{kl} = g\left( \pdifffrac{}{y^k}, \pdifffrac{}{y^l} \right) = \sum_{i,j} \pdifffrac{x^{i}}{y^k} \underbrace{\pdifffrac{x^j}{y^l}}_{\mathclap{\qquad \partial_l(\varphi^{i}\circ \psi^{-1})}} g_{ij} \]
\end{bem}

Eine Riemannsche Metrik induziert eine Metrik auf dem Kotangentialb"undel: Die Isomorphismen $\T_pM \to \T_p^*M$, $X \mapsto \langle X, \cdot \rangle_p$ einen Isomorphismus von $\T M$ nach $\T^*M$. F"ur $\omega \in \T_p^*M$ sei $X(\omega) \in \T_pM$ mit $\omega = \langle X(\omega), \cdot \rangle_p$. Man definiert nun durch
	\[ \langle \omega, \tilde \omega \rangle = \langle X(\omega), X(\tilde \omega) \rangle \]
ein Skalarprodukt auf $\T_p^*M$. F"ur $\omega = \sum \omega_i \dop x^{i}$, $X(\omega) = \xi^{i} \pdifffrac{}{x^{i}}$ gilt
	\[ \omega_i = \omega \left( \pdifffrac{}{x^{i}} \right) = \left\langle X(\omega), \pdifffrac{}{x^{i}} \right\rangle = \sum_j \xi^{i} g_{ij} \]
Also $\xi^{i} = \sum g^{ij} \omega_i$, wobei $(g^{ij})$ die zu $(g_{ij})$ inverse Matrix ist. Damit gilt:
\begin{align*}
	\langle \omega, \tilde \omega \rangle &= \langle X(\omega), X(\tilde \omega) \rangle \\
	&= \sum g_{kl} \xi^k \xi^l\\
	&= \sum g_{kl} g^{ki} \omega_i g^{lj} \tilde \omega_j\\
	&= \sum \delta_l^i g^{lj} \omega_i \tilde \omega_j\\
	&= \sum g^{ij} \omega_i \tilde \omega_j\\
\end{align*}