%% 
%% 2. Vorlesung <2012-10-19 Fri>, Fortsetung
%% 
%% Skript Differentialgeometrie im Wintersemester 12/13
%% Zur Vorlesung von Dr. Grensing am KIT Karlsruhe
%% 
%% Mitschrieb und Textsatz von Jan-Bernhard Kordaß.
%% 

\chapter{Tangentialvektoren und Tangentialräume}

% Abbildung 2-1
%\CmMarginSvg{2-1-tangentialvektoren-motivation}{3.5cm}
\marginnote{\begin{tikzpicture}[font=\scriptsize,scale = 0.5]
	%draw[step=0.25,gray!15] (-3,-3) grid (3,3); \draw[step=0.5,gray!30] (-3,-3) grid (3,3); \fill (0,0) circle(0.1); %Hilfsgitter
	% die Kurve
	\draw (0,0) circle(2); \draw (0,0) ellipse(2 and 0.9);
	\def\a{1}
	\def\b{1}
	\def\c{-0.8}
	\def\d{1}
	\coordinate (1) at (-1.5,1); \coordinate (2) at (0.5,1.25); \coordinate (3) at (0.5,-1.5);
	\coordinate (ctrl1) at (0.5,0.5); \coordinate (ctrl2) at (-0.75,0.5); \coordinate (ctrl3) at (0.75,1);% \tikzrichtung[\a]{(1)}{(ctrl1)};\tikzrichtung[\b]{(2)}{(ctrl2)};\tikzrichtung[\d]{(3)}{(ctrl3)};\tikzrichtung[\c]{(2)}{(ctrl2)};
	\draw (1) ..controls($(1) + \a*(ctrl1)$) and ($(2) + \b*(ctrl2)$).. (2) ..controls($(2) + \c*(ctrl2)$) and ($(3) + \d*(ctrl3)$).. (3) node[anchor=south west]{$c$};
	\filldraw[fill=white] (2) circle (0.1); \draw[->] (0,0) -- ($0.9*(2)$); \node[anchor=west] at (2) {$p$};
	% das Rechteck
	\coordinate (a) at (-3,2); \coordinate (b) at (0.125,-0.75); \coordinate (c) at (0.75,3); \coordinate (d) at ($(c) + (b) - (a)$);
	\draw (a) -- (c) -- (d) -- (b) -- cycle;
	% Beschriftungen
	\node at (2,3.5) {$\{p\}^\perp = E_p \cong \R^2$};
	\node at (-1.5,-2.5) {$S^2 \subset \R^2$};
\end{tikzpicture}}


Betrachte in der nebenstehenden Abbildung eine differenzierbare \gls{Kurve} $c \colon (-\epsilon,\epsilon) \to S^2$ mit $c(0) = p$. Dann gilt:
\begin{align*}
  0 = \difffrac[t=0]{}{t} \left<c(t),c(t)\right> = 2\left<\dot c(0),c(0)\right> = 2 \left<\dot c(0),p\right> 
  \Rightarrow \dot c(0) \in p^{\perp}.
\end{align*}

% Bemerke $1 = \left<c(t),c(t)\right>$

Die Kurven heißen \CmMark{äquivalent}, wenn es eine Karte $(\phi,U)$ von $M$ und $p$ gibt, so dass gilt
\begin{align*}
  \difffrac[t=0]{}{t}(\phi \circ c_1) = \difffrac[t=0]{}{t}(\phi \circ c_2)
\end{align*}

\begin{Lemma}
  Der oben definierte Begriff der Äquivalenz ist unabhängig von der Wahl der Karte.
\end{Lemma}

\begin{bew}
  Es sei $(\psi,V)$ eine weitere Karte von $M$ um $p$. Dann gilt:
  \begin{align*}
    \difffrac[t=0]{}{t}(\psi\circ c_1) & = \difffrac[t=0]{}{t}(\psi\circ\phi^{-1}\circ\phi \circ c_1) = \D (\psi \circ \phi^{-1})|_{\phi(p)} \cdot \difffrac[t=0]{}{t}(\phi \circ c_1)\\
    & = \D(\psi \circ \phi^{-1})|_{\phi(p)} \cdot \difffrac[t=0]{}{t}(\phi \circ c_2) = \ldots = \difffrac[t=0]{}{t}(\psi \circ c_2).
  \end{align*}
\end{bew}

\begin{Dfn}[Geometrische Definition des Tangentialraums]
  Es sei $M$ eine glatte Mannigfaltigkeit und $p \in M$. Ein (geometrischer) \CmMark[Tangential!-vektor]{Tangentialvektor} an $M$ in $p$ ist eine Äquivalenzklasse von Kurven $c$ mit $c(0) = p$. Die Menge
  \begin{align*}
    \T_{p}^{\text{geo}}M = \{ [c] \mid c \colon (-\epsilon,\epsilon) \to M \text{ glatt}, c(0) = p\}
  \end{align*}
  heißt (geometrischer) \CmMark[Tangential!-raum]{Tangentialraum} an $M$ in $p$.
\end{Dfn}

\begin{bem}
  Mit den Bezeichnungen wie oben ist die folgende Abbildung bijektiv:
  \begin{align*}
    A \colon \T_p^{\text{geo}}M \to \R^n, [c] \mapsto \difffrac[t=0]{}{t}(\phi \circ c).
  \end{align*}
\end{bem}

\begin{bew}
Zu jedem Vektor $v \in \R^n$ sei $B(v) = [t \mapsto \phi^{-1}(\phi(p) + tv)]$ die Äquivalenzklasse der abgebildeten Kurve auf der Mannigfaltigkeit.
% Abbildung 2-2
%\CmPutSvg{2-2-beweis-bijektivitaet-tpm-rn}{10cm}
\begin{center}\begin{tikzpicture}[font=\scriptsize]
	%\draw[step=0.25,gray!15] (-6,-4) grid (6,4); \draw[step=0.5,gray!30] (-6,-4) grid (6,4); \fill (0,0) circle(0.1); %Hilfsgitter
	
	\draw[->] (1.5,0) -- (5,0); \draw[->] (3,-1.5) -- (3,1.75); \draw (3,0) circle (1); \draw[->] ($(3,0) - (1.25,0.75)$) --node[above,xshift=0.25cm,yshift=0.25cm]{$v$} ($(3,0) + 1.2*(1.25,0.75)$); \filldraw[fill=white] (3,0) circle(0.05);
	\node at (3,-2) {$\Bild(t \mapsto \phi(p) + vt)$}; \draw[dashed] (3.5,-1.75) to[out=70,in=290] (3.5,0.25); \node at (4,1.5) {$\phi(p) = 0$};
	
	\coordinate (0) at (-3.25,2); \coordinate (1) at (-4.5,0.25); \coordinate (2) at (-2.5,-2); \coordinate (3) at (-0.25,0.5); \coordinate (4) at (0,1.25); \coordinate (5) at (-0.25,1.75);
	\coordinate (ctrl1) at (0,1); \coordinate (ctrl2) at (1,0); \coordinate (ctrl3) at (1,1); \coordinate (ctrl4) at (ctrl1);
	
	\draw (0) ..controls(0) and ($(1) + 1.25*(ctrl1)$).. (1) ..controls($(1) - (ctrl1)$) and ($(2) - (ctrl2)$).. (2) ..controls($(2) + 1.5*(ctrl2)$) and ($(3) - 1.25*(ctrl3)$).. (3) ..controls($(3) + 0.25*(ctrl3)$) and ($(4) - 0.25*(ctrl4)$).. (4) ..controls($(4) + 0.25*(ctrl4)$) and (5).. (5);
	
	\coordinate (cntr) at (-2,1.25);
	\begin{scope}
		\clip ($(cntr)-(1,0.6)$) rectangle ($(cntr)+(1,-0.1)$);
		\draw[name path=l] (cntr) ellipse(1 and 0.5);
	\end{scope}
	\path[name path=u] ($(cntr) - (0,0.5)$) ellipse(0.75 and 0.5);
	\path[name intersections={of=u and l}];
	\begin{scope}
		\clip (intersection-1) rectangle ($(intersection-2)+(0,0.5)$);
		\draw ($(cntr) - (0,0.5)$) ellipse(0.75 and 0.5);
	\end{scope}
	
	\coordinate (a) at (-4,0.75); \coordinate (b) at (-4.25,0); \coordinate (c) at (-3.75,-0.25); \coordinate (d) at (-3.25,-0.75); \coordinate (e) at (-2.75,-0.75); \coordinate (f) at (-1,0);
	\coordinate (ctrlb) at (0.5,-1); \coordinate (ctrlc) at (1,-1); \coordinate (ctrld) at (1,-0.5); \coordinate (ctrle) at (1,0);
	
	\draw (a) ..controls(a) and ($(b) - 0.25*(ctrlb)$).. (b) ..controls($(b) + 0.25*(ctrlb)$) and ($(c) - 0.25*(ctrlc)$).. (c) ..controls($(c) + 0.25*(ctrlc)$) and ($(d) - 0.25*(ctrld)$).. (d) ..controls($(d) + 0.25*(ctrld)$) and ($(e) - 0.25*(ctrle)$).. (e) ..controls($(e) + 1.25*(ctrle)$) and (f).. (f);
	
	\draw (e) circle(0.75); \filldraw[fill=white] (e) circle(0.05)node[anchor=north west]{$p$}; \node at (-3.25,0) {$U$};
	\draw[dashed] (-4,-2)node[below]{$\phi^{-1}(\phi(p) + tv$} to[out=70,in=220] (-3.25,-0.75);
	
	\coordinate (cntr) at (0.25,0);
	\def\lnge{2.0}
	\def\hohe{1}
	\def\angle{20}
	\draw[->] ($(cntr) + (\lnge, -\hohe)$) to[out=180+\angle,in=360-\angle] node[below]{$\phi^{-1}$} ($(cntr) + (-\lnge, -\hohe)$);
\end{tikzpicture}\end{center}
	\[ A B(v) = \difffrac[t=0]{}{t}(\phi \circ B(v)) = \difffrac[t=0]{}{t}(\phi \circ \phi^{-1}(\phi(p) + tv)) = \difffrac[t=0]{}{t}(\phi(p) + tv) = v. \]
	\[ B A (\underbrace{[c]}_{\ni c}) = B(v_c) = [t \mapsto \phi^{-1}(\phi(p) + tv_c)] \text{ wobei } v_c = \difffrac[t=0]{}{t}(\phi \circ c). \]
Die Kurven $c$ und $t \mapsto \phi^{-1}(\phi(p) + tv_c)$ sind äquivalent, also ist $B A[c] = [c]$ und somit $A$ bijektiv.
\end{bew}

Damit erhält $\T_p^{\text{geo}}M$ die Struktur eines reellen Vektorraumes vermöge der folgenden Verknüpfung:
\begin{align*}
  \lambda[c_1] + \mu[c_2] = A^{-1}(\lambda A[c_1]+ \mu A[c_2]).
\end{align*}
Dabei gilt $\lambda[c_1]+\mu[c_2] = [c]$ für $c(t) = \phi^{-1}(\phi(p) + t(\lambda v_1 + \mu v_2))$ mit $v_i = \difffrac[t=0]{}{t}(\phi \circ c_i)$.

\begin{Lemma}
  Die oben definierte Lineare Struktur ist unabhängig von der Wahl der Karte.
\end{Lemma}

\begin{bew}
  Es sei $(\psi, V)$ eine Karte von $M$ um $p$ und $A'[c] = \difffrac[t]{}{t}(\psi \circ c)$. Dann gilt:
  \begin{align*}
    A A'^{-1}(v) & = \difffrac[t=0]{}{t}(\phi \circ (\psi^{-1} (\psi(p) + tv)))\\
    & = \D(\phi \circ \psi^{-1})|_{\psi(p)} \cdot \difffrac[t=0]{}{t}(\psi \circ \psi^{-1}(\phi(p) + tv)) = \D (\phi \circ \phi^{-1}) \cdot v.
  \end{align*}
  Also ist $A A'^{-1}$ linear,
  \begin{align*}
    A'^{-1}(\lambda A'[c_1] + \mu A'[c_2]) & = A^{-1}(A A'^{-1}(\lambda A'[c_1] + \mu A'[c_2]))\\
    & = A^{-1} (\lambda A A'^{-1}[c_1] + \mu A A'^{-1} [c_2])\\
    & = A^{-1}(\lambda A [c_1] + \mu A [c_2]).
  \end{align*}
\end{bew}


% 3. Vorlesung <2012-10-23 Tue>

\paragraph{Motivation: Richtungsableitungen im $\R^n$}\hfill
\begin{bem}
Für $f,g \in C^{\infty}(\R^n), \ x,y \in \R^n$ ist die \CmMark{Richtungsableitung} wie folgt definiert:
\begin{align*}
	\partial_vf(x) = \D f(x) \cdot v = \difffrac[t=0]{}{t}f(x+tv).
\end{align*}
Diese erfüllt die Leibniz-Regel:
\begin{align*}
	\partial_v(fg)(x) = \partial_vf(x)\cdot g(x) + f(x) \cdot \partial_v(g)(x).
\end{align*}
\end{bem}

\begin{Dfn}[Algebraische Definition des Tangentialraumes]
  Es sei $M$ eine glatte Mannigfaltigkeit und $p\in M$. Ein (algebraischer) \CmMark[Tangential!-vektor]{Tangentialvektor} an $M$ in $p$ ist eine Lineare Abbildung $X_p \colon C^{\infty}(M) \to \R$, welche die Leibniz-Regel erfüllt:
  \begin{align*}
    X_p(fg) = X_p(f) \cdot g(p) + f(p) \cdot X_p(g).
  \end{align*}

  Die algebraischen Tangentialvektoren bilden einen reellen Vektorraum $\T_p^{\text{alg}}M$, den Tangentialraum an $M$ in $p$.
\end{Dfn}

\begin{Lemma}\label{lemma-2-5}
  Es sei U eine Umgebung von $p \in M$. Dann existiert eine Umgebung $V \subset U$ von $p$ und eine glatte reellwertige Funktion $\sigma \in C^{\infty}(M)$ mit den Eigenschaften $\sigma|_V = 1$ und $\gls{supp}(\sigma) \subset U$.
\end{Lemma}

%%%
%%% Abbildung 2-3
%%% 
\textcolor{red}{Abbildung 2-3}


\begin{bew}
  Man kann o.E. annehmen, dass $U$ Kartengebiet einer Karte $\phi$ von $M$ um $p$ ist und $\phi(p) = 0 \in \R^n$.\\

  Es sei $\epsilon > 0$ so, dass $\overline B_c(0) \subset \phi(U)$. \\

  %%% 
  %%% Abbildung 2-4
  %%% 
  \textcolor{red}{Abbildung 2-4}

  Ist dann $\eta$ eine glatte Funktion auf $\R$ mit $\eta \equiv 1$ auf $\left[\frac{-\epsilon^{2}}{2},\frac{\epsilon^2}{2}\right]$ und $\eta \equiv 0$ auf $\R \setminus (-\epsilon^2,\epsilon^2)$, so hat für $U_1 = \phi^{-1}(B_{\frac{\epsilon}{2}}(0))$ die Funktion
  \begin{align*}
    \sigma(q) =
    \begin{cases}
      \eta(\|\phi(q)\|^2) & \text{ für } q \in U_1\\
      0 & \text{ sonst }
    \end{cases}.
  \end{align*}
  die gewünschten Eigenschaften.
\end{bew}

% Lemma 2.
\begin{Lemma}
Für alle $X_p\in\T_p^{\text{alg}}M$ gilt:
\begin{enumerate}[label=(\roman*),widest=ii]
\item $X_p(f) = 0$ falls $f$ in einer Umgebung von $p$ konstant ist.
\item $X_p(f) = X_p(g)$ falls $f$ und $g$ auf einer Umgebung übereinstimmen.
\end{enumerate}
\end{Lemma}

\begin{bew}\begin{enumerate}[label=(\roman*),widest=ii,leftmargin=*]
\item[(ii)]
	Es sei $U$ eine Umgebung von $p$ mit $f|_U = g|_U$. Ist dann $\sigma$ wie in Lemma \ref{lemma-2-5}, so gilt $\sigma f = \sigma g$ und aus
	\begin{align*}
		X_p(\sigma)f(p)+\sigma(p)X_p(f) = X_{p}(\sigma f) = X_p(\sigma g) = X_p(\sigma) g(p) + \sigma(p) X_p(g)
	\end{align*}
	folgt $X_p(f) = X_p(g)$.\\
\item[(i)]
	Wegen der $\R$-Linearität und (ii) genügt es $f \equiv 1$ zu betrachten. Es gilt
	\begin{align*}
		X_p(1) = X_p(1 \cdot 1) = X_p(1) \cdot 1 + 1 \cdot X_p(1) = 2 \cdot X_p(1),
	\end{align*}
	also $X_p(1) = 0$.
\end{enumerate}\end{bew}

\begin{bem}
  Also gilt für $f \in C^{\infty}(M)$ und $g \in C^{\infty}(U)$ direkt:
  \begin{align*}
    & \sigma g =
    \begin{cases}
      \sigma g|_U & \textcolor{red}{\sigma g \in C^{\infty}(M)}\\
      0 & \text{ sonst }
    \end{cases},\\
    & \sigma g \in C^{\infty}(M) 
    \Rightarrow X_p(g) = X_p(\sigma g).
  \end{align*}
  Für eine Karte $\phi \colon U \to V$ von $M$ und $p$ seien algebraische Tangentialvektoren definiert:
  \begin{align*}
    \pdifffrac[p]{}{x^i} \in \T_p^{\text{alg}}M, \pdifffrac[p]{}{x^i}(f) = \partial_i(f \circ \phi^{-1})(\phi(p)) = \D(f \circ \phi^{-1})|_{\phi(p)}e_i.
  \end{align*}
\end{bem}

% Satz 2.7
\begin{Satz}\label{satz-2-7}
  Die Vektoren $\pdifffrac[p]{}{x^1},\ldots,\pdifffrac[p]{}{x^n}$ bilden eine Basis von $T_p^{\text{alg}}M$.
\end{Satz}

% Lemma 2.8
\begin{Lemma}
  Es sei $x_0 \in \R^n$ und $g \in C^{\infty}(B_{\rho}(x_0))$.
  Dann existieren glatte Funktionen $h_i \in C^{\infty}(B_{\rho}(x_0))$ mit $h_i(x_0) = \partial_ig(x_0)$ und 
  \begin{align*}
    g(x) = g(x_0) + \sum_{i=1}^n(x^i-x_0^i)h_i(x).
  \end{align*}
\end{Lemma}

\begin{bew}[Beweis des Satzes]
Die $j$-te Komponente $\phi^j$ der Karte ist glatt und es gilt:
\begin{align*}
	\pdifffrac[p]{}{x^i}(\phi^j) = \partial_i(\phi^j \circ \phi^i)(\phi(p)) = \partial_ix^j(\phi(p)) = \delta_i^j.
\end{align*}
Damit sind die Vektoren linear unabhängig.

Es sei $X_p\in \T_p^{\text{alg}}M$ und $f \in C^{\infty}(M)$.
Für $x_0=\phi(p) \in \R^n, \ B_{\rho}(x_0) \subset \phi(U)$ und für $g = f \circ \phi^{-1}|_{B_{\rho}(x_0)}$ gilt mit den Bezeichnungen wie im letzten Lemma:
\begin{align*}
	X_p(f) & = X_p(g \circ \phi) = X_p(g(\phi(p)) + \sum \left(\phi^i - \phi(p)^i)(h_i \circ \phi) \right)\\
	& = \underbrace{X_p(g(\phi(p)))}_{\mathclap{=0}} + \sum X_p((\phi^i-\phi(p)^i)(h_i \circ \phi))\\
	& = \sum X_p(\phi^i)(h_i\circ\phi)(p) - X_p(\phi(p)^i)(h_i\circ \phi)(p) + \sum (q^i-\phi(p)^i)(p) X_p(h_i \circ \phi)\\
	& = \sum_{i=1}^n X_p(\phi^i)\underbrace{(h_i \circ \phi)(p)}_{\mathclap{=h_i(\phi(p) = h_i(x_0) = \partial_ig(x_0) = \partial_i(f\circ \phi^{-1})(\phi(p)) = \pdifffrac[p]{}{x^i}(f)}}\\
	& = \sum_{i=1}^nX_p(\phi^i)\pdifffrac[p]{}{x^i}(f).
\end{align*}
\end{bew}

\begin{bem}
  Ist $X_p=\sum \xi^i\pdifffrac[p]{}{x^i}$, so gilt $\xi^i = X_p(\phi^i)$.
\end{bem}

\begin{bew}[Beweis des Lemmas]
Es gilt:
\begin{align*}
	g(x) - g(x_0) = \int_0^1\difffrac{}{t}g(tx + (1-t)x_0)dt = \sum_{i=1}^n(x^i-x_0^i)\underbrace{\int_0^1\partial_ig(tx + (1-t)x_0) dt}_{=: h_i(x)}.
\end{align*}
\end{bew}

% Satz 2.9
\begin{Satz}[Äquivalenz der Tangentialraumbegriffe]\label{satz-2-9}
  Die Abbildung
  \begin{align*}
    J_p \colon \T_p^{\text{geo}}M \to \T_p^{\text{alg}}M, \ J_{p}[c](f) = \difffrac[t=0]{}{t}(f\circ c)
  \end{align*}
  ist ein linearer \gls{Isomorphismus} \quot{$c(0)(f)$}.
\end{Satz}

\begin{bew}
  Wegen
  \begin{align*}
    J_p[c](f)& = \difffrac[t=0]{}{t}(f\circ c) = \difffrac[t=0]{}{t}(f \circ \phi^{-1} \circ \phi \circ c)\\
    &  = \D(f \circ \phi^{-1})|_{\phi(p)} \difffrac[t=0]{}{t} (\phi \circ c) = \D (f\circ \phi^{-1})|_{\phi(p)}A[c]
  \end{align*}
  ist $J_p = \D(\cdot)\circ A$ linear.

  Ist $[c] \in \Kern J_p$, so folgt aus $0 = J_p[c](\phi^i) = \difffrac[t=0]{}{t}(\phi^i \circ c)$, dass $\difffrac[t=0]{}{t}(\phi \circ c) = 0$ gilt, also $[c] = 0$. Damit ist $J_p$ injektiv, also ein Isomorphismus.
\end{bew}

\begin{bem}
  \begin{enumerate}[label=\arabic*)]
  \item Ist $X_p = \sum \xi^i\pdifffrac[p]{}{x^i}$, so gilt $X_p = \dot c(0)$ für $c(t) = \phi^{-1}(\phi(p) + t\xi)$.
\item Für jede glatte Kurve $c$ durch $p$ ist $\difffrac[t=0]{}{t}(\phi \circ c)$ der Koeffizientenvektor von $\dot c(0)$ in der Basis $\pdifffrac[p]{}{x^i}$.
  \end{enumerate}
\end{bem}


% Satz 2.10
\begin{Satz}[Transformationsverhalten bei Kartenwechsel]\label{satz-2-10}
  Es seien $\phi$ und $\psi$ Karten in $M$ um $p$ und es bezeichnen $\pdifffrac[p]{}{x^i}$ und $\pdifffrac[p]{}{y^i}$ die damit assoziierten Basen von $\T_pM$. Dann gilt
  \begin{align*}
    \pdifffrac[p]{}{x^i} = \sum_j \partial_i(\psi^j \circ \phi^{-1})(\phi(p)) \pdifffrac[p]{}{y^j}.
  \end{align*}
Es sei $X_p = \sum \xi^i \pdifffrac[p]{}{x^i} = \sum \eta^i\pdifffrac[p]{}{y^i}$. Dann gilt:
\begin{align*}
  \eta^j = \sum \partial_i(\psi^j \circ \phi^{-1})(\phi(p))\xi^i \text{ bzw. }
  \eta = \D(\psi \circ \phi^{-1})(\phi(p))\xi.
\end{align*}
\end{Satz}

\begin{bew}
  Es gelte $\pdifffrac[p]{}{x^i} = \sum \alpha_i^j\pdifffrac[p]{}{y^j}$ und nach obiger Bemerkung zum vorletzten Satz gilt:
  \begin{align*}
    \alpha_i^j = \pdifffrac[p]{}{x^i}(\psi^j) = \partial_i(\psi^j \circ \phi^{-1})(\phi(p))
  \end{align*}
\end{bew}

%%% Local Variables: 
%%% mode: latex
%%% TeX-master: "../skript-diffgeom"
%%% End: 

