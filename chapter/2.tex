%%
%% 2. Vorlesung 19.10.12, Fortsetung
%% 
%% Skript Differentialgeometrie im Wintersemester 12/13
%% Zur Vorlesung von Dr. Grensing am KIT Karlsruhe
%%
%% Mitschrieb und Textsatz von Jan-Bernhard Kordaß.
%%

\section{Tangentialvektoren und Tangentialräume}

% Abbildung 2-1
\CmMarginSvg{2-1-tangentialvektoren-motivation}{3.5cm}

Betrachte in der nebenstehenden Abbildung eine differenzierbare Kurve $c \colon (-\varepsilon,\varepsilon) \to S^2$ mit $c(0) = p$. Dann gilt:
\begin{align*}
  0 = \difffrac{t} \left<c(t),c(t)\right> = 2\left<c(0),c(0)\right> = 2 \left<c(0),p\right> 
  \Rightarrow c(0) \in p^{\perp}.
\end{align*}

% Bemerke $1 = \left<c(t),c(t)\right>$

Es sei $M$ eine glatte Mannigfaltigkeit und es seien glatte Kurven $c_i\colon (-\varepsilon_i,\varepsilon_i) \to M$ mit $c_1(0) = c_2(0) = p \in M$ gegeben.\\

Die Kurven heißen \CmMark{äquivalent}, wenn es eine Karte $(\varphi,U)$ von $M$ und $p$ gibt, so dass 
\begin{align*}
  \difffrac{t}|_{t=0}(\varphi \circ c_1) = \difffrac{t}|_{t=0}(\varphi \circ c_2)
\end{align*}
gilt.

\begin{lemma}
  Der oben definierte Begriff der Äquivalenz ist unabhängig von der Wahl der Karte.
\end{lemma}

\begin{proof}
  Es sei $(\psi,V)$ eine weitere Karte von $M$ um $p$. Dann gilt:
  \begin{align*}
    \difffrac{t}|_{t=0}(\psi\circ c_1) & = \difffrac{t}|_{t=0}(\psi\circ\varphi^{-1}\circ\varphi \circ c_1) = \D (\psi \circ \varphi^{-1})|_{\varphi(p)} \cdot \difffrac{t}|_{t=0}(\varphi \circ c_1\\
    & = \D(\psi \circ \varphi^{-1})|_{\varphi(p)} \cdot \difffrac{t}|_{t=0}(\varphi \circ c_2) ) = \ldots = \difffrac{t}|_{t=0}(\psi \circ c_2).
  \end{align*}
\end{proof}

\begin{dfn}[Geometrische Definition des Tangentialraums]
  Es sei $M$ eine glatte Mannigfaltigkeit und $p \in M$. Ein (geometrischer) \CmMark{Tangentialvektor} an $M$ in $p$ ist eine Äquivalenzklasse von Kurven $c$ mit $c(0) = p$. Die Menge
  \begin{align*}
    \T_{p}^{\text{geo}}M = \{ [c] \mid c \colon (-\varepsilon,\varepsilon) \to M \text{ glatt}, c(0) = p\}
  \end{align*}
  heißt (geometrischer) \CmMark{Tangentialraum} an $M$ in $p$.
\end{dfn}

\begin{bem}
  Mit den Bezeichnungen wie oben ist die folgende Abbildung bijektiv:
  \begin{align*}
    A \colon \T_p^{\text{geo}}M \to \R^n, [c] \mapsto \difffrac{t}|_{t=0}(\varphi \circ c).
  \end{align*}
\end{bem}

\begin{proof}
  Zu $v \in \R^n$ sei $B(v) = [t \mapsto \varphi^{-1}(\varphi(p) + tv)]$ - die Äquivalenzklasse der abgebildeten Kurve auf der Mannigfaltigkeit.

  % Abbildung 2-2
  \CmPutSvg{2-2-beweis-bijektivitaet-tpm-rn}{10cm}

  \begin{align*}
    & A B(v) = \difffrac{t}|_{t=0}(\varphi \circ B(v)) = \difffrac{t}|_{t=0}(\varphi \circ (\varphi^{-1}(\varphi(p) + tv)) = \difffrac{t}|_{t=0}(\varphi(p) + tv) = v.\\
    & v_c = \difffrac{t}|_{t=0}(\varphi \circ c)\\
    & B A (\underbrace{[c]}_{\ni c}) = B(v_c) = [t \mapsto \varphi^{-1}(\varphi(p + tv_c)].
  \end{align*}
  Die Kurven $c$ und $t \mapsto \varphi^{-1}(\varphi(p) + tv_c)$ sind äquivalent, also ist $B A[c] = [c]$ und somit $A$ bijektiv.
\end{proof}

Damit erhält $\T_p^{\text{geo}}M$ die Struktur eines reellen Vektorraumes vermöge der folgenden Verknüpfung:
\begin{align*}
  \lambda[c_1] + \mu[c_2] = A^{-1}(\lambda A[c_1]+ \mu A[c_2]).
\end{align*}
Dabei gilt $\lambda[c_1]+\mu[c_2] = [c]$ für $c(t) = \varphi^{-1}(\varphi(p) + t(\lambda v_1 + \mu v_2))$ mit $v_i = \difffrac{t}|_{t=0}(\varphi \circ c_i)$.

\begin{lemma}
  Die oben definierte Lineare Struktur ist unabhängig von der Wahl der Karte.
\end{lemma}

\begin{proof}
  Es sei $(\psi, V)$ eine Karte von $M$ um $p$ und $A'[c] = \difffrac{t}|_t(\psi \circ c)$. Dann gilt:
  \begin{align*}
    A A'^{-1}(v) & = \difffrac{t}|_{t=0}(\varphi \circ (\psi^{-1} (\psi(p) + tv)))\\
    & = \D(\varphi \circ \psi^{-1})|_{\psi(p)} \cdot \difffrac{t}|_{t=0}(\psi \circ \psi^{-1}(\varphi(p) + tv)) = \D (\varphi \circ \varphi^{-1}) \cdot v.
  \end{align*}
  Also ist $A A'^{-1}$ linear,
  \begin{align*}
    A'^{-1}(\lambda A'[c_1] + \mu A'[c_2]) & = A^{-1}(A A'^{-1}(\lambda A'[c_1] + \mu A'[c_2]))\\
    & = A^{-1} (\lambda A A'^{-1}[c_1] + \mu A A'^{-1} [c_2])\\
    & = A^{-1}(\lambda A [c_1] + \mu A [c_2]).
  \end{align*}
\end{proof}

%%% Local Variables: 
%%% mode: latex
%%% TeX-master: "../skript-diffgeom"
%%% End: 

