%%
%% 2. Vorlesung <2012-10-19 Fri>, Fortsetung
%% 
%% Skript Differentialgeometrie im Wintersemester 12/13
%% Zur Vorlesung von Dr. Grensing am KIT Karlsruhe
%%
%% Mitschrieb und Textsatz von Jan-Bernhard Kordaß.
%%

\section{Tangentialvektoren und Tangentialräume}

% Abbildung 2-1
\CmMarginSvg{2-1-tangentialvektoren-motivation}{3.5cm}

Betrachte in der nebenstehenden Abbildung eine differenzierbare Kurve $c \colon (-\varepsilon,\varepsilon) \to S^2$ mit $c(0) = p$. Dann gilt:
\begin{align*}
  0 = \difffrac{t} \left<c(t),c(t)\right> = 2\left<c(0),c(0)\right> = 2 \left<c(0),p\right> 
  \Rightarrow c(0) \in p^{\perp}.
\end{align*}

% Bemerke $1 = \left<c(t),c(t)\right>$

Es sei $M$ eine glatte Mannigfaltigkeit und es seien glatte Kurven $c_i\colon (-\varepsilon_i,\varepsilon_i) \to M$ mit $c_1(0) = c_2(0) = p \in M$ gegeben.\\

Die Kurven heißen \CmMark{äquivalent}, wenn es eine Karte $(\varphi,U)$ von $M$ und $p$ gibt, so dass 
\begin{align*}
  \difffrac{t}|_{t=0}(\varphi \circ c_1) = \difffrac{t}|_{t=0}(\varphi \circ c_2)
\end{align*}
gilt.

\begin{lemma}
  Der oben definierte Begriff der Äquivalenz ist unabhängig von der Wahl der Karte.
\end{lemma}

\begin{proof}
  Es sei $(\psi,V)$ eine weitere Karte von $M$ um $p$. Dann gilt:
  \begin{align*}
    \difffrac{t}|_{t=0}(\psi\circ c_1) & = \difffrac{t}|_{t=0}(\psi\circ\varphi^{-1}\circ\varphi \circ c_1) = \D (\psi \circ \varphi^{-1})|_{\varphi(p)} \cdot \difffrac{t}|_{t=0}(\varphi \circ c_1)\\
    & = \D(\psi \circ \varphi^{-1})|_{\varphi(p)} \cdot \difffrac{t}|_{t=0}(\varphi \circ c_2) = \ldots = \difffrac{t}|_{t=0}(\psi \circ c_2).
  \end{align*}
\end{proof}

\begin{dfn}[Geometrische Definition des Tangentialraums]
  Es sei $M$ eine glatte Mannigfaltigkeit und $p \in M$. Ein (geometrischer) \CmMark{Tangentialvektor} an $M$ in $p$ ist eine Äquivalenzklasse von Kurven $c$ mit $c(0) = p$. Die Menge
  \begin{align*}
    \T_{p}^{\text{geo}}M = \{ [c] \mid c \colon (-\varepsilon,\varepsilon) \to M \text{ glatt}, c(0) = p\}
  \end{align*}
  heißt (geometrischer) \CmMark{Tangentialraum} an $M$ in $p$.
\end{dfn}

\begin{bem}
  Mit den Bezeichnungen wie oben ist die folgende Abbildung bijektiv:
  \begin{align*}
    A \colon \T_p^{\text{geo}}M \to \R^n, [c] \mapsto \difffrac{t}|_{t=0}(\varphi \circ c).
  \end{align*}
\end{bem}

\begin{proof}
  Zu $v \in \R^n$ sei $B(v) = [t \mapsto \varphi^{-1}(\varphi(p) + tv)]$ - die Äquivalenzklasse der abgebildeten Kurve auf der Mannigfaltigkeit.

  % Abbildung 2-2
  \CmPutSvg{2-2-beweis-bijektivitaet-tpm-rn}{10cm}

  \begin{align*}
    & A B(v) = \difffrac{t}|_{t=0}(\varphi \circ B(v)) = \difffrac{t}|_{t=0}(\varphi \circ (\varphi^{-1}(\varphi(p) + tv)) = \difffrac{t}|_{t=0}(\varphi(p) + tv) = v.\\
    & v_c = \difffrac{t}|_{t=0}(\varphi \circ c)\\
    & B A (\underbrace{[c]}_{\ni c}) = B(v_c) = [t \mapsto \varphi^{-1}(\varphi(p + tv_c)].
  \end{align*}
  Die Kurven $c$ und $t \mapsto \varphi^{-1}(\varphi(p) + tv_c)$ sind äquivalent, also ist $B A[c] = [c]$ und somit $A$ bijektiv.
\end{proof}

Damit erhält $\T_p^{\text{geo}}M$ die Struktur eines reellen Vektorraumes vermöge der folgenden Verknüpfung:
\begin{align*}
  \lambda[c_1] + \mu[c_2] = A^{-1}(\lambda A[c_1]+ \mu A[c_2]).
\end{align*}
Dabei gilt $\lambda[c_1]+\mu[c_2] = [c]$ für $c(t) = \varphi^{-1}(\varphi(p) + t(\lambda v_1 + \mu v_2))$ mit $v_i = \difffrac{t}|_{t=0}(\varphi \circ c_i)$.

\begin{lemma}
  Die oben definierte Lineare Struktur ist unabhängig von der Wahl der Karte.
\end{lemma}

\begin{proof}
  Es sei $(\psi, V)$ eine Karte von $M$ um $p$ und $A'[c] = \difffrac{t}|_t(\psi \circ c)$. Dann gilt:
  \begin{align*}
    A A'^{-1}(v) & = \difffrac{t}|_{t=0}(\varphi \circ (\psi^{-1} (\psi(p) + tv)))\\
    & = \D(\varphi \circ \psi^{-1})|_{\psi(p)} \cdot \difffrac{t}|_{t=0}(\psi \circ \psi^{-1}(\varphi(p) + tv)) = \D (\varphi \circ \varphi^{-1}) \cdot v.
  \end{align*}
  Also ist $A A'^{-1}$ linear,
  \begin{align*}
    A'^{-1}(\lambda A'[c_1] + \mu A'[c_2]) & = A^{-1}(A A'^{-1}(\lambda A'[c_1] + \mu A'[c_2]))\\
    & = A^{-1} (\lambda A A'^{-1}[c_1] + \mu A A'^{-1} [c_2])\\
    & = A^{-1}(\lambda A [c_1] + \mu A [c_2]).
  \end{align*}
\end{proof}


% 3. Vorlesung <2012-10-23 Tue>

\begin{bem}
  \paragraph{Motivation: Richtungsableitungen im $\R^n$}\hfill

  Für $f,g \in C^{\infty}(\R^n), \ x,y \in \R^n$ ist die Richtungsableitung wie folgt definiert:
  \begin{align*}
    \partial_vf(x) = \D f|x \cdot v = \difffrac{t}|_{t=0}f(x+tv).
  \end{align*}
  Diese erfüllt die Leibniz-Regel:
  \begin{align*}
    \partial_v(fg)(x) = \partial_vf(x)\cdot g(x) + f(x) \cdot \partial_v(g)(x).
  \end{align*}
\end{bem}

\begin{dfn}[Algebraische Definition des Tangentialraumes]
  Es sei $M$ eine glatte Mannigfaltigkeit und $p\in M$. Ein (algebraischer) \CmMark{Tangentialvektor} an $M$ in $p$ ist eine Lineare Abbildung $X_p \colon C^{\infty}(M) \to \R$, welche die Leibniz-Regel erfüllt:
  \begin{align*}
    X_p(fg) = X_p(f) \cdot g(p) + f(p) \cdot X_p(g).
  \end{align*}

  Die algebraischen Tangentialvektoren bilden einen reellen Vektorraum $\T_p^{\text{alg}}M$, den Tangentialraum an $M$ in $p$.
\end{dfn}

\begin{lemma}
  Es sei U eine Umgebung von $p \in M$. Dann existiert eine Umgebung $V \subset U$ von $p$ und eine glatte reellwertige Funktion $\sigma \in C^{\infty}(M)$ mit den Eigenschaften $\sigma|_V = 1$ und $\supp(\sigma) \subset U$.
\end{lemma}

% Abbildung 2-3


\begin{proof}
  Man kann o.E. annehmen, dass $U$ Kartengebiet einer Karte $\varphi$ von $M$ um $p$ ist und $\varphi(p) = 0 \in \R^n$.\\

  Es sei $\varepsilon > 0$ so, dass $\overline B_c(0) \subset \varphi(U)$. \\

  % Abbildung 2-4

  Ist dann $\eta$ eine glatte Funktion auf $\R$ mit $\eta \equiv 1$ auf $\left[\frac{-\varepsilon^{2}}{2},\frac{\varepsilon^2}{2}\right]$ und $\eta \equiv 0$ auf $\R \setminus (-\varepsilon^2,\varepsilon^2)$, so hat für $U_1 = \varphi^{-1}(B_{\frac{\varepsilon}{2}}(0)$ die Funktion
  \begin{align*}
    \sigma(q) =
    \begin{cases}
      \eta(\|\varphi(q)\|^2 & \text{ für } q \in U_1\\
      0 & \text{ sonst }
    \end{cases}.
  \end{align*}
  die gewünschten Eigenschaften.
\end{proof}

% Lemma 2.
\begin{lemma}
  Für alle $X_p\in\T_p^{\text{alg}}M$ gilt:
  \begin{enumerate}
  \item $X_p(f) = 0$ falls $f$ in einer Umgebung von $p$ konstant ist.
  \item $X_p(f) = X_p(g)$ falls $f$ und $g$ auf einer Umgebung übereinstimmen.
  \end{enumerate}
\end{lemma}

\begin{proof}
  ad (ii). Es sei $U$ eine Umgebung von $p$ mit $f|_U = g|_U$. Ist dann $\sigma$ wie im Beweis des vorigen Lemmas, so gilt $\sigma f = \sigma g$ und aus
  \begin{align*}
    X_p(\sigma)f(p)+\sigma(p)X_p(f) = X_{p}(\sigma f) = X_p(\sigma g) = X_p(\sigma) g(p) + \sigma(p) X_p(g)
  \end{align*}
  folgt $X_p(f) = X_p(g)$.\\
  ad (i). Wegen der $\R$-Linearität und (ii) genügt es $f \equiv 1$ zu betrachten. Es gilt
  \begin{align*}
    X_p(1) = X_p(1 \cdot 1) = X_p(1) \cdot 1 + 1 \cdot X_p(1) = 2 \cdot X_p(1),
  \end{align*}
  also $X_p(1) = 0$.
\end{proof}

\begin{bem}
  Also gilt für $f \in C^{\infty}(M)$ und $g \in C^{\infty}(U)$ direkt:
  \begin{align*}
    & \sigma g =
    \begin{cases}
      \sigma g|_U & \\
      0 & \text{ sonst }
    \end{cases},\\
    & \sigma g \in C^{\infty}(M) 
    \Rightarrow X_p(g) = X_p(\sigma g).
  \end{align*}
  Für eine Karte $\varphi \colon U \to V$ von $M$ und $p$ seien algebraische Tangentialvektoren definiert:
  \begin{align*}
    \pdifffrac{x^i}|_p \in \T_p^{\text{alg}}M, \pdifffrac{x^i}|_p(f) = \partial_i(f \circ \varphi^{-1})(\varphi(p)) = \D(f \circ \varphi^{-1})|_{\varphi(p)}e_i.
  \end{align*}
\end{bem}

% Satz 2.7
\begin{satz}
  Die Vektoren $\pdifffrac{x^1}|_p,\ldots,\pdifffrac{x^n}|_p$ bilden eine Basis von $T_p^{\text{alg}}M$.
\end{satz}

% Lemma 2.8
\begin{lemma}
  Es sei $x_0 \in \R^n$ und $g \in C^{\infty}(B_{\rho}(x_0))$.
  Dann existieren glatte Funktionen $h_i \in C^{\infty}(B_{\rho}(x_0))$ mit $h_i(x_0) = \partial_ig(x_0)$ und 
  \begin{align*}
    g(x) = g(x_0) + \sum_{i=1}^n(x^i-x_0^i)h_i(x).
  \end{align*}
\end{lemma}

\begin{proof}
  (Beweis des Satzes).\\

  Die $j$-te Komponente $\varphi^j$ der Karte ist glatt und es gilt:
  \begin{align*}
    \pdifffrac{x^i}|_p(\varphi^j) = \partial_i(\varphi^j \circ \varphi^i)(\varphi(p)) = \partial_ix^j(\varphi(p)) = \delta_i^j.
  \end{align*}

  Damit sind die Vektoren linear unabhängig.\\

  Es sei $X_p\in \T_p^{\text{alg}}M$ und $f \in C^{\infty}(M)$.
  Für $x_0=\varphi(p) \in \R^n, \ B_{\rho}(x_0) \subset \varphi(U)$ und für $g = f \circ \varphi^{-1}|_{B_{\rho}(x_0)}$ gilt mit den Bezeichnungen wie im letzten Lemma:
  \begin{align*}
    X_p(f) & = X_p(g \circ \varphi) = X_p(g(\varphi(p)) + \sum \left(\varphi^i - \varphi(p)^i)(h_i \circ \varphi) \right)\\
    & = \underbrace{X_p(g(\varphi(p)))}_{\mathclap{=0}} + \sum X_p((\varphi^i-\varphi(p)^i)(h_i \circ \varphi))\\
    & = \sum X_p(\varphi^i)(h_i\circ\varphi)(p) - X_p(\varphi(p)^i)(h_i\circ \varphi)(p) + \sum (q^i-\varphi(p)^i)(p) X_p(h_i \circ \varphi)\\
    & = \sum_{i=1}^n X_p(\varphi^i)\underbrace{(h_i \circ \varphi)(p)}_{\mathclap{=h_i(\varphi(p) = h_i(x_0) = \partial_ig(x_0) = \partial_i(f\circ \varphi^{-1})(\varphi(p)) = \pdifffrac{x^i}|_p(f)}}\\
    & = \sum_{i=1}^nX_p(\varphi^i)\pdifffrac{x^i}|_p(f).
  \end{align*}
\end{proof}

\begin{bem}
  Ist $X_p=\sum \xi^i\pdifffrac{x^i}|_p$, so gilt $\xi^i = X_p(\varphi^i)$.
\end{bem}

\begin{proof}
  (Beweis des Lemmas). Es gilt:
  \begin{align*}
    g(x) - g(x_0) = \int_0^1\difffrac{t}g(tx + (1-t)x_0)dt = \sum_{i=1}^n(x^i-x_0^i)\underbrace{\int_0^1\partial_ig(tx + (1-t)x_0) dt}_{=: h_i(x)}.
  \end{align*}
\end{proof}

% Satz 2.9
\begin{satz}[Äquivalenz der Tangentialraumbegriffe]
  Die Abbildung
  \begin{align*}
    J_p \colon \T_p^{\text{geo}}M \to \T_p^{\text{alg}}M, \ J_{p}[c](f) = \difffrac{t}|_{t=0}(f\circ c)
  \end{align*}
  ist ein linearer Isomorphismus.
\end{satz}

\begin{proof}
  Wegen
  \begin{align*}
    J_p[c](f)& = \difffrac{t}|_{t=0}(f\circ c) = \difffrac{t}|_{t=0}(f \circ \varphi^{-1} \circ \varphi \circ c)\\
    &  = \D(f \circ \varphi^{-1})|_{\varphi(p)} \difffrac{t}|_{t=0} (\varphi \circ c) = \D (f\circ \varphi^{-1})|_{\varphi(p)}A[c]
  \end{align*}
  ist $J_p = \D(\cdot)\circ A$ linear.\\

  Ist $[c] \in \Kern J_p$, so folgt aus $0 ) J_p[c](\varphi^i) = difffrac{t}|_{t=0}(\varphi^i \circ c)$, dass $\difffrac{t}|_{t=0}(\varphi \circ c) = 0$ gilt, also $[c] = 0$.\\

  Damit ist $J_p$ injektiv, also ein Isomorphismus.
\end{proof}

\begin{bem}
  \begin{enumerate}
  \item Ist $X_p = \sum \xi^i\pdifffrac{x^i}|_p$,so gilt $X_p = c(0)$ für $c(t) = \varphi^{-1}(\varphi(p) + t\xi)$.
\item Für jede glatte Kurve $c$ durch $p$ ist $\difffrac{t}|_{t=0}(\varphi \circ c)$ der Koeffizientenvektor von $\dot c(0)$ in der Basis $\pdifffrac{x^i}|_p$.
  \end{enumerate}
\end{bem}


% Satz 2.10
\begin{satz}[Transformationsverhalten bei Kartenwechsel]
  Es seien $\varphi$ und $\psi$ Karten in $M$ um $p$ und es bezeichnen $\pdifffrac{x^i}|_p$ und $\pdifffrac{y^i}|_p$ die damit assoziierten Basen von $\T_pM$. Dann gilt
  \begin{align*}
    \pdifffrac{x^i}|_p = \sum_j \partial_i(\psi^j \circ \varphi^{-1})(\varphi(p)) \pdifffrac{y^j}|_p.
  \end{align*}
Es sei $X_p = \sum \xi^i \pdifffrac{x^i}|_p = \sum \eta^i\pdifffrac{y^i}|_p$. Dann gilt:
\begin{align*}
  \eta^j = \sum \partial_i(\psi^j \circ \varphi^{-1})(\varphi(p))\xi^i \text{ bzw. }
  \eta = \D(\psi \circ \varphi^{-1})(\varphi(p))\xi.
\end{align*}
\end{satz}

\begin{proof}
  Es gelte $\pdifffrac{x^i}|_p = \sum \alpha_i^j\pdifffrac{y^j}|_p$ und nach obiger Bemerkung zum vorletzten Satz gilt:
  \begin{align*}
    \alpha_i^j = \pdifffrac{x^i}|_p(\psi^j) = \partial_i(\psi^j \circ \varphi^{-1})(\varphi(p))
  \end{align*}
\end{proof}

%%% Local Variables: 
%%% mode: latex
%%% TeX-master: "../skript-diffgeom"
%%% End: 

